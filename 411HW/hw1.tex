\documentclass[12pt, letterpaper]{article}
\date{\today}
\usepackage[margin=1in]{geometry}
\usepackage{amsmath}
\usepackage{hyperref}
\usepackage{cancel}
\usepackage{amssymb}
\usepackage{fancyhdr}
\usepackage{pgfplots}
\usepackage{booktabs}
\usepackage{pifont}
\usepackage{amsthm,latexsym,amsfonts,graphicx,epsfig,comment}
\pgfplotsset{compat=1.16}
\usepackage{xcolor}
\usepackage{tikz}
\usetikzlibrary{shapes.geometric}
\usetikzlibrary{arrows.meta,arrows}
\newcommand{\Z}{\mathbb{Z}}
\newcommand{\N}{\mathbb{N}}
\newcommand{\R}{\mathbb{R}}
\newcommand{\Q}{\mathbb{Q}}
\newcommand{\C}{\mathbb{C}}

\newcommand{\Po}{\mathcal{P}}
\newcommand{\Pro}{\mathbb{P}}
\author{Alex Valentino}
\title{411 homework}
\pagestyle{fancy}
\renewcommand{\headrulewidth}{0pt}
\renewcommand{\footrulewidth}{0pt}
\fancyhf{}
\rhead{
	Homework 1\\
	411	
}
\lhead{
	Alex Valentino\\
}
\begin{document}
\begin{enumerate}
	\item Email sent!
	\item 
	\begin{enumerate}
	
	
		\item $\bigcap_{n \in \N} (\frac{-2}{\sqrt{n}}, \frac{3}{n^2})$\\
	Since these end points both converge to 0, and since the end points aren't included, then this set is empty.
		\item $\bigcap_{n \in \N} [\frac{-4}{n},\frac{4}{n}]$\\
		Since both end points go to 0, and since the end points included, means this set exactly contains the point 0.
		\item $\bigcup_{n \in \N} (\frac{-1}{5n}, \frac{1}{n^2+n})$\\
		Since the end points are monotonically increasing and decreasing respectively, all intervals being unioned together fit inside the first interval when $n=1$.  Therefore it evaluates to the interval $(\frac{-1}{5},\frac{1}{3})$.
		\item $\bigcup_{n \in \N} [\frac{-7}{n}, \frac{8}{n}]$\\
		Same as above, since the end points are monotonically increasing and decreasing respectively, then the union is equivalent to $[-7,8]$.   	
	\end{enumerate}
	\item For $n \in \N$ compute $\sum_{k=1}^n k^2$ and $\sum_{k=1}^n k^3$.
	\begin{itemize}
		\item $\sum_{k=1}^n k^2$.  
	\end{itemize}
	\item \textit{Prove that 1 is an upper bound of the following set}
	$$
	\{\frac{nm}{m^2 + 4n^2} : n,m \in \N\}
	$$
	We will show by cases:
	\begin{itemize}
		\item Assume $n = m$.  Therefore our fraction evaluates to $\frac{1}{5}$.
		Clearly $\frac{1}{5} < 1$.  
		\item Assume $n > m$.  Therefore $\frac{n}{m}> 1$.  Thus 
		$$
		n < \frac{n^2}{m} < 4\frac{n^2}{m} < 4\frac{n^2}{m} + m.
		$$
		Therefore $1 > \frac{n}{4\frac{n^2}{m}  + m} = \frac{nm}{n^2 + 4m^2}$.
		\item Assume $n < m$.  Therefore $\frac{m}{n}> 1$.  Thus 
		$$
		m < \frac{m^2}{n} < \frac{m^2}{n} + 4n.
		$$
		Therefore $1 > \frac{m}{\frac{m^2}{n} + 4n} = \frac{nm}{m^2 + 4n^2}$
	\end{itemize}
	Therefore our set is bounded above by 1.
	\item \textit{Prove that 1 is an upper bound of the following set}
	$
	\{\frac{nmk^2}{3n^3 + 9m^3 + k^6}\}
	$\\
	
	
	\item \textit{Let $A = \{1, 2, 3, 4, 5\}$. Verify if the following relations are equivalence relations:}
	
	
	\begin{enumerate}
		\item $R_1 = \{(1, 2), (2, 3), (3, 4), (4, 5), (5, 1)\}$\\
		
		$R_1$ is not an equivalence relation because no element is related to itself.
		
		\item $
		R_2 = \{(1, 1), (1, 3), (1, 4), (2, 2), (2, 5), (3, 1), (3, 3), (3, 4), (4, 1), (4, 3), (4, 4), (5, 2), (5, 5)\}		
		$\\
		
		 $R_2$ is an equivalence relation, every element is related to itself, and it is transitive.\\  The equivalence classes are $[1] \equiv [3] \equiv [4]$ and $[5] \equiv [2]$.
	\end{enumerate}
	\item \textit{Suppose $R_1$ and $R_2$ are equivalence relations on a set $A$. Define the relation $R$ on $A$ by $xRy$ if
$xR_1y$ and $xR_2y$. Prove that $R$ is an equivalence relation.}
	\begin{itemize}
		\item For all $x \in A$, $xR_1x$ and $xR_2x$.  Therefore $xRx$
		\item Suppose $xRy$.  Therefore $xR_1y$ and $xR_2y$.  Since both are equivalence relations then $yR_1x$ and $y R_2 x$.  Therefore $yRx$.
		\item Suppose $xRy$ and $yRz$.  Therefore $xR_1y, yR_1z$ and $xR_2y, yR_2z$.  Therefore $xR_1z, xR_2z$ since both relations are equivalence relations.
		Therefore by the definition of $R$, $xRz$.    
	\end{itemize}
	\item 
	\item 
	\item The proof that horses are all the same color fails is when one considers the $n=2$ case.  If we have two horses in a set, $\{h_1, h_2\}$, we know that the singleton sets of the horses have one color trivially.  However this does not imply that the singleton sets of horses have the same color.  Therefore the set of two horses is not guaranteed to have all the same color.   
	\item It is not a mathematical proof because the contradiction 
\end{enumerate}
\end{document}
