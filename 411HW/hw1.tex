\documentclass[12pt, letterpaper]{article}
\date{\today}
\usepackage[margin=1in]{geometry}
\usepackage{amsmath}
\usepackage{hyperref}
\usepackage{cancel}
\usepackage{amssymb}
\usepackage{fancyhdr}
\usepackage{pgfplots}
\usepackage{booktabs}
\usepackage{pifont}
\usepackage{amsthm,latexsym,amsfonts,graphicx,epsfig,comment}
\pgfplotsset{compat=1.16}
\usepackage{xcolor}
\usepackage{tikz}
\usetikzlibrary{shapes.geometric}
\usetikzlibrary{arrows.meta,arrows}
\newcommand{\Z}{\mathbb{Z}}
\newcommand{\N}{\mathbb{N}}
\newcommand{\R}{\mathbb{R}}
\newcommand{\Q}{\mathbb{Q}}
\newcommand{\C}{\mathbb{C}}

\newcommand{\Po}{\mathcal{P}}
\newcommand{\Pro}{\mathbb{P}}
\author{Alex Valentino}
\title{411 homework}
\pagestyle{fancy}
\renewcommand{\headrulewidth}{0pt}
\renewcommand{\footrulewidth}{0pt}
\fancyhf{}
\rhead{
	Homework 1\\
	411	
}
\lhead{
	Alex Valentino\\
}
\begin{document}
\begin{enumerate}
	\item Email sent!
	\item 
	\begin{enumerate}
	
	
		\item $\bigcap_{n \in \N} (\frac{-2}{\sqrt{n}}, \frac{3}{n^2})$\\
	Since these end points both converge to 0, and since the end points aren't included, then this set is empty.
		\item $\bigcap_{n \in \N} [\frac{-4}{n},\frac{4}{n}]$\\
		Since both end points go to 0, and since the end points included, means this set exactly contains the point 0.
		\item $\bigcup_{n \in \N} (\frac{-1}{5n}, \frac{1}{n^2+n})$\\
		Since the end points are monotonically increasing and decreasing respectively, all intervals being unioned together fit inside the first interval when $n=1$.  Therefore it evaluates to the interval $(\frac{-1}{5},\frac{1}{3})$.
		\item $\bigcup_{n \in \N} [\frac{-7}{n}, \frac{8}{n}]$\\
		Same as above, since the end points are monotonically increasing and decreasing respectively, then the union is equivalent to $[-7,8]$.   	
	\end{enumerate}
	\item For $n \in \N$ compute $\sum_{k=1}^n k^2$ and $\sum_{k=1}^n k^3$.\\
	We will be using the fact that $\sum_{k=1}^n k = \frac{n(n+1)}{2} = S_1$ 
	without proof.  
	\begin{itemize}
		\item $\sum_{k=1}^n k^2 = S_2$\\
		Consider the telescoping series $n^3 = \sum_{k=1}^n k^3 - (k-1)^3$.
		If we expand the sum on the right hand side we get
		$$
		n^3 = \sum_{k=1}^n k^3 - (k-1)^3 = \sum_{k=1}^n k^3 - k^3 +3k^2 - 3k + 1
		= 3*S_2 - 3*S_1 + n.  
		$$
		Therefore if we do some rearranging:
		\begin{align*}
			n^3 &=3*S_2 - 3*S_1 + n\\
			n^3 + \frac{3}{2}n(n+1) -n &= 3*S_2\\
			\frac{n^3}{3} + \frac{n(n+1)}{2} - \frac{n}{3}  &= S_2\\
			\frac{2n^3 + 3n^2 + 3n-2n}{6}&= \\
			\frac{n(2n^2 + 3n + 1)}{6}&=\\
			\frac{n(2n^2 + 3n + 1)}{6}&=\\
			\frac{n(n+1)(2n+1)}{6}
		\end{align*}
		\item $\sum_{k=1}^n k^3 = S_3$\\
		Consider the telescoping series $n^4 = \sum_{k=1}^n k^4-(k-1)^4$. 
		If we expand the sum on the right hand side we get\\
		$$n^4 = \sum_{k=1}^n k^4 - (k-1)^4 = \sum_{k=1}^n k^4-k^4 + 4k^3-6k^2 + 4k - 1 = 4S_3 - 6S_2 + 4S_1 - n.$$
		Therefore if we do some rearranging:
		\begin{align*}
			4S_3 - 6S_2 + 4S_1 - n &= n^4\\
			4S_3 &= n^4 + 6S_2 - 4S_1 + n\\
			S_3 &= \frac{n^4}{4} + \frac{3}{2}S_2 - S_1 + \frac{n}{4}\\
			&= \frac{n^4 + 2n^3 + 3n^2 + n - n^2 - n + n}{4}\\
			&= \frac{n^2(n^2+2n+1)}{4}\\
			&= \frac{n^2(n+1)^2}{4}\\
			&= \left(\frac{n(n+1)}{2}\right)^2
		\end{align*}
	\end{itemize}
	\item \textit{Prove that 1 is an upper bound of the following set}
	$$
	\{\frac{nm}{m^2 + 4n^2} : n,m \in \N\}
	$$
	We will show by cases:
	\begin{itemize}
		\item Assume $n = m$.  Therefore our fraction evaluates to $\frac{1}{5}$.
		Clearly $\frac{1}{5} < 1$.  
		\item Assume $n > m$.  Therefore $\frac{n}{m}> 1$.  Thus 
		$$
		n < \frac{n^2}{m} < 4\frac{n^2}{m} < 4\frac{n^2}{m} + m.
		$$
		Therefore $1 > \frac{n}{4\frac{n^2}{m}  + m} = \frac{nm}{n^2 + 4m^2}$.
		\item Assume $n < m$.  Therefore $\frac{m}{n}> 1$.  Thus 
		$$
		m < \frac{m^2}{n} < \frac{m^2}{n} + 4n.
		$$
		Therefore $1 > \frac{m}{\frac{m^2}{n} + 4n} = \frac{nm}{m^2 + 4n^2}$
	\end{itemize}
	Therefore our set is bounded above by 1.
	\item \textit{Prove that 1 is an upper bound of the following set}
	$
	\{\frac{nmk^2}{3n^3 + 9m^3 + k^6} : n,m,k \in \N \}
	$\\
	We will work by cases, and using the substitution $q = k^2$:
	\begin{itemize}
		\item Assume $n,m \leq q$.  Then $1 \leq \frac{q}{m},\frac{q}{n}$.
		Therefore
		$$
		q \leq q(\frac{q}{n})(\frac{q}{m}) \leq q(\frac{q}{n})(\frac{q}{m}) + 9(\frac{m^2}{qn}) + 3\frac{n^2}{mq}.
		$$
		Therefore $1 \geq \frac{q}{q(\frac{q}{n})(\frac{q}{m}) + 9(\frac{m^2}{qn}) + 3\frac{n^2}{mq}} = \frac{nmq}{3n^3 + 9m^3 + q^3}$ for the given condition.
		\item Assume $n,q \leq m$. Then $1\leq \frac{m}{q},\frac{m}{n}$.
		Therefore
		$$
		m \leq 9 m \frac{m^2}{qn} \leq 9 m \frac{m^2}{qn} + 3\frac{n^2}{qm} + \frac{q^2}{mn}.
		$$
		Therefore $\frac{m}{9 m \frac{m^2}{qn} + 3\frac{n^2}{qm} + \frac{q^2}{mn}} = \frac{nmq}{3n^3 + 9m^3 + q^3} \leq 1$ for the given condition.  
		\item Assume $m,q \leq n$.  Then $\frac{n}{q},\frac{n}{m} \geq 1$.
		Therefore
		$$
		n \leq 3n\frac{n}{q}\frac{n}{m} \leq 3n\frac{n}{q}\frac{n}{m} + 9\frac{m^2}{nq} + \frac{q^2}{mn}.
		$$
		Therefore $1 \geq \frac{n}{3n\frac{n}{q}\frac{n}{m} + 9\frac{m^2}{nq} + \frac{q^2}{mn}} = \frac{mnq}{3n^3 + 9m^3 + q^3}$
	\end{itemize}
	Since all of the conditions encompass all possible elements in the set, 1 is an upper bound.  
	
	\item \textit{Let $A = \{1, 2, 3, 4, 5\}$. Verify if the following relations are equivalence relations:}
	
	
	\begin{enumerate}
		\item $R_1 = \{(1, 2), (2, 3), (3, 4), (4, 5), (5, 1)\}$\\
		
		$R_1$ is not an equivalence relation because no element is related to itself.
		
		\item $
		R_2 = \{(1, 1), (1, 3), (1, 4), (2, 2), (2, 5), (3, 1), (3, 3), (3, 4), (4, 1), (4, 3), (4, 4), (5, 2), (5, 5)\}		
		$\\
		
		 $R_2$ is an equivalence relation, every element is related to itself, and it is transitive.\\  The equivalence classes are $[1] \equiv [3] \equiv [4]$ and $[5] \equiv [2]$.
	\end{enumerate}
	\item \textit{Suppose $R_1$ and $R_2$ are equivalence relations on a set $A$. Define the relation $R$ on $A$ by $xRy$ if
$xR_1y$ and $xR_2y$. Prove that $R$ is an equivalence relation.}
	\begin{itemize}
		\item For all $x \in A$, $xR_1x$ and $xR_2x$.  Therefore $xRx$
		\item Suppose $xRy$.  Therefore $xR_1y$ and $xR_2y$.  Since both are equivalence relations then $yR_1x$ and $y R_2 x$.  Therefore $yRx$.
		\item Suppose $xRy$ and $yRz$.  Therefore $xR_1y, yR_1z$ and $xR_2y, yR_2z$.  Therefore $xR_1z, xR_2z$ since both relations are equivalence relations.
		Therefore by the definition of $R$, $xRz$.    
	\end{itemize}
	\item Note that the first few elements are given by
	$a_0 = 0, a_1 = 1, a_2 = 3, a_3 = 7, a_4 = 15$\\
	 We want to show by induction that $2^n - 1 = a_n$ for all $n \in \N$.  Note that for the base case $n=0$
	we have $2^0 - 1 = 1 -1 = 0$.  Thus by the principle of mathematical induction for $k \in \N$ if $k \leq n$ then $a_k = 2^k - 1$.  
	Therefore if we want to find $a_{n+1}$ we can apply the recursion formula and get $3a_{n} - 2a_{n-1}$.  Therefore by the induction hypothesis we have that $$a_n = 3(2^n - 1) - 2(2^{n-1} = (2+1)2^{n} - 3 - 2^n + 2 = 2^{n+1} + 2^{n}-2^n - 1 = 2^{n+1}.$$
	Therefore $a_n = 2^n$ for all $n \in \N$.   
	  
	\item \textit{Prove that $133 \mid 11^{n+1} + 12^{2n-1}$}.
	Note that for $n=1$, $11^2 + 12^{2-1} = 121+12 = 133$.  Therefore the base case holds.  Therefore by the principle of mathematical induction
	for all $k \in \N$ if $k \leq n$ then $133 | 11^{k+1} + 12^{2k-1}$.  
	We want to show that the condition holds for $n+1$.  Note that since
	\begin{align*}
	11^{n+2} + 12^{2n+1} &= 11^{n+2} + 12^{2n+1}\\ &= 11\cdot 11^{n+1} + 12^2 \cdot 12^{2n-1}\\ &= 11\cdot 11^{n+1} + (133 + 11) \cdot 12^{2n-1}\\
	&= 11(11^{n+1} +  12^{2n-1}) + 133 \cdot 12^{2n-1}
	\end{align*}
	
	
	then we only need to know if $133 \mid 11^{n+1} +  12^{2n-1}$.  
	Since $n \leq n$ then by the induction hypothesis $133 \mid 11^{n+1} +  12^{2n-1}$ holds.  
	
	\item The proof that horses are all the same color fails is when one considers the $n=2$ case.  If we have two horses in a set, $\{h_1, h_2\}$, we know that the singleton sets of the horses have one color trivially.  However this does not imply that the singleton sets of horses have the same color.  Therefore the set of two horses is not guaranteed to have all the same color.   
	\item It is not a mathematical proof because the contradiction is of the form $n \not \in I \Rightarrow n \in I$.  This, logically speaking, is not well defined.  Therefore there isn't a logically meaningful contradiction, invalidating the proof.  
\end{enumerate}
\end{document}
