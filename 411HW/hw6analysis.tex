\documentclass[12pt, letterpaper]{article}
\date{\today}
\usepackage[margin=1in]{geometry}
\usepackage{amsmath}
\usepackage{hyperref}
\usepackage{cancel}
\usepackage{amssymb}
\usepackage{fancyhdr}
\usepackage{pgfplots}
\usepackage{booktabs}
\usepackage{pifont}
\usepackage{amsthm,latexsym,amsfonts,graphicx,epsfig,comment}
\pgfplotsset{compat=1.16}
\usepackage{xcolor}
\usepackage{tikz}
\usetikzlibrary{shapes.geometric}
\usetikzlibrary{arrows.meta,arrows}
\newcommand{\Z}{\mathbb{Z}}
\newcommand{\N}{\mathbb{N}}
\newcommand{\R}{\mathbb{R}}
\newcommand{\Q}{\mathbb{Q}}
\newcommand{\C}{\mathbb{C}}
\newcommand{\F}{\mathbb{F}}

\newcommand{\Po}{\mathcal{P}}
\newcommand{\Pro}{\mathbb{P}}
\author{Alex Valentino}
\title{411 homework}
\pagestyle{fancy}
\renewcommand{\headrulewidth}{0pt}
\renewcommand{\footrulewidth}{0pt}
\fancyhf{}
\rhead{
	Homework 6\\
	411	
}
\lhead{
	Alex Valentino\\
}
\begin{document}
	Lemma: Suppose $f: (a,b) \to \R$ is differentiable.  Then $f$ is lipschitz 
	if and only if $f'(x) < \infty$ for all $x \in (a,b)$
	\begin{itemize}
		\item Suppose $f$ is lipschitz, differentiable.  Since $f$ is lipschitz, 
		then there exists $c \in \R$ such that $|f(x) - f(y)| \leq L|x-y|$ 
		for all $x,y \in (a,b)$.  Then if we consider $x\neq y$ then we have 
		a bound on the quotient function for $f$: 
		$$
		\left| \frac{f(x) - f(y)}{x-y} \right| \leq L
		$$
		Therefore for a point $p \in (a,b)$ we can choose $x = p+h, y = p$, and 
		take the limit as $h \to 0$.  
		$$
		\lim_{h \to 0} \left| \frac{f(p+h) - f(p)}{h} \right| \leq L.
		$$
		Note that for every $h$, the quotient function satisfies the inequality,
		thus in the limit by the limit order theorem we have 
		that $|f'(p)| \leq L$.  Thus $f'$ is bounded on $(a,b)$.
		\item Suppose $f'$ is bounded.  
		Then for all $x \in (a,b), |f'(x)| \leq M$.  We will show that $f$ is 
		Lipschitz.  Suppose $l,r \in (a,b)$.  Then we know by the mean value 
		theorem that there exists $q \in (l,r)$ such that $\frac{f(r) - f(l)}{l -r} = f'(q)$.  Thus $\left|\frac{f(r) - f(l)}{r -l} \right| = |f'(q)| \leq M$.
		Thus $|f(r) - f(l) | \leq M|l-r|$, giving us that $f$ has lipschitz 
		constant $M$.  
		
	\end{itemize}
\begin{enumerate}
	\item 
	\begin{enumerate}
		\item 
		\begin{align*}
			\sum_{n=0}^\infty \frac{1}{(3n-2)(3n+1)} &= \frac{1}{3}\left(\sum_{n=0}^\infty \frac{1}{3n-2} -\sum_{n=0}^\infty \frac{1}{3n+1}\right)\\
			&= \lim_{s \to \infty }\frac{1}{3}\left(\sum_{n=0}^s \frac{1}{3n-2} -\sum_{n=0}^s \frac{1}{3n+1}  \right)\\
			&= \lim_{s \to \infty } \frac{1}{3}\left(\sum_{n=1}^s \frac{1}{3n-2} -\sum_{n=0}^{s-1} \frac{1}{3n+1} -\frac{1}{2} - \frac{1}{3s+1} \right)\\
			&= \lim_{s \to \infty } \frac{1}{3}\left(\sum_{m=0}^{s-1} \frac{1}{3m+1} -\sum_{n=0}^{s-1} \frac{1}{3n+1} -\frac{1}{2} - \frac{1}{3s+1} \right)\\
			&= \lim_{s \to \infty } \frac{-1}{3}\left( \frac{1}{2} + \frac{1}{3s+1}\right)\\
			&= \frac{-1}{6}
		\end{align*}
		Thus $\sum_{n=1}^\infty \frac{1}{(3n-2)(3n+1)} = \sum_{n=0}^\infty \frac{1}{(3n-2)(3n+1)} - \left( \frac{-1}{2} \right) = \frac{1}{2} - \frac{1}{6}
		= \frac{1}{3}$
		\item 
		\begin{align*}
			\sum_{n=1}^\infty \frac{2n+1}{n^2(n+1)^2} &= 
			\sum_{n=1}^\infty \frac{1}{n^2} - \frac{1}{(n+1)^2}\\
			&= \lim_{s \to \infty} \sum_{n=1}^s\frac{1}{n^2} - \frac{1}{(n+1)^2}\\
			&= \lim_{s \to \infty} \sum_{n=1}^s\frac{1}{n^2} -\sum_{n=1}^s\frac{1}{(n+1)^2}\\
			&= \lim_{s \to \infty} 1 + \sum_{n=2}^s\frac{1}{n^2} - \frac{1}{(s+1)^2}- \sum_{n=1}^{s-1}\frac{1}{(n+1)^2}\\
			&=\lim_{s \to \infty} 1 - \frac{1}{(s+1)^2} + \sum_{n=2}^s\frac{1}{n^2}- \sum_{m=2}^{s}\frac{1}{(m)^2}\\
			&= \lim_{s \to \infty} 1 - \frac{1}{(s+1)^2}\\
			&= 1
 		\end{align*}
	\end{enumerate}
	\item 
	\begin{enumerate}
		\item Clearly $\lim \frac{\sqrt{n}}{2\sqrt{n} - 1} = \frac{1}{2}$.  
		Thus the series does not converge
		\item Note that for $n \geq 3, n^3 - 5n + 1 \leq n^3$.  
		Thus $\frac{n^2}{n^3-5n+1}\geq \frac{1}{n}$.  Since $\sum_{n=0}^\infty \frac{1}{n}$ diverges then the sum diverges.
		\item Note that $\sqrt{n^2 + 1} - n = \frac{n^2 + 1 - n^2}{\sqrt{n^2 + 1} + n} = \frac{1}{\sqrt{n^2 + 1} + n}$.  Additionally $(\sqrt{2}+1)n = \sqrt{n^2 + n^2} + n \geq \sqrt{n^2+1}+n$.
		Therefore $\frac{1}{(\sqrt{2}+1)n} \leq \frac{1}{\sqrt{n^2 + 1} + n}$.
		Thus the sum diverges since $\sum \frac{1}{(\sqrt{2}+1)n}$ diverges.
		\item Note that by the ratio test $\lim \frac{2022 \cdot 2022^n}{(n+1)!} \frac{n!}{2022^n} = \lim \frac{2022}{n+1} = 0$, which being less than 1 ensures
		that the sum will converge.  
		\item Note that $\sqrt{n} + n \leq n + n$.  Thus 
		$\frac{1}{2n} \leq \frac{1}{\sqrt{n} + n}$.  Since $\frac{1}{2}\sum \frac{1}{n}$ is the harmonic series then it diverges.  Thus the sum diverges.  
		\item We know by the cauchy condensation test that $\sum \frac{1}{n\sqrt{n}}$
		converges if $\sum 2^n \frac{1}{2^n 2^{n/2}}$ converges.  After 
		algebraic manipulation we have the sum $\sum \frac{1}{\sqrt{2}^n}$.  Since
		$\sqrt{2} > 1$ then $\frac{1}{\sqrt{2}} < 1$.  Thus that geometric series 
		converges.  Thus $\frac{1}{n \sqrt{n}}$ converges.  
		\item Note that $\sqrt{n^6 + n} - n^3 = \frac{n}{\sqrt{n^6 + n} + n^3} = 
		\frac{1}{\sqrt{n^4 + \frac{1}{n}} + n^2}$.  Note that since 
		$n^2 \leq \sqrt{n^4 + \frac{1}{n}} + n^2$ then $\frac{1}{\sqrt{n^4 + \frac{1}{n}} + n^2} \leq \frac{1}{n^2}$.  Thus since $\sum \frac{1}{n^2}$ converges
		then our desired sum converges. 
		\item By the root test we have that $\lim \frac{2}{3}(\sqrt[n]{n})^2$.  
		Note that we proved in the slides that $\lim \sqrt[n]{n} = 1$.  Thus 
		the ratio test has that the sum converges.  
		\item Note that for $x \not \in \Z$ we have for all $n \in \N, x + n \neq 0$.  Thus the function $\frac{1}{x+n}$ is well defined for all $n \in \N$.  
		Note that since $\lim \frac{1}{x+n} = 0$ then we have by the alternating 
		series test that our desired sum converges
		\item By root test we have the limit $\lim \frac{(n!)^{\frac{2}{n}}}{2^n}$.  Note that since we have a ratio of positive numbers that our limit is bounded 
		below by 0.  Additionally since $\sqrt[n]{n!}\leq n$, then our limit is 
		bounded above by $\lim \frac{n^2}{2^n}$.  However by the leacture slides 
		we know that $\lim \frac{n^2}{2^n} = 0$ since exponentials are faster than
		any polynomial.  Thus our limit is equal to 0.  Thus our series converges 
	\end{enumerate}
	\item 
	\begin{enumerate}
		\item Note that since $[1,2]$ and $[4,7]$ are both closed, and the finite
		union of closed sets is closed implies $[1,2] \cup [4,7]$ is closed.
		$cl(A) = [1,2] \cup [4,7], int(A) = (1,2) \cup (4,7)$
		\item Note that $A = [0,1) \cup (4,5)$ is neither closed nor open since 
		$A$ contains a limit approaching 4 without having 4 and for any $r > 0$,
		$B(0,r) \not \subseteq [0,1)$.\\
		$cl(A) = [0,1] \cup [4,7], int(A) = (0,1) \cup (4,7)$.  
		\item $A = \Q$ is neither closed nor open since 
		the set $\{q^2 < 2: q \in \Q\} \subset \Q$, 
		however $\sqrt{2} \not \in \Q$.
		Additionally for any $q \in \Q, r > 0$ 
		we know that $B(q,r) \not \subset \Q$ since the set of irrationals is 
		dense in $\R$.  Furthermore, $int(\Q)  = \emptyset, cl(\Q) =\R$
		\item $A = \Q \cap [0,1]$ is nether closed nor open.  Note that $A$ 
		is not closed since\\ $\{q \in \Q : 0< q, q^2 < \frac{1}{2}\} \subset A$,
		however $\frac{1}{\sqrt{2}} \not \in \Q \cap [0,1]$.  Additionally, 
		for every $r > 0, B(0,r) \not \subset A$, thus $A$ is not open.  
		$cl(A) = [0,1], int(A) = \emptyset$
		\item Let $A = [0,3]\backslash \{\frac{2n+1}{3n} : n \in \N\}$.  
		$A$ is not open since for all $r > 0, B(3,r) \not \subseteq A$.  $A$ is 
		not closed since we can find a sequence which converges to 1, however 
		$1 \not \in A$.  Thus $A$ does not contain one of it's limit points.  
		$cl(A) = [0,3], int(A) = (0,3)\backslash \{\frac{2n+1}{3n} : n \in \N\}$.
	\end{enumerate}
	\item
	\begin{enumerate}
		\item  $B = [1,2)$ is not compact because it is not closed
		\item $B = \{\frac{n+1}{n} : n \in \N\}$ is not compact because it is not
		closed because it does not contain it's limit point. 
		\item $B = \{\frac{n+1}{n} : n \in \N\} \cup \{1\}$ is compact since it is 
		bounded by $M = 2$, and it contains the one limit point $1$.
		\item $B = \{\sqrt{n+1} - \sqrt{n} : n \in \N\} \cup $ is compact since 
		the set is bounded by $M = \sqrt{2}$ and $\lim \frac{1}{\sqrt{n+1} + \sqrt{n}} = 0$ thus $B$ is closed since it contains it's limit points.  
		\item $B = \bigcup_{n \in \N} [2n,2n+1]$ is not compact since for any 
		possible bound $M$, there exists $N \in \N$ such that $M < 2N$, and thus 
		not contain the interval $[2N,2N+1]$.  
	\end{enumerate}
	\item 
	\begin{enumerate}
		\item $\partial([1,2) \cup (3,5]) = \{1,2,3,5\}$
		\item $\partial \Z = \Z$
		\item $\partial (\{1\}\cup (2,3)) = \{1,2,3\}$
		\item $\partial \Q = cl(\Q) \backslash int(\Q) = \R \backslash \emptyset = \R$
		\item Let $C$ be the cantor set.  We know that the triadic numbers are dense in the unit interval.  Therefore for an arbitrary open interval in $[0,1]$
		we can place an infinite number of triatic numbers within it.  By the 
		construction of the cantor set between any two triadyic number eventually
		an interval is removed.  Thus for an arbitrary point $\alpha \in C,$
		there does not exists $\epsilon > 0$ such that $B(\alpha,\epsilon) \subset C$.  Thus $int(C) = \emptyset$.  Since $C$ is closed then $cl(C) = C$.  Thus $\partial C = C$.  
	\end{enumerate}
	\item \begin{enumerate}
		\item $[1,2]\cup [3,4]$ is compact since it is closed (finite union of closed sets is closed) and bounded (take $M = 4$), however 2.5 is not in the set.
		\item $(1,2)$ is connected by definition, and is not closed since it is 
		open and not $\R$ or $\emptyset$.
		\item $[1,2]$ is connected, and $\partial [1,2] = \{1,2\}$.    
		\item $[1,1]$ is compact since it is trivially closed and bounded, and 
		since no open interval exists in the singleton set then $[1,1]$ is nowhere dense.  
		\item $\Z$ is nowhere dense since it's closure is $\Z$ and $\Z$ lacks any 
		open intervals contained within.  Similarly $\Z$ is not compact because
		$\Z$ is unbounded.  
	\end{enumerate}	  
	\item 
	\begin{enumerate}
		\item $E = \bigcup_{n \in \N} [3n+1,3n+2]$.  Note that the first two 
		intervals of the set are $[4,5]$ and $[7,8]$, and since the the left 
		endpoints of each successive union'd interval is increasing then there are 
		no intervals to be added between two "adjacent" intervals.  Thus 6 is 
		between $5$ and $7$, yet $6 \not \in E$.  $E$ is not connected.  
		\item $E = \{1\} \cup [2,4]$.  Note that $1.5 \not \in E,$ yet $1 < 1.5 < 2$.  Thus $E$ is not connected.
	\end{enumerate}
	\item We know from class that $\sqrt{x},\sqrt[3]{x}$ are continuous. 
	Additionally both $\sqrt{x},\sqrt[3]{x}$ are positive on $(0,1)$, thus 
	their sum is never zero on $(0,1)$.  Therefore the fraction $f(x)  =\frac{1}{\sqrt[3]{x} + \sqrt{x}}$ is continuous.  Suppose for contradiction that $f$ is 
	uniformly continuous on $(0,1)$.  If we fix an $\epsilon > 0$, then there 
	exists $\delta > 0$ such that for all $x,y \in (0,1)$ if $|x-y| < \delta$
	then $|f(x) - f(y)| < \epsilon$.  Note that if we take $\delta$ and $0<x < \delta$, then we satisfy $|x-\delta| < \delta$.  Thus $|f(x) - f(\delta)|<\epsilon$.  Note that if we take $\lim_{x \to 0} f(x) = \infty$.  Since 
	$f(\delta)$ is finite then $|f(x)|$ exceeds $\epsilon$ at some point, otherwise
	contradicting the unbounded nature of $f$ at 0.  Therefore $f$ is not 
	uniformly continuous.  
	\item Suppose $f$ is lipschitz on $(a,b)$ with lipschitz constant $L$,
	$\epsilon > 0$, $x,y \in (a,b)$,\\ If $|x-y| < \frac{\epsilon}{L}$, we have
	that 
	$$
	|f(x) - f(y)| \leq L|x-y| < L \frac{\epsilon}{L} = \epsilon.
	$$
	Thus $f$ is uniformly continuous.  
	\item We know that $f(x) = \frac{1}{x}$ is differentiable on $(0,1)$, however, 
	if we consider that $f'(x) = \frac{-1}{x^2}$, then $\lim_{x\to 0} f'(x) = - \infty$.  We know by the lemma that if $f$ has an unbounded derivative then it is 
	not lipschitz.  
\end{enumerate}
\end{document}
