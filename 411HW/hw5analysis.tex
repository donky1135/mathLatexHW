\documentclass[12pt, letterpaper]{article}
\date{\today}
\usepackage[margin=1in]{geometry}
\usepackage{amsmath}
\usepackage{hyperref}
\usepackage{cancel}
\usepackage{amssymb}
\usepackage{fancyhdr}
\usepackage{pgfplots}
\usepackage{booktabs}
\usepackage{pifont}
\usepackage{amsthm,latexsym,amsfonts,graphicx,epsfig,comment}
\pgfplotsset{compat=1.16}
\usepackage{xcolor}
\usepackage{tikz}
\usetikzlibrary{shapes.geometric}
\usetikzlibrary{arrows.meta,arrows}
\newcommand{\Z}{\mathbb{Z}}
\newcommand{\N}{\mathbb{N}}
\newcommand{\R}{\mathbb{R}}
\newcommand{\Q}{\mathbb{Q}}
\newcommand{\C}{\mathbb{C}}
\newcommand{\F}{\mathbb{F}}

\newcommand{\Po}{\mathcal{P}}
\newcommand{\Pro}{\mathbb{P}}
\author{Alex Valentino}
\title{411 homework}
\pagestyle{fancy}
\renewcommand{\headrulewidth}{0pt}
\renewcommand{\footrulewidth}{0pt}
\fancyhf{}
\rhead{
	Homework 5\\
	411	
}
\lhead{
	Alex Valentino\\
}
\begin{document}
\begin{enumerate}
	\item Let $(W,\leq)$ be a linearly order set.  
	\begin{itemize}
		\item  $\Rightarrow$ Suppose $W$ is well ordered.  We want to show that there does not 
		exists a descending chain.  Suppose for contradiction that there is a sequence $(w_n)_{n \in \N}$ where $w_n > w_{n+1}$.  Since $W$ is well order than 
		$\min (w_n)$ exists.  Since $(w_n)$ has a minimum then there exists 
		$n' \in \N$ such that for all $n \in \N, w_{n'} \leq w_n$.  Note that 
		$w_{n'+1} < w_{n'}$ by the definition of $(w_n)$.  Therefore $w_{n'+1} < w_{n'}$ and $w_{n'+1} \geq w_{n'}$.  This is a contradiction.  Therefore a descending chain does not exists.  
		\item $\Leftarrow$ Suppose $W$ is not well ordered.  We want to show that there exists 
		a descending chain.  Since $W$ is not well ordered then there exists a set
		$S$ where $S \neq \emptyset, S \subseteq W$ where $\min S$ does not exists.  Since $S$ is nonempty then there exists $x_1 \in S$.  Note that $\min \{x_1\}$ exists, therefore there exists $x_2 \in S$ such that $x_2 < x_1$.  Therefore by induction $\{x_1,\cdots,x_n\} \subset S$, since $\min \{x_1,\cdots,x_n\}$ exists.  Therefore there exists $x_{n+1} \in S$ such that $x_{n+1} < x_n$.  Therefore by induction we have constructed a descending chain.  
	\end{itemize}
	\item 
	\begin{itemize}
		\item $(\Rightarrow)$ Suppose we have the hausdorff maximal principle, and 
		we want to show zorn's lemma.  Suppose $(X,\leq)$ is a poset, and all 
		chains in $X$ are bounded.  Then
		by our original assumption there exists a chain $L \subseteq X$ which is 
		maximal.  We know since $L$ is a chain then it is bounded above by assumption.  Let $a\in X$ be an upper bound of $L$.  We know that $a \in L$ since 
		otherwise $L \cup \{a\}$ would be a chain larger than $L$, contradicting
		the maximality of $L$.  Thus $a$ must be a maximal element of $X$.  
		Note that if we have $x \in X \backslash L, a \leq x$ then $a = x$ by maximality.  Otherwise $L \cup \{x\}$ would be a chain, contradicting the maximality of $L$
		\item $(\Leftarrow)$ Suppose we have zorn's lemma, and we want to prove 
		the hausdorff maximal principle.  Suppose $(X,\leq)$ is a poset, and we 
		have the set of all chains, $\mathcal{C} \subseteq \Po (X)$.  We want to 
		show that $\mathcal{C}$ has a maximal element.  Therefore we must show 
		all chains of chains (or 2chainz, if you will) of $\mathcal{C}$ are bounded, where we have the poset generated by $(\mathcal{C},\subseteq),$ the 
		set inclusion relation.  Suppose $C \subset \mathcal{C}$ is a 2-chain.  
		We claim that it is bounded above by $C^* = \bigcup_{c \in C} c$, and 
		that $C^* \in \mathcal{C}$.  Therefore we must show for all $x,y \in C^{*}$
		that $x \leq y$ or $x \geq y$.  Note that since $x,y \in C^{*}$, then 
		there exists $C_1,C_2 \in C$ such that $x \in C_1, y \in C_2$.  
		Since $C$ is a 2-chain, then either $C_1 \subseteq C_2$ or $C_1 \supseteq C_2$.  
		WLOG assume $C_1 \subseteq C_2$.  Thus since $x,y \in C_2$, then since 
		$C_2$ is a chain then $x\leq y$ or $x \geq y$.  Therefore $C^* \in \mathcal{C}$.
		Thus every chain has an upper bound.  Thus by Zorn's lemma $\mathcal{C}$
		has a maximal element.  Thus there is a maximal chain, hausdorff's principle holds.     
	\end{itemize}
	\item Suppose $\{A_i : i \in I\}$ is a set of sets.  Let $P$ be a non-empty
	poset defined by $f \in P$ being a function with $dom(f) \subseteq I$ and 
	$f : dom(f) \to \bigcup_{i \in I} A_i$ maintaing that for every $i \in dom(f)$
	one has $f(i) \in A_i$.  $P$ is a poset with the relation $\leq$ given by 
	for every $f,g \in P$ if $f \leq g$ then $dom(f) \subseteq dom(g)$ and 
	for all $i \in dom(f)$ one has $f(i) = g(i)$.  We want to show that 
	every chain of functions is bounded in $P$.  If one considers a chain 
	in $P$, given by $(f_i)_{i \in I'}$, then one can consider the upper bound to 
	trivially be the function $f: \bigcup_{i \in I'} dom(f_i) \to \bigcup_{i \in I} A_i $.
	Since for every $x \in \bigcup_{i \in I'} dom(f_i)$, every $g \in (f_i)_{i \in I'}$ with $x$ in it's domain has the exact same value on $x$.  Thus $f$ has unique values.  Thus the chain is bounded.  Therefore there exists a maximal $h$ by zorn's lemma.
	Since $h$ has a maximal domain, then it must be $I$ itself.  Otherwise, 
	if there exists $j \in I$ such that $j \not \in dom(h)$ then there exists
	a function defined on $j$, however one could make a new function which operates 
	on both $j$ and $dom(h)$, but this would contradict the maximality of $h$.  
	Thus there exists an element in $\bigcup_{i \in I}A_i$.     
	\item Since $A$ is finite then we can enumerate $A$ with $\{a_1,\cdots,a_n\}$.
	Since we have the axiom of choice then we have the choice function 
	$h : \Po(X) \backslash \{\emptyset\} \to X$ where for every $S \in \Po(X),$
	$h(S) \in S$.  Thus we can inductively define $a_{n+1} = h(X \backslash A)$,
	$a_{n+2} = h(X \backslash (A \cup \{a_{n+1}\})$, so on and so forth.  
	Thus the function $f(x) =  \begin{cases} a_{n+k} &\text{if}
	x = a_k \\ x & \text{otws} \end{cases}$ is necessarily a bijection from 
	$X \to X\backslash A$ since all enumerated elements map 1-1 to enumerated 
	elements in $X \backslash A$ and all unenumerated elements are mapped to 
	themselves.  Thus $X$ and $X \backslash A$ have the same cardinality  
	\item 
	\begin{align*}
		\Psi[\bigcup_{j \in J} A_j] &= A \backslash g [B \backslash f[\bigcup_{j \in J} A_j]]\\
		&= A \backslash g [B \backslash \bigcup_{j \in J} f[ A_j]]\\
		&= A \backslash g [B \cap (\bigcup_{j \in J} f[ A_j])^c)\\
		&= A \backslash g [B \cap \bigcap_{j \in J} f[A_j]^c]\\
		&= A \backslash (g[B] \cap \bigcap_{j \in J} g[f[A_j]^c])\\
		&= A \cap (g[B] \cap \bigcap_{j \in J} g[f[A_j]^c])^c\\
		&= A \cap (g[B]^c \cup \bigcup_{j \in J} g[f[A_j]^c]^c)\\
		&= (A \cap g[B]^c) \cup (\bigcup_{j \in J} A \cap g[f[A_j]^c]^c)\\
		&= \bigcup_{j \in J} (A \cap g[B]^c) \cup (A \cap g[f[A_j]^c]^c)\\
		&= \bigcup_{j \in J} (A \cap (g[B]^c \cup g[f[A_j]^c]^c))\\
		&= \bigcup_{j \in J} (A \cap (g[B \cap f[A_j]^c]^c))\\
		&= \bigcup_{j \in J} (A \backslash g[B \backslash f[A_j]]))\\
		&= \bigcup_{j \in J} \Psi[A_j]
	\end{align*}
	\item 
	\begin{itemize}
		\item We will show that $\Phi: \Po(\N) \to \R$ given by $\Phi(A) = \sum_{n \in A} \frac{2}{3^n}$ is an injection.
		Suppose $A,B \in \Po(\N), A\neq B$.  Since $A \neq B$ then the symmetric difference $A \Delta B$ is nonempty.  
		Since $A \Delta B \subseteq \N$, then $A \Delta B$ has a minimal element.  Let this element be denoted $m$, and 
		WLOG assume $m \in A$.  If we consider $\Phi(A) - \Phi(B) = \sum_{n \in A} \frac{2}{3^n}-\sum_{n \in B} \frac{2}{3^n} = \sum_{n \in A\backslash [m-1]} \frac{2}{3^n} - \sum_{n \in B\backslash [m-1]} \frac{2}{3^n}$ since $A$ and $B$
		must share every element less than $m$, otherwise contradicting $m$ is the smallest element in $A \Delta B$.
		Note that we can bound the difference below by considering $\sum_{n \in A \backslash [m-1]} \frac{2}{3^n} \geq \frac{2}{3^m}, -\sum_{n \in B \backslash [m-1]}\frac{2}{3^n}\geq -\sum_{n \in \N \backslash [m]}\frac{2}{3^n}$.  \\
		Since $-\sum_{n \in \N \backslash [m]}\frac{2}{3^n} = \frac{1}{3^m}$,
		then the difference $\Phi(A) - \Phi(B) \geq \frac{1}{3} > 0$.  Thus 
		$\Phi(A) > \Phi(B), \Phi(A) \neq \Phi(B)$.  Thus $\Phi$ is injective
		\item Suppose $x, y \in \R, x \neq y$.  Want to show that $\Psi(x) \neq \Psi(y)$.  WLOG assume that $x < y$.  Then by the density of $\Q$ in $\R$ 
		there exists $q \in \Q$ such that $x < q < y$.  Since $\Psi(x) = \{r \in \Q : r < x\}$ then by definition $q \not \in \Psi(x)$.  However since $q <y$
		and $q \in \Q$ then by definition $q \in \Psi(y)$.  Thus $\Psi(x) \neq \Psi(y)$.  
	\end{itemize}
	\item Since $\N$ has the same cardinality of $\Q$ we will consider 
	subsets of $\Q$ in place of subsets of $\N$.  The infinite family 
	of subsets of $\Q$ we will consider is the set of arbitrarily chosen 
	ration cauchy sequences of a given number.  
	$\mathcal{A} = \{(x_n) \in \Q^n: \exists ! r \in \R, x_n \to r\}$.  
	Since each element uniquely corresponds to a single real number, then $\mathcal{A}$ has the same cardinality of $\R$.  Note for any $(x_n), (y_n)$,
	since they converge to $x,y \in \R$, then by the topological definition of 
	convergence we can choose $\epsilon = \frac{x-y}{2}$ in which infinitely 
	many terms from each sequence would be in disjoint $\epsilon$ neighborhoods around 
	$x,y$ respectively.  Thus any two elements have a finite intersection.
	\item Note that the set of interval with rational endpoints corresponds to the 
	set 
	$$
	\bigcup_{a \in \Q} [a,a] \cup \bigcup_{a,b \in \Q, a < b} \{(a,b), (a,b],[a,b), [a,b]\}.
	$$
	Since each one of these has indexing over $\Q$ or $\Q^2$ then it's a countable
	union of countable sets.  Thus it is countable.  
	\item Suppose $\{B(x,r)\}$ is a set of disjoint balls.  
	Then uniquely for each $B(x,r)$, since $x-r < x+r$ then by the density of 
	$\Q$ in $\R$ there exists $q \in \Q$ such that $q \in B(x,r)$.  Since 
	the balls are disjoint then there is a unique rational number within each ball.
	Since $\Q$ is countable then the set of balls is countable.  
	\begin{enumerate}
		\item We claim that $card((\{0,1\}^\N)^\N) = card(\{0,1\}^{\N\times \N})$.
		Note that $(\{0,1\}^\N)^\N$ is the set of of sequences of infinite binary sequences.  Therefore for a given $f \in (\{0,1\}^\N)^\N$ we have that $f(n) = (b_{nk})_{ k\in \N}$ for all $n \in \N$.  If we define $g:(\{0,1\}^\N)^\N) \to \{0,1\}^{\N\times \N}$ by $g(f) = (b_{nk})_{(n,k) \in \N^2}$, then this is clearly a bijection.  Thus $card((\{0,1\}^\N)^\N) = card(\{0,1\}^{\N\times \N})$.
		Therefore: $$card(\R^\N) = card((\{0,1\}^\N)^\N) = card(\{0,1\}^{\N\times \N}) = card(\{0,1\}^{\N}) = card(\R)$$.    
		\item Let $S$ be some countable set, and let $X = \{S^n : n \in \N\}$.  We want to show that $X$ is countable.  
		Note that $S$ is countable, and therefore $S^n$ is countable by slide 22 of lecture 14.  Since $S^n$ and $\N$ is countable, then $\bigcup_{n \in \N}^\infty S^n$ is countable.  Since $\bigcup_{n=1}^\infty S^n = X$, then we're done.  
		\item Note that a polynomial is uniquely determined by it's coefficients.
		Therefore the set of polynomials over $\Z$ has the same cardinality as all of the finite integer sequences  Thus $card(\Z[x]) = card(\{\Z^n: n \in \N\}).$  
		Since $\{\Z^n: n \in \N\}$ is countable then $\Z[x]$ is countable
		\item Note that since $\Z[x]$ is countable and for $p \in \Z[x]$ the set
		$r(p) = \{p(x) = 0 : x \in \R \}$ is finite, then $\bigcup_{p \in \Z[x]} r(p)$ is countable.  Note that this is exactly the set of algebraic numbers.  Additionally, since we've found a countable subset of the real numbers, then 
		there are an uncountable number of numbers not in our set.  Thus real algebraic numbers exists \\ 
		\item Suppose for contradiction that there is a finite number of prime numbers.  Let the set of primes be denoted
		$\{p_1,\cdots,p_n\}$.  Consider the number $l = 1 + \prod_{i=1}^n p^i$.  Note that for each prime $p_i, l \equiv 1 \mod{p_i}$.  Thus $l$ is divisible by none of the prime numbers.  Since $l$ can't be divide by primes, it can't be divide by the product of any of the primes.  Thus $l$ is only divisible by 1 and itself.  Thus $l$ is prime.  This contradicts the fact that  $\{p_1,\cdots,p_n\}$ is the set of all primes.  Thus there is an infinite number of primes.
	\end{enumerate}
\end{enumerate}
\end{document}
