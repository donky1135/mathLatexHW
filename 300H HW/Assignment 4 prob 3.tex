\documentclass[12pt, letterpaper]{article}
\date{\today}
\usepackage[margin=1in]{geometry}
\usepackage{amsmath}
\usepackage{hyperref}

\usepackage{amssymb}
\usepackage{fancyhdr}
\usepackage{pgfplots}
\usepackage{booktabs}
\usepackage{pifont}
\usepackage{amsthm,latexsym,amsfonts,graphicx,epsfig,comment}
\pgfplotsset{compat=1.16}
\usepackage{xcolor}
\usepackage{tikz}
\usetikzlibrary{shapes.geometric}
\usetikzlibrary{arrows.meta,arrows}
\newcommand{\Z}{\mathbb{Z}}
\newcommand{\N}{\mathbb{N}}
\newcommand{\R}{\mathbb{R}}
\newcommand{\Po}{\mathcal{P}}

\author{Alex Valentino}
\title{Assignment 4}
\pagestyle{fancy}
\renewcommand{\headrulewidth}{0pt}
\renewcommand{\footrulewidth}{0pt}
\fancyhf{}
\rhead{
	Assignment 4 problem 3\\
	300H	
}
\lhead{
	Alex Valentino\\
}
\begin{document}
Below several pairs of predicates are given. For each pair do the following:
\begin{itemize}
	\item Identify the atomic assertions common to each pair of assertions and assign a variable to each of these assertions.
	\item  Find logical expressions for each sentence in terms of the variables.
	\item Determine whether the first can be logically deduced from the second, and whether the second can be logically deduced from the first. Explain your answers.
\end{itemize}
\begin{enumerate}
	\item \begin{enumerate}
	\item if $n$ is prime or $n+2$ is prime, then $n^2+2$ is prime or $n^2-2$ is prime
	\item $n^2+2$ is non-prime and $n^2-2$ is non-prime implies $n$ is non-prime and $n+2$ is non-prime.
	\end{enumerate}
	$A := n$ is prime.\\
	$B := n + 2$ is prime.\\
	$C := n^2 + 2$ is prime.\\
	$D := n^2 - 2$ is prime.\\
	\begin{enumerate}
	\item $(A \vee B) \Rightarrow (C \vee D)$
	\item $\neg C \wedge \neg D \Rightarrow \neg A \wedge \neg B$ 
	\end{enumerate}
	$(A \vee B) \Rightarrow (C \vee D) \leftrightarrow \neg (C \vee D) \Rightarrow \neg (A \vee B)$ (Modus Tollens)\\
	$\neg C \wedge \neg D \Rightarrow \neg A \wedge \neg B \leftrightarrow \neg (C \vee D) \Rightarrow \neg (A \vee B)$ (DeMorgan's rule)\\
	Therefore the two statements are logically equivalent.
	\item \begin{enumerate}
		\item  For all real numbers $x$, there is a real number $y$ such that $y^2+y+10 x=0$ or $x \leq 9$ and there is a real number $z$ such that $z^2+2 z+15 x=0$.
		\item For all real numbers $x, x \leq 9$ or there is both a real number $y$ such that $y^2+y+10 x=$ 0 and a real number $z$ such that $z^2+2 z+15 x=0$.
	\end{enumerate}
	$A(x,y) := y^2 + y + 10x = 0$\\
	$B(x) := x \leq 9$\\
	$C(x,z) := z^2 + 2z + 15x = 0$\\
	\begin{enumerate}
	\item $(\forall x \in \R, \exists y \in \R, (A(x,y)\vee B(x))) \wedge (\exists z \in \R, C(x,z))$
	\item ($\forall x \in \R, B(x)) \vee (\exists y \in \R,\exists z \in \R,( A(x,y) \wedge  C(x,z)))$
	\end{enumerate}
	(b) cannot imply (a) as if $A = T, B = T, C = F$ then $T \vee (T \wedge F) = T \vee F = T \not \Rightarrow (T \vee T) \wedge F = T \wedge F = F$.  However (a) implies (b) as if $(\forall x \in \R, \exists y \in \R, (A(x,y)\vee B(x))) \wedge (\exists z \in \R, C(x,z))$ is true then $(A \wedge B)$ is true.  Since it is an or statement we can choose $\forall x \in R B(x)$ to be true.  Since the highest level logical operator in (b) is an or, then $B(x) = T$ satisfies the statement.  
	
	\item \begin{enumerate}
		\item $f(x)>y$ and $g(y)>x$ implies $f(g(y))>y$ and $g(f(x))>x$.
		\item $f(g(y)) \leq y$ implies $f(x) \leq y$, and $g(f(x)) \leq x$ implies $g(y) \leq x$. \\
		
	\end{enumerate}
	$A(x,y) := f(x) > y$\\
		$B(x,y) := g(y) > x$\\
		$C(y) := f(g(y)) > y$\\
		$D(x) := g(f(x)) > x$
		\begin{enumerate}
			\item $(A(x,y) \wedge B(x,y)) \Rightarrow (C(y) \wedge D(x))$
			\item $(\neg C(y) \Rightarrow \neg A(x,y)) \wedge (\neg D(x) \Rightarrow \neg B(x,y))$		
		\end{enumerate}
		$(a) \neg (A \wedge B) \vee (C \wedge D)$ (Implication definiton)\\
%		$(a) (\neg (A \vee B) \vee C ) \wedge (\neg (A \vee B) \vee D)$ (or distribution) \\
		$(a) ((\neg A \vee \neg B) \vee C) \wedge ((\neg A \vee \neg B) \vee D$ (DeMorgan's rule)\\
		$(a) (\neg A \vee( \neg B \vee C)) \wedge (\neg A \vee (\neg B \vee D)) $ (or associativity)\\
		$(a) (A \Rightarrow (B \Rightarrow C)) \wedge (A \Rightarrow (B \Rightarrow D))$
%		$(a) (\neg A \vee C) \wedge (\neg B \vee C)) \wedge ((\neg A \vee D) \wedge (\neg B \vee D))$ (or distribution)\\
%		$(a) ((\neg A \vee D) \wedge (\neg C \vee D)) \wedge ((\neg B \vee C) \wedge (\neg B \vee A))$ (and commutative, distribution)\\
%		$(a) (A \Rightarrow (C \wedge D)) \wedge (B \Rightarrow (C \wedge D))$ (implication definition)\\
		$(b) (A \Rightarrow C) \wedge (B \Rightarrow D)$ (Contrapositive)\\
%		$(a) (A \Rightarrow (B \Rightarrow (C \wedge D))$ (Nested Implication)\\
%		$(b) (\neg A \vee (\neg B \vee (C \wedge D)))$\\
%		$(b) (\neg A \vee ((\neg B \wedge C) \vee (\neg B \wedge D))$
	a does not logically imply b as in the case where $A$ is true and the rest are false then $(T \wedge F) \Rightarrow (F \wedge F) = F \Rightarrow F = T \neq (T \Rightarrow (F \wedge F)) \wedge (F \Rightarrow (F \wedge F)) = (T \Rightarrow F) \wedge (F \Rightarrow F) = F \wedge T = F$.  However b implies a, as if $(A \Rightarrow C) \wedge (B \Rightarrow D) = T, (A \Rightarrow C) = T, (B \Rightarrow D) = T, $ then since we already know $(A \Rightarrow C) \Rightarrow (A \Rightarrow (B \Rightarrow C)) $ from problem 2, then we have $T \wedge (A \Rightarrow T) = T \wedge (\neg A \vee T) = T \wedge T = T.$  
	
	
	\item In this pair of assertions, $S, T , V ,$ and $W$ are all sets. \begin{enumerate}
		\item $S \subseteq T$ if and only if $S \subseteq V,$ or $S \subseteq T$ if and only if $W \subseteq T$
		\item $S \subseteq T$ if and only if	($S \subseteq T$ or $W \subseteq T$).
	\end{enumerate}
	$A(S,T) := S \subseteq T$\\
	$B(S,V) := S \subseteq V$\\
	$C(T,W) := S \subseteq W$\\
	\begin{enumerate}
		\item $A(S,T) \Leftrightarrow B(S,V) \vee A(S,T) \Leftrightarrow C(T,W)$
		\item $A(S,T) \Leftrightarrow (B(S,T) \vee C(T,W))$
	\end{enumerate}
	b doesn't imply a since if $A = F, B = T, C = F,$ then $T \Leftrightarrow F \vee F \Leftrightarrow F = F \vee T = T \neq F \Leftrightarrow (T \vee F) = F \Leftrightarrow T = F.$  However it does work the other way around, if we take $(b)$ to be true then we have $A \Leftrightarrow (B \vee C).$  Since we have an or statement, then we can just take $B$ out of it, and we're left with the statement $A \Leftrightarrow B,$ which if substituted into a as true, then we get $T \vee A \Leftrightarrow C = T.$  
\end{enumerate}
 
 
\end{document}