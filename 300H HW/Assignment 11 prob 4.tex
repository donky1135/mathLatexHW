\documentclass[12pt, letterpaper]{article}
\date{\today}
\usepackage[margin=1in]{geometry}
\usepackage{amsmath}
\usepackage{hyperref}
\usepackage{cancel}
\usepackage{amssymb}
\usepackage{fancyhdr}
\usepackage{pgfplots}
\usepackage{enumitem}
\usepackage{booktabs}
\usepackage{pifont}
\usepackage{amsthm,latexsym,amsfonts,graphicx,epsfig,comment}
\pgfplotsset{compat=1.16}
\usepackage{xcolor}
\usepackage{tikz}
\usetikzlibrary{shapes.geometric}
\usetikzlibrary{arrows.meta,arrows}
\newcommand{\Z}{\mathbb{Z}}
\newcommand{\N}{\mathbb{N}}
\newcommand{\R}{\mathbb{R}}
\newcommand{\Po}{\mathcal{P}}

\author{Alex Valentino}
\title{Assignment 11}
\pagestyle{fancy}
\renewcommand{\headrulewidth}{0pt}
\renewcommand{\footrulewidth}{0pt}
\fancyhf{}
\rhead{
	Assignment 11 problem 4\\
	300H	
}
\lhead{
	Alex Valentino\\
}
\begin{document}
	Here is an algorithm that takes as input two positive integers $m$ and $n$ and
outputs an integer.  (Recall that for numbers $a,b$, $\max(a,b)$ is the maximum of $a$
and $b$ and $\min(a,b)$ is the minimum of $a$ and $b$.)
\begin{description}
\item[1] Let $g=\max(m,n)$.
\item[2] Let $s=\min(m,n)$.
\item[3] If $s$ is a divisor of $g$ then output $s$ and stop.
\item[4] Otherwise, let $r$ be the remainder when $g$ is divided by $s$.
\item[5] Change the value of $g$ to the value of $s$.
\item[6] Change the value of $s$ to the value of  $r$.
\item[7] Go to line 3.
\end{description}
Prove that this algorithm outputs the greatest common divisor of $m$ and $n$.\\


Let the algorithm applied to two numbers, $a,b$ be denoted $E(a,b).$  Since at lines $5-7$ the algorithm sets $g$ to be the minimum, and $s$ to be the remainder of the division between $m,n$, which makes it strictly less than $n$, therefore having $g=max(n,r), s = min(s,r)$ is the same as evaluating $E(n,r)$, therefore $E(m,n) = E(min(m,n),r)$.\\


	We must show that this algorithm finds the $\gcd$.  By definition of the algorithm we must show $E(m,n) = gcd(m,n).$    By the principal of mathematical induction for all $j,k \in \N$ if $j,k < max(m,n),$ then $E(j,k) = \gcd(j,k).$  Suppose $m,n\in \N$.  Without loss of generality let $m = max(m,n), n = min(m,n).$  We must show that $E(m,n) = gcd(m,n).$  We have two cases, $n \mid m, n \nmid m.$
	\begin{itemize}
		\item Assume $n \mid m.$  Then $m = np, p \in \N.$  Therefore the algorithm terminates on the first pass on the third line, and $E(m,n) = n$.  Since between $n,m$, $n$ is the largest factor, then $gcd(m,n) = n.$  Therefore $E(m,n) = gcd(m,n).$
		\item Assume $n \nmid m.$  By the quotient remainder theorem there exists $q,r \in Z, 0 < r < n$ such that $m = qn + r.$  Then applying a first pass of the algorithm lines $1-7$ to $m,n$ yields $E(m,n) = E(n,r)$.  By the induction hypothesis we have that $E(n,r) = \gcd(n,r).$  Therefore we must show $\gcd(n,m) = \gcd(n,r).$  Let $l = \gcd(n,r).$  Therefore by the definition of $\gcd, l \mid n, l \mid r.$  By definition of divisibility $l n' = n, l r' = r.$  Therefore $l(qn' + r') = qn + r = m.$  By definition of divisibility $l m' = m, m' \in \Z.$  Therefore by definition of divisibility $l \mid m.$  By definition of $\gcd(n,r)=l$ there exists $x_1, x_2 \in \Z$ such that $n x_1 + r x_2 = l.$  By definition of $r = m - nq,$ we have that $(x_1 - qx_2)n + x_2 m = l.$  Dividing both sides by $l$ yields $(x_1 - qx_2)n' + x_2 m' = 1.$  Therefore by definition $\gcd(n',m') = 1.$  Therefore $l$ is the greatest common factor between $m,n.$  Thus $\gcd(n,r) = \gcd(m,n).$   
	\end{itemize}
	Therefore the requirements have been satisfied.  
\end{document}