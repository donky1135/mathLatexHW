\documentclass[12pt, letterpaper]{article}
\date{\today}
\title{Homework \# 15 }
\usepackage[margin=1in]{geometry}
\usepackage{amsmath}
\usepackage{hyperref}

\usepackage{amssymb}
\usepackage{fancyhdr}
\usepackage{pgfplots}
\usepackage{booktabs}
\usepackage{pifont}
\usepackage{amsthm,latexsym,amsfonts,graphicx,epsfig,comment}
\pgfplotsset{compat=1.16}
\usepackage{xcolor}
\usepackage{tikz}
\usetikzlibrary{shapes.geometric}
\usetikzlibrary{arrows.meta,arrows}
\newcommand{\Z}{\mathbb{Z}}
\newcommand{\N}{\mathbb{N}}
\newcommand{\R}{\mathbb{R}}
\newcommand{\Po}{\mathcal{P}}

\author{Alex Valentino}
\title{Assignment 1}
\pagestyle{fancy}
\renewcommand{\headrulewidth}{0pt}
\renewcommand{\footrulewidth}{0pt}
\fancyhf{}
\rhead{
	300H\\	
	Alex Valentino \\
	Assignment 1 problem 5 
}
\begin{document}
	

\textit{Now, suppose $g: B \to C$ is a 1-1 function. For each of the following two statements, prove
the statement or give a counterexample.}\\
\begin{enumerate}
	\item \textit{For any two functions $h_1 : A \to B$ and $h_2 : A \to B$ if $g \circ h_1 = g \circ h_2$ then $h_1 = h_2.$}
		\begin{itemize}
			\item Proof: By definition $b_1, b_2 \in B, g(b_1) = g(b_2) \implies b_1 = b_2$.  Therefore given that $g \circ h_1 = g \circ h_2$, then $h_1 = h_2.$
		\end{itemize}
	\item For any two functions $f_1 : C \to D$ and $f_2 : C \to D$ if $f_1 \circ g = f_2 \circ g$ then $f_1 = f_2.$ 
		\begin{itemize}
			\item Counterexample: Let $f_1 : \Z \to \Z, f_1(x) = sin(\frac{\pi}{2}x), f_2 : \Z \to \Z, f_2(x) = 0, g: \Z \to \Z, g(x) = 2x.$  While $sin(\pi x) = 0$ for all $x\in \Z,$ here $sin(\frac{\pi}{2}) = 1 \neq 0.$
		\end{itemize}
\end{enumerate}


\end{document}