\documentclass[12pt, letterpaper]{article}
\date{\today}
\usepackage[margin=1in]{geometry}
\usepackage{amsmath}
\usepackage{hyperref}

\usepackage{amssymb}
\usepackage{fancyhdr}
\usepackage{pgfplots}
\usepackage{booktabs}
\usepackage{pifont}
\usepackage{amsthm,latexsym,amsfonts,graphicx,epsfig,comment}
\pgfplotsset{compat=1.16}
\usepackage{xcolor}
\usepackage{tikz}
\usetikzlibrary{shapes.geometric}
\usetikzlibrary{arrows.meta,arrows}
\newcommand{\Z}{\mathbb{Z}}
\newcommand{\N}{\mathbb{N}}
\newcommand{\R}{\mathbb{R}}
\newcommand{\Po}{\mathcal{P}}

\author{Alex Valentino}
\title{Assignment 8}
\pagestyle{fancy}
\renewcommand{\headrulewidth}{0pt}
\renewcommand{\footrulewidth}{0pt}
\fancyhf{}
\rhead{
	Assignment 8 problem 1\\
	300H	
}
\lhead{
	Alex Valentino\\
}
\begin{document}
	\textit{Recall that is $k$ is a positive integer we define the relationship $\
equiv_k$ on $\mathbb{Z}$  by $m \equiv_k n$ provided that $k|(m-n)$.
Prove that for any  integers $m_1,m_2,n_1,n_2$ if $m_1 \equiv_k m_2$ and $n_1 \
equiv_k n_2$ then
(1) $m_1+n_1 \equiv_k m_2+n_2$
and (2) $m_1n_1 \equiv_k m_2n_2$}
\begin{enumerate}
	\item Proof: Suppose $m_1,m_2,n_1,n_2 \in \Z$ and $m_1 \equiv_k m_2, n_1 \equiv_k n_2$ . We must show that $m_1+n_1 \equiv_k m_2+n_2$.  By definition of congruence mod $k$ we must show $k \mid (m_1 + n_1 - (m_2 + n_2))$.  By definition of congruence mod $k$ we have $k \mid (m_1 - m_2)$ and $k \mid (n_1 - n_2).$  By the definition of divisibility we have $a,b \in \Z$ such that $ka = m_1 - m_2, kb = n_1 - n_2.$  Adding both equations together yields $k(a+b) = m_1+ n_1 - m_2  - n_2.$  Let $c = a+b,$ due to the closure of $\Z$ under addition, $c \in \Z.$  Therefore $kc = m_1+ n_1 - (m_2  + n_2).$  Therefore by the definition of divisibility $k \mid  (m_1+ n_1 - (m_2  + n_2)).$
	\item Proof: Suppose $m_1,m_2,n_1,n_2 \in \Z$ and $m_1 \equiv_k m_2, n_1 \equiv_k n_2$.  We must show that $m_1n_1 \equiv_k m_2n_2$.   By the definition of modular arithmetic, $0 \equiv_k m_1 - m_2, 0 \equiv_k n_1 - n_2.$ Multiplying the two expressions together yields $0 \equiv_k (m_1 - m_2)(n_1 - n_2).$  Therefore, \begin{align*}
	0 &\equiv_k (m_1 - m_2)(n_1 - n_2)\\
	0 &\equiv_k m_1(n_1 - n_2)-m_2(n_1 - n_2)\\
	m_2(n_1 - n_2) &\equiv_k m_1(n_1 - n_2)\\
	m_2 n_1 - m_2 n_2 &\equiv_k m_1 n_1 - m_1 n_2\\
	m_2 n_2 - m_1 n_2 &\equiv_k m_1 n_1 - m_1 n_2\\
	m_2 n_2 &\equiv_k m_1 n_1\\
	m_1 n_1 &\equiv_k m_2 n_2.
	\end{align*}
	% Expanding $(m_1 - m_2)$ yields $0 \equiv_k m_1(n_1 - n_2) - m_2(n_1 - n_2)$.  Adding $m_2(n_1 - n_2)$ to both sides yields $m_2(n_1 - n_2) \equiv_k m_1(n_1 - n_2).$  Distributing yields 
\end{enumerate}
\end{document}