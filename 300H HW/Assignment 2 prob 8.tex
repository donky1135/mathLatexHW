\documentclass[12pt, letterpaper]{article}
\date{\today}
\usepackage[margin=1in]{geometry}
\usepackage{amsmath}
\usepackage{hyperref}

\usepackage{amssymb}
\usepackage{fancyhdr}
\usepackage{pgfplots}
\usepackage{booktabs}
\usepackage{pifont}
\usepackage{amsthm,latexsym,amsfonts,graphicx,epsfig,comment}
\pgfplotsset{compat=1.16}
\usepackage{xcolor}
\usepackage{tikz}
\usetikzlibrary{shapes.geometric}
\usetikzlibrary{arrows.meta,arrows}
\newcommand{\Z}{\mathbb{Z}}
\newcommand{\N}{\mathbb{N}}
\newcommand{\R}{\mathbb{R}}
\newcommand{\Po}{\mathcal{P}}

\author{Alex Valentino}
\title{Assignment 2}
\pagestyle{fancy}
\renewcommand{\headrulewidth}{0pt}
\renewcommand{\footrulewidth}{0pt}
\fancyhf{}
\rhead{
	Assignment 2 problem 8\\
	300H	
}
\lhead{
	Alex Valentino\\
}
\begin{document}
	\textit{(a) If $S$ is a finite set, a permutation of $S$ is a function from $S$ to itself, whose range is all
of $S$. Give an example of a permutation of $\{1, 2, 3, 4, 5\}$.}\\
	Example: Let $id_S: S \to S$ be given as $x \in S, id_S(x) = x.$  This gives the permutation of $\{1,2,3,4,5\}$.\\
	\textit{(b) Recall that if $f$ and $g$ are permutations of $S$ then $f\circ g$ is the function from $S$ to $S$ given
by the rule $f \circ g(s) = f (g(s))$ for all $s \in S$. The identity permutation is the permutation
that maps every element to itself. Give an example of two different permutations $f$ and
$g$ of $\{1, 2, 3, 4, 5\}$ such that neither is the identity permutation, and $f\circ g = g \circ f$.} \\
Let $f : S \to S$ be given as $f(S) = \{2,1,3,4,5\}$ and $g : S \to S$ be given as $g(S) = \{1,2,3,5,4\}$.  Then $f(g(S)) = \{2,1,3,5,4\}$, and $g(f(S)) = \{2,1,3,5,4\}$.\\
	\textit{(c) Give an example of two different permutations $f$ and $g$ of $\{1, 2, 3, 4, 5\}$such that neither
is the identity permutation and such that $f \circ g = g \circ f$ and $f \circ f = g \circ g$.}\\
	The previous example functions work, as $f \circ f (S) = \{1,2,3,4,5\}$ and $g \circ g (S) = \{1,2,3,4,5\}$  
	
\end{document}