\documentclass[12pt, letterpaper]{article}
\date{\today}
\usepackage[margin=1in]{geometry}
\usepackage{amsmath}
\usepackage{hyperref}

\usepackage{amssymb}
\usepackage{fancyhdr}
\usepackage{pgfplots}
\usepackage{booktabs}
\usepackage{pifont}
\usepackage{amsthm,latexsym,amsfonts,graphicx,epsfig,comment}
\pgfplotsset{compat=1.16}
\usepackage{xcolor}
\usepackage{tikz}
\usetikzlibrary{shapes.geometric}
\usetikzlibrary{arrows.meta,arrows}
\newcommand{\Z}{\mathbb{Z}}
\newcommand{\N}{\mathbb{N}}
\newcommand{\R}{\mathbb{R}}
\newcommand{\Po}{\mathcal{P}}

\author{Alex Valentino}
\title{Assignment 3}
\pagestyle{fancy}
\renewcommand{\headrulewidth}{0pt}
\renewcommand{\footrulewidth}{0pt}
\fancyhf{}
\rhead{
	Assignment 3 problem 6\\
	300H	
}
\lhead{
	Alex Valentino\\
}
\begin{document}
	\textit{In calculus 1 you learned the following theorem: Any continuous function whose domain is
a closed bounded interval $[a, b]$ attains a maximum and minimum value. Given an example
of a function whose domain is a closed bounded interval that attains neither a maximum
or a minimum value. Try to choose your example to have as few points of discontinuity as
possible.}\\
Let $f: \R / \{0\} \to \R$ be given by $f(x) = \frac{1}{x}$ and evaluate $x \in [-1,1].$  No maximum exist as you can always go up the graph approaching from the right to the asymptote.  No minimum exist as you can always go farther down the well when approaching from the left.  
\end{document}