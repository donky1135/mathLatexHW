\documentclass[12pt, letterpaper]{article}
\date{\today}
\title{Assignment 2}
\usepackage[margin=1in]{geometry}
\usepackage{amsmath}
\usepackage{hyperref}

\usepackage{amssymb}
\usepackage{fancyhdr}
\usepackage{pgfplots}
\usepackage{booktabs}
\usepackage{pifont}
\usepackage{amsthm,latexsym,amsfonts,graphicx,epsfig,comment}
\pgfplotsset{compat=1.16}
\usepackage{xcolor}
\usepackage{tikz}
\usetikzlibrary{shapes.geometric}
\usetikzlibrary{arrows.meta,arrows}
\newcommand{\Z}{\mathbb{Z}}
\newcommand{\N}{\mathbb{N}}
\newcommand{\R}{\mathbb{R}}
\newcommand{\Po}{\mathcal{P}}

\author{Alex Valentino}
\title{Assignment }
\pagestyle{fancy}
\renewcommand{\headrulewidth}{0pt}
\renewcommand{\footrulewidth}{0pt}
\fancyhf{}
\rhead{
	Assignment 2 problem 4\\
	300H	
}
\lhead{
	Alex Valentino\\
}
\begin{document}
	\textit{Give an example of an infinite sequence $(A_j : j \in \Z_{>0})$ that satisfies the three requirements:
(1) Each $A_j$ is a nonempty subset of the interval set $[-100, 100]$. \\
(2) For each $j \geq 1,A_{j + 1} \subseteq A_j.$\\
(3) There is no number that belongs to all of the sets.\\}
Let $A_j = (0,\displaystyle \frac{1}{2^j}).$  Clearly the sequence satisfies (1), $\displaystyle \frac{1}{2^j}$ is comfortably under 100 for $j \geq 1$.  For (2), since $\displaystyle \frac{1}{2^j}$ is monotonically decreasing and the left side bound of 0 doesn't change, then $\displaystyle (0,\frac{1}{2^{j+1}}) \subseteq (0,\frac{1}{2^j } )$.  For property (3), one can show that for picking any value $\epsilon \in (0,1)$ that one can choose a value for $j$ such that $\displaystyle \frac{1}{2^j} < \epsilon,$ as $j$ can be constructed to be $j > \frac{ln(\epsilon)}{ln(2)}$ therefore for all the $j-1$ intervals that contain the element $\epsilon$, $\epsilon \notin A_j$.  Therefore there does not exist a number common to all of the intervals.  
\end{document}