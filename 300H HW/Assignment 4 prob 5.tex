\documentclass[12pt, letterpaper]{article}
\date{\today}
\usepackage[margin=1in]{geometry}
\usepackage{amsmath}
\usepackage{hyperref}

\usepackage{amssymb}
\usepackage{fancyhdr}
\usepackage{pgfplots}
\usepackage{booktabs}
\usepackage{pifont}
\usepackage{amsthm,latexsym,amsfonts,graphicx,epsfig,comment}
\pgfplotsset{compat=1.16}
\usepackage{xcolor}
\usepackage{tikz}
\usetikzlibrary{shapes.geometric}
\usetikzlibrary{arrows.meta,arrows}
\newcommand{\Z}{\mathbb{Z}}
\newcommand{\N}{\mathbb{N}}
\newcommand{\R}{\mathbb{R}}
\newcommand{\Po}{\mathcal{P}}

\author{Alex Valentino}
\title{Assignment 4}
\pagestyle{fancy}
\renewcommand{\headrulewidth}{0pt}
\renewcommand{\footrulewidth}{0pt}
\fancyhf{}
\rhead{
	Assignment 4 problem 5\\
	300H	
}
\lhead{
	Alex Valentino\\
}
\begin{document}
	\textit{Let $x$ and $y$ be real variables, and let $C(x, y)$ be a predicate involving $x$ and $y$.
Consider the two scenarios:\\
	\textbf{Scenerio 1}: Input variable $x$.  Assumption: For every $y \in \R, C(x,y)$ is true.\\
	\textbf{Scenerio 2}: Input variable $y$.  Assumption: For every $x \in \R, C(x,y)$ is false.\\
	Observe that in the first scenario $y$ is a bound variable, while in the second $x$ is a bound
variable. Consider the set $S_1$ of feasible instances to scenario 1 and $S_2$ of feasible instances
to scenario 2.
Can $S_1$ and $S_2$ both be nonempty? Explain.}  \\
	If one negates $S_2, $ then you get the statement $\neg S_2(y) := \exists x \in \R, C(x,y)$ is true.  The funny thing is this statement is roughly equivalent to $S_1$ if you let $S_2$ vary with regards to $y$.  If you do, that means that $S_1$ is non-empty, there has to exist an element which satisfies.  It gets even more interesting if you consider the double negation of $S_2,$ then there doesn't exist an element for which y makes $C(x,y)$ true.  This directly contradicts the first scenerio, in which for certain $x$ ALL $y$ make $C(x,y)$ true.  Therefore if one takes both to be true, then one must be empty.
	
\end{document}