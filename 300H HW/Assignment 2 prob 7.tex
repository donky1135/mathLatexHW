\documentclass[12pt, letterpaper]{article}
\date{\today}
\usepackage[margin=1in]{geometry}
\usepackage{amsmath}
\usepackage{hyperref}

\usepackage{amssymb}
\usepackage{fancyhdr}
\usepackage{pgfplots}
\usepackage{booktabs}
\usepackage{pifont}
\usepackage{amsthm,latexsym,amsfonts,graphicx,epsfig,comment}
\pgfplotsset{compat=1.16}
\usepackage{xcolor}
\usepackage{tikz}
\usetikzlibrary{shapes.geometric}
\usetikzlibrary{arrows.meta,arrows}
\newcommand{\Z}{\mathbb{Z}}
\newcommand{\N}{\mathbb{N}}
\newcommand{\R}{\mathbb{R}}
\newcommand{\Po}{\mathcal{P}}

\author{Alex Valentino}
\title{Assignment 2}
\pagestyle{fancy}
\renewcommand{\headrulewidth}{0pt}
\renewcommand{\footrulewidth}{0pt}
\fancyhf{}
\rhead{
	Assignment 2 problem 7\\
	300H	
}
\lhead{
	Alex Valentino\\
}
\begin{document}
	\begin{itemize}
		\item \textit{Give an example of a partition of $\Z_{>0}$ that has two parts, both of which have infinite
size.}\\
	$2\N$ and $2\Z_{\geq 0} + 1$.
		\item \textit{For each positive integer $k$, describe an example of a partition of $\Z_{>0}$ into $k$ parts, all
of which have infinite size.}\\
		The positive integers can be partitioned into $k$ parts starting from $k\Z_{>0}$ and then all of the cosets $k\Z_{\geq0} + 1$ to $k\Z_{\geq0} + k - 1$.   
		\item \textit{Give an example of a partition of $\Z_{>0}$ that has an infinite number of parts, each of
which has infinite size.}\\
You can partition the positive integers into prime partitions where their inclusion is based on which prime factor has the highest exponent, and then if there are multiple factors with the same highest exponent you go with the largest prime.

	\end{itemize}
\end{document}