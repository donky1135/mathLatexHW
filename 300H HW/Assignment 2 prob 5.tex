\documentclass[12pt, letterpaper]{article}
\date{\today}
\usepackage[margin=1in]{geometry}
\usepackage{amsmath}
\usepackage{hyperref}

\usepackage{amssymb}
\usepackage{fancyhdr}
\usepackage{pgfplots}
\usepackage{booktabs}
\usepackage{pifont}
\usepackage{amsthm,latexsym,amsfonts,graphicx,epsfig,comment}
\pgfplotsset{compat=1.16}
\usepackage{xcolor}
\usepackage{tikz}
\usetikzlibrary{shapes.geometric}
\usetikzlibrary{arrows.meta,arrows}
\newcommand{\Z}{\mathbb{Z}}
\newcommand{\N}{\mathbb{N}}
\newcommand{\R}{\mathbb{R}}
\newcommand{\Po}{\mathcal{P}}

\author{Alex Valentino}
\title{Assignment 2}
\pagestyle{fancy}
\renewcommand{\headrulewidth}{0pt}
\renewcommand{\footrulewidth}{0pt}
\fancyhf{}
\rhead{
	Assignment 2 problem 5\\
	300H	
}
\lhead{
	Alex Valentino\\
}
\begin{document}
	\begin{enumerate}
		\item $(4,5,6,1,2,3)$
		\item To construct a list of arbitrary $s$ and $t$, starting with the list $a = (1,2,\cdots,st-1,st)$, then we construct $\displaystyle \bigoplus_{i=1}^{t}(a_{s(t-i)+1},\cdots,a_{s(t-i)+s})$.  This satisfies the constraints of $s+1$ automatically by having these linearly increasing chunks of length $s$, then at the $s+1$st element, a numerical cliff is hit, and sinks from $a_{s(t-i)+s}$ to $a_{s(t-(i+1))+1}$.  This is also the only time a decrease occurs in the list, satisfying the $t+1$ decreasing list rule.  
		 
	\end{enumerate}
\end{document}