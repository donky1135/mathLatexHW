\documentclass[12pt, letterpaper]{article}
\date{\today}
\usepackage[margin=1in]{geometry}
\usepackage{amsmath}
\usepackage{hyperref}
\usepackage{cancel}
\usepackage{amssymb}
\usepackage{fancyhdr}
\usepackage{pgfplots}
\usepackage{booktabs}
\usepackage{pifont}
\usepackage{amsthm,latexsym,amsfonts,graphicx,epsfig,comment}
\pgfplotsset{compat=1.16}
\usepackage{xcolor}
\usepackage{tikz}
\usetikzlibrary{shapes.geometric}
\usetikzlibrary{arrows.meta,arrows}
\newcommand{\Z}{\mathbb{Z}}
\newcommand{\N}{\mathbb{N}}
\newcommand{\R}{\mathbb{R}}
\newcommand{\Po}{\mathcal{P}}

\author{Alex Valentino}
\title{Assignment 11}
\pagestyle{fancy}
\renewcommand{\headrulewidth}{0pt}
\renewcommand{\footrulewidth}{0pt}
\fancyhf{}
\rhead{
	Assignment 11 problem 2\\
	300H	
}
\lhead{
	Alex Valentino\\
}
\begin{document}
Recall that a 3-tuple $(a,b,c)$ of natural numbers is a \emph{Pythagorean triple} provided that
$a^2+b^2=c^2$.   The purpose of this problem is to prove the following theorem:
For any $(a,b,c) \in \mathbb{N}^3$, $(a,b,c)$ is a Pythagorean triple if and only if there exist natural numbers $m,n,k$ satisfying $m > n$ and $\gcd(m,n)=1$ such that $c=(m^2+n^2)k$ and either $a=(m^2-n^2)k$ and $b=2mnk$,  or $a=2mnk$ and $b=(m^2-n^2)k$.  (You will  most likely need to use
the Fundamental Theorem of Arithmetic, as stated in Chapter 11.)\\



(a) Prove the "if" direction of the theorem.\\
Suppose $m,n,k \in \N$ such that $m > n, gcd(m,n) = 1, c = k(m^2 + n^2)$ and $a = k(m^2 - n^2), b = 2mnk$ or $a = 2mnk,b = k(m^2 - n^2)$.  We must show that $a^2 + b^2 = c^2.$  Note that since addition of natural numbers is commutative, the choice we make for $a$ and $b$ can be swapped and not affect the proof.    Therefore WLOG we assume that $a= k(m^2 - n^2)$, $b=2mnk.$ Note that since $m > n,$ then $m^2 > n^2,$ and thus we have $m^2 - n^2 > 0,$ and thus makes the product of $k$ and $m^2 - n^2$ a natural number.  Therefore by algebraic manipulation we have:
\begin{align*}
	a^2 + b^2 &= k^2(m^2 - n^2)^2 + 4 m^2 n^2 k^2\\
		&= k^2 m^4 - 2 k^2 m^2 n^2 + k^2 n^4 + 4 m^2 n^2 k^2 \\
		&= k^2 (m^4 - 2 m^2 n^2 + n^4 + 4 m^2 n^2)\\
		&= k^2 (m^2 + 2 m^2 n^2 + n^4)\\
		&= k^2 (m^2 + n^2)^2\\
		&= (k(m^2 + n^2))^2\\
		&= c^2.		
\end{align*}


The rest of the problem is for the "only if" direction.   Suppose $a,b,c$ is a Pythagorean triple.  We must show there exists $m,n$ with the desired properties.   The proof will have two lemma.  Main Lemma: if $\gcd(a,b,c)=1$ then the conclusion holds.  Secondary lemma: if the result holds whenever $\gcd(a,b,c)=1$ then it also holds
for all $a,b,c$.\\  


(b)  Prove the secondary lemma:
Assume that the result is true whenever $\gcd(a,b,c)=1$.  Use this to prove that
the result is true for all Pythagorean triples $a,b,c$.\\


Suppose $(a,b,c) \in \N^3, a^2 + b^2 = c^2$ and that for all $(x,y,z) \in \N^3$ if $gcd(a,b,c) = 1,$ then the theorem holds.  We must show there exists $m,n,k \in \N$ such that $m > n, gcd(m,n) = 1, c = k(m^2 + n^2)$ and $a = k(m^2 - n^2), b = 2mnk$ or $a = 2mnk,b = k(m^2 - n^2)$.  Suppose $a \neq 2mnk, b \neq k(m^2 - n^2).$  We must show there exists $m,n,k \in \N$ such that $m > n, gcd(m,n) = 1, c = k(m^2 + n^2)$ and $a = k(m^2 - n^2), b = 2mnk$.  Let $l$ be given by $l = gcd(a,b,c).$  Therefore $l$ divides $a,b,c$.  Therefore by definition of divisibility $a = la', b= lb', c = lc'.$  Thus by algebraic manipulation we have $l^2 a'^2 + l^2 b'^2 = l^2 c'^2, a'^2 + b'^2 = c'^2.$  Therefore $(a',b',c')$ is a Pythagorean triple.  Note that since $l = gcd(a,b,c),$ then there exists $x_1, x_2, x_3 \in \Z$ such that $x_1 a + x_2 b + x_3 c = l$, therefore dividing out $l$ to get $a',b',c'$ yields $x_1 a' + x_2 b' + x_3 c' = 1,$ which by the definition of gcd means $gcd(a',b',c') = 1.$  Therefore since $a'^2 + b'^2 = c'^2$ and $gcd(a',b',c') = 1$ then there exists $m,n,k' \in \N$ such that $m > n, gcd(m,n) = 1, c' = k'(m^2 + n^2),  a' = k'(m^2 - n^2), b' = 2mnk'.$  Let $k = lk'.$  We claim that $m,n,k$ satisfy the requirements.  Since $a = la' = l = lk'(m^2 - n^2) = k(m^2 - n^2), b = lb' = lk'2mn = k2mn, c = lc' = lk'(m^2 + n^2) = k(m^2 + n^2),$ and all of the other requirements are satisfied by $m,n$ then the requirements have been satisfied.
\\


The remaining parts of the problem are for proving the main lemma. So we assume 
$\gcd(a,b,c)=1$, and prove
that the desired $m,n$ exist. \\


(c) Prove that $c$ is odd and exactly one of $a$ and $b$ is odd. \\ 
Suppose $(a,b,c) \in \N^3, a^2 + b^2 = c^2, gcd(a,b,c) = 1.$  We must show that $c$ is odd and exactly one of $a$ and $b$ is odd.  By definition we must show $c$ is odd, and $a \equiv 1 \mod 2$ and $b \equiv 0 \mod 2$ or $a \equiv 0 \mod 2$ and $b \equiv 1 \mod 2$.  Suppose $a \not\equiv 1 \mod 2, b \not \equiv 0 \mod 2.$  We must show that $c$ is odd, and $a \equiv 0 \mod 2, b \equiv 1 \mod 2$.  By definition of not congruent, $a \equiv 0 \mod 2, b \equiv 1 \mod 2.$  Therefore we must show $c$ is odd.  By definition of odd, we must show $c \equiv 1 \mod 2.$ Note that since if $x \equiv 0 \mod 2,$ then $x^2 \equiv 0^2 = 0 \mod 2,$ and if $x \equiv 1 \mod 2$ then $x^2 \equiv 1^2 = 1 \mod 2,$ then for $\Z / 2 \Z$ $x^2 \equiv x \mod 2.$  Therefore $c^2 \equiv c \mod 2, a^2 + b^2 \equiv a + b \mod 2.$ By algebraic manipulation we have: \begin{align*}
	c &\equiv a + b\\
	&\equiv 1 + 0\\
	&= 1 \mod 2.
\end{align*}

\iffalse
Suppose for contradiction that $c$ is even.  Therefore working over $\Z / 2 \Z,$ we have that $a^2 + b^2 \equiv 0 \mod 2.$  Note that since if $x \equiv 0 \mod 2,$ then $x^2 \equiv 0^2 = 0 \mod 2,$ and if $x \equiv 1 \mod 2$ then $x^2 \equiv 1^2 = 1 \mod 2,$ then for $\Z / 2 \Z$ $x^2 \equiv x \mod 2.$  Therefore $a^2 + b^2 \equiv a + b \equiv 0 \mod 2.$  Also note working $\Z / 2 \Z,$ $-1 \equiv 2 -1 \equiv 1 \mod 2,$ thus $a+b \equiv 0 \mod 2$ implies $a \equiv -b \mod 2$ which gives us $a \equiv b \mod 2.$  Since $gcd(a,b,c) = 1,$ then by definition there exists $x,y,z \in \Z$ such that $ax + by + cz = 1.$  Looking at the parity of this equation yields,
\begin{align*}
	1 &= ax + by + cz\\
	&\equiv ax + by + 0z\\
	&= ax + by\\
	&\equiv ax + ay\\
	&= a(x+y) \mod 2.
\end{align*}
Since $1 \equiv a(x+y) \mod 2,$ then both $a \equiv 1 \mod 2, x+y \equiv 1 \mod 2.$  Since $a \equiv b \mod 2,$ then $a + b \equiv a + a \equiv 2$
\fi
(d)  Without loss of generality assume that $a$ is odd.  Prove that $\gcd(c-a,c+a)=2$.\\
	Suppose $(a,b,c) \in \N^3, a^2 + b^2 = c^2, gcd(a,b,c) = 1,$ $a$ is odd.  We must show $\gcd(c-a,c+a) = 2.$ Suppose for contradiction  Note by the previous lemma $c$ is odd.  Since $a$ is odd, then $b$ is even.  By the definition of even and odd, let $a=2q -1, b=2r, c = 2s-1,$ where $q,r,s \in \N.$
	By the definition of $\gcd$ we must show there exists $x_1 , x_2 \in \Z$ such that $(c+a)x_1 + (c-a)x_2 = 2.$  By the parity of $a,c$ we must show that $2(s+q-1)x_1 + 2(s-q)x_2 = 2.$  Dividing out by $2$ we must show $(s+q-1)x_1 + (s-q)x_2 = 1.$  Suppose for contradiction that $(s+q-1)x_1 + (s-q)x_2 > 1.$  Let $k = gcd(s+q-1, s-q).$  Then $k \mid c-a, k \mid c+a.$  Therefore by algebraic manipulation \begin{align*}
		\frac{c-a}{k}\frac{c+a}{k} &= \frac{(c-a)(c+a)}{k^2}\\
		&= \frac{c^2-a^2}{k^2}\\
		&= \frac{c^2}{k^2} - \frac{a^2}{k^2}\\
		&= (\frac{c}{k})^2 - (\frac{a}{k})^2\\
		&= \frac{b^2}{k^2}\\
		&= (\frac{b}{k})^2
\end{align*}
Since $k$ divides $a,b,c$, then $k \mid gcd(a,b,c).$  This is a contradiction as $k > 1,$ and $gcd(a,b,c) = 1$.  Therefore $gcd(c+a,c-a)=2.$\\
 

(e) Show that the required integers $m$ and $n$ exist.
Suppose $a,b,c \in \N, gcd(a,b,c) = 1, a^2 + b^2 = c^2, gcd(c+a,c-a) = 2$. We must show that there exists $m,n \in \N$ such that $m>n, gcd(m,n) = 1, a = m^2 - n^2, b=2mn,c=m^2+n^2$.  
Since $b^2 = c^2 - a^2,$ then by algebraic manipulation 
\begin{align*}
	b^2 &= c^2 - a^2\\
	&= (c+a)(c-a)\\
	b &= \frac{(c+a)(c-a)}{b}\\
	\frac{b}{c+a} &= \frac{c-a}{b}
\end{align*}
Let $\frac{n}{m}$ be given by $\frac{n}{m} = \frac{b}{c+a}$ in lowest terms.  Therefore $gcd(m,n) = 1.$  Since $\frac{m}{n} = \frac{c+a}{b}, \frac{n}{m} = \frac{c-a}{b}$ then by algebraic manipulation we have 
\begin{align*}
	2\frac{c}{b} &= \frac{m}{n} + \frac{n}{m}\\
	&= \frac{m^2 + n^2}{nm}\\
	\frac{c}{b} &= \frac{m^2 + n^2}{2nm}\\
	2\frac{a}{b} &= \frac{m}{n} - \frac{n}{m}\\
	&= \frac{m^2-n^2}{nm}\\
	\frac{a}{b} &= \frac{m^2-n^2}{2nm}
\end{align*}
	Therefore $a=m^2 - n^2, b = 2mn, c=m^2 + n^2.$  We must show that $m > n.$  Suppose for contradiction that $m \leq n.$  Then $m^2 \leq n^2, m^2 - n^2 \leq 0.$  This is a contradiction as $a = m^2 - n^2, a \in \N.$  Therefore $m>n$.  Thus the requirements have been satisifed.       
\end{document}