\documentclass[12pt, letterpaper]{article}
\date{\today}
\usepackage[margin=1in]{geometry}
\usepackage{amsmath}
\usepackage{hyperref}

\usepackage{amssymb}
\usepackage{fancyhdr}
\usepackage{pgfplots}
\usepackage{booktabs}
\usepackage{pifont}
\usepackage{amsthm,latexsym,amsfonts,graphicx,epsfig,comment}
\pgfplotsset{compat=1.16}
\usepackage{xcolor}
\usepackage{tikz}
\usetikzlibrary{shapes.geometric}
\usetikzlibrary{arrows.meta,arrows}
\newcommand{\Z}{\mathbb{Z}}
\newcommand{\N}{\mathbb{N}}
\newcommand{\R}{\mathbb{R}}
\newcommand{\Po}{\mathcal{P}}

\author{Alex Valentino}
\title{Assignment 3}
\pagestyle{fancy}
\renewcommand{\headrulewidth}{0pt}
\renewcommand{\footrulewidth}{0pt}
\fancyhf{}
\rhead{
	Assignment 3 problem 7\\
	300H	
}
\lhead{
	Alex Valentino\\
}
\begin{document}
	Below are some false universal assertions. Find counterexamples to each one. For each one,
formulate a modified universal assertion that is true.
	\begin{itemize}
		\item \textit{For all choices of four real numbers $a,b,c,d$, if $a > b$ and $c > d$ then $ab > cd$}\\
		Let $a=10, b = 1, c = 1000, d = 100$.  While $a > b, c>d,$ the assertion that $10 > 100000$ is clearly false.  A correct assertion would be  if $a > b$ and $c > d$ then $ac > bd$.
		\item \textit{For any set $\mathcal{C}$ of intervals of $\R$, if no interval in $\mathcal{C}$ is disjoint from all of the other intervals of $\mathcal{C}$ then the union of the intervals in $\mathcal{C}$ is an interval}\\
		Let $\mathcal{C} = \{A,B,C,D\}, A \cap B \neq \emptyset, C \cap D \neq \emptyset, (A \cup B) \cap (C \cup D) = \emptyset.$  By this construction every set in $\mathcal{C}$ isn't disjoint to every other set , however their union does not form as a set as the two pairs of sets above form "islands", which is non-continuous if their union is taken.  To correct this, have the requirement that every interval in $\mathcal{C}$ not be disjoint to every other interval in the set.   \\
		(note: this one stumped me just by misreading the quantifiers, I thought you were specifying the correction I made to the rule, and with this frustration I went to the group chat, where the hints I received were "this doesn't involve the empty set" and that "the minimum number of  sets required for the counter-example was four", all else is my work.)
		\item \textit{For all lists $(a_1, \cdots , a_k)$ of real numbers the average of the squares of the numbers is
greater than the square of the average.}\\
Let $a=(1,1,1)$.  Therefore, \begin{align*}
	\displaystyle \frac{1^2+1^2+1^2}{3} &> (\frac{1+1+1}{3})^2\\
	1 &> 1.
\end{align*}
	This is a contradiction, as 1 is not greater than 1.  This statement would be true if you changed the greater than sign to a greater than or equal to sign.  
	\end{itemize}
\end{document}