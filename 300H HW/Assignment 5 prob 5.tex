\documentclass[12pt, letterpaper]{article}
\date{\today}
\usepackage[margin=1in]{geometry}
\usepackage{amsmath}
\usepackage{hyperref}

\usepackage{amssymb}
\usepackage{fancyhdr}
\usepackage{pgfplots}
\usepackage{booktabs}
\usepackage{pifont}
\usepackage{amsthm,latexsym,amsfonts,graphicx,epsfig,comment}
\pgfplotsset{compat=1.16}
\usepackage{xcolor}
\usepackage{tikz}
\usetikzlibrary{shapes.geometric}
\usetikzlibrary{arrows.meta,arrows}
\newcommand{\Z}{\mathbb{Z}}
\newcommand{\N}{\mathbb{N}}
\newcommand{\R}{\mathbb{R}}
\newcommand{\Po}{\mathcal{P}}

\author{Alex Valentino}
\title{Assignment 5}
\pagestyle{fancy}
\renewcommand{\headrulewidth}{0pt}
\renewcommand{\footrulewidth}{0pt}
\fancyhf{}
\rhead{
	Assignment 5 problem 5\\
	300H	
}
\lhead{
	Alex Valentino\\
}
\begin{document}
	We begin with some definitions. We say that $A$ is a neighbor of $B$ if $A \oplus B$ consists of exactly one element.
	\newline
	A list of sets is a \textit{neighborly list} of sets if each set is a neighbor of the set following it in the
list, and the last set is a neighbor of the first set.
	\newline
	
	Here is an interesting theorem: For all positive integers $n$, there is a list consisting of subsets
of $\{1, . . . , n\}$ such that (1) every subset of $\{1, . . . , n\}$ appears precisely once on the list and
(2) the list is neighborly.
\newline

	Prove the special cases of the theorem with $n = 1, n = 2, n = 3$ and $n = 4.$ (We'll prove the
theorem for all n later, but if you feel ambitious you can try it now.)

\begin{enumerate}
	\item $n=1$:\\  
	$\{\emptyset,\{1\}\}$
	\item $n=2$:\\
	$\{\emptyset,\{1\},\{1,2\},\{2\}\}$
	\item $n=3$:\\
	$\{\emptyset,\{2\},\{1,2\},\{1,2,3\},\{2,3\},\{3\},\{1,3\},\{1\}\}$
	\item $n=4$:\\
	$\{\{2\},\{1,2\},\{1,2,3\},\{2,3\},\{2,3,4\},\{1,2,3,4\},\{1,2,4\},\{2,4\},\{4\},\{1,4\},\{1,3,4\},\{3,4\},\{3\},\{1,3\},\{1\},\emptyset\}$
\end{enumerate}
\end{document}