\documentclass[12pt, letterpaper]{article}
\date{\today}
\usepackage[margin=1in]{geometry}
\usepackage{amsmath}
\usepackage{hyperref}

\usepackage{amssymb}
\usepackage{fancyhdr}
\usepackage{pgfplots}
\usepackage{booktabs}
\usepackage{pifont}
\usepackage{amsthm,latexsym,amsfonts,graphicx,epsfig,comment}
\pgfplotsset{compat=1.16}
\usepackage{xcolor}
\usepackage{tikz}
\usetikzlibrary{shapes.geometric}
\usetikzlibrary{arrows.meta,arrows}
\newcommand{\Z}{\mathbb{Z}}
\newcommand{\N}{\mathbb{N}}
\newcommand{\R}{\mathbb{R}}
\newcommand{\Po}{\mathcal{P}}

\author{Alex Valentino}
\title{Assignment 8}
\pagestyle{fancy}
\renewcommand{\headrulewidth}{0pt}
\renewcommand{\footrulewidth}{0pt}
\fancyhf{}
\rhead{
	Assignment 8 problem 5\\
	300H	
}
\lhead{
	Alex Valentino\\
}
\begin{document}
	For any finite set $A$ and full relation $R$ on $A$, there 
 is a Hamilton path in $R$.\\
 	Proof:  We must show the for any finite set $A$ and full relation $R$ on A, then there exists a Hamilton path in $R$.  Suppose $A$ is an arbitrary set, $|A| = k,$ and $R$ is a full relation on $A$.  We must show there exists a Hamilton path in $R$.  By definition of a Hamilton path, we must show that there is a non-repeating $R$-path through every element in $A$.  Assume for contradiction that the longest $R$-path does not include every element in $A$.  Let $(a_1,\cdots,a_t)$ be the longest $R$-path, where $t < k.$  Therefore exists an element $a^* \in A$ such that $a^* \not \in (a_1,\cdots,a_t).$   Since $R$ is full, then there exists 4 cases of how $a^*$ relates to $a_1$ and $a_t$:
 	\begin{itemize}
 		\item $a^* R a_1$, $a^* R a_t$\\
 		Since $a^* R a_1$, $a^*$ can be appended onto the beginning of $(a_1,\cdots, a_t)$ and form a valid $R$-path, contradicting our previous assumption.
 		\item $a^* R a_1$, $ a_tRa^*$\\
 		Since $a^* R a_1$, $a^*$ can be appended onto the beginning of $(a_1,\cdots, a_t)$ and form a valid $R$-path, contradicting our previous assumption.
 		\item $a_1 R a^*$, $a_t R a^*$\\
 		Since $a_t R a^*$, $a^*$ can be appended onto the end of $(a_1,\cdots, a_t)$ and form a valid $R$-path, contradicting our previous assumption.
 		\item $a_1 R a^*$, $a^* R a_t$\\
 		Since these two relations don't provide a clear insertion point for $a^*$, then we must look at the orbit of $a^*$: $O^* = \{a \in A: a^* R a\}$.  Since $a_1 R a^*$, we have at least one element that is in $O^{*c}.$  Since the relation is full by definition $A= O^* + O^{*c}.$  Therefore since there exists elements of both $O^*$ and $O^{*c}$ that lie in $(a_1,\cdots, a_t)$, then there must exists an element in $O^{*c}$ next to an element from $O^*.$  Suppose the element from $O^{*c}$ occurs at index $i.$  Therefore since $a_{i} R a^*$ and $a^* R a_i,$ then $a^*$ may be inserted at $i$, forming a valid $R$-path. 	
 	\end{itemize}
 	Therefore since all of the cases result in a contradiction, then $t=k,$ and $R$ has a Hamilton path.  
\end{document}