\documentclass[12pt, letterpaper]{article}
\date{\today}
\usepackage[margin=1in]{geometry}
\usepackage{amsmath}
\usepackage{hyperref}

\usepackage{amssymb}
\usepackage{fancyhdr}
\usepackage{pgfplots}
\usepackage{booktabs}
\usepackage{pifont}
\usepackage{amsthm,latexsym,amsfonts,graphicx,epsfig,comment}
\pgfplotsset{compat=1.16}
\usepackage{xcolor}
\usepackage{tikz}
\usetikzlibrary{shapes.geometric}
\usetikzlibrary{arrows.meta,arrows}
\newcommand{\Z}{\mathbb{Z}}
\newcommand{\N}{\mathbb{N}}
\newcommand{\R}{\mathbb{R}}
\newcommand{\Po}{\mathcal{P}}

\author{Alex Valentino}
\title{Assignment 7}
\pagestyle{fancy}
\renewcommand{\headrulewidth}{0pt}
\renewcommand{\footrulewidth}{0pt}
\fancyhf{}
\rhead{
	Assignment 7 problem 6\\
	300H	
}
\lhead{
	Alex Valentino\\
}
\begin{document}
	\textit{Let $a_1,\cdots,a_k$ be a list of intergers.  Define the function $f$ that maps a list of $k$ integers to an integer by the rule $f(x_1,\cdots,x_k) = a_1 x_1 + \cdots a_k x_k.$  Let R be the range of the function $f$. Prove:}
	\begin{itemize}
		\item \textit{For all $m,n \in R$ we have $m + n \in R$.}
		Suppose $m,n$ are arbitrary integers in the range $R$.  We must show that $m+n$ is in the range $R$.  By definition, $m,n$ have representations in the set of the list of integers of length $k$ such that $m = f(m_1,\cdots, m_k)$ and $n = f(n_1,\cdots,n_k).$  Therefore $m+n$ can be rewritten as:\\
		$m+n = f(m_1,\cdots,m_k) + f(n_1,\cdots,n_k) = a_1 m_1 + \cdots + a_k m_k + a_1 n_1 + \cdots a_k n_k = a_1 ( m_1 + n_1) \cdots a_k (m_k+n_k).$\\
		Since the integers are closed under addition and multiplication then the list $(m_1+n_1, \cdots, m_k +n_k )$ is a valid list of $k$ integers, which has been shown above defines $m+n,$ then $m+n \in R.$
		\item \textit{For all $n \in R$ and $c \in Z$ we have $cn \in R.$ }  Suppose $n,c$ are arbitrary integers, and $n \in R.$  By definition of being in the range of $R$ there exist a list $(n_1,\cdots, n_k)$ such that $n = f(n_1,\cdots,n_k) = a_1 n_1 + \cdots a_k n_k.$ Therefore multiplying $n$ by $c$ yields:\\
		$cn = c(a_1 n_1 + \cdots a_k n_k) = c a_1 n_1 + \cdots c a_k n_k.$  Since the integers are closed under multiplication, then $(cn_1,\cdots,cn_2)$ is a valid list of $k$ integers.  Therefore $f$ is defined over that list, and as shown above that is the representation of $cn$, therefore $cn \in R.$
	\end{itemize}
\end{document}