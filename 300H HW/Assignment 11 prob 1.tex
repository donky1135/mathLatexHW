\documentclass[12pt, letterpaper]{article}
\date{\today}
\usepackage[margin=1in]{geometry}
\usepackage{amsmath}
\usepackage{hyperref}
\usepackage{cancel}
\usepackage{amssymb}
\usepackage{fancyhdr}
\usepackage{pgfplots}
\usepackage{booktabs}
\usepackage{pifont}
\usepackage{amsthm,latexsym,amsfonts,graphicx,epsfig,comment}
\pgfplotsset{compat=1.16}
\usepackage{xcolor}
\usepackage{tikz}
\usetikzlibrary{shapes.geometric}
\usetikzlibrary{arrows.meta,arrows}
\newcommand{\Z}{\mathbb{Z}}
\newcommand{\N}{\mathbb{N}}
\newcommand{\R}{\mathbb{R}}
\newcommand{\Po}{\mathcal{P}}

\author{Alex Valentino}
\title{Assignment 11}
\pagestyle{fancy}
\renewcommand{\headrulewidth}{0pt}
\renewcommand{\footrulewidth}{0pt}
\fancyhf{}
\rhead{
	Assignment 11 problem 1\\
	300H	
}
\lhead{
	Alex Valentino\\
}
\begin{document}
Recall that a number is {\em perfect} if the sum of its proper divisors
is equal to the number itself.  Prove the following: if $n$ is a positive integer such
that $2^n-1$ is prime, then $2^{n-1}(2^n-1)$ is perfect.\\

Proof: Suppose $2^n -1$ is prime.  We must show that $2^{n-1}(2^n-1)$ is perfect.  Let $l = 2^{n-1}(2^n-1)$.  By definition of perfect we must show $\sum_{\substack{d \mid l\\d \neq l}} d = l$.  Since $2^n - 1$ is prime, then for all $d \in Div(l),$ either $2^n -1 \mid d$ or $2^n - 1 \nmid d.$  Let the set of all $d \in Div(l), 2^n -1 \nmid d$ be denoted $A$, and the set of all $d \in Div(l), 2^n -1 \mid d$ be denoted $B$.  Note by the definition of set union $A \cup B = Div(l).$  Since for all $a \in A, 2^n - 1 \nmid a, a \mid n,$ then $a \mid 2^{n-1}.$  Since $2$ is prime, then the only possible divisors of $2^{n-1}$ are $\{2^0,2^1,\ldots,2^{n-1}\}.$  Therefore $A = \{2^0,2^1,\ldots,2^{n-1}\}$.  Since the set $B$ is the set of divisors of $l$ who are divisable by $2^n -1,$ then aside from $1, b \in B$ should have the property $(b/(2^n -1)) \mid 2^{n-1}.$  Since $(b/(2^n -1)) \mid 2^{n-1},$ and the only divisors of $2^{n-1}$ are $\{2^0,\ldots, 2^{n-1}\}$, then all $b \in B \backslash \{1\}$ should have the form $b=2^k(2^n - 1),$ where $k \in \Z, n-1 \geq k \geq 0.$ Therefore $B = \{1,2^n -1, 2(2^n - 1),\ldots, 2^{n-1}(2^n - 1)\}$.  Also note that for any $r \in \Z_{\geq 0}, \sum^r_{i=0} 2^i= 2^{r+1} - 1,$ since $\sum^r_{i=0} 2^i$ is geometric series which when evaluated yields $\sum^r_{i=0} 2^i = \frac{1-2^{r+1}}{1-2} = \frac{1-2^{r+1}}{-1} = 2^{r+1} - 1.$
Therefore by algebraic manipulation
\begin{align*}
\sum_{\substack{d \mid l\\d \neq l}} d &= \sum_{a \in A} a + \sum_{b \in B \backslash \{1,l\}}b\\
&= \sum_{i=0}^{n-1} 2^i + (2^n - 1)\sum^{n-2}_{i=0}2^i \\
&= 2^n - 1 + (2^n - 1)(2^{n-1} - 1)\\
&= (2^n -1)(1 + 2^{n-1} - 1)\\
&= 2^{n-1}(2^n -1)\\
&= l.
\end{align*}

\end{document}