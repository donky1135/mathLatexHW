\documentclass[12pt, letterpaper]{article}
\date{\today}
\usepackage[margin=1in]{geometry}
\usepackage{amsmath}
\usepackage{hyperref}
\usepackage{cancel}
\usepackage{amssymb}
\usepackage{fancyhdr}
\usepackage{pgfplots}
\usepackage{booktabs}
\usepackage{pifont}
\usepackage{amsthm,latexsym,amsfonts,graphicx,epsfig,comment}
\pgfplotsset{compat=1.16}
\usepackage{xcolor}
\usepackage{tikz}
\usetikzlibrary{shapes.geometric}
\usetikzlibrary{arrows.meta,arrows}
\newcommand{\Z}{\mathbb{Z}}
\newcommand{\N}{\mathbb{N}}
\newcommand{\R}{\mathbb{R}}
\newcommand{\Po}{\mathcal{P}}

\author{Alex Valentino}
\title{Assignment 10}
\pagestyle{fancy}
\renewcommand{\headrulewidth}{0pt}
\renewcommand{\footrulewidth}{0pt}
\fancyhf{}
\rhead{
	Assignment 10 problem 1\\
	300H	
}
\lhead{
	Alex Valentino\\
}
\begin{document}
Prove: For any finite graph G if G is connected then $|E(G)|\geq|V (G)|-1$.\\
We must show for all finite graphs $G$ if $G$ is connected then $|E(G)|\geq|V (G)|-1$.  Suppose $G$ is a finite connected graph.  We must show that $|E(G)|\geq|V (G)|-1$.  By the principal of mathematical induction, for all finite connected graphs $H$ with $|V(H)| = k,$ if $k < |V(G)|,$ then $|E(H)| \geq |V(H)| - 1.$  We now have the cases $|V(G)| \leq 1, |V(G)| > 1.$
\begin{itemize}
	\item Assume $|V(G)| \leq 1$.  Since $G$ is anti-reflexive, and the edge set is the set of subsets of pairs of elements from $V(G),$ then $|E(G)| = 0.$  Since at most $|V(G)| = 1,$ then we have the inequality $0 \geq 1-1 = 0,$ which is true.
	\item Assume $|V(G)| > 1.$  Since $G$ is connected and has at minimum two nodes, then every node must have at least one neighbor.  Let $v$ be an arbitrary vertex of $G$, and let $N$ be the set of all of the neigbors of $v$.  Let $G'$ be the graph with $v$ removed.  Let the set $S$ be given by $S=\{\{n\} \cup G'(n): n \in N\}$ where $G'(n)$ is the set of all elements reachable by $n$ in $G'.$     	Let $W$ be the set of subgraphs such that $\{G'[s]: s \in S\}$.  Note that each subgraph in $W$ is connected since if they weren't then    We claim that \begin{enumerate}
	\item Each $s\in S$ is pairwise disjoint.
	\item Each $w \in W$ is connected.
	\item $|W| \leq |N|$.
	\item $\sum_{w\in W} |E(w)| + |N| = E(G)$	.
\end{enumerate}


Proof of claim 1.  We will prove this via contraposition.  Suppose $s_1, s_2 \in S, s_1 \cap s_2 = A$.  We must show that $s_1 = s_2.$ By definition of $S$, there exists $n_1, n_2 \in N, n_1 \neq n_2$ such that $G'(n_1) \subset s_1, n_1 \in s_1,G'(n_2) \subset s_2, n_2 \in s_2 .$ Suppose $a \in A.$ Since $a \in s_1, s_2,$ then $a$ is reachable by both $n_1,n_2.$  Therefore since $a \in G'(n_1), a \in s_2,$ then there exists a walk from $n_1$ to $n_2$ as there exists walks from both elements to $a$.  Therefore by the symmetry of $G'$, all elements reachable by $n_1$ include elements that are reachable by $n_2$ and all elements that are reachable by $n_2$ inlcude elements that are reachable by $n_2$.  Therefore $s_1 = s_2.$\\

Proof of claim 2.  Assume for contradiction that there exists a $w \in W$ such that $w$ is not connected.  Since $w$ is a subgraph of $G$, and there exists no other subgraph in $W$ which $w$ shares nodes with, then $G$ must be not connected since the unique contribution of $w$ to $G$ is not connected. This is a contradiction. \\


Proof of claim 3.  
	Since $S$ is pairwise disjoint, and each element of $W$ is defined by a distinct element in $S$, then $|S| = |W|.$  Therefore we must show $|S| \leq |N|.$  Suppose for contradiction that $|S| > |N|$.  Therefore there must exists a set in $S$ that cannot be paired with an element from $N$.  This is a contradiction as every set in S is defined to contain an element from $N$.
	
Proof of claim 4.  Since the all of the edges of $G'$ are contained in the elements of $W$, then we have $\sum_{w\in W}|E(w)| = |E(G')|$. We must show that $|E(G')| + |N| = |E(G)|.$  Since the only difference between $G$ and $G'$ is the removal of $v$, and $v$ was connected to $|N|$ elements, then $v$ had $|N|$ connections.  Therefore the number of missing connections between $G'$ and $G$ is $|N|.$  Therefore $|E(G')| + |N| = |E(G)|.$\\

Since we have established that each $w$ in $W$ is connected, and their union forms $G'$, whose vertex set $|V(G')| = |V(G)| -1,$ which noting $|V(G')| < |V(G)|$ lets us apply the induction hypothesis for all $w\in W, |E(w)| \geq |V(w)| - 1.$  Therefore adding 1 to both sides yields $|E(w)| +1 \geq |V(w)|.$  Therefore, \begin{align*}
|V(G)| - 1 &= |V(G')| = \sum_{w\in W} |V(w)| \\
&\leq \sum_{w\in W} (|E(w)| +1) \\
&= \sum_{w\in W}|E(w)| + |W| \\
&\leq \sum_{w\in W}|E(w)| + |N|\\
&= |E(G)|.
\end{align*}


	
\end{itemize}

\iffalse
Suppose for contradiction that $|E(G)| < |V (G)|-1$.  Since $G$ is connected, every node must be reachable.  We have two cases: \begin{itemize}
	\item Assume $|V(G)| = 1$.  Then $|E(G)| < 0,$ which is a contradiction as $|E(G)|$ is a non-negative integer valued function.
	\item Assume $|V(G)| > 1.$ Since $G$ is connected at minimum every vertex should have one connection to a different vertex except for one, which will be defined to be the end vertex. This corresponds to every vertex except one being uniquely paired to a subset in $|E(G)|$ which contains the vertex.  However since $|E(G)| < |V (G)|-1$ there must be more than one vertex, $v$ ,who doesn't have a unique pairing to a subset in  $|E(G)|$.  This creates two cases:
	\begin{itemize}
		\item Assume there does not exists a subset which contains $v$.  Then the vertex is isolated and lacks connections.  Therefore we have a contradiction.  
		\item Assume there exists a subset containing the vertex.  Since this isn't a unique pairing and no other elements connect to $v$ then we have a second end vertex, and 
	\end{itemize}
\end{itemize}
\fi
\end{document}