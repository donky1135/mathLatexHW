\documentclass[12pt, letterpaper]{article}
\date{\today}
\usepackage[margin=1in]{geometry}
\usepackage{amsmath}
\usepackage{hyperref}
\usepackage{cancel}
\usepackage{amssymb}
\usepackage{fancyhdr}
\usepackage{pgfplots}
\usepackage{booktabs}
\usepackage{pifont}
\usepackage{amsthm,latexsym,amsfonts,graphicx,epsfig,comment}
\pgfplotsset{compat=1.16}
\usepackage{xcolor}
\usepackage{tikz}
\usetikzlibrary{shapes.geometric}
\usetikzlibrary{arrows.meta,arrows}
\newcommand{\Z}{\mathbb{Z}}
\newcommand{\N}{\mathbb{N}}
\newcommand{\R}{\mathbb{R}}
\newcommand{\Po}{\mathcal{P}}

\author{Alex Valentino}
\title{Assignment 9}
\pagestyle{fancy}
\renewcommand{\headrulewidth}{0pt}
\renewcommand{\footrulewidth}{0pt}
\fancyhf{}
\rhead{
	Assignment 9 problem 4\\
	300H	
}
\lhead{
	Alex Valentino\\
}
\begin{document}
We say that {\em $A$ is a neighbor
of $B$} 
if $A \triangle B$ consists of exactly one element. 

A list of sets is a {\em neighborly list of sets}
if each  set is a neighbor of the set following it in the list,
and the last set is a neighbor of the first set.   Prove that 
for all positive integers
$n$, there is a list  consisting  of subsets of
$\{1,\ldots, n\}$ such that:\\
(1) every subset of $\{1,\ldots,n\}$ 
appears {\em precisely
once} on the list and \\
(2) the list is neighborly.\\
We must show that for all $n \in \N,$ there exists a list of subsets of $\{1,\ldots,n\}$ which contains all of the subsets exactly once and is neighborly.  Suppose $n \in \N.$  We must show there exists a list of subsets of $\{1,\ldots,n\}$ which contains all of the subsets exactly once and is neighborly.  By the principal of mathematical induction for all $k \in \N$ if $k<n$ then there exists a list of the subsets of $\{1,\ldots,k\}$ such that each subset occurs exactly once and the list is neighborly.  We now have two cases:
\begin{itemize}
	\item Assume $n=1$.  Then the list $(\emptyset, \{1\})$ is a neighborly list, and every subset of $\{1\}$ occurs exactly once.  
	\item Assume $n>1$.  Since $n-1 < n,$ by the induction hypothesis there exists a list of every subset of $\{1,\ldots, n-1\}$ exactly once and is neighborly.  Let this list be denoted $(S_1,\ldots,S_{2^n})$.
	%Since the only difference between $\{1,\ldots, n-1\}$ and $\{1,\ldots, n\}$ is the inclusion of $n$, then 
	We claim that the list satisfying the requirements for $\{1,\ldots,n\}$ is given by $(S_1,\ldots,S_{2^n},\{n\} \cup S_{2^n},\ldots, \{n\} \cup S_{1})$.  We must show this list satisfies the requirements.
	\begin{enumerate}
	\item Since the only difference between $\{1,\ldots, n-1\}$ and $\{1,\ldots, n\}$ is the inclusion of $n$, and $(S_1,\ldots, S_{2^n})$ contains every subset of $\{1,\ldots,n-1\}$ exactly once, then $(S_1,\ldots,S_{2^n},\{n\} \cup S_{2^n},\ldots, \{n\} \cup S_{1})$ contains all possible subsets that don't include $n$ and do include $n$.  Therefore every subset appears exactly once.
	\item Since $(S_1,\ldots,S_{2^n})$ is already neighborly, then $(\{n\} \cup S_{2^n},\ldots, \{n\} \cup S_{1})$ would be neighborly by virtual of the symmetric difference being symmetric, and the set difference between two sets both containing $n$ would not have $n$ by definition.  Then for the boundary interaction $ S_{2^n} \triangle \{n\} \cup S_{2^n} =(S_{2^n} \backslash \{n\} \cup S_{2^n}) \cup (\{n\} \cup S_{2^n} \backslash  S_{2^n}) = \emptyset \cap \{n\} = \{n\}.$  Therefore since all the possible symmetric differences between neighboring subsets have exactly one subset, then the list is neighborly.
	\end{enumerate}
\end{itemize}
\end{document}