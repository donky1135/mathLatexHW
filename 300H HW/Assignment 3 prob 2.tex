\documentclass[12pt, letterpaper]{article}
\date{\today}
\usepackage[margin=1in]{geometry}
\usepackage{amsmath}
\usepackage{hyperref}

\usepackage{amssymb}
\usepackage{fancyhdr}
\usepackage{pgfplots}
\usepackage{booktabs}
\usepackage{pifont}
\usepackage{amsthm,latexsym,amsfonts,graphicx,epsfig,comment}
\pgfplotsset{compat=1.16}
\usepackage{xcolor}
\usepackage{tikz}
\usetikzlibrary{shapes.geometric}
\usetikzlibrary{arrows.meta,arrows}
\newcommand{\Z}{\mathbb{Z}}
\newcommand{\N}{\mathbb{N}}
\newcommand{\R}{\mathbb{R}}
\newcommand{\Po}{\mathcal{P}}

\author{Alex Valentino}
\title{Assignment 3}
\pagestyle{fancy}
\renewcommand{\headrulewidth}{0pt}
\renewcommand{\footrulewidth}{0pt}
\fancyhf{}
\rhead{
	Assignment 3 problem 2\\
	300H	
}
\lhead{
	Alex Valentino\\
}
\begin{document}
	\textit{The entries of any matrix $M$ can be rearranged into a list $r(M)$ by forming the list row-by-row
starting from the first row, and also into a list c(M ) by forming the list column-by-column
starting from the first column. If m and n are positive integers and A is an $m\times n$ matrix
give a careful specification of the lists $r(M)$ and $c(M )$. (Your specification should include a
rule that for each index $i$ in the list gives the value of the $i$th item of the list.}\\
\begin{itemize}
	\item Specification for $c(M)$\\
	$c(M) := \displaystyle \bigoplus_{j=1}^n (M_{1j},\cdots,M_{mj}) = (a_1,...,a_{mn}).$
	$a_i \in c(M), a_i = M_{pq}, p = ((i-1 \mod m) + 1), q = (\lceil \frac{i}{m} \rceil)$
	\item Specification for $r(M)$\\
	$r(M) := \displaystyle \bigoplus_{j=1}^m (M_{j1},\cdots, M_{jn})$
	$a_i \in r(M), a_i = M_{qp}, p = ((i-1 \mod n) + 1), q = (\lceil \frac{i}{n} \rceil)$
	
\end{itemize}
	


\end{document}