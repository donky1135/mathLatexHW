\documentclass[12pt, letterpaper]{article}
\date{\today}
\usepackage[margin=1in]{geometry}
\usepackage{amsmath}
\usepackage{hyperref}

\usepackage{amssymb}
\usepackage{fancyhdr}
\usepackage{pgfplots}
\usepackage{booktabs}
\usepackage{pifont}
\usepackage{amsthm,latexsym,amsfonts,graphicx,epsfig,comment}
\pgfplotsset{compat=1.16}
\usepackage{xcolor}
\usepackage{tikz}
\usetikzlibrary{shapes.geometric}
\usetikzlibrary{arrows.meta,arrows}
\newcommand{\Z}{\mathbb{Z}}
\newcommand{\N}{\mathbb{N}}
\newcommand{\R}{\mathbb{R}}
\newcommand{\Po}{\mathcal{P}}

\author{Alex Valentino}
\title{Assignment 6}
\pagestyle{fancy}
\renewcommand{\headrulewidth}{0pt}
\renewcommand{\footrulewidth}{0pt}
\fancyhf{}
\rhead{
	Assignment 6 problem 5\\
	300H	
}
\lhead{
	Alex Valentino\\
}
\begin{document}
	\begin{itemize}
		\item Function examples:\\
			\begin{itemize}
				\item odd functions: $x^3$, $sin(x)$,
				\item even functions: $x^2, cos(x)$
			\end{itemize}
		\item Prove that for all functions $f: \R \to \R$, the function $s(x)$ given by the rule $s(x) = f(x) + f(-x)$ is even.  \\
		We must show that $s(x)$ is even.  By definition we must show $s(x) = s(-x)$  Suppose $x$ is an arbitrary real number.  Therefore, \begin{align*}
			s(-x) &= f(-x) + f(-(-x))\
			&= f(x) + f(-x)\\
			&= s(x).
		\end{align*}
		\item Prove that all functions $f: \R \to \R$ can be written as the sum of an odd and even function.  
		Lemma: For all functions $f:\R \to \R$, the function $O(x)$ given by the rule $O(x) = f(x)-f(-x)$ is even.  \\
		Proof: We must show that $O(x)$ is odd.  By definition we must show that $O(-x) = -O(x).$  Suppose $x$ is an arbitrary real number.  Therefore, \begin{align*}
		O(-x) &= f(-x) - f(-(-x))\\
		 &= -1*f(x) + -1*-f(-x)\\ 
		 &= -1*(f(x)-f(-x))\\ 
		 &= -O(x).
		\end{align*}
		Proof: We must show that $f$ is the sum of an odd function and an even function.  We claim that $f(x) = \frac{s(x) + O(x)}{2}$.  Suppose $x$ is an arbitrary real number.  Therefore, 
		\begin{align*}
			\frac{s(x) + O(x)}{2} &= \frac{f(x) + f(-x) + f(x) - f(-x)}{2}\\
			&= \frac{2f(x)}{2}\\
			&= f(x).
		\end{align*}			
	\end{itemize}
\end{document}