\documentclass[12pt, letterpaper]{article}
\date{\today}
\title{Assignment 2}
\usepackage[margin=1in]{geometry}
\usepackage{amsmath}
\usepackage{hyperref}

\usepackage{amssymb}
\usepackage{fancyhdr}
\usepackage{pgfplots}
\usepackage{booktabs}
\usepackage{pifont}
\usepackage{amsthm,latexsym,amsfonts,graphicx,epsfig,comment}
\pgfplotsset{compat=1.16}
\usepackage{xcolor}
\usepackage{tikz}
\usetikzlibrary{shapes.geometric}
\usetikzlibrary{arrows.meta,arrows}
\newcommand{\Z}{\mathbb{Z}}
\newcommand{\N}{\mathbb{N}}
\newcommand{\R}{\mathbb{R}}
\newcommand{\Po}{\mathcal{P}}

\author{Alex Valentino}
\title{Assignment 2}
\pagestyle{fancy}
\renewcommand{\headrulewidth}{0pt}
\renewcommand{\footrulewidth}{0pt}
\fancyhf{}
\rhead{
	Assignment 2 problem 10\\
	300H	
}
\lhead{
	Alex Valentino\\
}
\begin{document}
	\textit{Let $S$ denote the set $\{2, 10, 11, 18, 19, 27\}$.
For each of the following statements, determine whether the statement is true or false.}
\begin{enumerate}
	\item \textit{For all elements $x$ belonging to $S$ there is a $y$ belonging to $S$ such that $x + y$ is a multiple
of $5$}
	\begin{itemize}
		\item Reducing all the elements in the set $\mod 5$ we observe that $\{2, 10, 11, 18, 19, 27\} \equiv \{0,1,2,3,4\} \mod 5.$  Therefore since $S$ includes $5\Z$ and all of the cosets of $\Z / 5 \Z,$ whose addition operation is a group.  Therefore by definition every element has an additive inverse.  Therefore the statement is true. 
	\end{itemize}
	\item \textit{There exists an element $x$ belonging to $S$ such that for all elements $y$ belonging to $S$,
$x + y$ is a multiple of $5$.}
	\begin{itemize}
		\item The above statement is false because elements in a group, which are what $S$ forms under division by 5, by definition have unique inverses. 
	\end{itemize}
	\item \textit{For all $x$ belonging to $S$ there is a $y$ belonging to $S$ such that $x + y$ is a multiple of $7$.}
	\begin{itemize}
		\item Reducing $S \mod 7$ we observe that $\{2, 10, 11, 18, 19, 27\} \equiv \{2,3,4,5,6\}.$  Since $1 \mod 7 \notin S, $ then there exist no element which can be added to $27$ that would make it divisible by $7$.  Therefore the above statement is false. 
	\end{itemize}
\end{enumerate}
\end{document}