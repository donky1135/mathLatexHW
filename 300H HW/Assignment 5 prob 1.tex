\documentclass[12pt, letterpaper]{article}
\date{\today}
\usepackage[margin=1in]{geometry}
\usepackage{amsmath}
\usepackage{hyperref}

\usepackage{amssymb}
\usepackage{fancyhdr}
\usepackage{pgfplots}
\usepackage{booktabs}
\usepackage{pifont}
\usepackage{amsthm,latexsym,amsfonts,graphicx,epsfig,comment}
\pgfplotsset{compat=1.16}
\usepackage{xcolor}
\usepackage{tikz}
\usetikzlibrary{shapes.geometric}
\usetikzlibrary{arrows.meta,arrows}
\newcommand{\Z}{\mathbb{Z}}
\newcommand{\N}{\mathbb{N}}
\newcommand{\R}{\mathbb{R}}
\newcommand{\Po}{\mathcal{P}}

\author{Alex Valentino}
\title{Assignment 5}
\pagestyle{fancy}
\renewcommand{\headrulewidth}{0pt}
\renewcommand{\footrulewidth}{0pt}
\fancyhf{}
\rhead{
	Assignment 5 problem 1\\
	300H	
}
\lhead{
	Alex Valentino\\
}
\begin{document}
Recall the following definition: For any two sets $A$ and $B$, the difference set $A \backslash B$ is the set consisting of those objects that are members of $A$ but not members of $B$. Also $A \triangle B$ is equal to $A \backslash B \cup B \backslash A$.
	\begin{enumerate}
	\item Prove or disprove: For all sets $A, B, C$, if $A \backslash C=B \backslash C$ then $A=B$.\\
	Suppose $A,B,C$ are arbitrary sets.  We must show if $A \backslash C=B \backslash C$ then $A=B$.  By definition of set equality we must prove if $A \backslash C=B \backslash C$ then $A\subseteq B$ and  $B\subseteq A$.
		\begin{enumerate}
			\item Suppose $A,B,C$ are arbitrary sets.  We must show if $A \backslash C=B \backslash C$ then $A \subseteq B$.  By the definition of subset we must show $\forall x \in A \Rightarrow x \in B$. Suppose $x$ is an arbitrary element of $A \backslash C.$  By definition of set equality since $A \backslash C=B \backslash C$, then $x \in B \backslash C.$  By definition of set difference $x \in B$ and $x \not\in C$.  Therefore by the definition of subset, $A \subseteq B$.
			\item Suppose $A,B,C$ are arbitrary sets.  We must show if $A \backslash C=B \backslash C$ then $B \subseteq A.$ By definition of subset we must show $\forall x \in B \Rightarrow x \in A$. Suppose $x$ is an arbitrary element of $B \backslash C.$  By definition of set equality since $A \backslash C=B \backslash C$, then $x \in A \backslash C.$ By the definition of set difference $x \in A$ and $x \not\in C$. Therefore by the definition of subset, $B \subseteq A$.
		\end{enumerate}
		Therefore for all sets $A, B, C$, if $A \backslash C=B \backslash C$ then $A=B$.
	\item Prove or disprove: For all sets $A, B, C$, if $A \oplus C=B \oplus C$ then $A=B$.\\
	Suppose $A,B,C$ are arbitrary sets.  We must show if $A \oplus C=B \oplus C$ then $A=B$.  By definition of set equality we must prove if $A \oplus C=B \oplus C$ then $A\subseteq B$ and  $B\subseteq A$.
	\begin{enumerate}
		\item Suppose $A,B,C$ are arbitrary sets.  We must show if $A \oplus C=B \oplus C$ then $A \subseteq B$. By the definition of subset we must show $\forall x \in A \Rightarrow x \in B$. Suppose $x$ is an arbitrary element of $A \oplus B.$  By definition of set equality since $x \in A \oplus C$ then $x \in B \oplus C.$  By the definition of symmetric difference $x \in B \backslash C \cup C \backslash B.$  By the definition of set union $x \in B \backslash C$ or $x \in C \backslash B.$  Therefore since it's an or statement we can choose $x \in B \backslash C$.  By the definition of set difference $x \in B$ and $x \not\in C.$  Therefore $x \in B.$  This satisfies the requirement. 
		\item Suppose $A,B,C$ are arbitrary sets. We must show if $A \oplus C=B \oplus C$ then $B \subseteq A$. By the definition of subset we must show $\forall x \in B \Rightarrow x \in A$.  By definition of set equality since $x \in B \oplus C$ then $x \in A \oplus C.$ By the definition of symmetric difference $x \in A \backslash C \cup C \backslash A.$ By the definition of set union $x \in A \backslash C$ or $x \in C \backslash A.$ Therefore since it's an or statement we can choose $x \in A \backslash C$.  By the definition of set difference $x \in A$ and $x \not\in C.$  Therefore $x \in A.$  This satisfies the requirement. 
	\end{enumerate}
	Therefore for all sets $A, B, C$, if $A \oplus C=B \oplus C$ then $A=B$.
\end{enumerate}		
	
\end{document}