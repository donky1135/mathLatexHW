\documentclass[12pt, letterpaper]{article}
\date{\today}
\usepackage[margin=1in]{geometry}
\usepackage{amsmath}
\usepackage{hyperref}

\usepackage{amssymb}
\usepackage{fancyhdr}
\usepackage{pgfplots}
\usepackage{booktabs}
\usepackage{pifont}
\usepackage{amsthm,latexsym,amsfonts,graphicx,epsfig,comment}
\pgfplotsset{compat=1.16}
\usepackage{xcolor}
\usepackage{tikz}
\usetikzlibrary{shapes.geometric}
\usetikzlibrary{arrows.meta,arrows}
\newcommand{\Z}{\mathbb{Z}}
\newcommand{\N}{\mathbb{N}}
\newcommand{\R}{\mathbb{R}}
\newcommand{\Po}{\mathcal{P}}

\author{Alex Valentino}
\title{Assignment 3}
\pagestyle{fancy}
\renewcommand{\headrulewidth}{0pt}
\renewcommand{\footrulewidth}{0pt}
\fancyhf{}
\rhead{
	Assignment 3 problem 1\\
	300H	
}
\lhead{
	Alex Valentino\\
}
\begin{document}
	\begin{itemize}
		\item $\R \times \Z \times \Po_{fin}(\R)$\\
		This is a cartesian product with three different sets, so it will be written as a triple (x,y,z).  In the $x$ entry there will be a real number, in the $y$ entry there will be an integer, and in the $z$ entry there will be a finite subset of the real numbers.  Examples: $(1,1,\{1\}), (\pi^e,1792,\{\frac{\sqrt{2}}{2^1},\cdots,\frac{\sqrt{2}}{2^{1000}}\})$.
		\item $(\Po(\Z_{>0}))^{\R^2}$\\
		This is the set of functions from the real plane to subsets of the natural numbers.  Examples: $f:\R^2 / \{(0,0)\} \to \Po(\Z_{>0}), f(x,y) = \{|x|\}$, $g: \R^2/ \{(0,0)\} \to \Po(\Z_{>0}), g(x,y) = \{\{a_0 \cdot 10 ^ 1 + 1, \cdots, a_n \cdot 10 ^ {n+1} + 1\}: \frac{min(x,y)}{max(x,y)} = a_0.a_1 a_2 \cdots a_n\}$
	\end{itemize}
\end{document}