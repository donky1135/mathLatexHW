	\documentclass[12pt, letterpaper]{article}
\date{\today}
\usepackage[margin=1in]{geometry}
\usepackage{amsmath}
\usepackage{hyperref}

\usepackage{amssymb}
\usepackage{fancyhdr}
\usepackage{pgfplots}
\usepackage{booktabs}
\usepackage{pifont}
\usepackage{amsthm,latexsym,amsfonts,graphicx,epsfig,comment}
\pgfplotsset{compat=1.16}
\usepackage{xcolor}
\usepackage{tikz}
\usetikzlibrary{shapes.geometric}
\usetikzlibrary{arrows.meta,arrows}
\newcommand{\Z}{\mathbb{Z}}
\newcommand{\N}{\mathbb{N}}
\newcommand{\R}{\mathbb{R}}
\newcommand{\Po}{\mathcal{P}}
\usepackage{cancel}
\author{Alex Valentino}
\title{Assignment 7}
\pagestyle{fancy}
\renewcommand{\headrulewidth}{0pt}
\renewcommand{\footrulewidth}{0pt}
\fancyhf{}
\rhead{
	Assignment 7 problem 4 \\
	300H	
}
\lhead{
	Alex Valentino\\
}
\begin{document}
	Recall: a set $Y$ of real numbers has the no-gaps property provided
that for all $x, y, z \in \R$, if $x < y < z$ and $x, z \in Y$ then $y \in Y$ . In class we observed that
every interval has the no gaps property. It is also true that every set of real numbers with
the no-gaps property is an interval, but we have not proven this, and will not use it in this
problem.
Given a set $X$ of real numbers define a relation $N_X$ on the set $X$ according to the following:
for $x, y \in X, (x, y) \in pairs(N_X )$ provided that there is a set $I \subseteq X$ such that $x, y \in I$ and $I$
has the no-gaps property.
\begin{enumerate}
	\item Prove that this relation is an equivalence relation on X.
	\begin{itemize}
		\item Reflexivity proof: Suppose $x \in X.$  We want to show that for all $x \in X, (x,x) \in pairs(N_X).$  By definition of belonging to the relation we must show there exist $I \subseteq X$ such that $x,y \in I$, and $I$ has the no-gaps property.  Suppose $I = [x,x]$.  We must show that $I$ has the no-gaps property.  Since the only real number in the set $[x,x]$, the no gaps property requirement of of $x < y < z, x,y \in \R$ can only be written as $x < y < x,$ which is always false.  Therefore $I$ has the no gaps property.        
		\item Symmetry proof:  Suppose $x,y \in X, x,y \in I \subseteq X, $ $I$ has the no gaps property.  We must show that $(y,x) \in pairs(N_X).$  Therefore by definition we must show that there is a set $I \subseteq X$ such that $y,x \in I,$ and $I$ has the no gaps property.  Since $x,y \in I$, then $x \in I,$ and $y \in I,$ therefore $y,x \in I.$  Since $I\subseteq X$ exists and it has the no gaps property, then $(y,x) \in pairs(N_X).$   
		\item Transitivity proof.  Suppose $x,y,z \in X,$ $(x,y), (y,z) \in pairs(N_X).$  We must show $(x,z) \in pairs(N_X).$  Therefore by definition we must show that there exist $I \subset X$ such that $x,z \in I$ and $I$ has the no gaps property.  By definition of being in the relation, there exists $I_1, I_2 \subseteq X$ such that $x,y \in I_1, y,z \in I_2,$ and $I_1,I_2$ both have the no-gaps property.  Let $I = I_1 \cup I_2,$ since $I_1, I_2 \subseteq X,$ then $I \subseteq X.$  Therefore we must show that $I$ has the no-gaps property.  Since $x < y$ and $y < z,$ then by composing the inequalities we get $x<y<z,$ which since $x,y,z$ are defined to be in $I$, demonstrates that $I$ has the no-gaps property.        
	\end{itemize}
	\item Let $\mathcal{N} (X)$ be the set of equivalence classes. Prove that every equivalence class has the
no-gaps property.\\
		Suppose $C$ is an equivalence class in $\mathcal{N}(X).$  We must show that all $C \mathcal{N}(X)$ has the no gaps property.  By the definition of the no-gaps property we must show for all $x,y,z \in \R$ if $x,z \in C$ and $x < y < z$ then $y \in C.$  Suppose $x,y,z \in \R, $ $x,z \in C,$ and $x<y<z$.  We must show that $y \in C.$ Since $x,z \in C,$ then $x N_X z.$  By the definition of the relation there exists an interval $I \subseteq X$ such that $x,z\in I$, and $I$ has the no-gaps property.  Therefore since $I$ has the no gaps property, $x < y < z,$ and $x,z \in I,$ then $y \in I$.  Since $y \in I,$ then by the definition of the relation, $y N_X z$ and $y N_X x$, as they all exist in an interval with the no-gaps property.  Therefore since $y$ is related to elements in $C,$ then $y \in C.$
	\item Define a relationship $\ll$ on $\mathcal{N} (X)$ where for equivalence classes $A$ and $B, A \ll B$
provided that for all $x \in A$ and $y \in B$, $x < y$. Prove that the relation $\ll$ is transitive,
anti-reflexive, anti-symmetric, and full.
	\begin{itemize}
		\item Transitive proof:\\
		Suppose $A \ll B$ and $B \ll C.$  We must show that $A \ll C.$  By definition of the $\ll$, we must show for all $a \in A, c \in C$ that $a < c.$  By definition, all $a \in A, b \in B, c \in C, a < b, b < c.$  Therefore since $a < b$ and $b < c$ then by the transitivity of $<,$ $a<c.$
		\item Anti-reflexive and Anti-symmetry proof:\\
		Suppose $A \ll B.$  We must show that $B \cancel{\ll} A.$  By definition of $\ll$, we must show that there exist $a \in A, b \in B$ such that $a \leq b$.  By definition of $\ll$ we have for all $a \in A, b \in B, a < b.$  Therefore all $a,b$ satisfy the relationship $a \leq b.$ 
		\item Fullness proof: Suppose $A,B \in \mathcal{N}(X), A \neq B.$  We must show that $A \ll B \vee B \ll A.$  Since $A,B$ are seperate equivalence classes, then by definition they are sets in a partition of $X$, therefore $A \cap B = \emptyset.$  Assume $A \cancel{\ll} B.$   We must show that $B \ll A.$  Therefore by definition there exist $a \in A, b \in B$ such that $a \geq b.$  Since $A,B$ are disjoint then for all $a\in A, b \in B, a \neq b.$  Therefore we satisfy the condition $a > b.$  I'm now going to prove $B \ll A$ by contradiction.  Suppose not.  Let $a^* \in A, a^* < b.$  Since $a, a^* \in A,$ then there exists an inteval $a, a^* \in I$ with the no-gaps property.  Since $a^* < b,$ $b < a,$ and $a,a^* \in I,$ then by the definition of the no-gaps relation $b \in A.$  This is a contradiction, as $A \cap B = \emptyset$.  Therefore $B \ll A$      
	\end{itemize}
\end{enumerate}
\end{document}