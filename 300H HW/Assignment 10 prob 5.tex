\documentclass[12pt, letterpaper]{article}
\date{\today}
\usepackage[margin=1in]{geometry}
\usepackage{amsmath}
\usepackage{hyperref}
\usepackage{cancel}
\usepackage{amssymb}
\usepackage{fancyhdr}
\usepackage{pgfplots}
\usepackage{booktabs}
\usepackage{pifont}
\usepackage{amsthm,latexsym,amsfonts,graphicx,epsfig,comment}
\pgfplotsset{compat=1.16}
\usepackage{xcolor}
\usepackage{tikz}
\usetikzlibrary{shapes.geometric}
\usetikzlibrary{arrows.meta,arrows}
\newcommand{\Z}{\mathbb{Z}}
\newcommand{\N}{\mathbb{N}}
\newcommand{\R}{\mathbb{R}}
\newcommand{\Po}{\mathcal{P}}

\author{Alex Valentino}
\title{Assignment 10}
\pagestyle{fancy}
\renewcommand{\headrulewidth}{0pt}
\renewcommand{\footrulewidth}{0pt}
\fancyhf{}
\rhead{
	Assignment 10 problem 5\\
	300H	
}
\lhead{
	Alex Valentino\\
}
\begin{document}
	Let $T$ be a finite tournament on the ground set $X$ (Recall that a tournament is a relation that is anti-reflexive, anti-symmetric and full. Thus for any $x,y \in X$ if $x \neq y$ then exactly one of $xTy$ and $yTx$ holds.  
)   Prove that $T$ is transitive if and only if $T$ has no cycles.\\

We must show that $T$ is transitive if and only if $T$ has no cycles.  Therefore we have two cases.
\begin{itemize}
	\item Suppose $T$ is transitive.  We must show that $T$ has no cycles. Suppose for contradiction that $T$ has a cycle.   Since $T$ is transitive, then for all $a,b,c \in X$ if $aTb, bTc,$ then $aTc.$  Since $T$ is a cycle, then let $C$ be the cycle in $T$.  Since $C$ is a cycle, then $|C| \geq 3.$  We have two cases:  
	\begin{itemize}
		\item Assume $|C| = 3$.  Then there exists $x,y,z\in C$ such that $xTy, yTz, zTx,$ which contradicts the fact that $T$ is transitive.  
		\item Assume $|C| > 3$.  Then there exists $x,y,z \in C$ such that $x,y$ are adjacent and  $z$ is not adjacent to $y$ (This is to ensure that $x,y,z$ are not in a row, if they were then by the transitivity of $T$ there would not be a cycle).  Note that since $x,y$ are adjacent then $xTy$.  Since $T$ is transitive, and $x,y,z$ are in a cycle together where $z$ is after $y$, then $yTz,$ and $zTx$.  This is a contradiction since if $xTy, yTz,$ then $xTz.$
	\end{itemize}
	\item Suppose $T$ has no cycles. We must show that $T$ is transitive.  Suppose $a,b,c \in X, aTb, bTc.$  We must show that $aTc.$  Since $a,c \in X$ and $T$ is defined on $X$, then we have $aTc$ or $cTa.$  However if $cTa$ then we have a cycle, which contradicts the fact that $T$ has no cycles.  Therefore $aTc$.  
\end{itemize}
\end{document}