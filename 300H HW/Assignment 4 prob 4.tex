\documentclass[12pt, letterpaper]{article}
\date{\today}
\usepackage[margin=1in]{geometry}
\usepackage{amsmath}
\usepackage{hyperref}

\usepackage{amssymb}
\usepackage{fancyhdr}
\usepackage{pgfplots}
\usepackage{booktabs}
\usepackage{pifont}
\usepackage{amsthm,latexsym,amsfonts,graphicx,epsfig,comment}
\pgfplotsset{compat=1.16}
\usepackage{xcolor}
\usepackage{tikz}
\usetikzlibrary{shapes.geometric}
\usetikzlibrary{arrows.meta,arrows}
\newcommand{\Z}{\mathbb{Z}}
\newcommand{\N}{\mathbb{N}}
\newcommand{\R}{\mathbb{R}}
\newcommand{\Po}{\mathcal{P}}

\author{Alex Valentino}
\title{Assignment 4}
\pagestyle{fancy}
\renewcommand{\headrulewidth}{0pt}
\renewcommand{\footrulewidth}{0pt}
\fancyhf{}
\rhead{
	Assignment 4 problem 4\\
	300H	
}
\lhead{
	Alex Valentino\\
}
\begin{document}
	\begin{enumerate}
		\item Give an example of a collection of 4 sets such that any two distinct sets in the collection
intersect in exactly one element, and no element belongs to more than 2 sets.\\
	$S = \{S_1, S_2, S_3, S_4\}$\\
	$S_1 \cap S_2 = \{1\}$\\
	$S_1 \cap S_3 = \{2\}$\\
	$S_1 \cap S_4 = \{3\}$\\
	$S_2 \cap S_3 = \{4\}$\\
	$S_2 \cap S_4 = \{5\}$\\
	$S_3 \cap S_4 = \{6\}$\\
	$S_1 = \{1,2,3\}$\\
	$S_2 = \{1,4,5\}$\\
	$S_3 = \{2,4,6\}$\\
	$S_4 = \{3,5,6\}$
	\item Extend to $k$ sets.  Let $f: \Po_{fin}(\N) \to \N$ be given as $f(S) = \displaystyle \sum_{s \in S} 2^{s-1},$ it's a bijection, I swear.
	The first set $S_1 = \{1,2,\cdots, \frac{1}{2} \binom{k}{2}\}$. Then for the next $i = 1,2,\cdots,k-1$ sets be defined as $S_{i+1} = f^{-1} (f(1,\cdots,k) - f(S_1) + 2^{i-1} - 2^{\binom{k}{2}-i+1})$  The inversion takes the number found to it's binary expansion, then uses that as the choices to include or exclude the natural numbers in the set.  The idea is that the set can be constructed by "labeling" the intersections as was shown for the case where $k=4$.  Since we're choosing two sets out of $k$, where $k=4$ in this case we can extend to $\binom{k}{2}$ intersections.  Now to naively construct the sets you just find which sets have which intersection and just add it to the set, but this doesn't give the full mathematical structure at play.  Things become much clearer when you use the function to analyze them as bitmasks, for $k=4$ you get:
	$\begin{matrix}
	1 & 1 & 1 & 0 & 0 & 0\\
	1 & 0 & 0 & 1 & 1 & 0\\
	0 & 1 & 0 & 1 & 0 & 1\\
	0 & 0 & 1 & 0 & 1 & 1\\
\end{matrix}	 $  The symmetry becomes clear, when you look at the bottom right you see clearly an identity matrix with size of $k-1$, then you look to the right half and notice that it is a flipped and inverted copy of the left.  For me at least this is when it became clear how to program this via bitwise operations on intergers.  
		
	\end{enumerate}
\end{document}