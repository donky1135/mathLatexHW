\documentclass[12pt, letterpaper]{article}
\date{\today}
\usepackage[margin=1in]{geometry}
\usepackage{amsmath}
\usepackage{hyperref}

\usepackage{amssymb}
\usepackage{fancyhdr}
\usepackage{pgfplots}
\usepackage{booktabs}
\usepackage{pifont}
\usepackage{amsthm,latexsym,amsfonts,graphicx,epsfig,comment}
\pgfplotsset{compat=1.16}
\usepackage{xcolor}
\usepackage{tikz}
\usetikzlibrary{shapes.geometric}
\usetikzlibrary{arrows.meta,arrows}
\newcommand{\Z}{\mathbb{Z}}
\newcommand{\N}{\mathbb{N}}
\newcommand{\R}{\mathbb{R}}
\newcommand{\Po}{\mathcal{P}}

\author{Alex Valentino}
\title{Assignment 5}
\pagestyle{fancy}
\renewcommand{\headrulewidth}{0pt}
\renewcommand{\footrulewidth}{0pt}
\fancyhf{}
\rhead{
	Assignment 5 problem 4\\
	300H	
}
\lhead{
	Alex Valentino\\
}

\begin{document}
	(a) Prove: For any sets $A, B, C$ and $D,(A \times B) \cap(C \times D)=(A \cap C) \times(B \cap D)$.\\
	Suppose $A, B, C,D$ are arbitrary sets.  We must show $D,(A \times B) \cap(C \times D)=(A \cap C) \times(B \cap D)$.  By the defintion of set equality, we must show $(A \times B) \cap(C \times D) \subseteq (A \cap C) \times(B \cap D)$ and $(A \cap C) \times(B \cap D) \subseteq (A \times B) \cap(C \times D)$
	\begin{enumerate}
		\item We must show $(A \times B) \cap(C \times D) \subseteq (A \cap C) \times(B \cap D)$.   Suppose $(x,y) \in (A \times B) \cap(C \times D).$  By the definition of set intersection $(x,y) \in (A \times B)$ and $(x,y) \in (C \times D).$  Therefore by the definition of binary relation, $x \in A$ and $y \in B$ and $x \in C$ and $y \in D$.  By the definition of set intersection $x \in A \cap C$ and $y \in B \cap D.$  By the definition of binary relation $(x,y) \in (A \cap C) \times(B \cap D).$
		\item We must show  $(A \cap C) \times(B \cap D) \subseteq (A \times B) \cap(C \times D)$.  Suppose $(x,y) \in (A \cap C) \times(B \cap D).$  By the definition of binary relation $x \in A \cap C$ and $y \in B \cap D.$  By the definition of set intersection $x \in A$ and $x \in C$, and $y \in B$ and $y \in D$.  By the definition of binary relation $(x,y) \in A \times B$ and $(x,y) \in (C \times D).$  Therefore by the definition of set intersection $(A \times B) \cap(C \times D)$.
	\end{enumerate}
(b) Show that if we replace $\cap$ in all three places by $\cup$ in the previous assertion, then it is false.\\
Let $A = \{1,2\}, B= \{3\}, C = \{1\}, D = \{2,3\}$. $ \{(1,3),(2,3),(1,4)\} = (A \times B) \cup(C \times D) \neq (A \cup C) \times(B \cup D)= \{(1,3),(2,3),(1,4),(2,4)\}$.

(c) Prove: For any sets $A, B, C$, and $D,(A \times B) \cup(C \times D) \supseteq(A \cap C) \times(B \cup D)$.\\
Suppose $A, B, C, D$ are arbitrary sets.  We most show for all $(x,y) \in (A \cap C) \times(B \cup D) $ implies $(x,y) \in (A \times B) \cup(C \times D).$  Suppose $(x,y) \in (A \cap C) \times(B \cup D).$  By the definition of binary relation $x \in(A \cap C) $ and $y \in (B \cup D)$.  Then by the definition of set intersection $x\in A$ and $x \in C$, and by the definition of set union $y \in B$ or $y \in D$.  We now have two goals: $x \not \in A \times B$ and $x \not \in C \times D.$
\begin{enumerate}
	\item Assume $(x,y) \not \in A \times B.$  Therefore $x \not \in A$ or $y \not \in B.$  Since we know $x \in A, $ then $x \not in B.$  Thus since we have $y \in B$ or $y \in D,$ then $y$ must be a member of $D$.  Therefore by the definition of cartesean product $(x,y) \in C \times D.$
	\item Assume $(x,y) \not \in C \times D.$  Therefore $x \not \in C$ or $y \not \in D.$  Since we know $x \in C,$ then $y \not \in D.$  Since we also have $y \in B$ or $y \in D$, then $y \in B.$  Therefore since we already have $x \in A,$ then by the definition of cartesean product we have $(x,y) \in A \times B.$
	
\end{enumerate}
\end{document}