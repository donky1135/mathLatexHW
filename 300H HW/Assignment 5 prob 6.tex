\documentclass[12pt, letterpaper]{article}
\date{\today}
\usepackage[margin=1in]{geometry}
\usepackage{amsmath}
\usepackage{hyperref}

\usepackage{amssymb}
\usepackage{fancyhdr}
\usepackage{pgfplots}
\usepackage{booktabs}
\usepackage{pifont}
\usepackage{amsthm,latexsym,amsfonts,graphicx,epsfig,comment}
\pgfplotsset{compat=1.16}
\usepackage{xcolor}
\usepackage{tikz}
\usetikzlibrary{shapes.geometric}
\usetikzlibrary{arrows.meta,arrows}
\newcommand{\Z}{\mathbb{Z}}
\newcommand{\N}{\mathbb{N}}
\newcommand{\R}{\mathbb{R}}
\newcommand{\Po}{\mathcal{P}}

\author{Alex Valentino}
\title{Assignment 5}
\pagestyle{fancy}
\renewcommand{\headrulewidth}{0pt}
\renewcommand{\footrulewidth}{0pt}
\fancyhf{}
\rhead{
	Assignment 5 problem 6\\
	300H	
}
\lhead{
	Alex Valentino\\
}

\begin{document}
	\textit{The following theorem appears as part 1 of Corollary 6.37 in the book. For any two
functions $f : B \to C$ and $g : A \to B$ if $f$ is one-to-one and g is one-to-one then $f\circ g$ is one-
to-one. There is a proof given that uses the equivalence of one-to-one and left-invertibility.
Give an alternative proof that works directly from the definition of one-to-one without using
left-invertibility. (This is Exercise 6.4.8 in the book.)}\\

	Suppose $f : B \to C$ and $g : A \to B$ are arbitrary, injective, functions.  We must show that $f \circ g$ is injective.  By definition of injectivity we must show for arbitrary elements $x_1, x_2$ that if $f \circ g (x_1) = f \circ g (x_2)$ then $x_1 = x_2.$  By the definition of function composition we must show for arbitrary elements $x_1, x_2$ that if $f (g (x_1)) = f (g (x_2))$ then $x_1 = x_2.$ Suppose $x_1, x_2$ are arbitrary elements in $A$.  Then by the definition of a function $g(x_1), g(x_2)$ are elements in $range(g), $ which since $range(g) \subseteq B$ would make them elements in B.  Therefore by the definition of injectivity, since $f(g(x_1) = f(g(x_2))$ then $g(x_1)=g(x_2).$  Therefore by the definition of injectivity since  $g(x_1)=g(x_2)$ then $x_1 = x_2.$
\end{document}