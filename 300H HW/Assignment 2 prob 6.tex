\documentclass[12pt, letterpaper]{article}
\date{\today}
\usepackage[margin=1in]{geometry}
\usepackage{amsmath}
\usepackage{hyperref}

\usepackage{amssymb}
\usepackage{fancyhdr}
\usepackage{pgfplots}
\usepackage{booktabs}
\usepackage{pifont}
\usepackage{amsthm,latexsym,amsfonts,graphicx,epsfig,comment}
\pgfplotsset{compat=1.16}
\usepackage{xcolor}
\usepackage{tikz}
\usetikzlibrary{shapes.geometric}
\usetikzlibrary{arrows.meta,arrows}
\newcommand{\Z}{\mathbb{Z}}
\newcommand{\N}{\mathbb{N}}
\newcommand{\R}{\mathbb{R}}
\newcommand{\Po}{\mathcal{P}}

\author{Alex Valentino}
\title{Assignment 2}
\pagestyle{fancy}
\renewcommand{\headrulewidth}{0pt}
\renewcommand{\footrulewidth}{0pt}
\fancyhf{}
\rhead{
	Assignment 2 problem 6\\
	300H	
}
\lhead{
	Alex Valentino\\
}
\begin{document}
	If $S$ is a set, a partition of $S$ is a set $P$ of nonempty subsets of $S$ satisfying the following two
conditions: (i) for each $s \in S$, there is an $M \in P$ such that $s \in M$ , and (ii) For each pair
$M1, M2 \in P$ such that $M1 \neq M2$, we have $M1 \cap M2 = \emptyset$. The members of the partition are
called the parts of the partition
	\begin{itemize}
		\item \textit{Find all possible partitions of $\{1, 2, 3, 4\}$.}\\
		$\Po = \{\{\{1,2\},\{3\}\},\{\{1\},\{2,3\}\},\{\{1,3\},\{2\}\},\{\{1,2,3\}\}\}$
		\item \textit{Give an example of a set of subsets of $\{1, 2, 3, 4, 5\}$ that satisfies (ii) but not (i).}\\
		$\{\{\{1,2\}\},\{\{3,4\}\}\}$
		\item \textit{Give an example of a set of subsets of $\{1, 2, 3, 4, 5\}$ that satisfies (i) but not (ii).}\\
		$\{\{\{1\},\{2,3,4,5\}\},\{\{1\},\{2\},\{3,4,5\}\}\}$
		\item \textit{Give an example of a partition of the set $\{1\}$.}
		$\{\{1\}\}$
		\item \textit{Is it possible to have a partition of set $\{\}$. Why or why not?}
		It is.  Since a partition can only have non-empty subsets, there are no partitions, thus the empty set's partition is just the empty set in braces, or $\{\{\}\}$.  
		
	\end{itemize}
\end{document}