\documentclass[12pt, letterpaper]{article}
\date{\today}
\usepackage[margin=1in]{geometry}
\usepackage{amsmath}
\usepackage{hyperref}
\usepackage{cancel}
\usepackage{amssymb}
\usepackage{fancyhdr}
\usepackage{pgfplots}
\usepackage{booktabs}
\usepackage{pifont}
\usepackage{amsthm,latexsym,amsfonts,graphicx,epsfig,comment}
\pgfplotsset{compat=1.16}
\usepackage{xcolor}
\usepackage{tikz}
\usetikzlibrary{shapes.geometric}
\usetikzlibrary{arrows.meta,arrows}
\newcommand{\Z}{\mathbb{Z}}
\newcommand{\N}{\mathbb{N}}
\newcommand{\R}{\mathbb{R}}
\newcommand{\Po}{\mathcal{P}}

\author{Alex Valentino}
\title{Talk seminar}
\pagestyle{fancy}
\renewcommand{\headrulewidth}{0pt}
\renewcommand{\footrulewidth}{0pt}
\fancyhf{}
\rhead{
	Lucas' theorem and fractals\\
}
\lhead{
	Alex Valentino\\
}
\begin{document}
	\begin{enumerate}
		\item[definition] $\displaystyle \binom{n}{k} = \begin{cases} n \geq k & \frac{n!}{k! (n-k)! } \\
		n < k & 0
		\end{cases}$
	
			
		\item[Lemma 1] Given a prime $p$ and a natural number $k$ such that $0 < k < p$ then 
		$
		p \mid \binom{p}{k}.
		$\\
		Proof: Note that $\binom{p}{k} = \frac{p!}{k! (p-k)!}$.  Since $p$ is prime, and $k < p$ and $ p-k < p$, then there is no number which divides  $p$ in that range, therefore $p$ cannot be divided out of  the numerator.  
		\item[Coro] Given a prime $p$, then $(1 + x)^p \equiv 1 + x^p \mod p$.\\
		Proof: By the binomial theorem $(1+x)^p = \sum_{k=0}^p \binom{p}{k}x^k = \binom{p}{0} + \binom{p}{p} x^p + \sum_{k=1}^{p-1} \binom{p}{k}x^k \equiv 1 + x^p + 0 \mod p = 1 + x^p $
		\item[Coro 2] Given a prime $p$, for all $i \in \N \cup \{0\}$, $(1+x)^{p^i} \equiv 1 + x^{p^i}$.\\
		We have already shown the base case in the form of coro 1.  By the principle of mathematical induction for all $j \in \N$ if $j < i$ then $(1+x)^{p^j} \equiv 1 + x^{p^j} \mod p$.  Since $i - 1 < i$, then by the induction hypothesis $(1+x)^{p^{i-1}} \equiv 1 + x^{p^{i-1}} \mod p$.  Therefore $(1+x)^{p^i} = (1+x)^{p^{i-1}p} = ((1+x)^{p^{i-1}})^p \equiv (1+x^{p^{i-1}})^p \mod p = \sum_{k=0}^p \binom{p}{k} x^{p^{i-1}k} \equiv 1 + x^{p^i}$ .
		\item[Lemma 2] Given any natural number $n$ and $d$, there exists a unique $q,r \in \Z_{\geq 0}$ such that $0 \leq r < d $ where $n = dq + r$.\\
		Proof: We have two cases: if $d \mid n$ and $d \nmid n$.  If $d \mid n$, then by definition of divisibility there exists $k \in \N$ such that $n = kd$, which gives us our unique $q$ and $r = 0$.  Suppose $d \nmid n$.  Then we can define the set $S$ where $S = \{m \in \Z_+ : d \mid n - m\}$.  Note that this set is non-empty as $n \in S$ since $d$ divides $0$.  Therefore by the well ordering principle $S$ contains a smallest element.   Let's call the smallest element $g$. This element provides the existance for the element $q$ as $n -g = dq$.  We must check if $g$ checks our requirements.  Since $g$ is a positive integer then $g \geq 0$ by definition.  Therefore we must check $g < d$.  Suppose for contradiction that $g \geq d$.  This provides 2 cases for $g$, if $g=d$ and $g > d$.  Suppose $g = d$.  Then we have $dq + g = d(q+1)$, contradicting the fact that $d \nmid n$.  Suppose $g > d$.  Then we have that $g -d > 0$.  Let $g -d  = g_1$.  Therefore $n - g_1 = n -g + d = dq + g - g + d = d(q+1)$, thus $g_1 \in S$.  However this contradicts the fact that $g$ is the smallest element in $S$ as $g_1 < g$.  Therefore $0 \leq g < d$, thus setting $g = r$ and $q = \frac{n-g}{d}$ satisifes the requirements.\\
		
		Uniqueness:  Suppose $n = d b_1 + b_2 = d c_1 + c_2, 0 \leq b_2,c_2 < d$. Then $-d < b_2 - c_2 < d$.  Since $b_2 - c_2 = d (c_1 - b_1)$ then $-1 < c_1 - b_1 < 1$.  Therefore $c_1 - b_1 = 0$, thus $c_1 = b_1$.  Thus $c_2 = b_2$.  
		\item[Lemma 3] Given any natural number $d$ and non-negative integer $n$ then there exist $r_0,\ldots, r_m \in \Z_{\geq 0}, 0 \leq r_i < d$ for all $i$ such that $n = d^m r_m + \ldots d r_1 + r_0$.  \\
		
		By PMI for all $k \in \N$ if $k < n$ then there exists $s_0,\ldots s_l \in \Z_{\geq 0}, 0 \geq s_i < $ such that $k = d^l s_l + \ldots + d s_1 + s_0$.  By lemma $2$    \\
		
		\item[Lucas' Theorem] Given any $m,n \in \N \cup \{0\}$, $\binom{m}{n} = \prod_{i=0}^k \binom{m_i}{n_i}$, where $m = m_k p^k + \ldots + m_1 p + m_0, n = n_k p^k + \ldots + n_1 p + n_0$
	\end{enumerate}
\end{document}