\documentclass[12pt, letterpaper]{article}
\date{\today}
\usepackage[margin=1in]{geometry}
\usepackage{amsmath}
\usepackage{hyperref}

\usepackage{amssymb}
\usepackage{fancyhdr}
\usepackage{pgfplots}
\usepackage{booktabs}
\usepackage{pifont}
\usepackage{amsthm,latexsym,amsfonts,graphicx,epsfig,comment}
\pgfplotsset{compat=1.16}
\usepackage{xcolor}
\usepackage{tikz}
\usetikzlibrary{shapes.geometric}
\usetikzlibrary{arrows.meta,arrows}
\newcommand{\Z}{\mathbb{Z}}
\newcommand{\N}{\mathbb{N}}
\newcommand{\R}{\mathbb{R}}
\newcommand{\Po}{\mathcal{P}}
\newcommand{\Ss}{\mathcal{S}}


\author{Alex Valentino}
\title{Assignment 6}
\pagestyle{fancy}
\renewcommand{\headrulewidth}{0pt}
\renewcommand{\footrulewidth}{0pt}
\fancyhf{}
\rhead{
	Assignment 6 problem 4\\
	300H	
}
\lhead{
	Alex Valentino\\
}
\begin{document}
	A set $\Ss$ of sets is said to be intersecting if for any two members $A$ and $B$ of $\Ss$, we have $A \cap B \neq \emptyset$.  Prove that for any nonempty set $U$ and for any intersecting collection $\Ss$ of
subsets of $U$ and for any $X \subseteq U$ at least one of the two collections $\Ss \cup \{X\}$ and $\Ss \cup \{U \backslash X\}$ is intersecting.  \\
Proof: We must show that $\Ss \cup \{X\}$ or $\Ss \cup \{U \backslash X\}$ is intersecting.  Assume $X$ is an arbitrary subset of $U$.  Assume $\Ss \cup \{X\}$ is not intersecting.  We must show $\Ss \cup \{U \backslash X\}$ is intersecting.  By definition of being not intersecting, there exist $A,B \in \Ss \cup \{X\}$ such that $A \cap B = \emptyset$.  Since $\Ss$ was defined to be previously intersecting, we know that $X$ is disjoint with some set in $\Ss$, then without loss of generality $A = X.$  Since $B \subseteq U,$ and $A,B$ were disjoint, then $B \subseteq U \backslash X,$ and therefore $B \cap (U \backslash X) \neq \emptyset.$  Since $B \subseteq (U \backslash X)$ and $\Ss$ is intersecting, then all the other sets of $\Ss$ also have non-empty intersections.  Therefore $\Ss \cup \{U \backslash X\}$ is intersecting.  
\end{document}