\documentclass[12pt, letterpaper]{article}
\date{\today}
\usepackage[margin=1in]{geometry}
\usepackage{amsmath}
\usepackage{hyperref}

\usepackage{amssymb}
\usepackage{fancyhdr}
\usepackage{pgfplots}
\usepackage{booktabs}
\usepackage{pifont}
\usepackage{amsthm,latexsym,amsfonts,graphicx,epsfig,comment}
\pgfplotsset{compat=1.16}
\usepackage{xcolor}
\usepackage{tikz}
\usetikzlibrary{shapes.geometric}
\usetikzlibrary{arrows.meta,arrows}
\newcommand{\Z}{\mathbb{Z}}
\newcommand{\N}{\mathbb{N}}
\newcommand{\R}{\mathbb{R}}
\newcommand{\Po}{\mathcal{P}}

\author{Alex Valentino}
\title{Assignment 4}
\pagestyle{fancy}
\renewcommand{\headrulewidth}{0pt}
\renewcommand{\footrulewidth}{0pt}
\fancyhf{}
\rhead{
	Assignment 4 problem 2\\
	300H	
}
\lhead{
	Alex Valentino\\
}
\begin{document}
	\textit{Use a joint truth table to compare the following logical expressions. For which
pairs or expressions does one logically follow from the other. Which pairs are logically equivalent?}
$E=(A \Longrightarrow B) \Longrightarrow C$.\\
$F=A \Longrightarrow(B \Longrightarrow C)$.\\
$G=(B \Longrightarrow C) \Longrightarrow(A \Longrightarrow C)$.\\
$H=(A \Longrightarrow C) \Longrightarrow(B \Longrightarrow C)$.\\
$I=(A \Longrightarrow C) \Longleftrightarrow(B \Longrightarrow C)$.\\
E: \begin{array}{ccc|c@{}c@{}ccc@{}ccc@{}c}
a&b&c&(&(&a&\rightarrow&b&)&\rightarrow&c&)\\\hline
1&1&1&&&1&1&1&&\mathbf{1}&1&\\
1&1&0&&&1&1&1&&\mathbf{0}&0&\\
1&0&1&&&1&0&0&&\mathbf{1}&1&\\
1&0&0&&&1&0&0&&\mathbf{1}&0&\\
0&1&1&&&0&1&1&&\mathbf{1}&1&\\
0&1&0&&&0&1&1&&\mathbf{0}&0&\\
0&0&1&&&0&1&0&&\mathbf{1}&1&\\
0&0&0&&&0&1&0&&\mathbf{0}&0&
\end{array}\\
F: \begin{array}{ccc|c@{}ccc@{}ccc@{}c@{}c}
a&b&c&(&a&\rightarrow&(&b&\rightarrow&c&)&)\\\hline
1&1&1&&1&\mathbf{1}&&1&1&1&&\\
1&1&0&&1&\mathbf{0}&&1&0&0&&\\
1&0&1&&1&\mathbf{1}&&0&1&1&&\\
1&0&0&&1&\mathbf{1}&&0&1&0&&\\
0&1&1&&0&\mathbf{1}&&1&1&1&&\\
0&1&0&&0&\mathbf{1}&&1&0&0&&\\
0&0&1&&0&\mathbf{1}&&0&1&1&&\\
0&0&0&&0&\mathbf{1}&&0&1&0&&
\end{array}\\
G: \begin{array}{ccc|c@{}c@{}ccc@{}ccc@{}ccc@{}c@{}c}
a&b&c&(&(&b&\rightarrow&c&)&\rightarrow&(&a&\rightarrow&c&)&)\\\hline
1&1&1&&&1&1&1&&\mathbf{1}&&1&1&1&&\\
1&1&0&&&1&0&0&&\mathbf{1}&&1&0&0&&\\
1&0&1&&&0&1&1&&\mathbf{1}&&1&1&1&&\\
1&0&0&&&0&1&0&&\mathbf{0}&&1&0&0&&\\
0&1&1&&&1&1&1&&\mathbf{1}&&0&1&1&&\\
0&1&0&&&1&0&0&&\mathbf{1}&&0&1&0&&\\
0&0&1&&&0&1&1&&\mathbf{1}&&0&1&1&&\\
0&0&0&&&0&1&0&&\mathbf{1}&&0&1&0&&
\end{array}\\
H : \begin{array}{ccc|c@{}c@{}ccc@{}ccc@{}ccc@{}c@{}c}
a&b&c&(&(&a&\rightarrow&c&)&\rightarrow&(&b&\rightarrow&c&)&)\\\hline
1&1&1&&&1&1&1&&\mathbf{1}&&1&1&1&&\\
1&1&0&&&1&0&0&&\mathbf{1}&&1&0&0&&\\
1&0&1&&&1&1&1&&\mathbf{1}&&0&1&1&&\\
1&0&0&&&1&0&0&&\mathbf{1}&&0&1&0&&\\
0&1&1&&&0&1&1&&\mathbf{1}&&1&1&1&&\\
0&1&0&&&0&1&0&&\mathbf{0}&&1&0&0&&\\
0&0&1&&&0&1&1&&\mathbf{1}&&0&1&1&&\\
0&0&0&&&0&1&0&&\mathbf{1}&&0&1&0&&
\end{array}\\
I : \begin{array}{ccc|c@{}c@{}ccc@{}ccc@{}ccc@{}c@{}c}
a&b&c&(&(&a&\rightarrow&c&)&\leftrightarrow&(&b&\rightarrow&c&)&)\\\hline
1&1&1&&&1&1&1&&\mathbf{1}&&1&1&1&&\\
1&1&0&&&1&0&0&&\mathbf{1}&&1&0&0&&\\
1&0&1&&&1&1&1&&\mathbf{1}&&0&1&1&&\\
1&0&0&&&1&0&0&&\mathbf{0}&&0&1&0&&\\
0&1&1&&&0&1&1&&\mathbf{1}&&1&1&1&&\\
0&1&0&&&0&1&0&&\mathbf{0}&&1&0&0&&\\
0&0&1&&&0&1&1&&\mathbf{1}&&0&1&1&&\\
0&0&0&&&0&1&0&&\mathbf{1}&&0&1&0&&
\end{array}\\


Since $I = H \wedge G,$ $I \implies H$ and $I \implies G$ since as shown above none of the statements are tautologies, therefore since $I = H \wedge G, $ then $H \wedge G \implies H$ and $H \wedge G \implies G$.  Which happen to be tautologies, thus they logically follow.  However they aren't logically equivalent as $H \implies H \wedge G$ isn't a tautology: $(1 \implies 1 \wedge 0) = 0$.  E is a logical implication of H, and E is a logical implication of F.  None of the pairs are logically equivalent.  
\end{document}