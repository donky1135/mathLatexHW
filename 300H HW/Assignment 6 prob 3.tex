\documentclass[12pt, letterpaper]{article}
\date{\today}
\usepackage[margin=1in]{geometry}
\usepackage{amsmath}
\usepackage{hyperref}

\usepackage{amssymb}
\usepackage{fancyhdr}
\usepackage{pgfplots}
\usepackage{booktabs}
\usepackage{pifont}
\usepackage{amsthm,latexsym,amsfonts,graphicx,epsfig,comment}
\pgfplotsset{compat=1.16}
\usepackage{xcolor}
\usepackage{tikz}
\usetikzlibrary{shapes.geometric}
\usetikzlibrary{arrows.meta,arrows}
\newcommand{\Z}{\mathbb{Z}}
\newcommand{\N}{\mathbb{N}}
\newcommand{\R}{\mathbb{R}}
\newcommand{\Q}{\mathbb{Q}}
\newcommand{\Po}{\mathcal{P}}

\author{Alex Valentino}
\title{Assignment 6}
\pagestyle{fancy}
\renewcommand{\headrulewidth}{0pt}
\renewcommand{\footrulewidth}{0pt}
\fancyhf{}
\rhead{
	Assignment 6 problem 3\\
	300H	
}
\lhead{
	Alex Valentino\\
}
\begin{document}
	Claim: for $n \in \Z_+,$ if $a$ is the exponent of the largest power of two that divides $n$ and $b$ is $\lceil \frac{n}{2^{a+1}} \rceil$ then the function $f : \Z_+ \to \R$ given by $f(n) = \frac{a}{b}$ has a range of $\Q_{\geq0}$\\
	Proof: We must show that $f$ has a range of $Q_{\geq 0}$.  Suppose $n$ is an arbitrary natural number.  Then by definition it has a prime factorization $n=2^ap_1^{a_1}\cdots p_k^{a_k}$.  By definition of $range(f) = \Q_{\geq 0}$ we must show for all $n \in \Z_+, f(n) = \frac{a}{b}$ where $a \in \Z_{\geq 0}$ and $b \in \Z_+.$  
	\begin{itemize}
		\item (Proof of $a \in \Z_{\geq 0}$):   We claim that the exponent of $2$ in $n$'s prime factorization has a range of $\Z_{\geq 0}$.  Since $\Z_{+}$ contains the sequence $\{2^i\}_{i=0}^\infty$, and $2$ is prime, then every natural number is a multiple of that series.  Thus the range of $a$ is $\Z_+$ since $f$ is defined for every member of that sequence.  
		\item (Proof of $b \in \Z_+$):  By substituting in $n$'s prime factorization in $b$ we get $b = \lceil \frac{2^ap_1^{a_1}\cdots p_k^{a_k}}{2^{a+1}} \rceil = \lceil \frac{p_1^{a_1}\cdots p_k^{a_k}}{2} \rceil$.  Since $p_1^{a_1}\cdots p_k^{a_k}$ represents every odd natural number, as all primes greater than $2$  are odd, then we can say by definition $p_1^{a_1}\cdots p_k^{a_k} = 2l-1, l \in \Z_+$, therefore $b$ becomes $\lceil \frac{2l-1}{2} \rceil.$  
		Since the ceiling function is the closest integer greater than or equal to it's input, and $\lceil \frac{2l-1}{2} \rceil = \lceil l - \frac{1}{2} \rceil$, then $\lceil l - \frac{1}{2} \rceil = l$, as $l$ is the smallest integer larger than $l -\frac{1}{2}$.  Therefore since $l$ is an arbitrary positive integer, the range of $b$ is $\Z_+.$
	\end{itemize}
	Therefore since $a$ and $b$ independently range over $\Z_{\geq 0}$ and $\Z_+$, then $range(f) = \Q_{\geq 0}.$
\end{document}