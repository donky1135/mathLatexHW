\documentclass[12pt, letterpaper]{article}
\date{\today}
\usepackage[margin=1in]{geometry}
\usepackage{amsmath}
\usepackage{hyperref}

\usepackage{amssymb}
\usepackage{fancyhdr}
\usepackage{pgfplots}
\usepackage{booktabs}
\usepackage{pifont}
\usepackage{amsthm,latexsym,amsfonts,graphicx,epsfig,comment}
\pgfplotsset{compat=1.16}
\usepackage{xcolor}
\usepackage{tikz}
\usetikzlibrary{shapes.geometric}
\usetikzlibrary{arrows.meta,arrows}
\newcommand{\Z}{\mathbb{Z}}
\newcommand{\N}{\mathbb{N}}
\newcommand{\R}{\mathbb{R}}
\newcommand{\Po}{\mathcal{P}}

\author{Alex Valentino}
\title{Assignment 4}
\pagestyle{fancy}
\renewcommand{\headrulewidth}{0pt}
\renewcommand{\footrulewidth}{0pt}
\fancyhf{}
\rhead{
	Assignment 4 problem 1\\
	300H	
}
\lhead{
	Alex Valentino\\
}
\begin{document}
	For each of the following assertions, identify the free variables and the bound variables. Briefly explain your reasoning.\\
	\begin{itemize}
		\item \textit{(Variables: integers $n, m, r$,) For every positive integer $n$ the set $\{m \in \Z : m^2 - r \mid n \}$}\\
		$n$ is bound and $m,r$ are free.  Clearly $n$ is quantified with the "for every" statement and $m$ and $r$ are the variables for which inserting values for will turn this predicate into an assertion.  
		\item \textit{(Variables: real numbers $x, y, \epsilon$, and subset $S\subseteq \R$) $x$ is not a member of $S$ and for all real numbers $\epsilon > 0$, there exist a member $y$ of $S$ such that $|x-y|\geq \epsilon.$}  $\epsilon,y$ are bound as they come included with "for all" and "for every" respectively.  $x,S$ are the free variables as this statement is evaluating on $x$ and that evaluation is being stored in the set $S$.    
		\item \textit{(Variables: functions $f , g $and $h$ and real number $x$ ) There is a function $g$ and a
functon $h$ such that for every real number $x, f (x) = g(x) + h(x)$ and $g(x) = g(-x)$ and
$h(-x) = -h(x)$.}\\
		$h,g,x$ are bound variables, as for the functions $h,g$ the "there is" statement could have been rewritten as "there exist" and not change the meaning, and for the variable has a direct "for every" statement associated with it.  $f$ is the free variable, as all the other variables are constructed around this arbitrary $f$, thus it is the free variable on which this statement goes from a predicate to an assertion.
	\end{itemize}
 
\end{document}