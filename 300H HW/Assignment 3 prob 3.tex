\documentclass[12pt, letterpaper]{article}
\date{\today}
\usepackage[margin=1in]{geometry}
\usepackage{amsmath}
\usepackage{hyperref}

\usepackage{amssymb}
\usepackage{fancyhdr}
\usepackage{pgfplots}
\usepackage{booktabs}
\usepackage{pifont}
\usepackage{amsthm,latexsym,amsfonts,graphicx,epsfig,comment}
\pgfplotsset{compat=1.16}
\usepackage{xcolor}
\usepackage{tikz}
\usetikzlibrary{shapes.geometric}
\usetikzlibrary{arrows.meta,arrows}
\newcommand{\Z}{\mathbb{Z}}
\newcommand{\N}{\mathbb{N}}
\newcommand{\R}{\mathbb{R}}
\newcommand{\Po}{\mathcal{P}}

\author{Alex Valentino}
\title{Assignment 3}
\pagestyle{fancy}
\renewcommand{\headrulewidth}{0pt}
\renewcommand{\footrulewidth}{0pt}
\fancyhf{}
\rhead{
	Assignment 3 problem 3\\
	300H	
}
\lhead{
	Alex Valentino\\
}
\begin{document}
	{A finite arithmetic progression is a list of numbers with the property that the difference
between any two successive entries of the list is the same.}
	\begin{itemize}
		\item \textit{Give three examples of finite arithmetic progressions.}
		\begin{enumerate}
			\item $(1,2,3)$
			\item $(2,4,6)$
			\item $(1005, 1010, 1015)$
		\end{enumerate}
		\item \textit{What is the minimum information you need to completely specify an arithmetic progression? (The information should be represented by some input parameters that can
be used to fully describe the list.)}\\
		An arithmetic progression requires three parameters, the starting number $s$, the increment $p$, and the number of elements to be in the list besides the initial value, $n$.  
		\item \textit{In terms of the parameters given in the previous part, provide a specification for the
terms of the arithmetic progression (as a list).}
		The numbers above fully specify the list $(s, s + p, \cdots, s+np)$.
	\end{itemize}
\end{document}