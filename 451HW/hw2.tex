\documentclass[12pt, letterpaper]{article}
\date{\today}
\usepackage[margin=1in]{geometry}
\usepackage{amsmath}
\usepackage{hyperref}
\usepackage{cancel}
\usepackage{amssymb}
\usepackage{fancyhdr}
\usepackage{pgfplots}
\usepackage{booktabs}
\usepackage{pifont}
\usepackage{amsthm,latexsym,amsfonts,graphicx,epsfig,comment}
\pgfplotsset{compat=1.16}
\usepackage{xcolor}
\usepackage{tikz}
\usetikzlibrary{shapes.geometric}
\usetikzlibrary{arrows.meta,arrows}
\newcommand{\Z}{\mathbb{Z}}
\newcommand{\N}{\mathbb{N}}
\newcommand{\R}{\mathbb{R}}
\newcommand{\Q}{\mathbb{Q}}
\newcommand{\C}{\mathbb{C}}

\newcommand{\Po}{\mathcal{P}}
\newcommand{\Pro}{\mathbb{P}}
\author{Alex Valentino}
\title{451 homework}
\pagestyle{fancy}
\renewcommand{\headrulewidth}{0pt}
\renewcommand{\footrulewidth}{0pt}
\fancyhf{}
\rhead{
	Homework 2\\
	451	
}
\lhead{
	Alex Valentino\\
}
\begin{document}
\begin{enumerate}
	\item[4.8] \textit{Prove that a set $B = (v_1,\cdots,v_n)$ of vectors in $F^n$ is a basis if and only if the matrix obtained by assembling the coordinate vectors of $v_i$ is invertible}
	\begin{itemize}
		\item $\Rightarrow$ Suppose the set $(v_1,\cdots,v_n)$ is a basis of $F^n$.  We want to show that the matrix given by $(v_1,\cdots,v_n)$ is invertible.  
		Since $B$ is a basis, then it is linearly independent.  Therefore, 
		it has a non-zero determinate.  Therefore, since there are $n$ vectors with $n$ elements, then the matrix given by $B$ is invertible
		\item $\Leftarrow$ Suppose the matrix represented by $B$ in $F^n$ is invertible.  Since the matrix is invertible then it has a non-zero determinate.  
		Therefore it's columns are linearly independent.  Since there is a set of $n$ linearly independent vectors in $F^n$, then it must span $F^n$.  Therefore $B$ is a basis.  
	\end{itemize}
	\item[5.2]
	\begin{enumerate}
		\item The base change matrix is given by:\\
		$\begin{bmatrix}
		1 & 1\\
		1 & -1
		\end{bmatrix}.$
		\item The matrix which swaps all of the basis vectors to the opposite order is given by:
		$$
		\begin{bmatrix}
			0 &\cdots & 0 & 1\\
			0 &\cdots & 1 & 0\\
			\vdots  & \reflectbox{$\ddots$}& \vdots & \vdots\\
			1& \cdots & 0 & 0
		\end{bmatrix}
		$$
	\end{enumerate}
	\item[6.1] \textit{Prove that $M_n(\R)$ is the direct sum of the space of symmetry and the space of skew symmetric matrices}
	\begin{itemize}
		\item We will show that the intersection of the set of symmetric matrices and skew symmetric matricies is exactly the 0 matrix.  Suppose for contradiction that there is a non-zero matrix where is both skew symmetric and symmetric.  Let this matrix be denoted $E$.  Then $E = E^t, E = -E^t$.  Therefore $2E = 0$.  Therefore $E$ is identically the 0 matrix, giving us the desired contradiction.  Therefore the only element shared between the two subspaces is the 0 matrix.
		\item Suppose $A \in M_n(\R)$.  We will show there exists a symmetric matrix $S$ and a skew symmetric matrix $K$ such that $A = S + K$.  We claim that 
		$S = (\frac{a_{ij} + a_{ji}}{2}), K = (\frac{a_{ij} - a_{ji}}{2})$ satisfy the requirements.
		\begin{itemize}
			\item We claim that $S$ is symmetric.  Note that $(S)_{ij} = \frac{a_{ij} + a_{ji}}{2}$ and $(S)_{ji} = \frac{a_{ji} + a_{ij}}{2}$.
			Therefore $(S)_{ij} = (S)_{ji}$.  
			\item We claim that $K$ is skew symmetric.  Note that $(K)_{ij} = \frac{a_{ij} - a_{ji}}{2}$ and that $(K)_{ji} = \frac{a_{ji} - a_{ij}}{2}$.
			Therefore $-(K)_{ij} = -\frac{a_{ij} - a_{ji}}{2} = \frac{a_{ji} - a_{ij}}{2} = K_{ji}$.  Thus $K$ is skew symmetric.  
			\item We claim that $A = S + K$.  Note that 
			$$
			(A)_{ij}  = a_{ij} = \frac{2a_{ij}}{2} = \frac{a_{ij} + a_{ji}}{2} + \frac{a_{ij} - a_{ji}}{2} = (S)_{ij} + (K)_{ij}
			$$ 
			and
			$$
			(A)_{ji} = a_{ji} = \frac{2a_{ji}}{2} = \frac{a_{ji} + a_{ij}}{2} +
			\frac{a_{ji} - a_{ij}}{2} = (S)_{ji} + (K)_{ji}
			$$
			Therefore $A = S + K$.  
		\end{itemize}
	\end{itemize}
	\item[1.1] Suppose $A \in M_{l \times m}(F), B \in M_{n\times p}(F)$.
	We want to show for $M \in M_{m \times n} (F)$ that $T(M) = AMB$ is linear.
	Suppose $c \in F, N \in M_{m \times n} (F)$.  Therefore 
	\begin{align*}
		T(cM + N) &= A(cM + N)B\\
		&= (AcM + AN)B\\
		&= AcMB + ANB\\
		&= cAMB + ANB\\
		&= cT(M) + T(N).
	\end{align*}
	Thus $T$ is linear.  
	\item[1.3] \textit{Show using the dimension theorem that the dimension of the solutions to AX=0 is atleast $n-m$.}
	Note that $AX = 0$ is exactly the kernel of $A$.  
	Therefore by applying the dimension theorem we have that $n = dim(im(T)) + dim(ker(T))$.  Note that the rank is maximized by $m$, giving the formula
	$n \leq m + dim(ker(T))$.  Therefore
	$$
	dim(ker(T)) \geq n - m.  
	$$ 
	\item[2.1] \textit{Given two matrices $A = \begin{bmatrix} a_{11} & a_{12}\\	 a_{21}& a_{22}\end{bmatrix}, B = \begin{bmatrix} b_{11} & b_{12}\\	 b_{21}& b_{22}\end{bmatrix} \in F^{2 \times 2}$ show that for arbitrary $M \in F^{2 \times 2}$ the matrix representation of $AMB$.   }\\
	The matrix representation on the unit elements of $F^{2 \times 2}$ 
	with the ordered basis of $(e_{11},e_{12},e_{21},e_{22})$ is the following:\\
	
	\iffalse $\begin{bmatrix}
	\left (
  \begin {array} {cc}
           {a_{11}} {b_{11}} & {a_{11}} {b_{12}} \\
        {a_{21}} {b_{11}} & {a_{21}} {b_{12}} \\
    \end {array}
   \right)&
	\left (
  \begin {array} {cc}
           {a_{11}} {b_{21}} & {a_{11}} {b_{22}} \\
        {a_{21}} {b_{21}} & {a_{21}} {b_{22}} \\
    \end {array}
   \right)\\
	\left (
  \begin {array} {cc}
           {a_{12}} {b_{11}} & {a_{12}} {b_{12}} \\
        {a_{22}} {b_{11}} & {a_{22}} {b_{12}} \\
    \end {array}
   \right)
	& \left (
  \begin {array} {cc}
           {a_{12}} {b_{21}} & {a_{12}} {b_{22}} \\
        {a_{22}} {b_{21}} & {a_{22}} {b_{22}} \\
    \end {array}
   \right)\\
	\end{bmatrix}$\\ \fi
	$\
	\begin{bmatrix}
	{a_{11}} {b_{11}} & {a_{11}} {b_{12}} & {a_{21}} {b_{11}} & {a_{21}} {b_{12}} \\
	{a_{11}} {b_{21}} & {a_{11}} {b_{22}} &
        {a_{21}} {b_{21}} & {a_{21}} {b_{22}} \\
        {a_{12}} {b_{11}} & {a_{12}} {b_{12}} &
        {a_{22}} {b_{11}} & {a_{22}} {b_{12}} \\
        {a_{12}} {b_{21}} & {a_{12}} {b_{22}} &
        {a_{22}} {b_{21}} & {a_{22}} {b_{22}} \\
	\end{bmatrix}
	$
\end{enumerate}
\end{document}
