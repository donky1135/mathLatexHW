\documentclass[12pt, letterpaper]{article}
\date{\today}
\usepackage[margin=1in]{geometry}
\usepackage{amsmath}
\usepackage{hyperref}
\usepackage{cancel}
\usepackage{amssymb}
\usepackage{fancyhdr}
\usepackage{pgfplots}
\usepackage{booktabs}
\usepackage{pifont}
\usepackage{amsthm,latexsym,amsfonts,graphicx,epsfig,comment}
\pgfplotsset{compat=1.16}
\usepackage{xcolor}
\usepackage{tikz}
\usetikzlibrary{shapes.geometric}
\usetikzlibrary{arrows.meta,arrows}
\newcommand{\Z}{\mathbb{Z}}
\newcommand{\N}{\mathbb{N}}
\newcommand{\R}{\mathbb{R}}
\newcommand{\Q}{\mathbb{Q}}
\newcommand{\C}{\mathbb{C}}

\newcommand{\Po}{\mathcal{P}}
\newcommand{\Pro}{\mathbb{P}}
\author{Alex Valentino}
\title{451 homework}
\pagestyle{fancy}
\renewcommand{\headrulewidth}{0pt}
\renewcommand{\footrulewidth}{0pt}
\fancyhf{}
\rhead{
	Homework 6\\
	451	
}
\lhead{
	Alex Valentino\\
}
\begin{document}
\begin{enumerate}
	\item[7.3] Since we want to find the number of elements of order 5, and since the largest power of 5 that divides 10 is 5 then we can apply the third sylow theorem.  Let $s$ be the number of $5$-Sylow groups.  Then by the third $p$-sylow theorem $s \mid 2$ and $s \equiv 1 \mod 5$.  Therefore
	only 1 satisfies the equation.  Since $5$ is prime then our group is cyclic.  Thus every element aside from the identity is of order 5.  Thus our 
	group has 4 elements of order 5. 
	\item[7.5]
	\begin{enumerate}
		\item Since $D_{10}$ has order 20, we must find a subgroup of order 4.  
		$\{1,\theta^5, r, r\theta^5 \}$ is a subgroup of order 4
		\item Note that we have shown before that the homomorphism from $S_4 \to S_3$ has a kernel of order 4, and additionally every element has even parity.  
		Additionally each element of that kernel has order 2, therefore T contains
		an isomorphic copy of the Klein 4 group.
		\item Since $O$ has order 24, we seek a subgroup of order 8. 
		Note that if one fixes a pair of faces, one has a 90 degree rotation and a 
		flip.  Note that that is isomorphic to $D_4$, the set of symmetries of the 
		square.  Since $|D_4| = 8$, we are done
		\item Note that $I \cong A_5$, and we know that $|A_5| 3 \cdot 4 \cdot 5$
		therefore we need to find a subgroup of order 4. Since we found the klein 4 group 	inside of $A_4$ then we can find copies in $A_5$.  
	\end{enumerate}
	\item[7.6] We claim that $<(1234567),(124)(356)> = G$ is a non-abelian subgroup of order 21.  Note that our generators are of orders 7 and 3 respectively.  Additionally, in our non-abelian group by the third p-sylow theorem $<(1234567)>$ must be normal and $<(124)(356)>$ is not.  This is demonstrated by the fact that $(124)(356)(1234567)(142)(365) = (1357246) = (1234567)^2$, which was shown in artin to be the requirement on being in the 
	non-abelian isomorphism class of groups of order 21.  
	\item[7.7] We know by orbit stabilizer that for the action of conjugation by elements of $H$ on a given $s \in S$ satisfies $|H| = |C_H(s)||N_H(s)|$, where $C_H(s)$ is the conjugacy class of $s$ by elements of $H$ and 
	$N_H(s)$ is the normalizer of $s$ by elements in $h$.  Recognizing 
	that $|H| = p$ by construction gives us that either $|C_H(s)| = 1$ or $|C_H(s)| = 1$.  What does this say?  Either $s$ is fixed by $H$ or $s$ 
	after being conjugated by all elements of $H$ forms a cycle since $H$ 
	is isomorphic to $C_p$ then we can conjugate by successive powers of the generator of $H$ until we return to the original element $s$.    
	\item[7.8] 
	\begin{itemize}
		\item The order of $GL_n(F_p)$ is $\prod_{i=0}^{n-1}(p^n - p^i)$.  This is obtained by considering the rows of a given matrix with $F_p$ entries.  Note that the first row can be any element in $F_p^n$ except the 0 vector.  Thus the first row has $p^n-1$ possibilities.  Going a row down
		one can note that every element in $F_p^n$ satisfies except the 0 vector and the multiples of the first row.  Thus there are $p^n-p$ options 
		for the second.  For the $i$th row one has to consider that 
		any linear combination of the first $i-1$ rows.  Since there are 
		$i-1$ rows being multiplied by any element in $F_p$, then the $i$th row has $p^n-p^{i-1}$.  Thus since there are $n$ rows, $|GL_n(F_p)| = \prod_{i=1}^{n}(p^n - p^{i-1}) = \prod_{i=0}^{n-1}(p^n - p^i)$    
		\item Note that $\prod_{i=0}^{n-1}(p^n - p^i) = p^{\frac{n(n-1)}{2}}\prod_{i=0}^{n-1}(p^{n-i}-1)$.  Thus the $p$-sylow subgroups of $GL_n(F_p)$ are
		of order $p^{\frac{n(n-1)}{2}}$.  We claim that the subgroup of unitriangular matrices (the set of upper triangular matrices with 1s in the diagonal) is a $p$-sylow subgroup.  Note that the determinate of all unitriangular matrices is 1, so it is within $GL_n(F_p)$.  Additionally there are $\frac{n(n-1)}{2}$ entries with $p$ possible values per, so the subgroup has order $p^{\frac{n(n-1)}{2}}$.  Thus we have found a p-sylow subgroup.
		\item Since we want to find the number of $p$-sylow subgroups, and
		we know the subgroup of unitriangular matrices is a member, we need
		to find the number of conjugates.  Let $U$ be the set of unitriangular matrices.  We know by orbit stabilizer that $|GL_n(F_p)| = |N(U)||C(U)|$.  We know from problem 7.6.2 that the normalizer of unitriangular matrices is the group of upper triangular matrices.  
		Since upper triangular matrices have $\frac{n(n-1)}{2}$ entries above the diagonal that can take on $p$ values and $n$ entries along on the diagonal which can take on $p-1$ values (we exclude 0 to ensure the matrices remain inveritble).   Thus $|N(U)| = (p-1)^n p^{\frac{n(n-1)}{2}}$.  
		Therefore $|C(U)| = \frac{1}{(p-1)^n} \prod_{i=0}^{n-1}p^i - 1$
	\end{itemize}
	\item[7.9]
	\begin{enumerate}
		\item By the third $p$-sylow theorem, if we consider $s_p$ as the
		number of $p$-sylow groups, then $11 \mid s_3, s_3 \equiv 1 \mod{3}$ 
		and $3 \mid s_{11}, s_{11} \equiv 1 \mod{11}$ is only solved by $s_3 = s_{11} = 1$.  Thus they must be normal.  Since they are of prime order 
		then they are isomorphic to $\Z_3$ and $\Z_{11}$ respectively.
		Thus by the chinese remainder theorem $\Z_3 \times \Z_{11} \simeq \Z_{33}$.  Thus all groups of order $33$ are isomorphic to $\Z_{33}$.
		\item Note that by the third sylow theorem we know that the 3-sylow 
		subgroup is normal and there are either 1, 3, or 9 2-sylow subgroups (
		represented by the number $s_2$). 
		Note that there are two subgroups of order 9, 
		$\Z_9$ and $\Z_3 \times \Z_3$.  Therefore we must work by cases
		\begin{itemize}
			\item Suppose $s_2 = 1$.  If we have $\Z_9$ then by the normality of 
			the sylow 2-subgroup we have that $\Z_2 \times \Z_9$.  If we have 
			$\Z_3 \times Z_3$ then we have the group $\Z_3 \times \Z_3 \times \Z_2$.  
			\item Suppose that $s_2 = 3$.  If our sylow subgroup is $\Z_9$
			then we have a problem.  We know that if we have $x \in \Z_9$ and 
			our element of order 2, $y$ that $x^iy = yx^{-i}$.  This ensures that 
			$\{y,x^3y,x^6y\}$ are the only elements which have order 2.  
			Therefore $xy$ should have order 6, since it's not in $\Z_9$, and 
			doesn't have order 2 since it's not in the set of elements which have
			order 2, and can't be 18 since our group doesn't have a single 
			generator.  Thus $(xy)^2$ has order 3.  However the only elements 
			of order 3 are $x^3, x^6$.  Thus $(xy)^3 = x^4y$ or $(xy)^3 = x^7y$.
			However $(xy)^3$ must have order 2, which contradicts the fact that
			it multiplies to something which is not in the set of three elements 
			of order 2.  Thus we must consider our 3-sylow subgroup to be 
			$\Z_3 \times \Z_3$.  
			Note that since we have found all other possible matrices which
			satisfy $B^2 = I$, the only other possible matrix with a diagonal
			containing 1 and 2.   Note that if we consider the second entry 
			in $\Z_3 \times \Z_3$ and multiplication by the matrix, we have a 
			group of order 6 which is not commutative, yielding us $S_3$, and 
			then the first coordinate is unchanged and of order 3, thus our group
			is isomorphic to $S_3 \times \Z_3$. 
			\item Suppose that $s_2 = 9$.  If we have $\Z_9$ as our 3-sylow subgroup then for our element $y$ of order 2, we know that for $x \in \Z_9$
			we have that $yxy^{-1} = x^i$ since $\Z_9$ is normal.  Thus since 
			$y$ is of order 2 and $xy \not \in \Z_9$ then $1 = (xy)^2 = xyxy$
			implies $yxy = x^{-1}$.  Thus we have found $D_9$.  If we have 
			our sylow subgroup as $\Z_3 \times \Z_3$, then we must consider 
			the homomorphism from $\Z_2 \to Aut(\Z_3 \times \Z_3)$.  Since we 
			need to have 9 order 2 subgroups which satisfies the relation 
			$yxy = x^{-1}$.  Since every element is it's own inverse in $\Z_3 \times \Z_3$, and the trivial automorphism implies commutativity ($\Z_3 \times \Z_3 \times \Z_2)$ then we must have the matrix $2I$, where $I$ is the identity.  This
			corresponds to the general dihedral group for $\Z_3 \times \Z_3$ 
		\end{itemize}
		\item For all groups of order 20, the group of order 5 is normal via 
		the sylow theorem, or $\Z_5$, whose generator is $x$, and there are 2 groups of order 4, $\Z_4$ 
		and $V_4$.  We will consider where the elements of $\Z_5$ are taken 
		via conjugation by elements in our 2-sylow subgroup.
		\begin{itemize}
			\item Consider our group of order 4 to be $\Z_4$.  Let $y$ be a 
			generator of $\Z_4$
			\begin{itemize}
				\item If $yxy^{-1} = x$, then $\Z_4$ and $\Z_5$ commute, giving
				us $\Z_4 \times \Z_5$
				\item If $yxy^{-1} = x^3$, then $y^3xy^{-3} = x^2$, thus WLOG
				assume $yxy^{-1} = x^2$.  Since this homomorphism is well 
				defined, we have found another group of order 20
				\item Suppose $yxy^{-1} = x^4$.  Then $y^2 x y^{-2} = x$, thus 
				$x,y^2$ form a subgroup of order 10.  Note that since it has 
				index 2, then it is normal.  Let $z$ be a generator of the 
				subgroup.  Then $z^5 = y^2$.  Note that the only possible 
				way to conjugate $z$ with $y$ is $yzy^{-1} = x^{-1}$, since conjugating by $y$ twice would just be z, thus the power of $z$ which gets mapped
				from conjugation must square to 1, this is only satisfied by -1. 						Note 
				that this completely describes the multiplication table for our
				group.  
			\end{itemize}
			\item Consider our group of order 4 to be $V_4$ with generators $y,z$
			\begin{itemize}
				\item If both generators commute with x, then we have $\Z_2 \times \Z_10$
				\item If only (WLOG) $yxy^{-1} = x$, then $x,y$ generates a 
				subgroup of order 10, which is normal.  Let $x'$ be a generator
				of this subgroup.  Since $x,z$ don't commute, then $x',z$ don't 
				commute.  Thus $zx'z^{-1} = x'^{-1}$, for the same reason that 
				when our group of order 4 is $\Z_4$ for the automorphism $x \mapsto x^4$.  Thus we have exactly described $D_{10}$
				\item if neither $y$ nor $z$ commute with $x$ then set $y=yz$, 
				then repeat steps to get back $D_{10}$
			\end{itemize}
		\end{itemize}
		\item TODO order 30		 
	\end{enumerate}
\end{enumerate}
\end{document}
