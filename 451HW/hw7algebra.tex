\documentclass[12pt, letterpaper]{article}
\date{\today}
\usepackage[margin=1in]{geometry}
\usepackage{amsmath}
\usepackage{hyperref}
\usepackage{cancel}
\usepackage{amssymb}
\usepackage{fancyhdr}
\usepackage{pgfplots}
\usepackage{booktabs}
\usepackage{pifont}
\usepackage{amsthm,latexsym,amsfonts,graphicx,epsfig,comment}
\pgfplotsset{compat=1.16}
\usepackage{xcolor}
\usepackage{tikz}
\usetikzlibrary{shapes.geometric}
\usetikzlibrary{arrows.meta,arrows}
\newcommand{\Z}{\mathbb{Z}}
\newcommand{\N}{\mathbb{N}}
\newcommand{\R}{\mathbb{R}}
\newcommand{\Q}{\mathbb{Q}}
\newcommand{\C}{\mathbb{C}}

\newcommand{\Po}{\mathcal{P}}
\newcommand{\Pro}{\mathbb{P}}
\author{Alex Valentino}
\title{451 homework}
\pagestyle{fancy}
\renewcommand{\headrulewidth}{0pt}
\renewcommand{\footrulewidth}{0pt}
\fancyhf{}
\rhead{
	Homework 7\\
	451	
}
\lhead{
	Alex Valentino\\
}
\begin{document}
\begin{enumerate}
	\item[1.1]
	\item[1.8]
	\begin{enumerate}
		\item $\{1,5,7,11\}$
		\item $\{1,3,5,7\}$
		\item We claim that the set $\Phi(n) = \{k \in [n]: \gcd(n,k)=1\}$ is the set of units of $\Z/n\Z$.  Note that if $\gcd(a,n)=1$ then there exists 
		$x,y \in \Z$ such that $ax + ny = 1$.  Therefore $1 = ax + ny \equiv ax \mod{n}$.  Thus $x\mod n$ is the inverse of $a$.  However consider 
		for contradiction that $\Phi(n)$ does not contain all of the units.
		Thus there exists $u \in \Z/n\Z$ which $u \not \in \Phi(n)$ and there exists $w \in \Z/n\Z$ such that $uw \equiv 1 \mod{n}$.  Thus by the definition of modular arithmatic there exists $m \in \Z$ then $uw + my = 1$.
		Thus by definition of the $\gcd$, $\gcd(u,n) = 1$.  Thus $u \in \Phi(n)$.  This is a contradiction.  Thus $\Phi(n)$ contains all of the units of $\Z/n\Z$.   
	\end{enumerate}
	\item[2.2] Proving that $F[[x]]$ is a ring
	\begin{itemize}
		\item Addition is an abelian group.
		\begin{itemize}
			\item Commutativity: Suppose $a,b \in F[[x]]$ where 
			$a = \sum_{i=0}a_ix^i, b=\sum_{i=0} b_ix^i$.  Then 
			$$
			a+b = \sum_{i=0}a_ix^i + \sum_{i=0} b_ix^i = \sum_{i=0} (a_i + b_i ) x^i = \sum_{i=0} (b_i + a_i) x^i =  \sum_{i=0} b_ix^i + \sum_{i=0}a_ix^i = b + a
			$$
			\item Identity: Suppose $a \in F[[x]]$.  Then 
			$$
			0 + a = a + 0 = \sum_{i=0}a_ix^i + \sum_{i=0} 0 x^i = 
			\sum_{i=0} (a_i + 0)x^i = \sum_{i=0}a_ix^i = a.
			$$
			Thus $0$ is the additive identity for $F[[x]]$.
			\item Associativity: Suppose $a,b,c \in F[[x]]$.  Then
			\begin{align*}
			(a+b) + c &= \sum_{i=0}(a_i + b_i)x^i + \sum_{i=0}c_ix^i\\
			&= \sum_{i=0}(a_i + b_i + c_i)x^i\\ &= \sum_{i=0}a_ix^i + (b_i + c_i)x^i\\ &= \sum_{i=0}a_ix^i + \sum_{i=0}(b_i + c_i)x^i\\ &= a + (b+c)
			\end{align*}
			
			
			\item Additive inverses: Suppose $a \in F[[x]]$.  
			Then by definition $a = \sum_{i=0}a_ix^i$.  Since $F$ is a field
			then the sequence $(-a_0,-a_1,\cdots) \subseteq F$.  Therefore 
			we can construct $b = \sum_{i=0}-a_ix^i$.  
			Thus $a+b = \sum_{i=0} (a_i - a_i)x^i = \sum_{i=0} 0x^i = 0$.
			Thus $b$ is the inverse of $a$.  
		\end{itemize}
		\item Multiplication is commutative: Suppose $a,b \in F[[x]]$ Then 
		\begin{align*}
		(ab)_n  &= \sum_{i+j = n}a_ib_j\\
		&= \sum_{j+i = n}b_ja_i  \text{ commutativity of } F\\
		&= \sum_{l+k = n}b_la_k  \text{ let } l = j, k = i\\
		&= (ba)_n
		\end{align*}
		Since the $n$th coefficient is the same, then the power series is identical.  
		\item Multiplication is associative
		\item Distributive rule.  
	\end{itemize}
	The ideals of $F[[x]]$
	\item[3.2] Suppose $I \subset \Z[i]$ and consider $x \in I$.  
	By definition of $\Z[i]$ there exists $a,b \in \Z$ such that $x = a+ bi$ where at least one of the $a,b$ is non-zero.  Therefore the element $a-bi \in \Z[i]$ since $-b \in \Z$.  Thus
	by the definition of an ideal $(a-bi)(a+bi) \in I$.  Therefore
	$a^2 - b^2 \in I$.  Since $a,b \in \Z$ then $I$ contains an integer.	
	\item[3.6]
	\item[3.12]
	\item[4.1]
	\item[5.6]
	\item[6.1]
\end{enumerate}
\end{document}
