\documentclass[12pt, letterpaper]{article}
\date{\today}
\usepackage[margin=1in]{geometry}
\usepackage{amsmath}
\usepackage{hyperref}
\usepackage{cancel}
\usepackage{amssymb}
\usepackage{fancyhdr}
\usepackage{pgfplots}
\usepackage{booktabs}
\usepackage{pifont}
\usepackage{amsthm,latexsym,amsfonts,graphicx,epsfig,comment}
\pgfplotsset{compat=1.16}
\usepackage{xcolor}
\usepackage{tikz}
\usetikzlibrary{shapes.geometric}
\usetikzlibrary{arrows.meta,arrows}
\newcommand{\Z}{\mathbb{Z}}
\newcommand{\N}{\mathbb{N}}
\newcommand{\R}{\mathbb{R}}
\newcommand{\Q}{\mathbb{Q}}
\newcommand{\C}{\mathbb{C}}

\newcommand{\Po}{\mathcal{P}}
\newcommand{\Pro}{\mathbb{P}}
\author{Alex Valentino}
\title{451 homework}
\pagestyle{fancy}
\renewcommand{\headrulewidth}{0pt}
\renewcommand{\footrulewidth}{0pt}
\fancyhf{}
\rhead{
	Homework 7\\
	451	
}
\lhead{
	Alex Valentino\\
}
\begin{document}
\begin{enumerate}
	\item[1.1] 
	\begin{itemize}
		\item Note that the polynomial $f(x) = (x-7)^3 - 2$ satisfies 
		$f(7 + \sqrt[3]{2}) = 0$ since 
		$$
			f(7 + \sqrt[3]{2}) = (7 + \sqrt[3]{2} - 7)^3 - 2 = 2 - 2 = 0.
		$$
		\item Note that the polynomial $f(x) = (x^2 - 8)^2 - 60$ satisfies 
		$f(\sqrt{3}+\sqrt{5}) = 0$ since 
		$$
			f(\sqrt{3}+\sqrt{5}) = (5 + 3 + 2\sqrt{15} - 8)^2 - 60 = 60 - 60 = 0
		$$
	\end{itemize}
	\item[1.8]
	\begin{enumerate}
		\item $\{1,5,7,11\}$
		\item $\{1,3,5,7\}$
		\item We claim that the set $\Phi(n) = \{k \in [n]: \gcd(n,k)=1\}$ is the set of units of $\Z/n\Z$.  Note that if $\gcd(a,n)=1$ then there exists 
		$x,y \in \Z$ such that $ax + ny = 1$.  Therefore $1 = ax + ny \equiv ax \mod{n}$.  Thus $x\mod n$ is the inverse of $a$.  However consider 
		for contradiction that $\Phi(n)$ does not contain all of the units.
		Thus there exists $u \in \Z/n\Z$ which $u \not \in \Phi(n)$ and there exists $w \in \Z/n\Z$ such that $uw \equiv 1 \mod{n}$.  Thus by the definition of modular arithmatic there exists $m \in \Z$ then $uw + my = 1$.
		Thus by definition of the $\gcd$, $\gcd(u,n) = 1$.  Thus $u \in \Phi(n)$.  This is a contradiction.  Thus $\Phi(n)$ contains all of the units of $\Z/n\Z$.   
	\end{enumerate}
	\item[2.2] Proving that $F[[x]]$ is a ring
	\begin{itemize}
		\item Addition is an abelian group.
		\begin{itemize}
			\item Commutativity: Suppose $a,b \in F[[x]]$ where 
			$a = \sum_{i=0}a_ix^i, b=\sum_{i=0} b_ix^i$.  Then 
			$$
			a+b = \sum_{i=0}a_ix^i + \sum_{i=0} b_ix^i = \sum_{i=0} (a_i + b_i ) x^i = \sum_{i=0} (b_i + a_i) x^i =  \sum_{i=0} b_ix^i + \sum_{i=0}a_ix^i = b + a
			$$
			\item Identity: Suppose $a \in F[[x]]$.  Then 
			$$
			0 + a = a + 0 = \sum_{i=0}a_ix^i + \sum_{i=0} 0 x^i = 
			\sum_{i=0} (a_i + 0)x^i = \sum_{i=0}a_ix^i = a.
			$$
			Thus $0$ is the additive identity for $F[[x]]$.
			\item Associativity: Suppose $a,b,c \in F[[x]]$.  Then
			\begin{align*}
			(a+b) + c &= \sum_{i=0}(a_i + b_i)x^i + \sum_{i=0}c_ix^i\\
			&= \sum_{i=0}(a_i + b_i + c_i)x^i\\ &= \sum_{i=0}a_ix^i + (b_i + c_i)x^i\\ &= \sum_{i=0}a_ix^i + \sum_{i=0}(b_i + c_i)x^i\\ &= a + (b+c)
			\end{align*}
			
			
			\item Additive inverses: Suppose $a \in F[[x]]$.  
			Then by definition $a = \sum_{i=0}a_ix^i$.  Since $F$ is a field
			then the sequence $(-a_0,-a_1,\cdots) \subseteq F$.  Therefore 
			we can construct $b = \sum_{i=0}-a_ix^i$.  
			Thus $a+b = \sum_{i=0} (a_i - a_i)x^i = \sum_{i=0} 0x^i = 0$.
			Thus $b$ is the inverse of $a$.  
		\end{itemize}
		\item Multiplication is commutative: Suppose $a,b \in F[[x]]$ Then 
		\begin{align*}
		(ab)_n  &= \sum_{i+j = n}a_ib_j\\
		&= \sum_{j+i = n}b_ja_i  \text{ commutativity of } F\\
		&= \sum_{l+k = n}b_la_k  \text{ let } l = j, k = i\\
		&= (ba)_n
		\end{align*}
		Since the $n$th coefficient is the same, then the power series is identical.  
		\item Multiplication is associative: Suppose $a,b,c \in F[[x]]$. 
		Then 
		\begin{align*}
			(ab)c &= \left( \sum_{i=0}^\infty \sum_{k+j = i} a_k b_j x^i \right)\left( \sum_{l=0}^\infty c_l x^l \right)\\
			&= \sum_{i=0}^\infty \sum_{i = j+k+l} a_j b_k c_l x^i\\
			&= \left( \sum_{j=0}^\infty a_j x^j \right) \left(\sum_{n=0^\infty} \sum_{k+l = n}b_k c_l x^n \right)\\
			&= a(bc)
		\end{align*}
		\item Distributive rule: Suppose $a,b,c \in F[[x]]$.  Then
		\begin{align*}
			c(a+b) &= \left( \sum_{i=0}^\infty c_i x^i \right) \left( \sum_{i=0} a_i + \sum_{i=0} b_i \right)\\
			&= \sum_{i=0} \sum_{j+k = i} c_j (a_k + b_k)x^i \\
			&= \sum_{i=0} \sum_{j+k = i}c_j a_k x^i + \sum_{i=0} \sum_{j+k = i}c_j b_k x^i\\
			&= ca + cb
		\end{align*}
	\end{itemize}
	The units of $F[[x]]$ have to be pairs of power series of the form 
	$A = \sum_{i=0}^\infty a_i x^i$ and $B = \sum_{i=0}^\infty b_i x^i$ satisfying
	$AB = 1$.  Therefore $b_0$ must be $\frac{1}{a_0}$, giving us a requirement 
	that the original power series $A$ must have a non-zero constant term, and 
	that for $n > 0, \sum_{i+j = n} a_i b_j = 0$.  Note that we can use the 
	second condition to define $b_n$ recursively via $b_n = \frac{-1}{a_0}\sum_{i=0}^{n-1}a_{n-i} b_i$.  This gives us that the sum 
	$$
		\sum_{i+j = n} a_i b_j  = a_0 b_n + \sum_{i=0}^{n-1} a_{n-i}b_i = 
		-\sum_{i=0}^{n-1}a_{n-i} b_i + \sum_{i=0}^{n-1}a_{n-i} b_i = 0	
	$$
	Satisfying the requirements that all of the non-constant mononomial terms 
	have a coefficient of 0.  Therefore the only requirement on a given power series
	to be invertible is that there must be a non-zero constant term.  
	\item[3.2] Suppose $I \subset \Z[i]$ and consider $x \in I$.  
	By definition of $\Z[i]$ there exists $a,b \in \Z$ such that $x = a+ bi$ where at least one of the $a,b$ is non-zero.  Therefore the element $a-bi \in \Z[i]$ since $-b \in \Z$.  Thus
	by the definition of an ideal $(a-bi)(a+bi) \in I$.  Therefore
	$a^2 - b^2 \in I$.  Since $a,b \in \Z$ then $I$ contains an integer.	
	\item[3.6]  Let the automorphism be denoted $\psi : R[x,y] \to R[x,y]$, given 
	by $\psi(p(x,y)) = p(x+f(y),y)$
	\begin{itemize}
		\item Injectivity: Suppose $p,q \in R[x,y], \psi(p) = \psi(q)$.  We want to show 	that $p = q$.  Note that we can set $z = x + f(y)$.  From our original 
		supposition we have that $p(z,y) = q(z,y)$.  Therefore trivially $p(x,y) = q(x,y)$  
		\item Surjectivity: Suppose $p \in R[x,y]$.  We want to show there exists 
		$p' \in R[x,y]$ such that $\psi(p') = p$.  We claim that $p' = p(x - f(y), y)$.  Observe that $\psi(p') = p((x-f(y)) + f(y), y) = p(x,y)$. 
		\item Homomorphism requirements: Suppose $p,q,r \in R[x,y], g = pq, w = g + r$ then 
		$$\psi(pq + r) = \psi(w(x,y)) = w(x + f(y), y) = g(x + f(y),y) + r(y)$$ 
		$$= 
		p(x+f(y),y)q(x+f(y),y) + r(x+f(y),y)= \psi(p)\psi(q) + \psi(r)$$.\\
		Additionally, since $\psi$ is a substitution map on the variables x,y, 
		then it does not affect constants, thus $\psi(1) = 1$.  
	\end{itemize}
	\item[3.12] We must show that $I + J = {i+j: i \in I, j \in J}$ is an ideal in the ring $R$
	\begin{itemize}
		\item Suppose $a,b \in I + J $.  Then there exists $a_i, b_i \in I, a_j, b_j \in J$ such that $a = a_i + a_j, b= b_i + b_j$.   Therefore $a+b = a_i + a_j + b_i + b_j = (a_i + b_i) + (a_j + b_j)$.  Since $I,J$ are closed under addition then 
	$a_i + b_i \in I, a_j + b_j \in J$.  Since $a+b$ is the sum of an element 
	from $I$ and an element from $J$ then it's an element of $I + J$.
	\item Suppose $c \in R, a \in I + J$.  Then there exists $a_i \in I, a_j \in J$
	such that $a_i + a_j = a$.  Thus $ca = c(a_i + a_j) c a_i + c a_i$.  Since 
	ideals "absorb" multiplication then $ca_i \in I, ca_j \in J$.  Thus by the 
	definition of $I+J, ca \in I+J$.  
	\end{itemize}
	Therefore $I+J$ is an ideal of $R$.
	\item[4.1] Since the substitution homomorphism is surjective from $\Z[x]$ to 
	$\Z$ then we can apply the correspondence theorem.  This gives us a bijective
	correspondence between ideals in $\Z[x]$ containing $x-1$ and ideals in $\Z$.  
	Note that the ideals in $\Z$ are exactly $(n)$ where $n \in \Z$.  Since constants are unaffected by substitution then this says that all of the ideals 
	which contain $(x-1)$ are exactly the ideals of the form $(n,x-1)$.  
	\item[5.6] 
	\begin{enumerate}
		\item Let $\phi: R[x] \to R[\alpha]$ be the substitution homomorphism 
		$\phi(x) = \alpha$, $\pi: R[x] \to R[x] / (ax-1)$ be the projection
		map, and $\psi : R[x]/(ax-1) \to R[\alpha]$ be the isomorphism guaranteed by the first isomorphism theorem.  We know by the first isomorphism theorem that $R[x] / (ax-1) \cong R[\alpha]$.  Therefore we must show that all elements in $R[x] / (ax-1)$ are 
		equal to $cx^k$, where $c \in R$.  Given a polynomial in $R[x]$, 
		$p(x) = b_0 + b_1 x + \cdots + b_n x^n,$ one can remove the constant term
		$b_0$ by adding $b_0(ax-1)$.  Note that our new polynomial is equivalent in $R[x]/(ax-1)$.  Therefore we can continue the process via adding $(b_1 + b_0 a x)(ax-1)$, which will elimate the $x$.  This process can be continued til we end up 
		with a polynomial of the form $b x^n$ where $b \in R$.  Since $\psi$ is a bijection between $R[x]/(ax-1)$ and $R[\alpha]$ then we have 
		that all elements in $R[\alpha]$ are of the form $b \alpha^n$.  
		\item Note that if $b \in \ker \psi$ then there exists $p(x) \in R[x]$
		such that $b = (ax-1)p(x)$.  Since $p(x) \in R[x]$ then it takes the form
		$p(x) = c_0 + c_1 x + \cdots + c_n x^n$.  Therefore we have the equation
		$b = (ax-1)(c_0 + c_1 x + \cdots + c_n x^n)$, which must satisfy 
		$-c_0  =b, c_0 a = c_1, c_1 a = c_2,\cdots, c_{n-1} a = c_{n}, 0 = c_n a$.
		Thus going up the equations we end with $0 = a^n b$.  
		
		\iffalse Suppose $b \in R$ has the property that there exists $n \in \N$ 
		such that $a^n b = 0$, $g: R \to R[x]$ is the inclusion map from the ring 
		to the ring on the set of polynomials with coefficients in the ring.
		The map we care about is equivalent to $\psi \circ \pi \circ g$.  Let 
		$\gamma = \psi \circ \pi \circ g$  We know 
		that $g$ and $\psi$ have trivial kernels, so we must evaluate whether $b$ 
		is in the kernel of $\pi$.  Note that $b \equiv b + b(ax-1) \equiv abx$.
		Furthermore $abx \equiv abx + ab(ax-1) \equiv a^2 bx^2$.  We can continue this process until we hit $a^n bx^n$.  We know this is equal to 0 by assumption.  
		Thus the isomorphism will just map 0 to 0. Thus the set of elements $\{x \in R: \exists k \in \N, a^k x = 0\} \subseteq \ker \gamma$.  
		Suppose for all $n \in \N, a^n b \neq 0 $.  We want to show that 
		$\gamma(b) \neq 0$.  Note that for an arbitrary $m$, 
		\fi
		\item 
		\begin{itemize}
			\item $(\Rightarrow)$ Suppose $R'$ is the zero ring.  Then 
			trivially all elements are in the kernel.  Thus there exists an $n$
			for every element $b \in R$ where $a^n b = 0$.  This implies that all 
			elements are either zero divisors or $a^n = 0$.  If every element is 
			a zero divisor then $R$ is the zero ring, making $a$ trivially
			 nilpotent.  Otherwise $a^n = 0$, thus making $a$ nilpotent.  
			\item $(\Leftarrow)$ Suppose $a$ is nilpotent.  We want to show 
			that $R'$ is the zero ring.  
			Then by definition there exists $n \in \N$ such that $a^n = 0$.
			Thus all elements $b \in R$ satisfy $a^n b = 0$.  Thus all elements 
			from $R$ are in the kernel of $\gamma$.  Thus $R'$ is the zero ring.
		\end{itemize}		  
	\end{enumerate}
	\item[6.1] Let $\varphi : \R[x] \to \C \times \C$ be the homomorphism given by 
	$\varphi(x) = (1,i), \varphi(r) = (r,r), r \in \R$.  
	\begin{itemize}
		\item We claim that $im \varphi = \R \times \C$.  Observe that no matter 
		the polynomial, the first coordinate must be a real number because both 
		$\varphi(x), \varphi(r)$ have a real number in the first coordinate, and 
		we are taking linear combinations of $x$ and $r \in \R$.  Furthermore the
		second coordinate spans $\C$ since $ax+b \in \R[x]$ 
		$$\varphi(ax+b) = \varphi(a) \varphi(x) + \varphi(b) = (a,a)(1,i) + (b,b) = (a + b, a + bi)$$
		Clearly the second coordinate spans all of $\C$.  
		\item We claim that $\ker \varphi = (x^4 - 1)$.  Note that $\varphi(x^4) = \varphi(x)^4 = (1,i)^4 = (1^4,i^4) = (1,1)$, therefore $\varphi(x^4) = \varphi(1)$.
		Thus $\varphi(x^4 - 1) = 0$.  Note that if there exists a smaller polynomial which generates the kernel then it must divide $x^4 - 1$.  Note that 
		$\varphi(x^2+1) = (2,0)$ and $\varphi(x^2 - 1) = (0,-2)$, thus the divisors are non-zero.  Therefore $x^4 - 1$ is the smallest polynomial in the kernel.  Therefore $\ker \varphi = (x^4 - 1)$.  
	\end{itemize}
\end{enumerate}
\end{document}
