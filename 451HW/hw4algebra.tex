\documentclass[12pt, letterpaper]{article}
\date{\today}
\usepackage[margin=1in]{geometry}
\usepackage{amsmath}
\usepackage{hyperref}
\usepackage{cancel}
\usepackage{amssymb}
\usepackage{fancyhdr}
\usepackage{pgfplots}
\usepackage{booktabs}
\usepackage{pifont}
\usepackage{amsthm,latexsym,amsfonts,graphicx,epsfig,comment}
\pgfplotsset{compat=1.16}
\usepackage{xcolor}
\usepackage{tikz}
\usetikzlibrary{shapes.geometric}
\usetikzlibrary{arrows.meta,arrows}
\newcommand{\Z}{\mathbb{Z}}
\newcommand{\N}{\mathbb{N}}
\newcommand{\R}{\mathbb{R}}
\newcommand{\Q}{\mathbb{Q}}
\newcommand{\C}{\mathbb{C}}

\newcommand{\Po}{\mathcal{P}}
\newcommand{\Pro}{\mathbb{P}}
\author{Alex Valentino}
\title{451 homework}
\pagestyle{fancy}
\renewcommand{\headrulewidth}{0pt}
\renewcommand{\footrulewidth}{0pt}
\fancyhf{}
\rhead{
	Homework 4\\
	451	
}
\lhead{
	Alex Valentino\\
}
\begin{document}
\begin{enumerate}
	\item[2.6]\textit{Let $G$ be a group. Define an opposite group $G^o$ with law of composition $a * b$ as follows:
The underlying set is the same as $G$, but the law of composition is $a * b = ba$. Prove that $G^o$ is a group.}\\
	\begin{itemize}
		\item Associativity: For all $a,b,c \in G^o$
		$$
		a * (b*c) = a*(cb) = cba = c(ba) = (ba) * c = (a*b)*c
		$$
		\item Identity: For all $a \in G^o$ 
		$$
			a*e = ea = a = ae = e * a 
		$$
		\item Inverses:  For all $a \in G^o$
		$$a * a^{-1} = a^{1}a = e = a a^{-1} = a^{-1} * a$$
	\end{itemize}
	\item[4.9]\textit{How many elements of order 2 does the symmetric group $S_4$ contain?}\\
	There are 6 elements of order 2 in $S_4$, they are all of the 
	transpositions, which is equivalent to $\binom{4}{2} = 6$.
	Note that for $a, b \in [4]$ if $a\neq b$, then $(ab)(ab)$, when evaluated on $a$ yields $(ab)(ab)a = (ab)b = a$ and on $b$ becomes
	$(ab)(ab)b = (ab)a = b$.  Therefore $(ab)(ab) = 1$.  
	\item[6.1]\textit{Let $G'$ be the group of real matrices of the form
	$\begin{bmatrix}1 & x\\ 0 & 1\end{bmatrix}$.  Is the map $\R^+ \to G'$ that sends $x$ to this matrix an isomorphism?	}\\
	Let $f : \R^+ \to G'$ be given by $x \mapsto \begin{bmatrix}1 & x\\ 0 & 1\end{bmatrix}$.
	\begin{itemize}
		\item $f$ is a homomorphism:\\
		$f(x)f(y) = \begin{bmatrix}1 & x\\ 0 & 1\end{bmatrix} \begin{bmatrix}1 & y\\ 0 & 1\end{bmatrix} = \begin{bmatrix}1 & x+y\\ 0 & 1\end{bmatrix} = f(x+y)$
		\item $f$ is injective:\\
		Suppose $x,y \in \R^+, f(x) = f(y)$. Then
		$$
		 \begin{bmatrix}1 & x\\ 0 & 1\end{bmatrix} =  \begin{bmatrix}1 & y\\ 0 & 1\end{bmatrix}
		$$
		$$
			x = y.		
		$$
		\item $f$ is surjective:\\
		Suppose $A \in G'$.  Then $A =  \begin{bmatrix}1 & a\\ 0 & 1\end{bmatrix}$, where $a \in \R$.  Therefore $f(a) =  \begin{bmatrix}1 & a\\ 0 & 1\end{bmatrix}$.  
\end{itemize}	 
	\item[6.4] \textit{Prove that in a group, the products ab and ba are conjugate elements.}\\
	Note that $(ba^{-1})ab(b^{-1}a) = b(a^{-1}a)(b b^{-1})a = ba$	
	Therefore $ab$ and $ba$ are conjugates of each other by the element 
	$ba^{-1}$.  
	\item[7.1] \textit{Let G be a group. Prove that the relation $a \sim b$ if $b = gag^{-1}$ for some $g$ in $G$ is an
equivalence relation on $G$}
	\begin{itemize}
		\item Reflexivity:   Suppose $a \in G$. Note that $e a e^{-1} = a$.
		Therefore $a \sim a$.
		\item Symmetry: Suppose $a \sim b$.  Then there exists $g \in G$ such that $g a g^{-1} = b$.  Then $g^{-1} b g = g^{-1} g a g^{-1} g = a$
		\item Transitivity:  Suppose $a,b,c \in G,a \sim b, b \sim c$.  
		Then there exists $g,h \in G$ such that $g a g^{-1} = b, h b h^{-1} = c$.  \\
		Then $hgag^{-1}h^{-1} = hbh^{-1} = c$.  Thus $a~c$ by the element $gh$.   
	\end{itemize}
	\item[8.1]\textit{Let $H$ be the cyclic subgroup of the alternating group $A_4$ generated by the permutation
$(123)$. Exhibit the left and the right cosets of $H$ explicitly.}\\
		\begin{itemize}
			\item Right cosets:
			\begin{itemize}
				\item $H$ acting upon the identity operation $(123)$ or $(132)$: $\{e, (123),(132)\}$
				\item $H$ acting upon $(12)(34), (234)$ or $(134)$: $\{(12)(34), (234),(134)\}$
				\item $H$ acting upon $(13)(24), (124)$ or $(243)$: $\{(13)(24), (124),(243)\}$
				\item $H$ acting upon $(14)(23), (142)$ or $(143)$: $\{(14)(23), (142),(143)\}$
			\end{itemize}
			\item Left cosets:
			\begin{itemize}
				\item The identity operation, $(123)$, or $(132)$ acting upon $H$: $\{e, (123),(132)\}$
				\item $(12)(34),(243)$ or $(143)$ acting upon $H$: $\{(12)(34),(243), (143)\}$
				\item $(13)(24), (142)$ or $(234)$ acting upon $H$: $\{(13)(24), (142), (234)\}$
				\item $(14)(23), (124)$ or $(134)$ acting upon $H$: $\{(14)(23), (124), (134)\}$
			\end{itemize}
		\end{itemize}
	\item[8.12]\textit{Let S be a subset of a group G that contains the identity element 1, and such that the left
cosets aS, with a in G, partition G. Prove that S is a subgroup of G}\\
		\begin{itemize}
			\item Identity:  We assume this.  
			\item Closure:  Suppose $a,b \in S$.  We want to show that
			$ab \in S$.  Note that $ab \in aS$.  Since the 
			cosets of $S$ partition $G$, then $a \in aS$ implies that 
			$aS = S$.   Therefore $ab \in S$.  
			\item Inverses: Suppose $a \in S$.  We want to show that 
			$a^{-1} \in S$.  Note that $a^{-1}S$ contains the identity.  
			Since the cosets of $S$ partition $G$, and $e \in S, a^{-1}S$ then that implies $S = a^{-1}S$.  Therefore $a^{-1} \in S$.  
		\end{itemize}
	\item[9.7]\textit{Determine the order of each of the matrices $A =\begin{bmatrix}1 & 1\\ 0 & 1\end{bmatrix},  B = \begin{bmatrix}1 & 1\\ 1 & 0\end{bmatrix}$ }\\
	\begin{itemize}
		\item $A$ has order 3\\
		$$
		A^3 = \begin{bmatrix}1 & 1\\ 0 & 1\end{bmatrix}^3 = \begin{bmatrix}1 & 2\\ 0 & 1\end{bmatrix} \begin{bmatrix}1 & 1\\ 0 & 1\end{bmatrix} = \begin{bmatrix}1 & 0\\ 0 & 1\end{bmatrix}
		$$
		\item $B$ has order 8\\
		\begin{align*}
		B^8 &= \begin{bmatrix}1 & 1\\ 1 & 0\end{bmatrix} \begin{bmatrix}1 & 1\\ 1 & 0\end{bmatrix}^7\\
		&= \begin{bmatrix}2 & 1\\ 1 & 1\end{bmatrix} \begin{bmatrix}1 & 1\\ 1 & 0\end{bmatrix}^6\\
		&= \begin{bmatrix}0 & 2\\ 2 & 1\end{bmatrix} \begin{bmatrix}1 & 1\\ 1 & 0\end{bmatrix}^5\\
		&= \begin{bmatrix}2 & 0\\ 0 & 2\end{bmatrix} \begin{bmatrix}1 & 1\\ 1 & 0\end{bmatrix}^4\\
		&= \begin{bmatrix}2 & 2\\ 2 & 0\end{bmatrix} \begin{bmatrix}1 & 1\\ 1 & 0\end{bmatrix}^3\\
		&= \begin{bmatrix}1 & 2\\ 2 & 2\end{bmatrix} \begin{bmatrix}1 & 1\\ 1 & 0\end{bmatrix}^2\\
		&= \begin{bmatrix}0 & 1\\ 1 & 2\end{bmatrix} \begin{bmatrix}1 & 1\\ 1 & 0\end{bmatrix}\\
		&= \begin{bmatrix}1 & 0 \\ 0 & 1\end{bmatrix}
		\end{align*}
	\end{itemize}
\end{enumerate}
\end{document}
