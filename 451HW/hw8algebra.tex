\documentclass[12pt, letterpaper]{article}
\date{\today}
\usepackage[margin=1in]{geometry}
\usepackage{amsmath}
\usepackage{hyperref}
\usepackage{cancel}
\usepackage{amssymb}
\usepackage{fancyhdr}
\usepackage{pgfplots}
\usepackage{booktabs}
\usepackage{pifont}
\usepackage{amsthm,latexsym,amsfonts,graphicx,epsfig,comment}
\pgfplotsset{compat=1.16}
\usepackage{xcolor}
\usepackage{tikz}
\usetikzlibrary{shapes.geometric}
\usetikzlibrary{arrows.meta,arrows}
\newcommand{\Z}{\mathbb{Z}}
\newcommand{\N}{\mathbb{N}}
\newcommand{\R}{\mathbb{R}}
\newcommand{\Q}{\mathbb{Q}}
\newcommand{\C}{\mathbb{C}}

\newcommand{\Po}{\mathcal{P}}
\newcommand{\Pro}{\mathbb{P}}
\author{Alex Valentino}
\title{451 homework}
\pagestyle{fancy}
\renewcommand{\headrulewidth}{0pt}
\renewcommand{\footrulewidth}{0pt}
\fancyhf{}
\rhead{
	Homework 8\\
	451	
}
\lhead{
	Alex Valentino\\
}
\begin{document}
\begin{enumerate}
	\item[6.7]
	\begin{itemize}
		\item We want to show that $(2) \cap (x) = (2x)$.  Suppose $a \in (2) \cap (x)$.  Then there exists $b,c \in \Z[x]$ such that $bx = a, 2c = a$.  From the second equivalence of a we know that $2 \mid a$.  Therefore $2 \mid cx$.  By 
		Euclid's lemma we have that $2 \mid c$ or $2 \mid x$.  Since $2 \nmid x$
		then $2 \mid c$.  Thus there exists $d \in \Z[x]$ such that $a = 2xd$.  
		Thus $a \in (2x)$.  The other direction is trivial.  Thus $(2) \cap (x) = (2x)$.  
		\item Consider the map $\phi : \Z[x] \to F_2 [x] \times \Z$ given by 
		$\varphi(f) = (f \mod{2}, f(0))$.  Note that $\ker \varphi$ is all of the 
		polynomials which are both divisible by 2 and have x as a factor.  This 
		is exactly $(2) \cap (x) = (2x)$, the ideal mentioned above.  Since the 
		projection from $\Z[x] \to \Z[x]/(2x)$ is surjective, then by the first 
		isomorphism theorem for rings we have that $\Z[x]/(2x) \cong im \varphi$.
		Additionally $im \varphi$ satisfies the requirements put forward in the 
		question since trivially if $f(0) = n$ then $\overline{f}(0) \equiv n \mod{2}$, since $n \equiv \overline{n} = \overline{f}(n)$.  
	\end{itemize}
	\item[7.1] Let $D$ be a finite integral domain.  Suppose for contradiction 
	that $D$ is not a field.  Then there exists $a \in D$ where $a$ is not a unit,
	$a \neq 0$.
	Then $aD \subset D$, otherwise there would exists $d \in D$ where $ad = 1$, 
	contradicting the fact that $a$ is not a unit.  Since $D$ is finite and $aD$
	is not injective then there exists $b,c \in D$ where $b \neq c, ab = ac$.
	Therefore $a(b-c) = 0$.  Since $D$ is a domain and  $a \neq 0$ then $b-c =0$.
	This contradicts the fact that $b \neq c$, therefore $D$ is a field.
	\item[7.2] Let $R$ be a domain.  We will show that $R[x]$ is a domain.  
	Suppose for contradiction that $R[x]$ is not a domain.  Then there exists
	$p,q \in R[x], p(x)q(x) = 0$.  Since $p,q\in R[x]$ then $p(x) = \sum_{i=0}^n p_i x^i, q(x) = \sum_{i=0}^m q_i x^i$.  Therefore $p_n q_m x^{n+m} = 0$.  This implies that $p_n q_m = 0$.  Since $p,q$ are of degree $n,m$ respectively then 
	their leading coefficients must be non-zero.  This contradicts the fact that 
	$R$ is a domain.  
	\item[8.1] We will work by cases:
	\begin{itemize}
		\item $(1) = \Z[x]$, and $n \neq 1, (n) \subset (n,x) \subset \Z[x]$, 
		since $(n,x)$ does no contain 1, and entirely contains $(n)$.
		\item Consider $p(x) \in \Z[x]$.  There must exists $r \in \Z$ where 
		$p(r) \neq -1,0,1$.  Then for any $q(x) \in (p(x),p(r))$, then $p(r)/q(x)$.
		Thus $1 \not \in (p(x),p(r))$.
	\end{itemize}
	Thus no principle ideal of $\Z[x]$ is maximal.  (credit to brian bi)
	
	
	\item[8.4] We know that the maximal ideals of $\R[x]$ are the irreducible 
	elements of $\R[x]$.  Additionally we have the fact that the only irreducible 
	polynomials in $\R[x]$ are of degree 1 or 2.  Therefore we have trivially 
	that there is a bijective correspondence between the polynomials of degree one
	and the real line.  We must look to the quadratic polynomials.   The degree 2
	polynomials which are irreducible over $\R$ factor over $\C$.  Since they 
	factor over $\C,$ then they have a complex root.  Consider $p(x) \in \R[x], 
	deg(p) = 2, \alpha \in \C, p(\alpha) = 0$.  Since $p \in \R[x]$, then it is 
	invariant under complex conjugation, giving us that $\bar{p(x)} = p(\bar{x})$.  Thus 
	$\bar{\alpha}$ is the second root of $p(x)$, and $p(x) = (x-\alpha)(x-\bar{\alpha})$.  Therefore we can uniquely associate each degree two polynomial to 
	a point in the upper half plane, since it is entirely defined by that single root and it's conjugate.  
	\item[9.1] Consider the ideal $I = (y^2 + x^3 - 17)$ and the quotient ring 
	$R = \C[x,y]/I$.  
	\begin{itemize}
		\item The ideal $(x-1,y-4)$ is maximal since by the theorem 11.9.1 in the textbook, $(x-1,y-4)$ corresponds with the point $(1,4)$, which satisfies $1 + 16 - 17 = 0$, thus the ideal is maximal in $R$
		\item $(x+1,y+4)$ is not maximal since $-1 + 16 - 17 \neq 0$,  thus 
		the ideal does not correspond to a point on the curve $y^2 = 17 - x^3$
		\item The ideal $(x^3 - 17,y^2)$ is not maximal since the zeros of $x^3 - 17$ in $\C$ are $\{\sqrt[3]{17}\omega^n: n \in \N \}$ where $\omega$ is the third 
		root of unity.  Thus it is not maximal since it has a correspondence with 
		3 points instead of 1.  
	\end{itemize}
	\item[9.5] Suppose $V$ is the variety for a set of polynomials $\{f_1,\cdots,f_r\}$, and that $I = (f_1,\cdots,f_r)$.  Suppose $x \in V$.  Then 
	for every $p(x) \in I$, it is a linear combination of polynomials which are all 
	zero at $x$.  Thus $p(x) = 0$.  Suppose $x \in \C^n$ has the property that
	for all $p \in I, p(x) = 0$.  Then it is true for the generators that $f_i(x) = 0$ for all $i \in [r]$.  Thus $x \in V$.  Therefore $V$ and $I$ depend on just each 
	other.   
\end{enumerate}
\end{document}
