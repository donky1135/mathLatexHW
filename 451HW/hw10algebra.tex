\documentclass[12pt, letterpaper]{article}
\date{\today}
\usepackage[margin=1in]{geometry}
\usepackage{amsmath}
\usepackage{hyperref}
\usepackage{cancel}
\usepackage{amssymb}
\usepackage{fancyhdr}
\usepackage{pgfplots}
\usepackage{booktabs}
\usepackage{pifont}
\usepackage{amsthm,latexsym,amsfonts,graphicx,epsfig,comment}
\pgfplotsset{compat=1.16}
\usepackage{xcolor}
\usepackage{tikz}
\usetikzlibrary{shapes.geometric}
\usetikzlibrary{arrows.meta,arrows}
\newcommand{\Z}{\mathbb{Z}}
\newcommand{\N}{\mathbb{N}}
\newcommand{\R}{\mathbb{R}}
\newcommand{\Q}{\mathbb{Q}}
\newcommand{\C}{\mathbb{C}}

\newcommand{\Po}{\mathcal{P}}
\newcommand{\Pro}{\mathbb{P}}
\author{Alex Valentino}
\title{451 homework}
\pagestyle{fancy}
\renewcommand{\headrulewidth}{0pt}
\renewcommand{\footrulewidth}{0pt}
\fancyhf{}
\rhead{
	Homework 10\\
	451	
}
\lhead{
	Alex Valentino\\
}
\begin{document}
\begin{enumerate}
	\item[1.5] Lemma: if $\gcd(a,b) = 1$ then $\gcd(ab,a+b) = 1$.  Note that if 
	any number divides $ab$ then it divides either $a$ or $b$, otherwise contradicting that $\gcd(a,b) = 1$ if it is larger.  Suppose $d \mid a, d \nmid b$.
	Then since $d \nmid b$ then $d \nmid a + b$.  The alternate case is identical.
	Therefore $\gcd(ab,a+b)$.\\
	Since $\gcd(ab,a+b) = 1$, then by applying Euler's theorem there exists $n \in \N$ 	such that $(a+b)^n \equiv 1 \mod{ab}$($<a+b>$ forms a subgroup of $\Z/ab$, 
	therefore it will cycles back to 1).  Thus noting that $(a+b)^n = \sum_{i=0}^n \binom{n}{i}a^i b^{n-i} \equiv a^n + b^n \mod{ab}$ since all cross terms are of the 
	form $ka^r b^q$ where both $r,q \geq 1$.  Thus all of the cross terms 
	\item[2.2] $\gcd(x^5 + 2x^3+x^2 + x + 1, x^6 + x^4 + x^3 + x^2 + x + 1) = x^2 + 1$  
	\item[2.10]  Note that the set of all units is all power series with non-zero 
	constant term.  Consider $f(x) = \sum_{i=0}^\infty a_i x^i$ with first non-zero 
	term being $a_n$.   Therefore since $\sum_{i=0}^\infty a_{i+n} x^i$ is a unit 
	then we have that $f(x) = x^n\sum_{i=0}^\infty a_{i+n} x^i$.  Since both the 
	unit and the power of $x$ is uniquely determined by the sequence, then the 
	factorization is unique.  Additionally since this takes exactly one step then 
	factoring terminates.  Thus the ring of power series is a UFD.  
	\item[3.1]
	\begin{enumerate}
		\item Let $\phi: \Z[x] \to \R$ be the homomorphism given by $\phi(x) = 1 + \sqrt{2}$.  Therefore the kernel contains 
		\begin{align*}
			x &= 1 + \sqrt{2}\\
			x-1 &= \sqrt{2}\\
			(x-1)^2 &= 2\\
			(x-1)^2 - 2 &= 0\\
			x^2 -2x + 1 - 2 &= 0\\
			x^2 -2x -1 &= 0\\
		\end{align*}
		Since this polynomial found has an irrational root by construction, and 
		since it has another root which also must be irrational then our
		polynomial is irreducible.  Therefore our kernel is principle.  
		\item Let $\phi: \Z[x] \to \R$ be the homomorphism given by $\phi(x) = \frac{1}{2} + \sqrt{2}$.  Therefore the kernel contains 
		\begin{align*}
			x &= \frac{1}{2} + \sqrt{2}\\
			2x-1 &= 2\sqrt{2}\\
			(2x-1)^2 &= 8\\
			(2x-1)^2 - 8 &= 0\\
			4x^2 -4x -7 &= 0\\
		\end{align*}
		Since this polynomial found has an irrational root by construction, and 
		since it has another root which also must be irrational then our
		polynomial is irreducible.  Therefore our kernel is principle.  
	\end{enumerate}
	\item[3.2] Let $f, g \in \Z[x]$
	\begin{itemize}
		\item $\Rightarrow$ Suppose the $\gcd_{\Q[x]}(f(x), g(x)) = 1$.  Then
		there exists $a(x), b(x) \in \Q[x]$ such that $a(x)f(x) + b(x) g(x) = 1$.
		Since $a(x), b(x) \in \Q[x]$ then they take the form $a(x) = \sum_{i = 0}^n \frac{a_i}{a'_i}x^i, b(x) = \sum_{i = 0}^m \frac{b_i}{b'_i}x^i$.  If we multiply 
		$a(x)f(x) + b(x) g(x) = 1$ by $a'_1\cdots a'_n b'_1 \cdots b'_m$ then we 
		guarantee $a(x), b(x)$ to now be integer polynomials.  Thus we have found 
		a combination in $\Z[x]$ of $f,g$ which is an integer.  Thus $a'_1\cdots a'_n b'_1 \cdots b'_m \in (f(x),g(x))$.  
		combination  
		\item $\Leftarrow$ Suppose that there are $a(x), b(x) \in \Z[x]$ such that 
		$a(x)f(x) + b(x)g(x) = n, n \in \Z$.  Note that $a(x)/n, b(x)/n \in \Q[x]$
		therefore we have found polynomials satisfying the requirements for $f,g$
		to be coprime over $\Q[x]$  
	\end{itemize}
	\item[4.1]
	\begin{enumerate}
		\item 		
		\begin{itemize}
			\item $x^9 - 1 = (x-1)^9$ over $\mathbb{F}_3[x]$
			\item $x^9 - x = x(x+1)(x+2)(x^2+1)(x^2 + x + 2)(x^2 + 2x + 2)$ over $\mathbb{F}_3[x]$
		\end{itemize}
		\item $x^{16} - x = x(x+1)(x^2+x+1)(x^4+x+1)(x^4 + x^3 + 1)(x^4 + x^3 + x^2 + x + 1)$ over $\mathbb{F}_2[x]$
	\end{enumerate}
\end{enumerate}
\end{document}
