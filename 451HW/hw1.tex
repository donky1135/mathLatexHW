\documentclass[12pt, letterpaper]{article}
\date{\today}
\usepackage[margin=1in]{geometry}
\usepackage{amsmath}
\usepackage{hyperref}
\usepackage{cancel}
\usepackage{amssymb}
\usepackage{fancyhdr}
\usepackage{pgfplots}
\usepackage{booktabs}
\usepackage{pifont}
\usepackage{amsthm,latexsym,amsfonts,graphicx,epsfig,comment}
\pgfplotsset{compat=1.16}
\usepackage{xcolor}
\usepackage{tikz}
\usetikzlibrary{shapes.geometric}
\usetikzlibrary{arrows.meta,arrows}
\newcommand{\Z}{\mathbb{Z}}
\newcommand{\N}{\mathbb{N}}
\newcommand{\R}{\mathbb{R}}
\newcommand{\C}{\mathbb{C}}
\newcommand{\Q}{\mathbb{Q}}
\newcommand{\Po}{\mathcal{P}}
\newcommand{\Pro}{\mathbb{P}}

\newcommand{\grstep}[2][\relax]{%
   \ensuremath{\mathrel{
       {\mathop{\longrightarrow}\limits^{#2\mathstrut}_{
                                     \begin{subarray}{l} #1 \end{subarray}}}}}}
\newcommand{\swap}{\leftrightarrow}


\author{Alex Valentino}
\title{451 homework}
\pagestyle{fancy}
\renewcommand{\headrulewidth}{0pt}
\renewcommand{\footrulewidth}{0pt}
\fancyhf{}
\rhead{
	Homework 1\\
	451	
}
\lhead{
	Alex Valentino\\
}
\begin{document}
\begin{enumerate}
	\item[2.1] \textit{Prove that the numbers of the form $a + \sqrt{2}b$, where $a$ and $b$ are rational numbers, form a 
subfield of $\C$.}\\
	Let our alleged subfield be denoted $\Q[\sqrt{2}]$
	\begin{enumerate}
		\item \textit{If $a$ and $b$ are in $\Q[\sqrt{2}]$, then a + b is in $\Q[\sqrt{2}]$.}\\
		Suppose $a+\sqrt{2}b, c + \sqrt{2}d \in \Q[\sqrt{2}]$.  Therefore
		$$
		a+\sqrt{2}b + c + \sqrt{2}d = (a+c) + \sqrt{2}(b+d) \in  \Q[\sqrt{2}].
		$$
		
		\item \textit{If $a$ is in $\Q[\sqrt{2}]$, then $-a$ is in $\Q[\sqrt{2}]$}\\
		Suppose $a + \sqrt{2}b \in \Q[\sqrt{2}]$.  Since $a,b \in \Q$ then 
		$-a,-b \in \Q$.  Therefore $-a - \sqrt{2}b = -(a+\sqrt{2}b) \in \Q[\sqrt{2}]$.
		\item \textit{If $a$ and $b$ are in  $\Q[\sqrt{2}]$, then $ab$ is in $\Q[\sqrt{2}]$.}\\
		Suppose $a+\sqrt{2}b, c + \sqrt{2}d \in \Q[\sqrt{2}]$.  Therefore
		\begin{align*}
			(a+\sqrt{2}b)(c + \sqrt{2}d) &= ac + \sqrt{2}ad + \sqrt{2}bc + 2bd\\
			&= (ac + 2bd) + \sqrt{2}(ad + bc) \in \Q[\sqrt{2}].
		\end{align*}
		\item \textit{If $a$ is in $\Q[\sqrt{2}]$ and $a\neq0$, then $a^{-1}$ is in $\Q[\sqrt{2}]$.}\\
		Suppose $a + \sqrt{2}b \in \Q[\sqrt{2}], a+ \sqrt{2}b  \neq 0$.  
		Note that since $a+ \sqrt{2}b  \neq 0$ then $a-\sqrt{2}b \neq 0$, because if $a-\sqrt{2}b = 0$ then $a+\sqrt{2}b = 2\sqrt{2}b$, contradicting $a$ being an arbitrary rational number.  Since $a+\sqrt{2}b, a-\sqrt{2}b$ are non-zero then their product, $a^2 + 2b^2$, is non-zero.  Therefore, 
		$$
		(a+\sqrt{2}b)\frac{a-\sqrt{2}b}{a^2+2b^2} = \frac{a^2+2b^2}{a^2+2b^2} = 1,
		$$
		thus $\frac{a-\sqrt{2}b}{a^2+2b^2}$ is an inverse for $a+\sqrt{2}b$.	
		\item \textit{1 is in $\Q[\sqrt{2}]$.}\\
		Since $1,0 \in \Q$, then $1 = 1 + \sqrt{2}*0 \in \Q[\sqrt{2}]$.
		Therefore $1 \in \Q[\sqrt{2}]$.
	\end{enumerate}
	\iffalse	
	\item[2.2]\textit{Find the inverse of 5 modulo $p$, for $p = 7, 11, 13, \text{and } 17$.}
	\begin{enumerate}
		\item[$p=7$] $3*5 = 15 \equiv 1 \mod{7}$
		\item[$p=11$] $ 9*5 = 45 \equiv 1 \mod{11}$
		\item[$p=13$] $8*5 = 40 \equiv 1 \mod{13}$
		\item[$p=17$] $7*5 = 35 \equiv 1 \mod{17}$
	\end{enumerate}
	\fi
	\item[2.3]\textit{Compute the product of the polynomial  $(x^3 + 3x^2 + 3x + 1) (x^4 + 4x^3 + 6x^2 + 4x + 1)$ when the coefficents are elements in $\mathbb{F}_7$.  Explain your answer. }\\
	\begin{align*}
	(x^3 + 3x^2 + 3x + 1) (x^4 + 4x^3 + 6x^2 + 4x + 1) &= 
	x^3(x^4 + 4x^3 + 6x^2 + 4x + 1)\\ &+ 3x^2(x^4 + 4x^3 + 6x^2 + 4x + 1)\\
	&+ 3x(x^4 + 4x^3 + 6x^2 + 4x + 1)\\ &+ (x^4 + 4x^3 + 6x^2 + 4x + 1)\\
	&= (x^7 + 4x^6 + 6x^5 + 4x^4 + x^3)\\ &+ (3x^6 + 12x^5 + 18x^4 + 12x^3 + 3x^2)\\
	&+ (3x^5 + 12x^4 + 18x^3 + 12x^2 + 3x)\\ &+ (x^4 + 4x^3 + 6x^2 + 4x + 1)\\
	&= x^7 +(3 + 4)x^6\\ &+ (6 + 12 + 3)x^5 + (4+ 18 + 12 + 1)x^4\\
	&+ (1+12 + 18 + 4)x^3 + (3+12+6)x^2 + (3+4)x + 1\\
	&= x^7 + (7*1 + 0) + (7*3 + 0)x^5 + (7*5 + 0)x^4\\ &+ (7*5 + 0)x^3 + (7*3 + 0)x^2 +(7*1 + 0)x + 1\\
	&= x^7 + 1
	\end{align*}
	Explanation: If you look at the right hand side, the first two equal signs is 
	just multiplying out the polynomial then regrouping the terms.
	After the third equals sign I employ the fact that $\mathbb{F}_7$ has 
	characteristic 7 and further reduce the coefficients of the polynomial.  
	\item[2.4]\textit{Consider the system of linear equations 
	$\begin{bmatrix}6 & -3\\ 2 & 6 \end{bmatrix} \begin{bmatrix}x_1 \\x_2 \end{bmatrix} = \begin{bmatrix} 3 \\ 1\end{bmatrix}	 $}\\
	Note: the inverse of $\begin{bmatrix}6 & -3\\ 2 & 6 \end{bmatrix}$ in $\R$ is 
	$\frac{1}{42}\begin{bmatrix}6 & 3\\ -2 & 6 \end{bmatrix}$.
	\begin{enumerate}
		\item[$p=5$] 
			Note that 42 reduces to $2 \mod 5$, therefore having inverse is $3$.
			Thus our inverted matrix is 
			$$
			3 \begin{bmatrix}6 & 3\\ -2 & 6 \end{bmatrix} = \begin{bmatrix}18 & 9\\ -6 & 18 \end{bmatrix} \equiv \begin{bmatrix}3 & 4\\ 4 & 3 \end{bmatrix} \mod{5}.
			$$
			Therefore the solution to our equation is 
			$$
			\begin{bmatrix}3 & 4\\ 4 & 3 \end{bmatrix} \begin{bmatrix} 3 \\ 1\end{bmatrix} = \begin{bmatrix} 9 + 4\\12 + 3	\end{bmatrix} \equiv \begin{bmatrix} 3\\0	\end{bmatrix} \mod{5}.
			$$
		\item[$p=7$] Since the determinate above is a multiple of 7, the matrix is not invertable. Therefore
		\begin{align*}
			\begin{bmatrix}6 & -3 & 3\\ 2 & 6 & 1 \end{bmatrix} \grstep[]{R_1 \swap R_2} 
			\begin{bmatrix}2 & 6 & 1\\ 6 & -3 & 3  \end{bmatrix}
			\grstep[R_1 * 4]{R_2 *4} \begin{bmatrix}1 & 3 & 4\\ 3 & 2 & 5 \end{bmatrix}\\
			\grstep[]{R_2 - 3*R_1}
			\begin{bmatrix}
			1 & 3 & 4\\
			0 & 0 & 0
			\end{bmatrix}
\end{align*}		 
	Since we have a row of 0s, then we can have $x_1 = 4 - 3x_2 = 4 + 4x_2$.
	Therefore our solution is $\begin{bmatrix} x_1 \\ x_2 	\end{bmatrix} =
	\begin{bmatrix} 4 + 4a \\ a 	\end{bmatrix}= \begin{bmatrix} 4 \\ 0 	\end{bmatrix} + \begin{bmatrix} 4 \\ 1 	\end{bmatrix}a, a \in \mathbb{F}_7$
	\end{enumerate}
	
	\item[3.2]\textit{Which of the following subsets is a subspace of the vector space $F^{n\times n}$ of $n \times n$ matrices
with coefficients in $F$?}
	\begin{enumerate}
		\item \textit{Symmetric Matrices}\\
		Symmetric matrices form a subspace as the transpose operation satisfies the following:
		\begin{enumerate}
			\item Suppose $A,B$ are symmetric matrices, therefore
			$$(A+B)^T = A^T + B^T = A + B.$$  Thus their sum is symmetric.
			\item Suppose $A$ is a symmetric matrix, $c \in F$, therefore
			$$(cA)^T = cA^T.$$
			Thus scalar multiples of symmetric matrices are symmetric.
			\item The $n \times n$ matrix of all 0s is trivially symmetric  
		\end{enumerate}
		Therefore the set of symmetric matrices forms a subspace.
		\item \textit{Invertible Matrices }\\
		These do not form a subspace as the 0 matrix has determinate 0, and is thus not invertible. 
		\item \textit{Upper Triangular Matrices}\\
		Upper triangular matrices form a subspace as they satisfy the following:
		\begin{enumerate}
			\item Suppose $A,B$ are upper triangular matrices.  If we consider $A+B$ element wise, 
			then if $i > j$ then $a_{ij} = b_{ij} = 0$, thus $(A+B)_{ij} = a_{ij} + b_{ij} = 0 + 0 = 0$.  Therefore their sum is upper triangular.
			\item Suppose $A$ is upper triangular, $c \in F$.  Therefore if $i > j$ then $(cA)_{ij} = c a_{ij} = c * 0 = 0$.  Thus $cA$ is upper triangular.
			\item The 0 matrix clearly ensures for all entries $a_{ij}$ where $i>j$ that $a_{ij} = 0$.  Thus the 0 matrix is upper triangular.    
		\end{enumerate}
	\end{enumerate}
\end{enumerate}
\end{document}
