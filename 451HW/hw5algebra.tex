\documentclass[12pt, letterpaper]{article}
\date{\today}
\usepackage[margin=1in]{geometry}
\usepackage{amsmath}
\usepackage{hyperref}
\usepackage{cancel}
\usepackage{amssymb}
\usepackage{fancyhdr}
\usepackage{pgfplots}
\usepackage{booktabs}
\usepackage{pifont}
\usepackage{amsthm,latexsym,amsfonts,graphicx,epsfig,comment}
\pgfplotsset{compat=1.16}
\usepackage{xcolor}
\usepackage{tikz}
\usetikzlibrary{shapes.geometric}
\usetikzlibrary{arrows.meta,arrows}
\newcommand{\Z}{\mathbb{Z}}
\newcommand{\N}{\mathbb{N}}
\newcommand{\R}{\mathbb{R}}
\newcommand{\Q}{\mathbb{Q}}
\newcommand{\C}{\mathbb{C}}

\newcommand{\Po}{\mathcal{P}}
\newcommand{\Pro}{\mathbb{P}}
\author{Alex Valentino}
\title{451 homework}
\pagestyle{fancy}
\renewcommand{\headrulewidth}{0pt}
\renewcommand{\footrulewidth}{0pt}
\fancyhf{}
\rhead{
	Homework 5\\
	451	
}
\lhead{
	Alex Valentino\\
}
\begin{document}
\begin{enumerate}
	\item[8.3] Suppose $|G| = p^n, n > 1$. If we have an element of the order $|a| = p^k$ where $1 < k < n$ then we can simply take the element
	$a^{p^{k-1}}$ then we have that $(a^{p^{k-1}})^p = 1$.   Therefore we have an element of order $p$.
	\item[8.10]
	\begin{itemize}
		\item Suppose for a subgroup $H$ of $G$ that $[G:H]=2$.  
		Then we have that $\{H, aH\}$ partitions $G$ for some $a \in G \backslash H$.  Suppose $r \in aH$.  Therefore $r \not \in H$.  Therefore since $\{H,Ha\}$ partitions $G$ then $r \in HA$. A similar proof exists for 
		the other direction.  Thus $aH = Ha$.  
		\item Consider the subgroup of $S_3$  $\{e, (12)\}$.  Clearly 
		the index is 3 since $3 = 6/2$ by Lagrange's theorem.  
		Note that $(123)(12) = (13)$ and $(12)(123) = (23)$.  Thus the 
		subgroup is not normal.  
	\end{itemize}
	\item[9.4] 
	\begin{itemize}
		\item Note that 	$2^{-1} \equiv 5 \mod{9}$.  Thus $25 \equiv 7 \equiv x \mod{9}$.   For all $n \in \Z$ $x = 9n + 7$
		\item Since $2x-5$ is always odd, and $6$ is an even number then 
		$6 \nmid 2x -5$.  Thus $2x \not \equiv 5 \mod 6$
	\end{itemize}
	\item[10.5] 
	\begin{itemize}
		\item $ker \phi$ is a subgroup w/ $ker \phi$.
		\item $S_4$ is a subgroup w/ $ker \phi$.
		\item $A_4$ contains $\ker \phi$.  
		\item The group generated by $(1234), (12)(34)$ contains $\ker \phi$.
		\item The group generated by $(1324), (13)(24)$ contains $\ker \phi$.
		\item The group generated by $(1342), (13)(42)$ contains $\ker \phi$.
	\end{itemize}
	\iffalse Since $|\ker \phi| = 4$ and we know by Lagrange's theorem that any subgroup $H$ containing $\ker \phi$ must satisfy $4 \mid |H|$.
	Further more since $H$ is a subgroup of $S_4$ then we must have that 
	$24/|H|$.  Thus our subgroups can only be of order $4,8,12, 24$.
	Clearly $ $ \fi
	\item[11.5]  We want to show that if $Z_1$ is the center of $G_1$ and
	$Z_2$ is the center of $G_2$ then $Z_1 \times Z_2$ is the center of $G_1 \times G_2$.  Suppose $g \in G_1 \times G_2$ and $z \in Z_1 \times Z_2$.
	Then by definition $g = (g_1,g_2)$ and $z = (z_1,z_2)$.  Therefore 
	$g \cdot z = (g_1 z_1, g_2 z_2) = (z_1 g_1, z_2 g_2) = z \cdot g$.\\
	To show that $Z_1 \times Z_2$ is uniquely the center of $G_1 \times G_2$ 
	suppose for contradiction that there is an element 
	$h \in G_1 \times G_2 \backslash Z_1 \times Z_2$ such that
	for all $g \in G_1 \times G_2$, $g \cdot h = h \cdot g$.
	Therefore by definition of being a member of the product group
	there exists $h_1 \in G_1, h_2 \in G_2$ such that $h = (h_1, h_2)$.
	Thus if $h \cdot g = g \cdot h$ then $(h_1g_1, h_2 g_2) = (g_1h_1, g_2h_2)$.  However this implies that $h_1 \in Z_1, h_2   \in Z_2$.
	Thus $h \in Z_1 \times Z_2$.  Thus we have found the unique centralizer of $G_1 \times G_2$.    
	\item[12.2] 
	\begin{itemize}
		\item $H$ is a subgroup of $GL_3(\R)$ since given two elements 
		$A,B \in H$ of the form 
		$$ A = \begin{bmatrix} 1 & x & y\\ 0 & 1 & z\\ 0 & 0 & 1 \end{bmatrix}, B = \begin{bmatrix} 1 & a & b\\ 0 & 1 & c\\ 0 & 0 & 1\end{bmatrix}		 $$, then $AB = \begin{bmatrix}
		1 & x + a & y + b + cx\\ 0 & 1 & z + c\\ 0 & 0 & 1
\end{bmatrix}		 $.  This gives us that it's closed under multiplication, and if we set $z,y,z,a,b,c$ to 0 then we have the identity matrix.\\
If for a given matrix with $x,y,z$ as entries as shown above 
then if we set $a=-x, c=-z, b = zx-y$ then we have the identity matrix, giving us the inverse.\\  Thus $H$ is a group
		\item Clearly $K$ is a subgroup since by the calculation above 
		if we set $x=z=a=c=0$ then we have two matrices which are in $K$, 
		and clearly their product is also in $K$.  Similarly the identity is in $K$. \\
		Note that if $c=x=z=a=0$ and $b=-y$ then we get the identity matrix.		
		This makes it a subgroup of $H$.
		\item The quotient group of $H / K$ is of the form
		$\{\begin{bmatrix} 1 & l & 0\\ 0 & 1 & r\\ 0 & 0 & 1 \end{bmatrix}: l,r \in \R\}$ since by our formula above we have that the multiplication of 
		any two of these matrices from this group yields the matrix
		$\begin{bmatrix} 1 & x + a & cx\\ 0 & 1 & z+c\\ 0 & 0 &1\end{bmatrix}$, which we can quotient out by the matrix $\begin{bmatrix}
		1 & 0 & cx\\ 0 & 1 & 0\\ 0 & 0 & 1
		\end{bmatrix}$ to get the matrix$\begin{bmatrix} 1 & x + a & 0\\ 0 & 1 & z+c\\ 0 & 0 &1\end{bmatrix}$, which is in the set I defined to be 
		the quotient group.  
		\item In order to find the centralizer, if we assume $A \in Z$, fixing $x,y,z$ and allow $B$ to vary, we can find the solution to the equation $AB = BA$ by swapping all $x,y,z$ for $a,b,c$.  This will yield 
		the equation $y + b+cx = y+baz$ in top left entry.  Since $a,c$ vary and are independent of each other, then the only fixed solution for $x,z$ would be $x=z=0$.  Thus $A$ is in $K$.  Similarly it's clear that all $K$ commute with elements in $H$, by taking $B$ to have $a=c=0$
		$$
			AB = \begin{bmatrix}
		1 & x & y + b\\ 0 & 1 & z\\ 0 & 0 & 1
\end{bmatrix} = BA
		$$
		
		, thus $K = Z$, $K$ is the centralizer.  
	\end{itemize}
\end{enumerate}
\end{document}
