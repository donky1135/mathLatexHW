	\documentclass[12pt, letterpaper]{article}
\date{\today}
\usepackage[margin=1in]{geometry}
\usepackage{amsmath}
\usepackage{hyperref}
\usepackage{cancel}
\usepackage{amssymb}
\usepackage{fancyhdr}
\usepackage{pgfplots}
\usepackage{booktabs}
\usepackage{pifont}
\usepackage{amsthm,latexsym,amsfonts,graphicx,epsfig,comment}
\pgfplotsset{compat=1.16}
\usepackage{xcolor}
\usepackage{tikz}
\usetikzlibrary{shapes.geometric}
\usetikzlibrary{arrows.meta,arrows}
\newcommand{\Z}{\mathbb{Z}}
\newcommand{\N}{\mathbb{N}}
\newcommand{\R}{\mathbb{R}}
\newcommand{\Q}{\mathbb{Q}}
\newcommand{\C}{\mathbb{C}}
\newcommand{\F}{\mathbb{F}}

\newcommand{\Po}{\mathcal{P}}
\newcommand{\Pro}{\mathbb{P}}
\author{Alex Valentino}
\title{412 homework}
\pagestyle{fancy}
\renewcommand{\headrulewidth}{0pt}
\renewcommand{\footrulewidth}{0pt}
\fancyhf{}
\rhead{
	Homework 5\\
	412	
}
\lhead{
	Alex Valentino\\
}
\begin{document}
\begin{enumerate}
	\item[8.16] (rudin) Let $\sigma_N [f](x) = 
	\int_{-\pi}^\pi f(x-t) K_N(t) dt$.  We want to show that 
	$\sigma_N[f](x) \to \frac{f(x-) + f(x+)}{2}$. Let $\epsilon > 0$
	be given.  By using the evenness 
	of $K_N$ and that $\frac{1}{2 \pi}\int_0^\pi K_N(t)dt = 1/2$ we have that,
	\begin{align*}
	&|\frac{1}{2 \pi}\int_{-\pi}^\pi f(x-t) K_N(t) dt - \frac{f(x-) + f(x+)}{2}|\\
	 &= |\frac{1}{2 \pi}\int_{0}^\pi [f(x-t) + f(x+t)] K_N(t) dt - \frac{f(x-) + f(x+)}{2}|\\ &= |\int_{0}^\pi [f(x-t) + f(x+t)] K_N(t) dt - \frac{1}{2 \pi}\int_0^{\pi}f(x-)K_N(t) dt - \frac{1}{2 \pi}\int_0^{\pi} f(x+)K_N(t) dt  |\\ + 
	 &\leq |\frac{1}{2 \pi}\int_{0}^\pi |f(x-t) - f(x-) | K_N(t) dt 
	 + \frac{1}{2 \pi}\int_{0}^\pi |f(x+t) - f(x+) | K_N(t) dt.
	\end{align*}	
	Looking just at the integral 
	$\frac{1}{2 \pi} \int_{0}^\pi |f(x-t) - f(x-) | K_N(t) dt$, we know
	that $\lim_{t \to x^-} f(t)$ exists, therefore there exists 
	$\delta > 0$ such that $|f(x-t) - f(x-)| < \epsilon$ for all 
	$t \in (0,\delta)$.  Note that if we split the integral into
	$\frac{1}{2 \pi} \int_{0}^\delta [f(x-t) - f(x-) ] K_N(t) dt
	+ \frac{1}{2\pi} \int_{\delta}^\pi K_N(t) dt$ 
	that the first integral will be bounded above by $\frac{e \delta}{2\pi}$, and by the proof of fejer's theorem in your (Han's) notes 
	the second integral is bounded above by $M\epsilon$ for a 
	sufficently large $N$, and the bound $|f(t)| < M$ on $[-\pi, \pi]$,
	as $f$ is Riemann integrable.  Note that all the previous steps can 
	be applied to the other integral approaching $x$ from the left.  
	Therefore the proposition holds.  
	\item[4.3.6] The fourier sine series of $g(x) = 1$ is given 
	by the fourier coefficents $\frac{1}{\pi}\int_0^\pi \sin(nx)dx = \frac{1}{\pi}\frac{\sin(n\pi) - \sin(0)}{n} = 0$, and $\frac{1}{\pi}\int_0^\pi dx = 1$.  Thus the fourier series of $g(x)$ is $g(x)$, 
	it is uniformly convergent.  
	\item[4.3.8] Let $g \approx a_0 + \sum_{n=1}^\infty a_n \cos \left(\frac{n \pi}{l}x \right) + b_n \sin \left(\frac{n \pi}{l}x \right)$, by the conditions stated in the problem we know this holds on $[-l,l]$.   Let $f(x) = g(\frac{l}{\pi}x)$.  Then 
	$f \approx a_0 + \sum_{n=1}^\infty a_n \cos \left(nx \right) + b_n \sin \left(nx \right)$ on $[-\pi,\pi]$
	 With 
	$f'$ defined on all but a finite number of points and is piecewise
	continuous.  Then by theorem 4.3.2 
	$f' \approx \sum_{n=1}^\infty n[a'_n \cos \left(\frac{n \pi}{l}x \right)+  b'_n \sin \left(\frac{n \pi}{l}x \right)]$ where 
	$a'_n = nb_n, b'_n = -na_n$.  Therefore by parseval's equality 
	we have that $\int_{-\pi}^\pi |f'(t)|^2 dt = 
	\pi \sum_{n=1}^\infty [{a'_n}^2 + {b'_n}^2] = \pi \sum_{n=1}^\infty n^2[{a_n}^2 + {b_n}^2]$.  Observe that 
	$\int_{-l}^l |g'(t)|^2 dt = \int_{-\pi}^\pi |g'(\frac{l}{\pi}t)|^2 
	\frac{l}{\pi}dt
	=\int_{-\pi}^\pi \frac{l}{\pi}|\frac{\pi}{l}f'(t)|^2 dt = 
	\frac{\pi}{l}\int_{-\pi}^\pi|f'(t)|^2 dt =
	\frac{\pi^2}{l} \sum_{n=1}^\infty n^2[{a_n}^2 + {b_n}^2]
	= \sum_{n=1}^\infty \left(\frac{n\pi}{l}\right)^2 l[{a_n}^2 + {b_n}^2].
	  $\\
	  This demonstrates the desired result.  
	\item[4.3.9] By the conditions gave in the problem then there 
	exists a fourier series such that 
	$g(x) \approx  a_0 +   \sum_{i=1}^n a_n \cos(\frac{n \pi}{L} x) + \sin(\frac{n \pi}{L} x) $.  Therefore $g(x) - \bar{g} \approx \sum_{i=1}^n a_n \cos(\frac{n \pi}{L} x)$.  Thus applying parseval's 
	equality yields 
	$\int_0^L |g - \bar{g}|^2 dx = L\sum_{i=1}^n [a_n^2 + b_n^2]$.  Furthermore, by the previous problem we know that   
	$\int_0^L |g'(x)|^2 dx = \sum_{n=1}^\infty \left(\frac{n2\pi}{L}\right)^2 L[{a_n}^2 + {b_n}^2]$.  Therefore if we multiply this series 
	by $\left(\frac{L}{2\pi}\right)^22$ we get that 
	$\left(\frac{L}{2\pi}\right)^2 \int_0^L |g'(x)|^2 dx
	= \sum_{n=1}^\infty  n^2L[{a_n}^2 + {b_n}^2]$.  Note 
	that term by term in the series representation we have 
	$L[{a_n}^2 + {b_n}^2]$ and $Ln^2[{a_n}^2 + {b_n}^2]$.  Note 
	that $n^2 \geq 1$, therefore for each term the inequality 
	$L[{a_n}^2 + {b_n}^2] \leq Ln^2[{a_n}^2 + {b_n}^2]$ holds.  
	Therefore, since each $L[{a_n}^2 + {b_n}^2] > 0$, the sums maintain
	the inequality, therefore 
	$\int_0^L |g - \bar{g}|^2 dx \leq \left(\frac{L}{2\pi}\right)^2 \int_0^L |g'(x)|^2 dx$.  Furthermore if $g = \bar{g} +
	 a_1 \cos\left(\frac{2 \pi}{L}x\right) + b_1 \sin\left(\frac{2 \pi}{L}x\right)$, then additionally $g' = \frac{2\pi}{L}[b_1 \cos\left(\frac{2 \pi}{L}x\right) - a_1 \sin\left(\frac{2 \pi}{L}x\right)]$, which gives us the identity proved previously 
	 that $\int_0^L |g'(x)|^2 dx = \frac{(2\pi)^2}{L^2} L[a_1^2 + a_2^2]$.  
	 Therefore 
	 \begin{align*}
	 &\int_0^L |g - \bar{g}|^2 dx\\ &= \int_0^L a_1^2 \sin\left(\frac{2 \pi}{L}x\right) + b_1^2 \cos\left(\frac{2 \pi}{L}x\right) + a_1 b_2 |\sin \left(\frac{2 \pi}{L}x\right)\cos \left(\frac{2 \pi}{L}x\right)|dx\\
	 &= L[a_1^2 + b_1^2]\\
	 &= \left(\frac{L}{2\pi}\right)^2 \int_0^L |g'(x)|^2 dx
	 \end{align*}
	 Note that in the only if direction, since the terms in 
	 $\frac{L^2}{4\pi^2} \int_0^L |f'|dx$ have a factor of $n^2$, 
	 then equality can only occur if $n=1$, and this can only occur 
	 if $g = c + a_1 \cos(\frac{2\pi}{L}x) + b_1 \sin(\frac{2\pi}{L}x)$,
	 and furthermore we need $c = \bar{g}$ to have it be properly 
	 subtracted.  
	\item[4.3.10] In this scenario we consider $g$ approximated by 
	a cosine fourier series, $g(x) \approx a_0 + \sum_{n=1}^\infty a_n \cos \left(\frac{\pi}{L}x\right) $.  Note that by definition $a_0 = \bar{g}$,
	thus allowing us to apply parseval's identity and get us 
	$\int_0^L |g - \bar{g}|^2 dx = \frac{L}{2} \sum_{i=1}^n a_n^2 $.
	Note that $g'$ is piecewise continuous, therefore there exists a 
	sine series such that $g' \approx \sum_{i=1}^n a'_n \sin\left(\frac{n\pi}{L}x\right)$ where $a'_n = -\frac{n\pi}{L}a_n $.
	Note that applying parseval's for this series we get that 
	$\int_0^L |g'|^2 dx = \frac{L}{2}\sum_{i=1}^n \left(\frac{n\pi}{L} \right)^2 a_n^2$.  Note that term by term in the respective series we 
	have $a_n^2 \leq n^2 a_n^2$.  Thus 
	$\int_0^L |g-\bar{g}|^2 dx = \frac{L}{2} \sum_{i=1}^n a_n^2 \leq \frac{L}{2}\sum_{i=1}^n n^2 a_n^2 = \left(\frac{L}{\pi} \right)^2 \frac{L}{2}\sum_{i=1}^n \left(\frac{n\pi}{L} \right)^2 a_n^2
	= \left(\frac{L}{\pi} \right)^2 \int_0^L |g'|dx$.  Additionally
	by nearly the same proof as above, we have equality iff
	$g = \bar{g} + a_1 \cos(\frac{\pi}{L}x)$.  
\end{enumerate}
\end{document}
