\documentclass[12pt, letterpaper]{article}
\date{\today}
\usepackage[margin=1in]{geometry}
\usepackage{amsmath}
\usepackage{hyperref}
\usepackage{cancel}
\usepackage{amssymb}
\usepackage{fancyhdr}
\usepackage{pgfplots}
\usepackage{booktabs}
\usepackage{pifont}
\usepackage{amsthm,latexsym,amsfonts,graphicx,epsfig,comment}
\pgfplotsset{compat=1.16}
\usepackage{xcolor}
\usepackage{tikz}
\usetikzlibrary{shapes.geometric}
\usetikzlibrary{arrows.meta,arrows}
\newcommand{\Z}{\mathbb{Z}}
\newcommand{\N}{\mathbb{N}}
\newcommand{\R}{\mathbb{R}}
\newcommand{\Q}{\mathbb{Q}}
\newcommand{\C}{\mathbb{C}}
\newcommand{\F}{\mathbb{F}}

\newcommand{\Po}{\mathcal{P}}
\newcommand{\Pro}{\mathbb{P}}
\author{Alex Valentino}
\title{412 homework}
\pagestyle{fancy}
\renewcommand{\headrulewidth}{0pt}
\renewcommand{\footrulewidth}{0pt}
\fancyhf{}
\rhead{
	Homework 3\\
	412	
}
\lhead{
	Alex Valentino\\
}
\begin{document}
\begin{enumerate}
	\iffalse \item[3.24] Let $X$ be a metric space.
	\begin{enumerate}
		\item Let $\{p_n\}, \{q_n\}, \{r_n\}$ be Cauchy sequences in $X$, and 
		define two sequences to be equivalent in if 
		$\lim_{n \to \infty}d(p_n,q_n) = 0$.  We will show this is an equivalence
		relation
		\begin{itemize}
			\item Reflexivity:  Note that for each $n \in \N, d(p_n,p_n) = 0$ by 
			the definition of a metric.  Therefore $\lim d(p_n,p_n) = 0$.
			\item Symmetry:  Suppose $\lim d(p_n,q_n) = 0$.  
			Note by the definition of a metric that $d(p_n,q_n) = d(q_n,p_n)$.  
			Therefore $\lim d(q_n,p_n) = 0$
			\item Transitivity:  Suppose that $\lim d(p_n,q_n) = 0$ and 
			$\lim(q_n,r_n) = 0$.  We know by the definition of a metric that 
			$d(p_n,r_n) \leq d(p_n,q_n) + d(q_n,r_n)$.  Therefore in the limit 
			we have that $\lim d(p_n,r_n) \leq 0 + 0 = 0$.  
			Since $0 \leq d(p_n,r_n)$ then by the squeeze theorem 
			$\lim d(p_n,r_n) = 0$.
		\end{itemize}
		Thus equivalence of cauchy sequences in $X$ is an equivalence relation.
		\item Let $\{p_n\} \equiv \{r_n\} \in P, \{q_n\} \equiv \{s_n\} \in Q$ where $\equiv$
		is the equivalence in $X$ defined before, and $P,Q$ are equivalence classes in $X*$, and let 
		$\lim d(p_n,q_n) = \Delta(P,Q))$.  
		We claim that $\lim d(r_n,s_n) = \Delta(P,Q)$.  Let $\epsilon > 0$
		be given.  Note that, since 
		$\{p_n\} \equiv \{r_n\}$ then there exists $N_1 \in \N$ such that for 
		all $n \geq N_1, d(p_n,r_n) < \epsilon/3$.  Since $\{q_n\} \equiv \{s_n\}$
		then there exists $N_2 \in \N$ such that for 
		all $n \geq N_1, d(q_n,s_n) < \epsilon/3$.  Since $\lim d(p_n,q_n) = \Delta(P,Q)$ then there exists $N_3 \in \N$ 
		such that for 
		all $n \geq N_3, |d(p_n,q_n) - \Delta(P,Q)| < \epsilon/3$.  
		
		Therefore, taking $N = \max\{N_1,N_2,N_3\}$ we have for $n\geq N$
		\begin{align*}
			|d(r_n,s_n) - \Delta(P,Q)| &\leq |d(r_n,p_n) + d(p_n,s_n) - \Delta(P,Q)|\\
			&\leq |d(r_n,p_n) + d(p_n,q_n) + d(q_n,s_n) - \Delta(P,Q)|\\
			&\leq d(r_n,p_n) + d(q_n,s_n) + |d(p_n,q_n) - \Delta(P,Q)|\\
			&< \epsilon.
		\end{align*}
	\end{enumerate} \fi
	\item[1.2.3] Suppose $G = \cup_l I_l, a_{k,l} = \int_{I_l} |c_k(x)|^p dx$.
	Let $\lim_{k \to \infty} a_{k,l} = b_l$.  Then 
	\begin{align*}
		|\sum_l a_{k,l} - \sum_l b_l| &\leq |\sum_{i=1}^N a_{k,l} - b_l| + 
		|\sum_{l = N+1}^\infty a_{k,l}| + |\sum_{l = N+1}^\infty b_l|. 
	\end{align*}
	Note since both $\sum_{l = 1}^\infty a_{k,l},\sum_{l = 1}^\infty b_l$ converge
	then necessarily there must exist respective $N_1,N_2 \in \N$ such that for 
	all $N \geq N_1, N_2,|\sum_{l = N+1}^\infty a_{k,l}| < \epsilon,
	|\sum_{l = N+1}^\infty b_l|< \epsilon$.  Additionally since $|\sum_l a_{k,l} - \sum_l b_l| \leq \sum_{l=1}^N \int_{I_l} |c_k(x) - \phi(x)|^p dx$, and since 
	$\lim_{k \to \infty} |c_k - \phi|_{L^p[a,b]} \to 0$ then there exists $K > 0$
	such that for all $k \geq K, \sum_{l=1}^N \int_{I_l} |c_k(x) - \phi(x)|^p dx < \epsilon$.  
	Therefore the entire sum is bounded above by $3\epsilon$.  
	Since we have proved the limit interchange, we know that 
	$\int_G |\phi(x)|^p dx = \lim_{k \to \infty}\int_G |c_k(x)|^pdx$.  
	Therefore since it was already proven that the integrals of Cauchy sequences 
	satisfy the UAC property, then for the $\epsilon$ already given, there exists 
	$\delta > 0$ such that for arbitrary open $|G| < \delta, 
	\int_G |c_k(x)|^pdx < \epsilon$.  Therefore by taking the limit we find that
	 $\int_G |\phi(x)|^p dx < \epsilon$
	\item[1.2.4]
	Consider ${b_{k,l}}$ where if $k \leq l $ then $b_{k,l} = 2^{k-l}$, otherwise
	if $k > l, b_{k,l} = 0$.  Note for an arbitrary fixed $k$ the sum 
	$\sum_{l=k}^\infty 2^{k-l} = 2$.  Note that since this is true for any $k$
	then $\lim_{k \to \infty} \sum_{l} b_{k,l} = 2$.  However, if the limit
	is taken on the inside first then we get that $\lim_{k \to \infty} b_{k,l} = 0$
	.  Therefore $\sum_{l} \lim_{k \to \infty} b_{k,l} = 0$
	\item[1.2.5] Consider the function 
	$$
	c_k(x) = 
		\begin{cases} 
			12k^3x^2 & x \in [0,1/2k]\\
			12k^3(x-\frac{1}{k})^2 & x \in [1/2k,1/k]\\
			0 &\text{ otws}		
		\end{cases} 
	$$
	At $x=0, c_k(0) = 12k^3 0^2 = 0$, and if $x \in (0,1]$ then there exists 
	$K \in \N$ such that $\frac{1}{k} < x$ which ensures for all $k \geq K$, 
	$c_k(x) = 0$.  Additionally, for every $k \in \N$ we get that
	$$\int_0^1 c_k(x)dx = \int_0^\frac{1}{2k} 12k^3 x^2 dx + 
	\int_\frac{1}{2k}^\frac{1}{2k} 12k^3 (x-\frac{1}{k})^2 = 
	\frac{1}{2} + \frac{1}{2} = 1.$$
	Therefore the integral limit swap does not work.  Additionally by construction
	we have that $\int_0^\frac{1}{k}f(x)dx = 1$, therefore $f$ is not UAC since
	we can take the arbitrarily small neighborhood of $(0,1/k)$ and still have 
	an integral of 1.   
	\item[1.2.10]  Let $f \in L^1 [a,b]$, with $x \not \in [a,b]$ taking on the 
	value $f(x) = 0$.  We want to show that 
	$\lim_{h \to 0} \int_a^b |f(x+h) - f(x)| dx = 0$.  Let $\epsilon > 0$ be given
	and $\int_a^b |f(x)|dx = M$.
	Since $f \in L^1 [a,b]$ then $|f| \in R[a,b]$.  Therefore there exists a 
	partition $P_1$ such that $|U(|f|,P_1) - L(|f|,P_1)| < \epsilon$
	Additionally note that 
	$$\int_a^b |f(x) - f(x+h)| dx \leq \int_a^b |f(x)| dx + \int_a^b |f(x+h)|dx \leq 2M.$$
	Therefore $|f(x) - f(x+h)| \in R[a,b]$.  Applying the previous fact, there 
	exists a partition $P_2 = \{y_0,\cdots,y_m\}$ such that for 
	arbitrary $s_i \in [y_{i-1},y_{i}],$
	$$ |\sum_{i=1}^m |f(s_i) - f(s_i+h)| \Delta y_i - \int_a^b |f(x) - f(x+h)|dx| < \epsilon.$$
	Let $P = P_1 \cup P_2$.  Since $P$ is a refinement of both $P_1$ and $P_2$
	then both properties hold for $P$.  
	For $P = \{x_0,\cdots,x_n\}$, $s_i \in [x_{i-1},x_i]$, and $\Delta x_i = x_i - x_{i-1}$ with a bit of rearrangement we get that 
	$$
	\int_a^b |f(x) - f(x+h)|dx < \epsilon + |\sum_{i=1}^n ||f(s_i)| - |f(s_i+h)|| \Delta x_i|.
	$$
	Since there is a finite number of $\Delta x_i$'s, we can choose 
	$d = \min\{\Delta x_1,\cdots \Delta x_n\}$.  Therefore if we choose $c_i's$
	such that $c_i = (x_{i} - x_{i-1})/2$ and $|h| < d/2$ then that guarantees that
	both $c_i$ and $c_i + h$ will be within their given $[x_{i-1},x_{i}]$.  
	Therefore for every $||f(c_i)| - |f(c_i+h)| \leq M_i - m_i$, and if $c_i + h < a$ 
	or $c_i + h  > b$ then $|f(c_i)| \leq M_i$, which ensures that 
	$$|\sum_{i=1}^n ||f(s_i)| - |f(s_i+h)|| \Delta x_i| \leq |U(|f|,P) - L(|f|,P)| < \epsilon$$.  Therefore $\int_a^b |f(x) - f(x+h)|dx  < 2 \epsilon$.  
	\item[1.2.12] Consider $f \in L^1[a,b]$ with the extension that 
	$x \not \in [a,b]$ corresponding to $f(x) = 0$.  Additionally let 
	$\{c_k\} \subset C[a,b]$ be a Cauchy sequence for which 
	$|c_k|_{L^1[a,b]} \to f, f_h(x) = h^{-1}\int_x^{x+h} f(y)dy, 
	c_{k,h} = h^{-1}\int_x^{x+h} c_{k,h}(y)dy, 
	C_k(x) = \int_0^x c_k(y)dy$.  Then one can write
	$$\int_a^b |f_h(x) - f(x)|dx \leq 
	\int_a^b |f_h(x) - c_{k,h}(x)|dx + \int_a^b |c_{k,h}(x) - c_k(x)|dx
	+ \int_a^b |c_k(x) - f(x)|.$$
	Note that for both $\int_a^b |f_h(x) - c_{k,h}(x)|dx,\int_a^b |c_k(x) - f(x)|$ for 
	sufficently large $k$, one has that $|c_k(x) - f(x)| < \epsilon$.  Since 
	$f_h(x) - c_{k,h}(x) \leq h^{-1}\int_x^{x+h} |f(x) - c_k(x)|dx < h^{-1} \epsilon(x+h-x) = 
	\epsilon$, then both terms are bounded above by $\epsilon(b-a)$.  For the term 
	$c_{k,h}(x) - c_k(x),$ since $c_{k,h}(x) = h^{-1}\int_x^{x+h} c_k(y)dy = 
	\frac{C(x+h)-C(x)}{h}$, and since $c_k \in C[a,b]$ implies that $C_k(x)$ is 
	differentiable.  Therefore there exists $x' \in (x,x+h)$ such that $\frac{C_k(x+h)-C_k(x)}{h} = c_k(x')$ by the mean value theorem.  Therefore for sufficiently small $h$, by the 
	continuity of $c_k, |c_k(x')-c_k(x)| <\epsilon $ bounding $\int_a^b |c_{k,h}(x) - c_k(x)|dx < \epsilon(b-a)$.  Thus 
	$\int_a^b |f_h(x) - f(x)|dx < 3(b-a)\epsilon$.  
\end{enumerate}
\end{document}
