\documentclass[12pt, letterpaper]{article}
\date{\today}
\usepackage[margin=1in]{geometry}
\usepackage{amsmath}
\usepackage{hyperref}
\usepackage{cancel}
\usepackage{amssymb}
\usepackage{fancyhdr}
\usepackage{pgfplots}
\usepackage{booktabs}
\usepackage{pifont}
\usepackage{amsthm,latexsym,amsfonts,graphicx,epsfig,comment}
\pgfplotsset{compat=1.16}
\usepackage{xcolor}
\usepackage{tikz}
\usetikzlibrary{shapes.geometric}
\usetikzlibrary{arrows.meta,arrows}
\newcommand{\Z}{\mathbb{Z}}
\newcommand{\N}{\mathbb{N}}
\newcommand{\R}{\mathbb{R}}
\newcommand{\Q}{\mathbb{Q}}
\newcommand{\C}{\mathbb{C}}
\newcommand{\F}{\mathbb{F}}

\newcommand{\Po}{\mathcal{P}}
\newcommand{\Pro}{\mathbb{P}}
\author{Alex Valentino}
\title{412 homework}
\pagestyle{fancy}
\renewcommand{\headrulewidth}{0pt}
\renewcommand{\footrulewidth}{0pt}
\fancyhf{}
\rhead{
	Final\\
	412	
}
\lhead{
	Alex Valentino\\
}
\begin{document}
\begin{enumerate}
	\item Suppose $f$ is twice differentiable.  Then 
	\begin{align*}
		d(df) &= d \left(\sum_{i=1}^n \frac{\partial f}{\partial x_i}dx_i \right)\\
		&= \sum_{i=1}^n \left( \sum_{j=1}^n 
		\frac{\partial f}{\partial x_j \partial x_i} dx_j
		\right) \wedge dx_i\\
		&= \sum_{i=1}^n \sum_{j=1}^n\frac{\partial f}{\partial x_j \partial x_i} dx_j \wedge dx_i\\
		&= \sum_{i < j}^n  \frac{\partial f}{\partial x_j \partial x_i} dx_j \wedge dx_i
		+ \sum_{i < j}^n \frac{\partial f}{\partial x_i \partial x_j} dx_i \wedge dx_j\\
		&= \sum_{i < j}^n  \frac{\partial f}{\partial x_j \partial x_i} dx_j \wedge dx_i
		- \sum_{i < j}^n \frac{\partial f}{\partial x_j \partial x_i} dx_j \wedge dx_i\\
		&= 0
	\end{align*}
	Now for the map $\gamma : [0,1] \to \R^n$ with 
	$\gamma(0) = \gamma(1)$, we have by the fundamental 
	theorem of calculus that 
	$\int_\gamma df = \int_0^1 \left( \frac{\partial f (\gamma(t))}{\partial x_1},\cdots,\frac{\partial f (\gamma(t))}{\partial x_n} \right)\cdot \gamma'(t) = 
	\int_0^1 (f(\gamma(t))'dt = f(\gamma(1)) - f(\gamma(0)) = f(\gamma(0)) - f(\gamma(0)) = 0$.  
	\item Let $\omega = 
	\frac{-y}{x^2 + y^2}dx + \frac{x}{x^2 + y^2}dy$
	\begin{enumerate}
		\item \begin{align*}
			d\omega &= d\left(\frac{-y}{x^2 + y^2} \right) \wedge dx 
			+ d\left(\frac{x}{x^2 + y^2} \right)\wedge dy\\
			&= \left( \frac{2xy}{(x^2+y^2)^2}
			dx + \frac{y^2 - x^2}{(x^2+y^2)^2}dy \right)\wedge dx + \left(\frac{y^2 - x^2}{(x^2 + y^2)^2}dx 
		+ \frac{-2xy}{x^2 + y^2}dy  \right) \wedge dy\\
		&= 0 + \frac{y^2 - x^2}{(x^2 + y^2)^2}dy \wedge dx + 
		\frac{y^2 - x^2}{(x^2 + y^2)^2}dx \wedge dy + 0\\
		&= \frac{y^2 - x^2}{(x^2 + y^2)^2}dx \wedge dy 
		- \frac{y^2 - x^2}{(x^2 + y^2)^2} dx \wedge dy\\
		&= 0
		\end{align*}
		\item Let $\gamma(t) = (\cos(2 \pi t), \sin(2 \pi t))$.  The integral $\int_\gamma \omega = 
		2 \pi \int_0^{1} (-\sin(2 \pi t))( - \sin(2 \pi t)) 
		+ \cos(2 \pi t) \cos(2 \pi t) dt = 
		2 \pi \int_0^1 dt = 2 \pi $.  Note that the 
		given integral evaluates to $2 \pi$ contradicts 
		that there exists a differentiable $g$ such that 
		$dg = \omega$ since we showed above that for any differentiable 
		function $g$ we have that $\int_\gamma dg = 0$ 
		where $\gamma$ is a closed differentiable loop.
		Since our given $\gamma$ is a closed differentiable loop and  $0 \neq 2 \pi$,  then it is impossible to 
		find such a $g$.  
	\end{enumerate}
	\item Let $x = r \cos (\theta), y = r \sin (\theta)$
	\begin{enumerate}
		\item 
		\begin{align*}
			dx \otimes dx + dy \otimes dy &= (\cos (\theta) dr + - r \sin (\theta) d \theta) \otimes (\cos (\theta) dr + - r \sin (\theta) d \theta)\\ &+ (\sin (\theta) dr + r \cos (\theta) d \theta) \otimes (\sin (\theta) dr + r \cos (\theta) d \theta)\\
			&= \cos^2(\theta) dr \otimes dr - r \cos(\theta) \sin (\theta) dr \otimes d\theta - r \cos(\theta) \sin (\theta) d\theta \otimes dr\\
			&+ r^2 \sin^2(\theta) d\theta \otimes d \theta + 
			\sin^2(\theta)dr \otimes dr + r \cos(\theta) \sin (\theta) dr \otimes d\theta\\
			& + r \cos(\theta) \sin (\theta) d\theta \otimes dr	+ r^2 \cos^2(\theta) d \theta \otimes d \theta\\
			&= (\cos^2(\theta) + \sin^2(\theta))dr \otimes dr + r^2(\cos^2(\theta) + \sin^2(\theta))d\theta \otimes d\theta\\
			&= dr \otimes dr + r^2 d\theta \otimes d\theta
		\end{align*}
		\item 
		\begin{align*}
		dx \wedge dy &= (\cos (\theta) dr + - r \sin (\theta) d \theta) \wedge (\sin (\theta) dr + r \cos (\theta) d \theta)\\
		&= \cos(\theta)\sin(\theta)dr \wedge dr +
		 r \cos^2(\theta)dr \wedge d\theta -\sin^2(\theta)
		 d\theta \wedge dr - r^2 \cos(\theta)\sin(\theta) 
		 d \theta \wedge d \theta\\
		 &= r \cos^2(\theta)dr \wedge d\theta -r\sin^2(\theta)
		 d\theta \wedge dr\\
		 &= r \cos^2(\theta)dr \wedge d\theta +r\sin^2(\theta)
		 dr \wedge d\theta\\
		 &= r(\cos^2(\theta) + \sin^2(\theta))dr \wedge d\theta\\
		 &= r dr \wedge d\theta
		\end{align*}
	\end{enumerate}
	\item
	\begin{enumerate}
		\item Note that\\ $$dx = \sin (\phi) \cos(\theta)dr 
		+ r\cos(\phi)\cos(\theta) d\phi - r \sin(\phi) \sin(\theta)d \theta,$$ \\
		$$dy = \sin (\phi) \sin(\theta)dr 
		+ r\cos(\phi)\sin(\theta) d\phi + r \sin(\phi) \cos(\theta)d \theta,$$ \\
		$$dz = \cos(\phi)dr - r\sin(\phi)d\phi.$$  Therefore,
		\begin{align*}
		dx \wedge dy \wedge dz &= (\sin (\phi) \cos(\theta)dr 
		+ r\cos(\phi)\cos(\theta) d\phi - r \sin(\phi) \sin(\theta)d \theta)\\ &\wedge (\sin (\phi) \sin(\theta)dr
		+ r\cos(\phi)\sin(\theta) d\phi + r \sin(\phi) \cos(\theta)d \theta)\\ &\wedge(\cos(\phi)dr - r\sin(\phi)d\phi)\\
		&= \sin (\phi) \cos(\theta) (r \sin(\phi) \cos(\theta)) (r \sin (\phi)) dr \wedge d \phi \wedge d \theta\\
		&- r \cos(\phi)\cos(\theta)(-r\sin(\phi)\cos(\theta))
		(\cos(\phi(dr) dr \wedge d \phi \wedge d \theta\\
		&- r \sin(\phi)\sin(\theta)(-r \sin^2(\phi)\sin(\theta) - r \cos^2(\phi)\sin(\theta))dr \wedge d \phi \wedge d \theta\\
		&= r^2\sin(\phi)(\cos^2(\theta)\sin^2(\phi) + 
		\cos^2(\phi)\cos^2(\theta)\\ &+ \sin^2(\theta)(\cos^2(\phi) + \sin^2(\phi)))dr \wedge d \phi \wedge d \theta\\
		&= r^2 \sin(\phi)(\cos^2(\theta) + \sin^2(\theta))
		dr \wedge d \phi \wedge d \theta\\
		&= r^2 \sin(\phi)\wedge d \phi \wedge d \theta
		\end{align*}
		\item 
		\begin{itemize}
			\item $\frac{\partial f}{\partial r} 
			= \frac{\partial f}{\partial x} \frac{\partial x}{\partial r} + \frac{\partial f}{\partial y} \frac{\partial y}{\partial r} + \frac{\partial f}{\partial z} \frac{\partial z}{\partial r} = \frac{\partial f}{\partial x} \sin(\phi)\cos(\theta) + \frac{\partial f}{\partial y}  \sin(\phi)\sin(\theta) + \frac{\partial f}{\partial z} \cos(\phi)$
			\item $\frac{\partial f}{\partial \phi} 
			= \frac{\partial f}{\partial x} \frac{\partial x}{\partial \phi} + \frac{\partial f}{\partial y} \frac{\partial y}{\partial \phi} + \frac{\partial f}{\partial z} \frac{\partial z}{\partial \phi} = \frac{\partial f}{\partial x} r\cos(\phi)\cos(\theta) + \frac{\partial f}{\partial y} r \cos(\phi)\sin(\theta) - \frac{\partial f}{\partial z} r\sin(\phi)$
			\item $\frac{\partial f}{\partial \theta} 
			= \frac{\partial f}{\partial x} \frac{\partial x}{\partial \theta} + \frac{\partial f}{\partial y} \frac{\partial y}{\partial \theta} + \frac{\partial f}{\partial z} \frac{\partial z}{\partial \theta} = -\frac{\partial f}{\partial x}r\sin(\phi)\cos(\theta) + \frac{\partial f}{\partial y} r \sin(\phi)\cos(\theta)$
		\end{itemize}
	\end{enumerate}
	\item 
	\item Let 
	$\omega = \frac{x dy \wedge dz + y dz \wedge dx + z dx \wedge dy}{(x^2 + y^2 + z^2)^{3/2}}$, and $C$ be 
	the sphereically-parameterized sphere $x^2 + y^2 + z^2 = R^2$ with $(\phi, \theta) \in [0,\pi] \times [0,2 \pi]$.
	Note that by example 6.1 in the stokes theorem notes 
	that a two form exactly of the form above 
	(where $P = x/(x^2 + y^2 + z^2)^{3/2}, Q = y/(x^2 + y^2 + z^2)^{3/2}, R = z/(x^2 + y^2 + z^2)^{3/2}$) with a parameterization 
	from $\gamma : (u,v) \in \square \to \R^3$(which is $c$ in our case
	can be computed via $\int_\square (P \circ \gamma, 
	Q \circ \gamma,R \circ \gamma)\cdot (D_u \gamma \times
	D_v \gamma)$.   Therefore, we first compute 
	$D_\phi \times D_\theta = (R \cos(\phi) \cos(\theta), 
	R \cos(\phi)\sin(\theta), -R\sin(\phi)) \times 
	(R \sin(\phi)\cos(\theta), R  \sin(\phi)\cos(\theta),0)
	= (R^2\sin^2(\phi)\cos(\theta), R^2\sin^2(\phi)\sin(\theta), R^2 \sin(\theta)\cos(\phi))$.  Furthermore, the other vector 
	we're dotting is \\
	$(P \circ \gamma, 
	Q \circ \gamma,R \circ \gamma) = (R \sin(\phi)\cos(\theta), R\sin(\phi)\sin(\phi), R\cos(\phi))$
	\begin{align*}
		\int_c \omega &= \int_{0}^\pi \int_0^{2 \pi} 
		(R \sin(\phi)\cos(\theta), R\sin(\phi)\sin(\phi), R\cos(\phi))\\ &\cdot (R^2\sin^2(\phi)\cos(\theta), R^2\sin^2(\phi)\sin(\theta), R^2 \sin(\theta)\cos(\phi))
		d\theta d \phi\\
		&= \int_0^\pi \int_0^{2 \pi} \sin(\phi) d \theta d \phi\\
		&= 4\pi
	\end{align*}
\end{enumerate}
\end{document}
