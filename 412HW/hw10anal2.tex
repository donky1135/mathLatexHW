\documentclass[12pt, letterpaper]{article}
\date{\today}
\usepackage[margin=1in]{geometry}
\usepackage{amsmath}
\usepackage{hyperref}
\usepackage{cancel}
\usepackage{amssymb}
\usepackage{fancyhdr}
\usepackage{pgfplots}
\usepackage{booktabs}
\usepackage{pifont}
\usepackage{amsthm,latexsym,amsfonts,graphicx,epsfig,comment}
\pgfplotsset{compat=1.16}
\usepackage{xcolor}
\usepackage{tikz}
\usetikzlibrary{shapes.geometric}
\usetikzlibrary{arrows.meta,arrows}
\newcommand{\Z}{\mathbb{Z}}
\newcommand{\N}{\mathbb{N}}
\newcommand{\R}{\mathbb{R}}
\newcommand{\Q}{\mathbb{Q}}
\newcommand{\C}{\mathbb{C}}
\newcommand{\F}{\mathbb{F}}

\newcommand{\Po}{\mathcal{P}}
\newcommand{\Pro}{\mathbb{P}}
\author{Alex Valentino}
\title{412 homework}
\pagestyle{fancy}
\renewcommand{\headrulewidth}{0pt}
\renewcommand{\footrulewidth}{0pt}
\fancyhf{}
\rhead{
	Homework 10\\
	412	
}
\lhead{
	Alex Valentino\\
}
\begin{document}
\begin{enumerate}
	\item[6.7.2.1] Note that the surface we care about is defined 
	via 
	$\begin{bmatrix}
		1 & 0 & -a & -c\\ 0 & 1 & -b & -d
\end{bmatrix} x = \begin{bmatrix}
0\\ 0
\end{bmatrix}	 $.  It is parameterized via 
$\{\begin{bmatrix}
 a & c\\ b & d\\ 1 & 0\\ 0 & 1
\end{bmatrix} \begin{bmatrix}
x_3\\ x_4
\end{bmatrix} : x_3,x_4 \in \R \} $.  Let $A$ be the matrix 
of the parameterized kernel.  Therefore to compute 
the area of the projected unit square in $x_3,x_4$ coordinates 
we need to compute $\sqrt{\det(A^T A)}$.  This computation 
comes out to 
$\sqrt{\det(A^T A)} = 
\sqrt{a^2+b^2+c^2+d^2 +(ad-bc)^2 + 1}$.  Therefore 
we have to solve the integral 
$\int_0^1 \int_0^1\sqrt{\det(A^T A)} dx_3 dx_4 = \sqrt{\det(A^T A)}
= \sqrt{a^2+b^2+c^2+d^2 +(ad-bc)^2 + 1}$
	\item[6.7.2.2] For the triangle defined by 
	$\{(1,0,0), (\cos(\frac{\pi}{n}), \pm \sin(\frac{\pi}{n}), \frac{1}{2m})\}$, we can just recenter to the origin with the new 
	coordinates $\{(0,0,0), (\cos(\frac{\pi}{n})-1, \pm \sin(\frac{\pi}{n}), \frac{1}{2m})\}$.  Now we can compute the area of the parallelogram spanned by the two non-zero vectors, 
	and then halve it to find the area of the triangle.   
	If we let $a = (\cos(\frac{\pi}{n})-1, \sin(\frac{\pi}{n}), \frac{1}{2m}), b=(\cos(\frac{\pi}{n})-1, - \sin(\frac{\pi}{n}), \frac{1}{2m})$ the area computation becomes 
	$$\frac{1}{2}\sqrt{(a\cdot a)(b \cdot b) - (a \cdot b)^2}
	= $$ $$\frac{1}{2} \sqrt{(\frac{1 + 4m^2 - 8m^2 \cos(\frac{\pi}{n}) 
	+ 4m^2\cos^2(\frac{\pi}{n})}{m^2})\sin^2(\frac{\pi}{n})} 
	$$ $$= \frac{1}{2}\sin(\frac{\pi}{n})\sqrt{\frac{1}{m^2} + 
	2^2(1-\cos(\frac{\pi}{n})))^2} = 
	\frac{1}{2}\sin(\frac{\pi}{n})\sqrt{\frac{1}{m^2} + 
	4^2\sin^4(\frac{\pi}{2n})}.
	$$
	For the triangle $\{(1,0,0), (\cos(\frac{\pi}{n}),  \sin(\frac{\pi}{n}), \pm\frac{1}{2m})\}$ we do the same process, 
	where instead we have 
	$a = (\cos(\frac{\pi}{n})-1, \sin(\frac{\pi}{n}), \frac{1}{2m}), b=(\cos(\frac{\pi}{n})-1, \sin(\frac{\pi}{n}), -\frac{1}{2m})$,
	yielding the computation 
	\begin{align*}
	\frac{1}{2}\sqrt{(a\cdot a)(b \cdot b) - (a \cdot b)^2}
	&= \sqrt{\frac{1}{m^2} - \frac{2\cos(\frac{\pi}{n})}{m^2}+ 
	\frac{\cos^2(\frac{\pi}{n})}{m^2}+ 
	\frac{\sin^2(\frac{\pi}{n})}{m^2}} \\ 
	&= \sqrt{\frac{1}{m^2} \sin^2(\frac{\pi}{2n})}\\
	&= \frac{\sin(\frac{\pi}{2n})}{m}
	\end{align*}
	\item[6.7.2.3] To compute the volume of the parallelpiped between 3 of the four verticies of 
	$\{(1,0,0,0),(0,1,0,0),(0,0,1,0),(0,0,0,1)\}$ one 
	must take 3 of the vectors and form a matrix, take
	$A = \begin{bmatrix}
	1 & 0 & 0 & 0\\
	0 & 1 & 0 & 0\\
	0 & 0 & 0 & 1
\end{bmatrix}	 $, Observe that 
$\sqrt{\det(A^T A)} = 1$.  Since one parallelpiped is 
$\frac{1}{6}$th of the total area of the tetrahedron then 
it's 6 times the area, thus the total area is 6.  
	\item[6.7.3.3] 
	\begin{enumerate}
		\item Since $f(x_1,x_2,x_3) = \sqrt{R^2 - x_1^2- x_2^2- x_3^2}$ is a 
	graph over $x_1,x_2,x_3$, we can find the surface area via 
	computing $\sqrt{1 + |Df(x)|^2}$ and computing over 
	$x_1^2 + x_2^ + x_3^2 \leq R^2$.  Note that 
	$Df(x) = \left(\frac{-x_1}{\sqrt{R^2 - x_1^2- x_2^2- x_3^2}},
	\frac{-x_2}{\sqrt{R^2 - x_1^2- x_2^2- x_3^2}},
	\frac{-x_3}{\sqrt{R^2 - x_1^2- x_2^2- x_3^2}}\right)$, 
	thus $Df(x)^T Df(x) = 
	\frac{x_1^2 + x_2^2 + x_3^2}{R^2 - (x_1^2 + x_2^2 + x_3^2)}$.
	Therefore to compute the general surface area we have to compute
	\begin{align*}
	2\int \int \int_{x_1^2 + x_2^ + x_3^2 \leq R^2} \sqrt{1+ |Df(x)|^2} &=  2\int \int \int_{x_1^2 + x_2^ + x_3^2 \leq R^2} 
	\sqrt{\frac{R^2 - x_1^2- x_2^2- x_3^2 +x_1^2+
	x_2^2 + x_3^2}{R^2 - x_1^2- x_2^2- x_3^2}}\\ 
	&= 2\int \int \int_{x_1^2 + x_2^ + x_3^2 \leq R^2} 
	\sqrt{\frac{R^2 }{R^2 - x_1^2- x_2^2- x_3^2}}\\ 
	&= 2R \int_0^{2\pi}\int_0^\pi \int_0^R \frac{r^2 \sin(\theta)}{\sqrt{R^2 -r^2}} d\varphi d\theta d r\\
	&= 8R^2 \pi \int_0^R \frac{r^2}{\sqrt{R^2 - r^2}}  d r\\
	&= 2\pi^2 	R^3
	\end{align*}
		\item Using the sphereical coordinates we have 
		that the surface  is parameterized by 
		$S(\phi, \theta_1,\theta_2) = 
		(R\sin(\theta_2) \sin(\theta_1) \cos(\phi),
		R \sin(\theta_2) \sin(\theta_1) \sin(\phi),
		R \sin(\theta_2) \cos(\theta_1),
		R \cos(\theta_2)		
		)$.  Observe that \\
		$D_\phi S = (-R\sin(\theta_2) \sin(\theta_1) \sin(\phi),
		R \sin(\theta_2) \sin(\theta_1) \cos(\phi),
		0,
		0
		)$\\
		$D_{\theta_1} S = 
		(R\sin(\theta_2) \cos(\theta_1) \cos(\phi),
		R \sin(\theta_2) \cos(\theta_1) \sin(\phi),
		-R \sin(\theta_2) \sin(\theta_1),
		0
		)$\\
		$D_{\theta_2} S = (R\cos(\theta_2) \sin(\theta_1) \cos(\phi),
		R \cos(\theta_2) \sin(\theta_1) \sin(\phi),
		R \cos(\theta_2) \cos(\theta_1),
		-R \sin(\theta_2)		
		)$
		$\|D_{\phi}\|^2 = R^2 \sin^2(\theta_2)\sin^2(\theta_1)$\\
		$\|D_{\theta_1}\|^2 = R^2\sin^2(\theta_2)$\\
		$\|D_{\theta_2}\|^2 = R^2$.\\
		Since the angles have orthogonal tangents then the 
		dot product between any two non-same terms will be zero.
		Thus $\sqrt{(\det((\nabla S)^T \nabla S))} = 
		R^3 \sin^2(\theta_2)\sin(\theta_1)$.  Observe that 
		$\int_0^\pi \int_0^\pi \int_0^{2\pi} R^3 \sin^2(\theta_2)\sin(\theta_1) d \phi d \theta_1 d \theta_2 = R^3 2 \pi^2$
	\end{enumerate}
	
	\item[6.7.3.4] 
	\begin{itemize}
		\item Let our parameterization 
		be denoted $S_r(z,\theta) = (r(z)\cos(\theta),
		r(z)\sin(\theta),z)$.  
		Observe that 
		$D_z S_r = (r'(z)\cos(\theta), r'(z) \sin(\theta),1)$ and that $D_\theta S_r = 
		(-r(z)\sin(\theta), r(z) \cos(\theta), 0)$.
		Therefore 
		$\| D_z\|^2 = |r'(z)|^2 + 1, \| D_\theta\|^2 = 
		r^2(z), D_z \cdot D_r = 0$.  Thus the 
		integral of the surface becomes 
		$\int_a^b \int_0^{2 \pi} \sqrt{r^2(z)(|r'(z)|^2 + 1)}d\theta dz = 2 \pi \int_a^b r(z) \sqrt{|r'(z)|^2 + 1}dz $.
		Note that surface generated via $r(z) =cosh(1)$ 
		is $4 \pi \cosh(1)$.  For $r(z) = cosh(z)$ we 
		have the integral 
		$2 \pi \int_{-1}^1 cosh(z)^2 dz = 2 \pi(1+cosh(1)sinh(1))$.  Note that $2 \pi(1+cosh(1)sinh(1)$
		is less then $4 \pi \cosh(1)$, thus the radius 
		of $cosh(z)$ is the smaller.  
	\end{itemize}
	\item[6.7.4.3] Since we're integrating with the condition 
	that we're on $x^2 + y^2 = 1$ implies that 
	$x = \cos(\theta), y = \sin(\theta)$ is a valid 
	parameterization.  Furthermore, the condition on the 
	planes gives us that $z = \pm1 - \cos(\theta) - \sin(\theta)$.
	Therefore the integral can be translated into 
	$\int_0^{2\pi} \int_{-1-\cos(\theta) - \sin(\theta)}^{1-\cos(\theta) - \sin(\theta)} z^2 dz d \theta = 
	\frac{8}{3}\int_0^{2 \pi} d \theta = \frac{16\pi}{3}$ \\
	
	
\end{enumerate}
\end{document}
