\documentclass[12pt, letterpaper]{article}
\date{\today}
\usepackage[margin=1in]{geometry}
\usepackage{amsmath}
\usepackage{hyperref}
\usepackage{cancel}
\usepackage{amssymb}
\usepackage{fancyhdr}
\usepackage{pgfplots}
\usepackage{booktabs}
\usepackage{pifont}
\usepackage{amsthm,latexsym,amsfonts,graphicx,epsfig,comment}
\pgfplotsset{compat=1.16}
\usepackage{xcolor}
\usepackage{tikz}
\usetikzlibrary{shapes.geometric}
\usetikzlibrary{arrows.meta,arrows}
\newcommand{\Z}{\mathbb{Z}}
\newcommand{\N}{\mathbb{N}}
\newcommand{\R}{\mathbb{R}}
\newcommand{\Q}{\mathbb{Q}}
\newcommand{\C}{\mathbb{C}}
\newcommand{\F}{\mathbb{F}}

\newcommand{\Po}{\mathcal{P}}
\newcommand{\Pro}{\mathbb{P}}
\author{Alex Valentino}
\title{412 homework}
\pagestyle{fancy}
\renewcommand{\headrulewidth}{0pt}
\renewcommand{\footrulewidth}{0pt}
\fancyhf{}
\rhead{
	Homework 10\\
	412	
}
\lhead{
	Alex Valentino\\
}
\begin{document}
\begin{enumerate}
	\item[6.7.2.1] Note that the surface we care about is defined 
	via 
	$\begin{bmatrix}
		1 & 0 & -a & -c\\ 0 & 1 & -b & -d
\end{bmatrix} x = \begin{bmatrix}
0\\ 0
\end{bmatrix}	 $.  It is parameterized via 
$\{\begin{bmatrix}
 a & c\\ b & d\\ 1 & 0\\ 0 & 1
\end{bmatrix} \begin{bmatrix}
x_3\\ x_4
\end{bmatrix} : x_3,x_4 \in \R \} $.  Let $A$ be the matrix 
of the parameterized kernel.  Therefore to compute 
the area of the projected unit square in $x_3,x_4$ coordinates 
we need to compute $\sqrt{\det(A^T A)}$.  This computation 
comes out to 
$\sqrt{\det(A^T A)} = 
\sqrt{a^2+b^2+c^2+d^2 +(ad-bc)^2 + 1}$.  Therefore 
we have to solve the integral 
$\int_0^1 \int_0^1\sqrt{\det(A^T A)} dx_3 dx_4 = \sqrt{\det(A^T A)}
= \sqrt{a^2+b^2+c^2+d^2 +(ad-bc)^2 + 1}$
	\item[6.7.2.2] For the triangle defined by 
	$\{(1,0,0), (\cos(\frac{\pi}{n}), \pm \sin(\frac{\pi}{n}), \frac{1}{2m})\}$, we can just recenter to the origin with the new 
	coordinates $\{(0,0,0), (\cos(\frac{\pi}{n})-1, \pm \sin(\frac{\pi}{n}), \frac{1}{2m})\}$.  Now we can compute the area of the parallelogram spanned by the two non-zero vectors, 
	and then halve it to find the area of the triangle.   
	If we let $a = (\cos(\frac{\pi}{n})-1, \sin(\frac{\pi}{n}), \frac{1}{2m}), b=(\cos(\frac{\pi}{n})-1, - \sin(\frac{\pi}{n}), \frac{1}{2m})$ the area computation becomes 
	$$\frac{1}{2}\sqrt{(a\cdot a)(b \cdot b) - (a \cdot b)^2}
	= $$ $$\frac{1}{2} \sqrt{(\frac{1 + 4m^2 - 8m^2 \cos(\frac{\pi}{n}) 
	+ 4m^2\cos^2(\frac{\pi}{n})}{m^2})\sin^2(\frac{\pi}{n})} 
	$$ $$= \frac{1}{2}\sin(\frac{\pi}{n})\sqrt{\frac{1}{m^2} + 
	2^2(1-\cos(\frac{\pi}{n})))^2} = 
	\frac{1}{2}\sin(\frac{\pi}{n})\sqrt{\frac{1}{m^2} + 
	4^2\sin^4(\frac{\pi}{2n})}.
	$$
	For the triangle $\{(1,0,0), (\cos(\frac{\pi}{n}),  \sin(\frac{\pi}{n}), \pm\frac{1}{2m})\}$ we do the same process, 
	where instead we have 
	$a = (\cos(\frac{\pi}{n})-1, \sin(\frac{\pi}{n}), \frac{1}{2m}), b=(\cos(\frac{\pi}{n})-1, \sin(\frac{\pi}{n}), -\frac{1}{2m})$,
	yielding the computation 
	\begin{align*}
	\frac{1}{2}\sqrt{(a\cdot a)(b \cdot b) - (a \cdot b)^2}
	&= \sqrt{\frac{1}{m^2} - \frac{2\cos(\frac{\pi}{n})}{m^2}+ 
	\frac{\cos^2(\frac{\pi}{n})}{m^2}+ 
	\frac{\sin^2(\frac{\pi}{n})}{m^2}} \\ 
	&= \sqrt{\frac{1}{m^2} \sin^2(\frac{\pi}{2n})}\\
	&= \frac{\sin(\frac{\pi}{2n})}{m}
	\end{align*}
	\item[6.7.2.3]
	\item[6.7.3.3] Since $\sqrt{R^2 - x_1^2- x_2^2- x_3^2}$ is a 
	graph over $x_1,x_2,x_3$, we can find the surface area via 
	computing 
	\item[6.7.3.4]
	\item[6.7.4.3]
\end{enumerate}
\end{document}
