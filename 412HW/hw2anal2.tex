\documentclass[12pt, letterpaper]{article}
\date{\today}
\usepackage[margin=1in]{geometry}
\usepackage{amsmath}
\usepackage{hyperref}
\usepackage{cancel}
\usepackage{amssymb}
\usepackage{fancyhdr}
\usepackage{pgfplots}
\usepackage{booktabs}
\usepackage{pifont}
\usepackage{amsthm,latexsym,amsfonts,graphicx,epsfig,comment}
\pgfplotsset{compat=1.16}
\usepackage{xcolor}
\usepackage{tikz}
\usetikzlibrary{shapes.geometric}
\usetikzlibrary{arrows.meta,arrows}
\newcommand{\Z}{\mathbb{Z}}
\newcommand{\N}{\mathbb{N}}
\newcommand{\R}{\mathbb{R}}
\newcommand{\Q}{\mathbb{Q}}
\newcommand{\C}{\mathbb{C}}
\newcommand{\F}{\mathbb{F}}

\newcommand{\Po}{\mathcal{P}}
\newcommand{\Pro}{\mathbb{P}}
\author{Alex Valentino}
\title{412 homework}
\pagestyle{fancy}
\renewcommand{\headrulewidth}{0pt}
\renewcommand{\footrulewidth}{0pt}
\fancyhf{}
\rhead{
	Homework 2\\
	412	
}
\lhead{
	Alex Valentino\\
}
\begin{document}
\begin{enumerate}
	\item[6.7]
	\begin{enumerate}
		\item We want to show for $f \in R[0,1]$ that 
		$\int_0^1 f(x) dx = \lim_{c \to 0} \int_c^1 f(x)dx$.  
		Note that $\int_0^1 f(x) dx - \lim_{c \to 0} \int_c^1 f(x)dx = 
		\int_0^c f(x) dx$.  Note that since as a function 
		$F(x) = \int_0^x f(y) dy$ is continuous and that 
		$\lim_{c \to 0} \int_0^c f(x) dx =
		\lim_{c \to 0} F(c) = F(\lim_{c \to 0} c) = F(0) = 0$
		then equality holds.
		\item Consider the function $f(x)$ being defined piecewise on 
		$x \in [\frac{1}{n+1},\frac{1}{n}]$ such that $f(x) = (-1)^{n+1}(n+1)$.
		Note that since the function is simply a countable collection of constant 
		functions that $f$ is in fact Riemann integrable.   
		Therefore if one takes the limit $\lim_{c \to 0} \int_c^1 f(x)dx = 
		\sum_{i=1}^\infty \frac{(-1)^{n+1}}{n} = \log(2)$.  However, if one 
		considers $\int_0^1 |f(x)|dx$, all of the constant functions are made 
		positive, and one is left with $\sum_{n=1}^\infty \frac{1}{n} \leq \int_0^1 |f(x)|dx$.  Therefore the integral with the function $|f(x)|$ does 
		not exists.  
	\end{enumerate}
	\item[6.17] Suppose $P = \{x_0,\cdots,x_n\}$ is a partition of $[a,b]$, 
	and choose $t_i \in [x_{i-1},x_{i}]$ such that $g(t_i)\Delta x_i = G(x_i) - G(x_{i-1})$.  Therefore one can apply the abel summation formula to the 
	approximation to the integral $\int_a^b g(x) \alpha(x) dx, \sum_{i=1}^n g(t_i)\alpha(x_i)\Delta x_i$ as follows:
	
	\begin{align*}
	\sum_{i=1}^n \alpha(x_i)g(t_i)\Delta x_i &= \sum_{i=1}^n \alpha(x_i) (G(x_i)-G(x_{i-1})\\
	&= \sum_{i=1}^n \alpha(x_i)G(x_i) - \sum_{i=1}^n \alpha(x_i)G(x_{i-1})\\
	&= G(b)\alpha(b) + \sum_{i=1}^{n-1} \alpha(x_i)G(x_i)- \sum_{i=1}^n \alpha(x_i)G(x_{i-1})\\
	&= G(b)\alpha(b) - G(a)\alpha(a) + \sum_{i=1}^{n} \alpha(x_{i-1})G(x_{i-1})\sum_{i=1}^n \alpha(x_i)G(x_{i-1})\\
	&= G(b)\alpha(b) - G(a)\alpha(a) + \sum_{i=1}^{n} G(x_i) (\alpha(x_{i-1}) - \alpha(x_{i}))\\
	&= G(b)\alpha(b) - G(a)\alpha(a) - \sum_{i=1}^{n} G(x_i) \Delta \alpha_i
	\end{align*}
	Note that in the limit the final sum on the right converges to $\int_a^b G(x) d\alpha(x)$.  Additionally the original sum on the left converges to $\int_a^b g(x) \alpha(x) dx$.  Thus $\int_a^b g(x) \alpha(x) dx = G(b)\alpha(b) - G(a)\alpha(a) -\int_a^b G(x) d\alpha(x)$
	\item[7.12] Let $h_n(x) = f(x) - f_n(x)$.  Note that since $f_n \leq g$ then 
	$\int_0^\infty f_n(x) dx \leq \int_0^\infty g(x) dx < \infty$, and additionally
	the above inequality would imply that $\int_0^\infty f(x)dx \leq \int_0^\infty g(x) dx$.    Therefore if we consider taking the integral and splitting it up 
	we get $\lim_{n \to \infty} \int_0^\infty h_n(x) dx = 
	\lim_{n \to \infty} \int_0^t h_n(x) dx + \int_t^T h_n(x) dx + \int_T^\infty h_n(x) dx$.  First, independent of $n$ if we take the limit as $t \to 0$ for 
	$\int_0^t h_n(x) dx$ then as shown in exercise 6.7 (a) we can arbitrarily 
	choose a $t$ such that $\int_0^t h_n(x) dx < \epsilon/3$.  Additionally for 
	$\int_T^\infty h_n(x) dx$ since $\int_0^\infty f_n(x) dx = A_n$ and $\int_0^\infty f(x) dx = A$ then for each respective integral we can choose a $T$ 
	large enough such that $|A - A_n - \int_0^T h_n(x) | < \epsilon/3$ since they 
	converge.  Now, we know since $f_n \to f$ uniformly there exists $N\in \N$ 
	such that for all $n \geq N$, $|f_n(x) - f(x)| < \frac{\epsilon}{3(T-t)}$ so that 
	$\int_t^T h_n(x) dx \leq \int_t^T \frac{\epsilon}{3(T-t)} dx = \epsilon/3$.
	Therefore $\lim_{n \to \infty} \int_0^\infty h_n(x) dx = \epsilon$.  
	Therefore $\lim_{n \to \infty} \int_0^\infty f_n(x) = \int_0^\infty f(x) dx$.
	\item[7.20] Note that by the Stone-Weierstrass theorem there exists a sequence
	of polynomials such that $\lim_{n \to \infty} P_n(x) = f(x)$ uniformly on $[a,b]$.  Note that since
	the constituent parts of $P(n)$ are just mononomials, then 
	$\int_0^1 P_n(x) f(x) = 0$.  Therefore by theorem 7.16 in rudin, 
	$0 = \lim_{n \to \infty} \int_0^1 P_n(x) f(x) dx = \int_0^1 f^2(x)$.
	Note that since $\int_0^1 f^2(x) = 0$ then $\bar{\int_0^1} |f(x)|^2 = 0$.
	This implies that the largest value on every interval of $f$ is 0.  
	Additionally since $f^2(x)$ is bounded below by zero implies that $f$ is 0 
	at every point.  Thus $f$ is 0 on $[0,1]$.  
	\item[1.1.6] 
	\begin{itemize}
		\item $\bar{\int_{-1}^1} H(x)dH(x)$.  Note for $y,x > 0$ that 
		$H(x)-H(y) = 1-1=0$ and for $x,y < 0$ 
		additionally $H(x) - H(y) = 0 - 0 = 0$.  Therefore we only have to 
		consider the partitions which either straddles the origin or contains
		the origin. So if $x_i < 0 < x_{i+1}$ then $sup_{[x_i,x_{i+1}]} H(x) (H(x_{i+1}-H(x_i)) = 1 * 1 = 1$.  If we have $[x_i,x_{i+1}], x_{i+1} = 0$ then 
		we have the same scenario of $sup_{[x_i,x_{i+1}]} H(x) (H(x_{i+1})-H(x_i)) = 1 * 1 = 1$.  Thus $\bar{\int_{-1}^1} H(x)dH(x) = 1$.
		\item $\underline{\int_{-1}^1} H(x)dH(x)$.  Note by a similar construction
		above we have the scenario where the smallest value of the straddled 
		interval is 0 and the smallest value of the interval on the origin is 0.
		Thus $\underline{\int_{-1}^1} H(x)dH(x) = 0$.
		\item $\bar{\int_{-1}^1} G(x)dH(x)$.  Note that now if we choose to 
		straddle the origin it matters in the case of the sum.  If $x_i < 0 < x_{i+1}$ then the maximum value $G$ attains on $[x_i,x_{i+1}]$ is 1, as opposed to 
		having $x_{i-1}< x_i = 0 < x_{i+1}$, where on $x \in [x_{i-1},x_i], G(x)=0$.  However since the upper sum chooses the smallest value here, then
		$\bar{\int_{-1}^1} G(x)dH(x) = 0$.
		\item   $\underline{\int_{-1}^1} G(x)dH(x)$.  Trivially for the lower 
		sum, the only situation listed above in which the Stieltjets integral
		could be 1 is when taking the sup on the interval straddling the origin.
		Therefore taking the inf will result in 0.  Thus all lower sums are 0, 
		and the sup of the lower sums is 0.  Therefore $\underline{\int_{-1}^1} G(x)dH(x) = 0$
	\end{itemize}
	Since both the upper and lower sums agree for $\int_{-1}^1 G(x)dH(x)$ then 
	$G(x) \in R(dH)$.  
\end{enumerate}
\end{document}
