<<<<<<< Updated upstream
\documentclass[12pt, letterpaper]{article}
\date{\today}
\usepackage[margin=1in]{geometry}
\usepackage{amsmath}
\usepackage{hyperref}
\usepackage{cancel}
\usepackage{amssymb}
\usepackage{fancyhdr}
\usepackage{pgfplots}
\usepackage{booktabs}
\usepackage{pifont}
\usepackage{amsthm,latexsym,amsfonts,graphicx,epsfig,comment}
\pgfplotsset{compat=1.16}
\usepackage{xcolor}
\usepackage{tikz}
\usetikzlibrary{shapes.geometric}
\usetikzlibrary{arrows.meta,arrows}
\newcommand{\Z}{\mathbb{Z}}
\newcommand{\N}{\mathbb{N}}
\newcommand{\R}{\mathbb{R}}
\newcommand{\Q}{\mathbb{Q}}
\newcommand{\C}{\mathbb{C}}
\newcommand{\F}{\mathbb{F}}

\newcommand{\Po}{\mathcal{P}}
\newcommand{\Pro}{\mathbb{P}}
\author{Alex Valentino}
\title{412 homework}
\pagestyle{fancy}
\renewcommand{\headrulewidth}{0pt}
\renewcommand{\footrulewidth}{0pt}
\fancyhf{}
\rhead{
	Homework 8\\
	412	
}
\lhead{
	Alex Valentino\\
}
\begin{document}
\begin{enumerate}
	\item[5.5.13] Since the Jacobian of $f(x,y,z) = x^2 + y^2 - z^2, g(x,y,z) = x-y$ is 
	$\begin{bmatrix} 2x & 2y & -2z\\ 1 & -1 & 0\end{bmatrix}$ has a submatrix corresponding to 
	$\begin{bmatrix} 2x & 2y\\ 1 & -1 \end{bmatrix}$ with determinant $-2y-2x$ then we can imply the inverse function theorem 
	whenever $-x \neq y$.  Furthermore when we can take partial 
	derivatives, we have that for $(y+c)^2 + y^2 - 1 = z^2$
	$2(y+c) \frac{\partial y}{\partial z} + 2y \frac{\partial y}{\partial z} = 2 z $, thus $\frac{\partial y}{\partial z}  = \frac{z}{c + 2y}$.  For $x$ we have the equation $x^2 + (x-c)^2 - 1 = z^2$, therefore $2x\frac{\partial x}{\partial z} + 2(x-c) \frac{\partial x}{\partial z} = 2z$, therefore 
	$\frac{\partial x}{\partial z} = \frac{z}{2x - c}$
	\item[5.5.14]
	\begin{itemize}
		\item The lemniscate intersects the $x$-axis at 
		$x = 1 \pm a$.  Note that if $a \neq \mp 1 $ then 
		where the lemniscate crosses the origin is at a cusp, 
		thus it is not differentiable there, thus it fails 
		to be invertible.  Otherwise the implicit function theorem 
		applies.  
		\item Note that if we convert the function to polar 
		we find that $r = a + \cos(\theta)$, if we have $|a| > 1$
		then it's impossible for $r=0$, thus the solution around 
		the origin will be isolated.  Otherwise if $|a|\leq 1$ 
		then we have a continuous path in and out of the origin
		by the continuity of $\cos$.  
	\end{itemize}
	\item[5.5.15] Observe that the partial derivative of 
	$\|\vec{x}\|$ with respect to $x_l$ is 
	$$
	\frac{\partial \|\vec{x}\|}{\partial x_l} = p \frac{|x_l|^{p-1}}{(\sum_{i=1}^n |x_i|^p)^{1-\frac{1}{p}}}
	$$
	The conditions under which we can use the 
	implicit function theorem are as follows
	\begin{itemize}
		\item Consider $p \leq 1$.  In this scenario, suppose 
		that we have $\| \vec{x}\| = \| \vec{x}_0 \|$ with 
		$x_j= 0$ in $\vec{x}$, then $(\sum_{i=1}^n |x_i|^p )^{\frac{1}{p}-1} = 0$ since $x_l^p = \infty$, then $\infty^{\frac{1}{p} - 1} = \infty$, thus the derivative is not 
		invertible.  Also if we have $p=1$ then the denominator 
		will still evaluate to 
		\item Consider $p > 1$.  Note for any coordinates which 
		are not $0$, so long as they all aren't zero, we have that 
		$0 < (\sum_{i=1}^n |x_i|^p )^{\frac{1}{p}-1} < \infty$, 
		as $0^p = 0$.  This also gives us the condition under 
		which we cannot apply the implicit function theorem, because 
		if $\vec{x} = \vec{0}$ then the denominator is $0$, which 
		gives us that the derivatives are undefined.  
	\end{itemize}
	Furthermore outside of these circumstances, the derivative 
	as defined above is continuous, thus the function is 
	representable as a differentiable graph on $n-1$ variables.  
\end{enumerate}
\end{document}
=======
\documentclass[12pt, letterpaper]{article}
\date{\today}
\usepackage[margin=1in]{geometry}
\usepackage{amsmath}
\usepackage{hyperref}
\usepackage{cancel}
\usepackage{amssymb}
\usepackage{fancyhdr}
\usepackage{pgfplots}
\usepackage{booktabs}
\usepackage{pifont}
\usepackage{amsthm,latexsym,amsfonts,graphicx,epsfig,comment}
\pgfplotsset{compat=1.16}
\usepackage{xcolor}
\usepackage{tikz}
\usetikzlibrary{shapes.geometric}
\usetikzlibrary{arrows.meta,arrows}
\newcommand{\Z}{\mathbb{Z}}
\newcommand{\N}{\mathbb{N}}
\newcommand{\R}{\mathbb{R}}
\newcommand{\Q}{\mathbb{Q}}
\newcommand{\C}{\mathbb{C}}
\newcommand{\F}{\mathbb{F}}

\newcommand{\Po}{\mathcal{P}}
\newcommand{\Pro}{\mathbb{P}}
\author{Alex Valentino}
\title{412 homework}
\pagestyle{fancy}
\renewcommand{\headrulewidth}{0pt}
\renewcommand{\footrulewidth}{0pt}
\fancyhf{}
\rhead{
	Homework 8\\
	412	
}
\lhead{
	Alex Valentino\\
}
\begin{document}
\begin{enumerate}
	\item[5.5.13] Since the Jacobian of $f(x,y,z) = x^2 + y^2 - z^2, g(x,y,z) = x-y$ is 
	$\begin{bmatrix} 2x & 2y & -2z\\ 1 & -1 & 0\end{bmatrix}$ has a submatrix corresponding to 
	$\begin{bmatrix} 2x & 2y\\ 1 & -1 \end{bmatrix}$ with determinant $-2y-2x$ then we can imply the inverse function theorem 
	whenever $-x \neq y$.  Furthermore when we can take partial 
	derivatives, we have that for $(y+c)^2 + y^2 - 1 = z^2$
	$2(y+c) \frac{\partial y}{\partial z} + 2y \frac{\partial y}{\partial z} = 2 z $, thus $\frac{\partial y}{\partial z}  = \frac{z}{c + 2y}$.  For $x$ we have the equation $x^2 + (x-c)^2 - 1 = z^2$, therefore $2x\frac{\partial x}{\partial z} + 2(x-c) \frac{\partial x}{\partial z} = 2z$, therefore 
	$\frac{\partial x}{\partial z} = \frac{z}{2x - c}$
	\item[5.5.14]
	\begin{itemize}
		\item The lemniscate intersects the $x$-axis at 
		$x = 1 \pm a$.  Note that if $a \neq \mp 1 $ then 
		where the lemniscate crosses the origin is at a cusp, 
		thus it is not differentiable there, thus it fails 
		to be invertible.  Otherwise the implicit function theorem 
		applies.  
		\item Note that if we convert the function to polar 
		we find that $r = a + \cos(\theta)$, if we have $|a| > 1$
		then it's impossible for $r=0$, thus the solution around 
		the origin will be isolated.  Otherwise if $|a|\leq 1$ 
		then we have a continuous path in and out of the origin
		by the continuity of $\cos$.  
	\end{itemize}
	\item[5.5.15] Observe that the partial derivative of 
	$\|\vec{x}\|$ with respect to $x_l$ is 
	$$
	\frac{\partial \|\vec{x}\|}{\partial x_l} = p \frac{|x_l|^{p-1}}{(\sum_{i=1}^n |x_i|^p)^{1-\frac{1}{p}}}
	$$
	The conditions under which we can use the 
	implicit function theorem are as follows
	\begin{itemize}
		\item Consider $p \leq 1$.  In this scenario, suppose 
		that we have $\| \vec{x}\| = \| \vec{x}_0 \|$ with 
		$x_j= 0$ in $\vec{x}$, then $(\sum_{i=1}^n |x_i|^p )^{\frac{1}{p}-1} = 0$ since $x_l^p = \infty$, then $\infty^{\frac{1}{p} - 1} = \infty$, thus the derivative is not 
		invertible.  Also if we have $p=1$ then the denominator 
		will still evaluate to 
		\item Consider $p > 1$.  Note for any coordinates which 
		are not $0$, so long as they all aren't zero, we have that 
		$0 < (\sum_{i=1}^n |x_i|^p )^{\frac{1}{p}-1} < \infty$, 
		as $0^p = 0$.  This also gives us the condition under 
		which we cannot apply the implicit function theorem, because 
		if $\vec{x} = \vec{0}$ then the denominator is $0$, which 
		gives us that the derivatives are undefined.  
	\end{itemize}
	Furthermore outside of these circumstances, the derivative 
	as defined above is continuous, thus the function is 
	representable as a differentiable graph on $n-1$ variables.  
\end{enumerate}
\end{document}
>>>>>>> Stashed changes
