<<<<<<< Updated upstream
\documentclass[12pt, letterpaper]{article}
\date{\today}
\usepackage[margin=1in]{geometry}
\usepackage{amsmath}
\usepackage{hyperref}
\usepackage{cancel}
\usepackage{amssymb}
\usepackage{fancyhdr}
\usepackage{pgfplots}
\usepackage{booktabs}
\usepackage{pifont}
\usepackage{amsthm,latexsym,amsfonts,graphicx,epsfig,comment}
\pgfplotsset{compat=1.16}
\usepackage{xcolor}
\usepackage{tikz}
\usetikzlibrary{shapes.geometric}
\usetikzlibrary{arrows.meta,arrows}
\newcommand{\Z}{\mathbb{Z}}
\newcommand{\N}{\mathbb{N}}
\newcommand{\R}{\mathbb{R}}
\newcommand{\Q}{\mathbb{Q}}
\newcommand{\C}{\mathbb{C}}
\newcommand{\F}{\mathbb{F}}

\newcommand{\Po}{\mathcal{P}}
\newcommand{\Pro}{\mathbb{P}}
\author{Alex Valentino}
\title{412 homework}
\pagestyle{fancy}
\renewcommand{\headrulewidth}{0pt}
\renewcommand{\footrulewidth}{0pt}
\fancyhf{}
\rhead{
	Homework 9\\
	412	
}
\lhead{
	Alex Valentino\\
}
\begin{document}
\begin{enumerate}
	\item[6.2.1] Exercise 3.  Let $\epsilon > 0$.  Then 
	there exists $k \in \N$ such that $\frac{1}{2^k} < 2 \epsilon$.  Let $\Po_y = \{s_0,\cdots s_{2^k}\}$ be the partition of evenly spaced intervals of length $\frac{1}{2^k}$ for the $y$ axis, 
	where $I_{yi} = [s_{i-1},s_i]$.  Furthermore, for any 
	$(x,y) \in [0,1]^2$, $0 \leq f(x,y) \leq \frac{1}{2}$.  Therefore for any $U \subseteq [0,1]^2$,
	$osc(f,U) \leq \frac{1}{2}$.  Thus for an arbitrary 
	$\Po_x$ of the $x$ axis with $2^k$ intervals denoted 
	$I_{xi}$, 
	$\sum_{i=1}^{2^k} osc(f,I_{xi}\times I_{yi}) \frac{1}{2^k}|I_{xi}| \leq \frac{1}{2^{k+1}} \sum_{i=1}^{2^k} |I_{xi}| = 
	\frac{1}{2^{k+1}} < \frac{1}{2}2\epsilon = \epsilon$.  
	Therefore the 2d thomae function $f(x,y)$ is Riemann integrable.
	Since the irrationals are dense in $[0,1]^2$, 
	then for an arbitrary $I_x \times I_y \subseteq [0,1]^2$,
	$m(f,I_x \times I_y) = 0$, thus for an arbitrary partition
	$\Po, L(f,\Po) = 0$.  Thus $\underline{\int}_{[0,1]^2}f = 0$, 
	giving us that $\int_{[0,1]^2} f = 0$.  
	\item[6.2.3] Exercise 1.  Given that $f,g$ are Riemann integrable then both of their sets of discontinuities are of 
	measure 0.  Then the union of those sets is also of measure 0.
	Since the discontinuities of $f\cdot g$ is at most the union 
	of the previous sets, and that is measure zero implies the 
	discontinuities of $f\cdot g$ is measure 0.  Thus $f \cdot g$
	is Riemann integrable.  
	\item[6.2.3] Exercise 2.  Note that for all $x$, $0 \leq 
	\sum_{i=2}^\infty \frac{x^i}{}i!$, therefore $x + 1 \leq e^x$,
	and finally $\log(x+1) \leq x$.  Therefore since $\sum_{k=1}^\infty r_k < \infty$, then $\sum_{k=1}^\infty \log(1+r_k) < \infty$.  This implies that $\prod_{k=1}^\infty (1+r_k) < \infty$.  
	Therefore we get that $\prod_{k=1}^\infty (1-r_k) < \infty$.  
	Note that by induction, if we remove ratio after 
	ratio of the unit interval we get that $\mathcal{K}$ has 
	length $\prod_{k=1}^\infty (1-r_k)$.  If we consider the fact 
	that $r_k \to 0$ as $k \to \infty$ implies that $1-r_k$ will 
	converge to 1.  If $\prod_{k=1}^\infty (1-r_k) = 0$ then 
	that would imply past some $K \in \N$, for all $k \geq K, 
	1 - r_k < r < 1$.  However this is impossible for the 
	afformentioned limit argumentation.  Thus our given set 
	isn't measure 0.  Therefore, an upper sum can attain 1 for 
	intervals of lengths which will sum to the length determined 
	earlier by the product $\prod_{k=1}^infty (1-r_k)$.  Additionally, the lower sums can trivially be made 
	0, as any open set can eventually have a multiple of $\frac{r_k}{2^k}$ put inside of it (end points of the k-th order cantor set process).  Therefore the upper sums and the lower sums disagree, making
	$\chi_\mathcal{K}$ not Riemann integrable.  Note that the character of 
	$\mathcal{K}^c$ is also not Riemann integrable since it is
	equivalent to $1 - \chi_\mathcal{K}$, which is not Riemann 
	integrable.  
	
	\item[6.3.4]	 Note that the function defined as 
	$f(x,y)$ in this problem is equivalent to $1-f(y,x)$ as defined 
	in Exercise 3 of 6.2.1 (first problem in the homework), and in 
	this exercise $f$ is shown to be Riemann integrable, therefore
	$1-f(y,x)$ is Riemann integrable.  Therefore the definition of $f$ for this problem is Riemann integrable.  Now to consider 
	$\bar{\int}_0^1 f(x,y) dy$, for any possible partition of 
	$\Po$ for $[0,1]$ in $y$ and any $x$, $U(f,\Po) = 1$, as the 
	irrationals are dense in $[0,1]$, therefore for any subinterval 
	the max of $1$ can always be attained.  Therefore 
	$\bar{\int}_0^1 f(x,y) dy = 1$.  For $\underline{\int}_0^1 f(x,y) dy$, if $x = \frac{p}{q}, \gcd(p,q) = 1$ then 
	in every possible interval of $y$ for an arbitrary partition 
	$\Po$, $f(x,y) = 1-\frac{1}{q}$, by the density of the rationals.  Thus $L(f,\Po) =1-\frac{1}{q} $.  If $x$ is irrational 
	then $L(f,\Po) = 1$ since every $f(x,y) = 1$.  Thus 
	$\underline{\int}_0^1 f(x,y) dy = \begin{cases}
	1- \frac{1}{q} & x = \frac{p}{q}, \gcd(p,q) = 1\\
	1 & \text{ otherwise}
	\end{cases}$.  
	Therefore based off of this analysis it appears that 
	$\int_0^1 f(x,y) dy$ is Riemann integrable for irrational $x$.  
	\item[6.3.9] Suppose for contradiction that 
	$g(x)\neq 0$ almost everywhere. Note that $0 = \bar{\int}_{R_1} g(\bold{x}) \geq 
	\underline{\int}_{R_1} g(\bold{x}) \geq \min_{\bold{x} \in R_1} g(x) |R_1| \geq 0$, therefore $0 = {\int}_{R_1} g(\bold{x})
	 = \bar{\int}_{R_1} g(\bold{x}) = 
	\underline{\int}_{R_1} g(\bold{x})$, making $g$ Riemann 
	integrable on $R_1$.  Since $g$ 
	is riemann integrable on $R_1$ then it is bounded, therefore 
	there exists $m > 0$ such that $g(\bold{x}) \geq m$, as $g(\bold{x}) \geq 0, \neq 0$ by assumption.  Thus, by definition 
	of Riemann integrability, there exists $\delta>0$ such 
	that for any partition $\lambda(\Po) < \delta$ we have that 
	$\left|\sum_{S_\alpha \in \Po} g(\bold{x}_\alpha) |S_\alpha| \right| < m|R_1|$.  Since $g(x) \neq 0$ almost everywhere in $R_1$, then 
	for each $S_\alpha$ we can choose an $\bold{x}_\alpha$ such that 
	$g(\bold{x}_\alpha) \neq 0$.  Therefore 
	$m|R_1| \leq \left|\sum_{S_\alpha \in \Po} g(\bold{x}_\alpha) |S_\alpha| \right| < m|R_1| $.  This is a contradiction.  Therefore 
	$g(\bold{x}) = 0$ almost everywhere.  
\end{enumerate}
\end{document}
=======
\documentclass[12pt, letterpaper]{article}
\date{\today}
\usepackage[margin=1in]{geometry}
\usepackage{amsmath}
\usepackage{hyperref}
\usepackage{cancel}
\usepackage{amssymb}
\usepackage{fancyhdr}
\usepackage{pgfplots}
\usepackage{booktabs}
\usepackage{pifont}
\usepackage{amsthm,latexsym,amsfonts,graphicx,epsfig,comment}
\pgfplotsset{compat=1.16}
\usepackage{xcolor}
\usepackage{tikz}
\usetikzlibrary{shapes.geometric}
\usetikzlibrary{arrows.meta,arrows}
\newcommand{\Z}{\mathbb{Z}}
\newcommand{\N}{\mathbb{N}}
\newcommand{\R}{\mathbb{R}}
\newcommand{\Q}{\mathbb{Q}}
\newcommand{\C}{\mathbb{C}}
\newcommand{\F}{\mathbb{F}}

\newcommand{\Po}{\mathcal{P}}
\newcommand{\Pro}{\mathbb{P}}
\author{Alex Valentino}
\title{412 homework}
\pagestyle{fancy}
\renewcommand{\headrulewidth}{0pt}
\renewcommand{\footrulewidth}{0pt}
\fancyhf{}
\rhead{
	Homework 9\\
	412	
}
\lhead{
	Alex Valentino\\
}
\begin{document}
\begin{enumerate}
	\item[6.2.1] Exercise 3.  Let $\epsilon > 0$.  Then 
	there exists $k \in \N$ such that $\frac{1}{2^k} < 2 \epsilon$.  Let $\Po_y = \{s_0,\cdots s_{2^k}\}$ be the partition of evenly spaced intervals of length $\frac{1}{2^k}$ for the $y$ axis, 
	where $I_{yi} = [s_{i-1},s_i]$.  Furthermore, for any 
	$(x,y) \in [0,1]^2$, $0 \leq f(x,y) \leq \frac{1}{2}$.  Therefore for any $U \subseteq [0,1]^2$,
	$osc(f,U) \leq \frac{1}{2}$.  Thus for an arbitrary 
	$\Po_x$ of the $x$ axis with $2^k$ intervals denoted 
	$I_{xi}$, 
	$\sum_{i=1}^{2^k} osc(f,I_{xi}\times I_{yi}) \frac{1}{2^k}|I_{xi}| \leq \frac{1}{2^{k+1}} \sum_{i=1}^{2^k} |I_{xi}| = 
	\frac{1}{2^{k+1}} < \frac{1}{2}2\epsilon = \epsilon$.  
	Therefore the 2d thomae function $f(x,y)$ is Riemann integrable.
	Since the irrationals are dense in $[0,1]^2$, 
	then for an arbitrary $I_x \times I_y \subseteq [0,1]^2$,
	$m(f,I_x \times I_y) = 0$, thus for an arbitrary partition
	$\Po, L(f,\Po) = 0$.  Thus $\underline{\int}_{[0,1]^2}f = 0$, 
	giving us that $\int_{[0,1]^2} f = 0$.  
	\item[6.2.3] Exercise 1.  Given that $f,g$ are Riemann integrable then both of their sets of discontinuities are of 
	measure 0.  Then the union of those sets is also of measure 0.
	Since the discontinuities of $f\cdot g$ is at most the union 
	of the previous sets, and that is measure zero implies the 
	discontinuities of $f\cdot g$ is measure 0.  Thus $f \cdot g$
	is Riemann integrable.  
	\item[6.2.3] Exercise 2.  Note that for all $x$, $0 \leq 
	\sum_{i=2}^\infty \frac{x^i}{}i!$, therefore $x + 1 \leq e^x$,
	and finally $\log(x+1) \leq x$.  Therefore since $\sum_{k=1}^\infty r_k < \infty$, then $\sum_{k=1}^\infty \log(1+r_k) < \infty$.  This implies that $\prod_{k=1}^\infty (1+r_k) < \infty$.  
	Therefore we get that $\prod_{k=1}^\infty (1-r_k) < \infty$.  
	Note that by induction, if we remove ratio after 
	ratio of the unit interval we get that $\mathcal{K}$ has 
	length $\prod_{k=1}^\infty (1-r_k)$.  If we consider the fact 
	that $r_k \to 0$ as $k \to \infty$ implies that $1-r_k$ will 
	converge to 1.  If $\prod_{k=1}^\infty (1-r_k) = 0$ then 
	that would imply past some $K \in \N$, for all $k \geq K, 
	1 - r_k < r < 1$.  However this is impossible for the 
	afformentioned limit argumentation.  Thus our given set 
	isn't measure 0.  Therefore, an upper sum can attain 1 for 
	intervals of lengths which will sum to the length determined 
	earlier by the product $\prod_{k=1}^infty (1-r_k)$.  Additionally, the lower sums can trivially be made 
	0, as any open set can eventually have a multiple of $\frac{r_k}{2^k}$ put inside of it (end points of the k-th order cantor set process).  Therefore the upper sums and the lower sums disagree, making
	$\chi_\mathcal{K}$ not Riemann integrable.  Note that the character of 
	$\mathcal{K}^c$ is also not Riemann integrable since it is
	equivalent to $1 - \chi_\mathcal{K}$, which is not Riemann 
	integrable.  
	
	\item[6.3.4]	 Note that the function defined as 
	$f(x,y)$ in this problem is equivalent to $1-f(y,x)$ as defined 
	in Exercise 3 of 6.2.1 (first problem in the homework), and in 
	this exercise $f$ is shown to be Riemann integrable, therefore
	$1-f(y,x)$ is Riemann integrable.  Therefore the definition of $f$ for this problem is Riemann integrable.  Now to consider 
	$\bar{\int}_0^1 f(x,y) dy$, for any possible partition of 
	$\Po$ for $[0,1]$ in $y$ and any $x$, $U(f,\Po) = 1$, as the 
	irrationals are dense in $[0,1]$, therefore for any subinterval 
	the max of $1$ can always be attained.  Therefore 
	$\bar{\int}_0^1 f(x,y) dy = 1$.  For $\underline{\int}_0^1 f(x,y) dy$, if $x = \frac{p}{q}, \gcd(p,q) = 1$ then 
	in every possible interval of $y$ for an arbitrary partition 
	$\Po$, $f(x,y) = 1-\frac{1}{q}$, by the density of the rationals.  Thus $L(f,\Po) =1-\frac{1}{q} $.  If $x$ is irrational 
	then $L(f,\Po) = 1$ since every $f(x,y) = 1$.  Thus 
	$\underline{\int}_0^1 f(x,y) dy = \begin{cases}
	1- \frac{1}{q} & x = \frac{p}{q}, \gcd(p,q) = 1\\
	1 & \text{ otherwise}
	\end{cases}$.  
	Therefore based off of this analysis it appears that 
	$\int_0^1 f(x,y) dy$ is Riemann integrable for irrational $x$.  
	\item[6.3.9] Suppose for contradiction that 
	$g(x)\neq 0$ almost everywhere. Note that $0 = \bar{\int}_{R_1} g(\bold{x}) \geq 
	\underline{\int}_{R_1} g(\bold{x}) \geq \min_{\bold{x} \in R_1} g(x) |R_1| \geq 0$, therefore $0 = {\int}_{R_1} g(\bold{x})
	 = \bar{\int}_{R_1} g(\bold{x}) = 
	\underline{\int}_{R_1} g(\bold{x})$, making $g$ Riemann 
	integrable on $R_1$.  Since $g$ 
	is riemann integrable on $R_1$ then it is bounded, therefore 
	there exists $m > 0$ such that $g(\bold{x}) \geq m$, as $g(\bold{x}) \geq 0, \neq 0$ by assumption.  Thus, by definition 
	of Riemann integrability, there exists $\delta>0$ such 
	that for any partition $\lambda(\Po) < \delta$ we have that 
	$\left|\sum_{S_\alpha \in \Po} g(\bold{x}_\alpha) |S_\alpha| \right| < m|R_1|$.  Since $g(x) \neq 0$ almost everywhere in $R_1$, then 
	for each $S_\alpha$ we can choose an $\bold{x}_\alpha$ such that 
	$g(\bold{x}_\alpha) \neq 0$.  Therefore 
	$m|R_1| \leq \left|\sum_{S_\alpha \in \Po} g(\bold{x}_\alpha) |S_\alpha| \right| < m|R_1| $.  This is a contradiction.  Therefore 
	$g(\bold{x}) = 0$ almost everywhere.  
\end{enumerate}
\end{document}
>>>>>>> Stashed changes
