\documentclass[12pt, letterpaper]{article}
\date{\today}
\usepackage[margin=1in]{geometry}
\usepackage{amsmath}
\usepackage{hyperref}
\usepackage{cancel}
\usepackage{amssymb}
\usepackage{fancyhdr}
\usepackage{pgfplots}
\usepackage{booktabs}
\usepackage{pifont}
\usepackage{amsthm,latexsym,amsfonts,graphicx,epsfig,comment}
\pgfplotsset{compat=1.16}
\usepackage{xcolor}
\usepackage{tikz}
\usetikzlibrary{shapes.geometric}
\usetikzlibrary{arrows.meta,arrows}
\newcommand{\Z}{\mathbb{Z}}
\newcommand{\N}{\mathbb{N}}
\newcommand{\R}{\mathbb{R}}
\newcommand{\Q}{\mathbb{Q}}
\newcommand{\C}{\mathbb{C}}
\newcommand{\F}{\mathbb{F}}

\newcommand{\Po}{\mathcal{P}}
\newcommand{\Pro}{\mathbb{P}}
\author{Alex Valentino}
\title{412 homework}
\pagestyle{fancy}
\renewcommand{\headrulewidth}{0pt}
\renewcommand{\footrulewidth}{0pt}
\fancyhf{}
\rhead{
	Homework 7\\
	412	
}
\lhead{
	Alex Valentino\\
}
\begin{document}
\begin{enumerate}
	\item[5.5.3] Let $S(x,y) = (x^2 - y^2, 2xy)$.  Note that 
	$DS(x,y) = \begin{bmatrix} 2x & -2y\\ 2y & 2x \end{bmatrix}.$ 
	Therefore the determinate of the jacobian is as follows:
	$|DS(x,y)| = 4(x^2 + y^2)$.  The map can only be non-invertible 
	when $|DS(x,y)| = 0$, however this only occurs when $x=y=0$.  
	Furthermore, the structure of the jacobian is all linear continuous
	terms, thus the jacobian is continuous everywhere.  Therefore 
	the inverse function theorem holds for $S$ everywhere but the 
	origin.  Additionally, solving $(u,v) = S(x,y)$ for undetermined 
	$(x,y)$ generally, one finds that $x = \sqrt{\frac{\sqrt{u^2 + v^2} + u}{2}}, y = \sqrt{\frac{\sqrt{u^2 + v^2} - u}{2}}$.  Note that if 
	we consider the inverse near $(0,0)$ that the approximation still 
	holds since $\sqrt{u^2 + v^2} - u \geq u - u = 0$, thus the function
	is always defined since it never goes negative under the square root.
	\item[5.5.4] Let $(u,v) = f(x,y) = (e^x \cos(y), e^x \sin(y))$.
	Then $Df(x,y) = e^x \begin{bmatrix}  \cos(y) & - \sin(y)\\ 
	\sin(y) & \cos(y)
	\end{bmatrix}$.  Note that the determinate $|Df(x,y)| = e^x(\cos^2(y) + \sin^2(y)) = e^x > 0$, therefore the Jacobian is always
	invertible. Furthermore, each term of the Jacobian is the product of 
	continuous functions, thus making it continuous.  Therefore 
	the inverse function theorem holds everywhere.  Note that 
	generally the inverse is defined as $(\frac{1}{2}\log ({u^2 + v^2}), 
	\arctan(\frac{u}{v}))$.  Note that issues don't arise the the 
	inverse of the $x$ coordinate, only the $y$, as 
	$- \frac{\pi}{2} < \arctan(\frac{u}{v}) < \frac{\pi}{2}$.  Therefore 
	the inverse is well defined within $U = B((0,0),\frac{\pi}{2})$.  For 
	the point $(1,0)$ the same condition on $y$ holds, however the disk 
	will be centered at $(1,0)$, thus $V = B((1,0),\frac{\pi}{2})$.  
	If we want to pick a larger open set then $U$ or $V$, one could 
	take $\R \times (-\frac{\pi}{2},\frac{\pi}{2})$.  Since $\log$ is 
	onto for all real numbers uniquely, then the inverse defined for $x$ works
	everywhere.  Note that the given set is open since for arbitrary 
	$(a,b) \in \R \times (-\frac{\pi}{2},\frac{\pi}{2})$, for any 
	chosen $r$ such that $(b-r,b+r) \subseteq (-\frac{\pi}{2},\frac{\pi}{2})$ then any points $(c,d) \in B((a,b),r)$ have that $|d-b| \leq \sqrt{(c-a)^2 + (d-b)^2} < r$, thus $d \in (b-r,b+r)$, and trivially 
	$c \in \R$, thus our given set is open.  
	\item[5.5.6]
	\item[5.5.7]
\end{enumerate}
\end{document}
