\documentclass[12pt, letterpaper]{article}
\date{\today}
\usepackage[margin=1in]{geometry}
\usepackage{amsmath}
\usepackage{hyperref}
\usepackage{cancel}
\usepackage{amssymb}
\usepackage{fancyhdr}
\usepackage{pgfplots}
\usepackage{booktabs}
\usepackage{pifont}
\usepackage{amsthm,latexsym,amsfonts,graphicx,epsfig,comment}
\pgfplotsset{compat=1.16}
\usepackage{xcolor}
\usepackage{tikz}
\usetikzlibrary{shapes.geometric}
\usetikzlibrary{arrows.meta,arrows}
\newcommand{\Z}{\mathbb{Z}}
\newcommand{\N}{\mathbb{N}}
\newcommand{\R}{\mathbb{R}}
\newcommand{\Q}{\mathbb{Q}}
\newcommand{\C}{\mathbb{C}}
\newcommand{\F}{\mathbb{F}}

\newcommand{\Po}{\mathcal{P}}
\newcommand{\Pro}{\mathbb{P}}
\author{Alex Valentino}
\title{412 homework}
\pagestyle{fancy}
\renewcommand{\headrulewidth}{0pt}
\renewcommand{\footrulewidth}{0pt}
\fancyhf{}
\rhead{
	Homework 7\\
	412	
}
\lhead{
	Alex Valentino\\
}
\begin{document}
\begin{enumerate}
	\item[5.5.3] Let $S(x,y) = (x^2 - y^2, 2xy)$.  Note that 
	$DS(x,y) = \begin{bmatrix} 2x & -2y\\ 2y & 2x \end{bmatrix}.$ 
	Therefore the determinate of the jacobian is as follows:
	$|DS(x,y)| = 4(x^2 + y^2)$.  The map can only be non-invertible 
	when $|DS(x,y)| = 0$, however this only occurs when $x=y=0$.  
	Furthermore, the structure of the jacobian is all linear continuous
	terms, thus the jacobian is continuous everywhere.  Therefore 
	the inverse function theorem holds for $S$ everywhere but the 
	origin.  Additionally, solving $(u,v) = S(x,y)$ for undetermined 
	$(x,y)$ generally, one finds that $x = \pm \sqrt{\frac{\pm\sqrt{u^2 + v^2} + u}{2}}, y = \pm\sqrt{\frac{\pm\sqrt{u^2 + v^2} - u}{2}}$.  Since we're effectively  
	doing complex exponentiation then swapping the sign should result in 
	the same value, therefore if we consider approximations close to the origin, 
	say within a ball $B((0,0),\delta)$ for $\delta > 0$, then the points 
	$a = (\delta/2,\delta/2)$ and $b = (-\delta/2,-\delta/2)$ are in $B((0,0),\delta)$.
	Observe that $u(a) = u(b) = 0$ and $v(a) = v(b) = \frac{\delta^2}{2}$.  
	Thus we fail to have a well defined inverse.  
	\item[5.5.4] Let $(u,v) = f(x,y) = (e^x \cos(y), e^x \sin(y))$.
	Then $Df(x,y) = e^x \begin{bmatrix}  \cos(y) & - \sin(y)\\ 
	\sin(y) & \cos(y)
	\end{bmatrix}$.  Note that the determinate $|Df(x,y)| = e^x(\cos^2(y) + \sin^2(y)) = e^x > 0$, therefore the Jacobian is always
	invertible. Furthermore, each term of the Jacobian is the product of 
	continuous functions, thus making it continuous.  Therefore 
	the inverse function theorem holds everywhere.  Note that  the inverse is defined as $(\frac{1}{2}\log ({u^2 + v^2}), 
	\arctan(\frac{u}{v}))$.  Note that issues don't arise for the 
	inverse of the $x$ coordinate, only the $y$, as 
	$- \frac{\pi}{2} < \arctan(\frac{u}{v}) < \frac{\pi}{2}$.  \\
	In order to find the largest disc $U$ around the origin in which $f$ has 
	an inverse, we only must consider the restrictions on $y$, as $e^x$ is always 
	invertible.  Since we have $\cos(y)$ and $\sin(y)$, we claim that the largest 
	ball around $(0,0)$ can be $B((0,0),\pi)$.  Note if we have for a small $\delta$
	the point $(x,\pi+ \delta)$, then we have $u,v$ such that 
	$(u,v) = e^x(-\cos(\delta),-\sin(\delta))$, which is equivalent to 
	$e^x(\cos(\delta - \pi),\sin(\delta - \pi))$, corresponding to the point 
	$(x,\delta - \pi) \in B((0,0),\pi)$, for a small enough choice of $x$.  Thus 
	we have found the largest $U$.\\
	To find the disk $V$ For 
	the point $(1,0)$, we have the $\arctan$ restriction which would make one think 
	that  $V = B((1,0),\frac{\pi}{2})$, however the point $(u,v) = (0,0)$ 
	yields $x = \frac{1}{2}\log(0) = -\infty$, thus the largest disk we can 
	have is $V = B((1,0),1)$.  
	If we want to pick a larger open set for $U$, one could 
	take $\R \times (-\pi,\pi)$.  Since $\log$ is 
	onto for all real numbers uniquely, then the inverse defined for $x$ works
	everywhere.  Note that the given set is open since for arbitrary 
	$(a,b) \in \R \times (-\frac{\pi}{2},\frac{\pi}{2})$, for any 
	chosen $r$ such that $(b-r,b+r) \subseteq (-\frac{\pi}{2},\frac{\pi}{2})$ then any points $(c,d) \in B((a,b),r)$ have that $|d-b| \leq \sqrt{(c-a)^2 + (d-b)^2} < r$, thus $d \in (b-r,b+r)$, and trivially 
	$c \in \R$, thus our given set is open.  
	For a larger $V$, note that the image of $f$ under $U$ would be 
	$\R^2 \backslash \{(x,0): y \leq 0\}$, as the origin can't be mapped since 
	$e^x > 0$, and the point $(-1,0) = e^0(\cos(\pi),\sin(\pi))$ can't be 
	reached since $(x,\pi) \not \in U$.  
	\item[5.5.6] We want to find $\frac{\partial}{\partial y_j} g_k (y_1,\cdots,y_m) = \frac{\partial}{\partial y_j}  f_k(\Phi(y_1,\cdots,y_m))$, where $k=m+1\ldots n$, $\Phi$ is the inverse of 
	$(x_1,\cdots,x_m) \mapsto (f_1(x_1,\cdots,x_m),\ldots,f_m(x_1,\cdots,x_m))$ around $(y_{10},\cdots,y_{m0})$ which satisfies the inverse 
	function theorem.  Let $\vec{y} = (y_1,\cdots,y_m)$  Observe by the general chain rule that 
	$Dg_k(\vec{y}) = D f_k(\Phi(\vec{y})) = 
	[D f_k]_{\Phi(\vec{y})} [D \Phi]_{\vec{y}}$.  Observe that 
	$[D f_k]_{\Phi(\vec{y})} = \left(\frac{\partial f_k(\Phi(\vec{y}))}{\partial x_1 },\cdots,\frac{\partial f_k(\Phi(\vec{y}))}{\partial x_m } \right)$ and since the partials of $f$ are continuous then by the 
	inverse function theorem we have that 
	$[D \Phi]_{\vec{y}} = [D (f_1,\cdots,f_m)]^{-1}_{\Phi(\vec{y})}$.  
	Therefore $\frac{\partial}{\partial y_j} g_k (y_1,\cdots,y_m) = D f_k(\Phi(\vec{y}))\hat{e}_j = \left(\frac{\partial f_k(\Phi(\vec{y}))}{\partial x_1 },\cdots,\frac{\partial f_k(\Phi(\vec{y}))}{\partial x_m } \right) \begin{bmatrix}
	\frac{\partial f_1(\Phi(\vec{y}))}{\partial x_1} & \cdots & \frac{\partial f_1(\Phi(\vec{y}))}{\partial x_m}\\
	\vdots & \ldots & \vdots\\
	\frac{\partial f_m(\Phi(\vec{y}))}{\partial x_1} & \cdots & \frac{\partial f_m(\Phi(\vec{y}))}{\partial x_m}
\end{bmatrix}^{-1} \hat{e}_j	 $
	\item[5.5.7]
	\begin{itemize}
		\item Inversion with respect to the $x$ variable at $t=\pi$:
		$x'(\pi) = \cos(\pi) = -1$, thus the inversion formula can be applied.  
		Since $\arcsin$ only has a domain of $[-\pi/2,\pi/2]$, we must consider 
		the inversion at $x = \sin(t+\pi) = - \sin(t)$, thus $y = - x\sqrt{1-x^2}$.
		Since $y(0) = 0$ then $\frac{dy}{dx}y(0) = \frac{d}{dx}(- x\sqrt{1-x^2})|_{x=0} = -\sqrt{1-x^2}|_{x=0} + \frac{x^2}{\sqrt{1-x^2}}|_{x=0} = -1$.  
		\item See if the inverse function theorem can be applied at $y$ for the 
		following $x$ values:
		\begin{itemize}
			\item $t=0$: $y'(0) = \cos^2(0) - \sin^2(0) = 1$, thus it can be inverted, and 
			$\frac{dx}{dy} = \frac{1}{1} = 1$
			\item $t = \frac{\pi}{2}$: $y'(\frac{\pi}{2}) = \cos^2(\frac{\pi}{2}) - \sin^2(\frac{\pi}{2}) = -1$, thus it can be inverted and 
			$\frac{dx}{dy} = \frac{0}{-1} = 0$
			\item $t=\pi$: $y'(\pi) = \cos^2(\pi) - \sin^2(\pi) = 1$, thus it can 
			be inverted and $\frac{dx}{dy} = \frac{-1}{1} = -1$
		\end{itemize}
		\item For $t = \frac{\pi}{2}$, $x'(\frac{\pi}{2}) = 
		\cos(\frac{\pi}{2}) = 0$, thus the inverse function theorem 
		can't be applied. 
	\end{itemize}
\end{enumerate}
\end{document}
