\documentclass[12pt, letterpaper]{article}
\date{\today}
\usepackage[margin=1in]{geometry}
\usepackage{amsmath}
\usepackage{hyperref}
\usepackage{cancel}
\usepackage{amssymb}
\usepackage{fancyhdr}
\usepackage{pgfplots}
\usepackage{booktabs}
\usepackage{pifont}
\usepackage{amsthm,latexsym,amsfonts,graphicx,epsfig,comment}
\pgfplotsset{compat=1.16}
\usepackage{xcolor}
\usepackage{tikz}
\usetikzlibrary{shapes.geometric}
\usetikzlibrary{arrows.meta,arrows}
\newcommand{\Z}{\mathbb{Z}}
\newcommand{\N}{\mathbb{N}}
\newcommand{\R}{\mathbb{R}}
\newcommand{\Q}{\mathbb{Q}}
\newcommand{\C}{\mathbb{C}}
\newcommand{\F}{\mathbb{F}}

\newcommand{\Po}{\mathcal{P}}
\newcommand{\Pro}{\mathbb{P}}
\author{Alex Valentino}
\title{411 homework}
\pagestyle{fancy}
\renewcommand{\headrulewidth}{0pt}
\renewcommand{\footrulewidth}{0pt}
\fancyhf{}
\rhead{
	Homework 1\\
	412	
}
\lhead{
	Alex Valentino\\
}
\begin{document}
\begin{enumerate}
	\item[6]
	Let $\epsilon > 0$ be given.  We will show that $\sum_{n=1}^\infty (-1)^n \frac{x^2 + n}{n^2}$ converges uniformly on
	the interval $(a,b)$.  Let $B = \max\{|a|,|b|\}$   Note that the series $\sum_{n=1}^\infty \frac{B^2}{n^2} = \frac{\pi^2}{6}$ and $\sum_{n=1}^\infty (-1)^n \frac{1}{n} = -\log(2)$.  Since both of these separate series converges 
	then they both satisfy the Cauchy criterion for series.  Therefore there exists $N_1 \in \N$ such that for all 
	$k \geq m \geq N_1, \sum_{n=m}^k \frac{B^2}{n^2}  < \frac{\epsilon}{2}$ and there exists $N_2 \in \N$ 
	such that $q \geq p \geq N_2, \lvert \sum_{n = p}^q (-1)^n \frac{1}{n} \rvert < \frac{\epsilon}{2}$.  Therefore if
	we take $N = \max \{N_1, N_2\}$ then for all $r \geq s \geq N$, 
	$$
	\lvert \sum_{n = s}^r (-1)^n \frac{x^2 + n}{n^2} \rvert \leq \sum_{n = s}^r \frac{B^2}{n^2} + |\sum_{n = s}^r (-1)^n \frac{1}{n}| < \frac{\epsilon}{2} + \frac{\epsilon}{2} = \epsilon
	$$  
	Since the series satisfies the Cauchy criterion, then it is uniformly convergent.  \\
	The function does not converge absolutely since 
	$$
	\sum_{n=1}^\infty \frac{x^2 + n}{n^2} = \sum_{n=1}^\infty \frac{x^2}{n^2} + \frac{1}{n} > \sum_{n=1}^\infty \frac{1}{n} = \infty.
	$$
	\item[8] Note that for each function in the sum we have that $|c_n I(x - x_n)| \leq |c_n|$.  Therefore by the 
	Weierstrass M-test the series $\sum_{n=1}^\infty c_n I(x-x_n)$ converges uniformly.  To show the continuity of 
	the series when $x \neq x_n$ we must consider two cases.  If $x$ is not a limit point of the sequence $\{x_n\}$ then 
	there must exists $\delta > 0$ such that $V(x,\delta) \cap \{x_n\} = \emptyset$.	Therefore if we consider 
	the subsequence $\{x_{n_k}\}$ that is to the left of $x$ then the value function within $V(x,\delta)$ is simply the 
	constant function with value $\sum_{n_k} c_{n_k}$, thus making it continuous. 
	\\ If $x$ is a limit point of $\{x_n\}$ then for all of the partial sums $\sum_{n=1}^m I(x-x_n) c_n$ is constant for some $\delta > 0$ around $x$.  Therefore 
	since it is true for the partial sums and since the series converges uniformly
	then we can apply the limit interchange theorem and get that 
	$\lim_{t \to x} \lim_{m \to \infty} \sum_{n=0}^m c_n I(t - x_n)$ exists.
	Note that the value is exactly the sum of all the $c_n$ to the left of $x$,
	which is exactly $\sum_{n=0}^\infty c_n I(x - x_n)$.  Thus on all points such 
	that $x \neq x_n$ the function is continuous.  
	
	\item[11] Let $M$ be the bound on all of the partial sums such that $|\sum_{n=1}^k f_n(x)| = |F_k(x)| \leq M$ for all $x \in E$ and let $\epsilon$ be
	given.  Since $g_n \to 0$ uniformly then choose $N \in \N$ such that $|g_m(x)| < 2M \epsilon$ for all $m \geq N, x \in E$.  Therefore, for all $N \leq p \leq q$, 
	we have that 
	\begin{align*}
		\lvert \sum_{n = p}^q f_n(x) g_n(x) \rvert &= 
		\lvert \sum_{n=p}^{q-1} F_n(x) (g_n(x) - g_{n+1}(x)) + F_q(x) g_q (x) - F_{p-1}(x) g_p(x)  \rvert\\ & \text{ abel summation}\\
		&\leq M|\sum_{n=p}^{q-1} (g_n(x) - g_{n+1}(x)) + g_q(x) + g_p(x)|\\ 
		&\text{ uniform bound on } F_n(x)\\ 
		&= 2Mg_p(x)\\
		&< \epsilon
	\end{align*}
	\item[16] Let $K$ be a compact set, $\{f_n\}$ be a set of equicontinuous 
	functions which converge pointwise on $K$.  We want to show that the functions
	converge uniformly.  Let $\epsilon$ be given.  Since $\{f_n\}$ is 
	equicontinuous then there exists $\delta > 0$ such that for all $x,y \in K$,
	$|x-y| < \delta$ implies $|f_n(x) - f_n(y)| < \epsilon/3$ for all $n \in \N$.  
	Since $K$ is compact then there exists a finite set of points 
	$\{x_1,\cdots, x_r\} \subset K$ such that 
	$K \subseteq \cup_{i=1}^r V(x_i,\delta)$.  Therefore for an arbitrary $x\in K$,
	there exists $x_p$ such that $x \in V(x_p,\delta)$.  Since $\{f_n\}$ converges
	pointwise for every point in $K$ then there exists $N \in \N$ such that for 
	all $n \geq m \geq N, |f_n(x_p) - f_m(x_p)| < \epsilon/3$.  Now if take the 
	same indicies and test whether $\{f_n\}$ converges at $x$ we get that
	\begin{align*}
		|f_n(x) - f_m(x)| &= |f_n(x) -f_n(x_p) + f_n(x_p) - f_m(x_p) + f_n(x_p) - f_m(x)|\\
		&\leq |f_n(x) -f_n(x_p)| + |f_n(x_p) - f_m(x_p)| + |f_m(x_p) - f_m(x)|\\
		&< \frac{\epsilon}{3} + \frac{\epsilon}{3} + \frac{\epsilon}{3}\\
		&= \epsilon
	\end{align*}
	\item[18] Let $\{f_n\}$ be a set of Riemann integrable functions on $[a,b]$ 
	which are uniformly bounded by $M > 0$, and let $F_n$ denote $F_n(x) \int_a^x f(t) dt$.  We want to show that $\{F_n\}$ has a uniformly convergent subsequence.
	We claim that $\{F_n(x)\}$ is equicontinuous.  Note that if $x,y \in [a,b]$
	and $|x-y| < \frac{\epsilon}{M}$ then for all 
	$n \in \N,|F_n(x) - F_n(y)| \leq M \frac{\epsilon}{M} = \epsilon $ 
	(by theorem 6.12, Rudin).  Therefore since $\{F_n\}$ is equicontinuous and 
	uniformly bounded by $M(b-a)$ then we can apply theorem 7.25 of Rudin which 
	gives us a uniformly convergent subsequence of $\{F_n\}$.  
\end{enumerate}
\end{document}
