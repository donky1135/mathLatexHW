\documentclass[12pt, letterpaper]{article}
\date{\today}
\usepackage[margin=1in]{geometry}
\usepackage{amsmath}
\usepackage{hyperref}
\usepackage{cancel}
\usepackage{amssymb}
\usepackage{fancyhdr}
\usepackage{pgfplots}
\usepackage{booktabs}
\usepackage{esvect}
\usepackage{pifont}
\usepackage{amsthm,latexsym,amsfonts,graphicx,epsfig,comment}
\pgfplotsset{compat=1.16}
\usepackage{xcolor}
\usepackage{tikz}
\usetikzlibrary{shapes.geometric}
\usetikzlibrary{arrows.meta,arrows}
\newcommand{\Z}{\mathbb{Z}}
\newcommand{\N}{\mathbb{N}}
\newcommand{\R}{\mathbb{R}}
\newcommand{\Po}{\mathcal{P}}
\author{Alex Valentino}
\title{Calc 291}
\pagestyle{fancy}
\renewcommand{\headrulewidth}{0pt}
\renewcommand{\footrulewidth}{0pt}
\fancyhf{}
\rhead{
	Challenge Problem Set 3\\
	Calc 3 291
}
\lhead{
	Alex Valentino\\
}
\begin{document}
	\textbf{Exercises:}\\
	\begin{enumerate}
		\item For any points $\vec{b} \in \R^n$, define $\vec{b}_{proj} = \displaystyle \sum_{j=1}^r (\vec{b}\cdot \hat{u}_j) \hat{u}_j$.  Show for all $\vec{b}_{proj}$ belongs to the span of the columns of $A$, and that for all $y$ in the span of the
columns of $A$,  $\| \vec{b}_{proj} - \vec{b} \|^2 \leq \|\vec{y}-\vec{b}\|^2$.\\
		\begin{itemize}
			\item Show that $\vec{b}_{proj} \in Span\{Cols(A)\}.$\\
			Since $\vec{b}_{proj} = \displaystyle \sum_{j=1}^r (\vec{b}\cdot \hat{u}) \hat{u}$, and each $\hat{u}_j$ is a linear combination of the rows of $A$, then $\vec{b}_{proj}$ is a linear combination of the columns of $A$. 
%			Since $Span\{Cols(A)\} = Q, rank(A) = rank(Q) = r, A = QR,$ and R is an $r \times n$ matrix, then $R$ must be onto $\R^r.$  Suppose not, then there exists a vector $\vec{y} \in \R^r$ such that for all $x \in \R^n$, $\vec{y} \neq R \vec{x}$.  Thus multiplying both sides by $Q$ yields $Q\vec{y} \neq QR \vec{x} = A \vec{x}.$  Since $Q\vec{y} \neq A \vec{x}$ then by definition of span $Span\{Q\} \neq Span\{A\}$.  This is a contradiction. Therefore since $Q^T \vec{b}$ is in $\R^r,$ there must exists $\vec{x} \in \R^n$ such that $Q^T \vec{b} = R \vec{x}$.  Since multiplying both sides by $Q$ yields $QQ^T \vec{b} = QR \vec{x},$ and by definition $QR = A,$  then $QQ^T \vec{b}$ lies in the span of $A$.
			\item $\| \vec{b}_{proj} - \vec{b} \|^2 \leq \|\vec{y}-\vec{b}\|^2$.\\
			Suppose $\vec{y} \in Span\{A\}$.  Then $\vec{y}-\vec{b} = \vec{y} - \vec{b}_{proj} + \vec{b}_{proj} - \vec{b}.$  Since $ \vec{b}_{proj} \in Span\{A\},$ and $\vec{b} \not \in Span\{A\},$ then $\vec{y} - \vec{b}_{proj}$ is orthogonal to $\vec{b}_{proj} - \vec{b}$.  Therefore by the Pythagorean theorem,  $ \|\vec{y}-\vec{b}\|^2 = \|\vec{y} - \vec{b}_{proj}\|^2 + \|\vec{b}_{proj} - \vec{b}\|^2.$  Since $\|\vec{b}_{proj} - \vec{b}\|^2$ is constant then the only function that can be minimized is $\|\vec{y} - \vec{b}_{proj}\|^2$, which defines the square of the distance to $\vec{b}_{proj}.$  Therefore it is minimized when $\vec{y} = \vec{b}_{proj},$ thus $ \| \vec{b}_{proj} - \vec{b}\|^2 \leq \|\vec{y}-\vec{b}\|^2$.
		\end{itemize}
		\item Show that for all $\vec{x} \in \mathbb{R}^n$,
		$$
			\|A \vec{x}_0-\vec{b} \|^2 \leq\|A \vec{x}-\vec{b}\|^2 .
		$$
			Let $\vec{y} = A \vec{x},$ for arbitrary $\vec{x} \in \R^n.$  Then the above equation may be rewritten as $\| \vec{b}_{proj} - \vec{b} \|^2 \leq \|\vec{y}-\vec{b}\|^2$.  Therefore as demonstrated by the problem above the inequality is true.  
	
		\item Show that for $s+1 \leq j \leq n, A \hat{w}_j=0$. Note that $\left\{\hat{w}_{s+1}, \ldots, \hat{w}_n\right\}$ is an orthonormal basis of the null space ${Null}(A)$ of $A$. Next show that if $\vec{x}_0$ satisfies $A \vec{x}_0=\vec{b}_{{proj }}$, then so does
$$
\vec{x}_1:=\vec{x}_0-\sum_{j=s+1}^n\left(\mathbf{x}_0 \cdot \vec{w}_j\right) \vec{w}_j=\sum_{j=1}^s\left(\vec{x}_0 \cdot \vec{w}_j\right) \vec{w}_j .
$$
Moreover, show that $\left\|\vec{x}_1\right\| \leq\left\|\vec{x}_0\right\|$, and there is equality if and only if $\vec{x}_0$ is in the span of the rows of $A$.
		\begin{itemize}
			\item Show that for $s+1 \leq j \leq n, A \hat{w}_j=0$.\\
			Since $Span\{A\} = Q,$ and $Q$ is given by $Q=[\hat{w}_1,\cdots,\hat{w}_s],$ then $Q^\perp = [\hat{w}_{s+1},\cdots,\hat{w}_n]$.  Therefore by the definition of orthogonality $A \hat{w}_j = [\hat{w}_1 \cdot  \hat{w}_j,\cdots,\hat{w}_s \cdot  \hat{w}_j]= [0,\cdots, 0].$
			\item Show that if $\vec{x}_0$ satisfies $A \vec{x}_0=\vec{b}_{\text {proj }}$, then so does
$$
\vec{x}_1:=\vec{x}_0-\sum_{j=s+1}^n\left(\vec{x}_0 \cdot \vec{w}_j\right) \vec{w}_j=\sum_{j=1}^s\left(\vec{x}_0 \cdot \vec{w}_j\right) \vec{w}_j .
$$
		\begin{align*}
			A \vec{x}_1 &= A(\vec{x}_0-\sum_{j=s+1}^n\left(\vec{x}_0 \cdot \vec{w}_j\right) \vec{w}_j)\\
			&= A\vec{x}_0 - A \sum_{j=s+1}^n\left(\vec{x}_0 \cdot \vec{w}_j\right) \vec{w}_j\\
			&=  A\vec{x}_0 - \sum_{j=s+1}^n\left(\vec{x}_0 \cdot \vec{w}_j\right) A \vec{w}_j\\
			&= A \vec{x}_0 - 0= \vec{b}_{proj}.
		\end{align*}
		\item Show that $\left\|\vec{x}_1\right\| \leq\left\|\vec{x}_0\right\|$, and there is equality if and only if $\vec{x}_0$ is in the span of the rows of $A$\\
		Since $\vec{x}_0-\sum_{j=s+1}^n\left(\mathbf{x}_0 \cdot \vec{w}_j\right) \vec{w}_j=\sum_{j=1}^s\left(\vec{x}_0 \cdot \vec{w}_j\right) \vec{w}_j $, then $x_0$ can be expressed in terms of parallel and orthogonal components with respect to $A$: $\vec{x}_0= \sum_{j=s+1}^n\left(\mathbf{x}_0 \cdot \vec{w}_j\right) \vec{w}_j +\sum_{j=1}^s\left(\vec{x}_0 \cdot \vec{w}_j\right) \vec{w}_j $.  Therefore by the Pythagorean theorem ,  $\|\vec{x}_0\|^2= \|\sum_{j=s+1}^n\left(\mathbf{x}_0 \cdot \vec{w}_j\right) \vec{w}_j \|^2 +\|\sum_{j=1}^s\left(\vec{x}_0 \cdot \vec{w}_j\right) \vec{w}_j \|^2 = \|\sum_{j=s+1}^n\left(\vec{x}_0 \cdot \vec{w}_j\right) \vec{w}_j \|^2 + \|\vec{x}_1\|^2 \geq \|\vec{x}_1\|^2$.  Since the norm of a vector has a range of $\R_{\geq 0},$ and $x^2$ is a monotonically increasing function on that set, then $\|\vec{x}_1\| \leq \|\vec{x}_0\|$.  In addition, if $\vec{x}_0$ is in the span of the rows of $A$, then $\sum_{j=s+1}^n\left(\vec{x}_0 \cdot \vec{w}_j\right) \vec{w}_j = 0,$ and thus $\|\vec{x}_0\| = \|\vec{x}_1\|.$  If we assume $\|\vec{x}_1\| = \|\vec{x}_0\|,$ then by definition of $\vec{x}_1,$ $\|\vec{x}_0\| = \| \sum_{j=1}^s\left(\vec{x}_0 \cdot \vec{w}_j\right) \vec{w}_j \| = \|QQ^T \vec{x}_0\|.$  Therefore since the magnitude of $\vec{x}_0$ is equivalent to it's magnitude projected into $A$, then it is in the span of the rows of $A$.
		\end{itemize}
		\item 
			\begin{enumerate}
				\item $\vec{b} = (1,2,3,4), \|\vec{b} - \vec{b}_{proj}\| = 1, $ unique solution $\vec{x}$ to $A \vec{x} = \vec{b}_{proj}, \vec{x} = (2,\frac{-5}{2},0,\frac{3}{2})$. 
				\item $\vec{b} = (1,1,1,-1), \|\vec{b} - \vec{b}_{proj}\| = 2, $ unique solution $\vec{x}$ to $A \vec{x} = \vec{b}_{proj}, \vec{x} = (0,0,0,0)$.
				\item $\vec{b} = (1,1,-1,1), \|\vec{b} - \vec{b}_{proj}\| = 0, $ unique solution $\vec{x}$ to $A \vec{x} = \vec{b}_{proj}, \vec{x} = (1,0,0,0)$. 
			\end{enumerate}
		\item 
			\begin{itemize}
				\item Show that as long as all of the $\vec{x}_j$ are not the same, then $rank(A) = 2$
and the 2 × 2 matrix $A^T A$ is invertible.  By definition of invertible, we must show that $det(A^T A) \neq 0.$  Since $rank(A) = 2,$ then when we perform gram schmidt we will get $\{\hat{u}_1, \hat{u}_2\} = Span\{A\}.$  Therefore let $Q = \{\hat{u}_1, \hat{u}_2\}.$  Then by QR factorization $A^T A = R^T Q^T Q R = R^T R.$  By definition of $R$, \[R = \begin{bmatrix}
Col_1 (A)  \cdot  \hat{u}_1 & \cdots & Col_n (A) \cdot \hat{u}_1 \\
Col_1 (A)  \cdot  \hat{u}_2 & \cdots & Col_n (A) \cdot \hat{u}_2 
\end{bmatrix}.\]
	Therefore \[R^T R = \begin{bmatrix}
	\sum^n_{j=1}(Col_j (A)  \cdot  \hat{u}_1)^2 & \sum^n_{j=1}(Col_j (A)  \cdot  \hat{u}_1)(Col_j (A)  \cdot  \hat{u}_2)\\
	\sum^n_{j=1}(Col_j (A)  \cdot  \hat{u}_1)(Col_j (A)  \cdot  \hat{u}_2) & \sum^n_{j=1}(Col_j (A)  \cdot  \hat{u}_2)^2
	\end{bmatrix} \]\\	
	\[
	=\begin{bmatrix}
		\|A^T \hat{u}_1\|^2 & (A^T \hat{u}_1) \cdot (A^T \hat{u}_2)\\
		(A^T \hat{u}_1) \cdot (A^T \hat{u}_2) & \|A^T \hat{u}_2\|^2 
	\end{bmatrix} .\]\\
	Therefore $det(A^T A) = (\|A^T \hat{u}_1\|\|A^T \hat{u}_2\| )^2 -((A^T \hat{u}_1) \cdot (A^T \hat{u}_2))^2 = (\|A^T \hat{u}_1\|\|A^T \hat{u}_2\| + (A^T \hat{u}_1) \cdot (A^T \hat{u}_2))(\|A^T \hat{u}_1\|\|A^T \hat{u}_2\| - (A^T \hat{u}_1) \cdot (A^T \hat{u}_2)) $.  The only way for this determinate to be 0 is if $A^T \hat{u}_1 = A^T \hat{u}_2$.  However this implies that the first row of $R$ is equal to the second.  In that case $R$ would have a rank of 1, which would contradict the fact that $rank(A) = 2.$  Therefore $det(A) \neq 0,$ and thus $A^T A$ has an inverse.
	\item Show for all $\vec{y} \in \R^m, \exists \vec{x} \in \R^n $ such that for all $\vec{z} \in \R^n, \|A\vec{x} - \vec{y}\| \leq \|A\vec{z} - \vec{y}\|$.\\
	Since $\vec{y} \in \R^m,$ then $\vec{y}$ decomposes into parts that are parallel and orthogonal to $A: \vec{y} = \vec{y}_\perp + \vec{y}_\parallel.$  The portion which is parallel to $A$, $\vec{y}_\parallel,$ by definition can be written as the product of a vector $\vec{x}_0$ in $\R^n$ and $A: \vec{y}_\parallel = A \vec{x}_0.$  Therefore by taking the decomposition of $\vec{y}$ and the formula found for the parallel portion, we can solve for the orthogonal part: $\vec{y} = \vec{y}_\perp + \vec{y}_\parallel = \vec{y}_\perp + A\vec{x}_0$, $\vec{y} - A \vec{x}_0 = \vec{y}_\perp.$  Therefore $\vec{y} - A\vec{z} = \vec{y}_\perp + \vec{y}_\parallel - A \vec{z}.$  Since by definition $\vec{y}_\perp$ is orthogonal to $A,$ and $A\vec{z}$ lies explicitly within $A$, then by the pythagorean theorem $\|\vec{y} - A\vec{z}\|^2 = \|\vec{y}_\perp\|^2 + \|\vec{y}_\parallel - A \vec{z}\|^2.$  Therefore to minimize the above equation, we must choose $\vec{x} = \vec{x_0}$ so that $\vec{y}_\parallel - A \vec{z} = \vec{0}.$  
	\item Show that $A\vec{x} - \vec{y} \perp Col_j(A),$ for $ j = 1,\cdots, n.$\\
	Since it's established that $A\vec{x} - \vec{y}$ is orthogonal to the columns of $A$, then by definition is is orthogonal to each individual columns $Col_j (A)$ 
	\item Finally show that $(A^T A)^{-1}A^T \vec{y}$ is the unique least squares solution of $A\vec{x} = \vec{y}.$\\
	Suppose there exists $\vec{x}^*$ such that $A\vec{x}^* = \vec{y}$ in addition to $\vec{x} = (A^T A)^{-1}A^T \vec{y}.$  We must show that $\vec{x}^* = (A^T A)^{-1}A^T \vec{y}$.  Since both $A\vec{x}^* = \vec{y}, A \vec{x} = \vec{y},$ then $A\vec{x}^* = A\vec{x}.$  By definition of $\vec{x}$, $A\vec{x}^* = A(A^T A)^{-1}A^T \vec{y}$.  Multiplying both sides by $A^T$ yields $A^T A \vec{x}^* = A^T A(A^T A)^{-1}A^T \vec{y}$.  By the associativity of matrix multiplication and the definition of inverse we have, $A^T A \vec{x}^* = A^T \vec{y}.$  Since $A^T A$ has been shown to have an inverse, multiplying both sides by $(A^T A)^{-1}$ yields $\vec{x}^* = (A^T A)^{-1}A^T \vec{y}.$
			\end{itemize}
		\item $y= 1.3369x-1.13274$
	\end{enumerate}
\end{document}