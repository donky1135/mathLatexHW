\documentclass[12pt, letterpaper]{article}
\date{\today}
\usepackage[margin=1in]{geometry}
\usepackage{amsmath}
\usepackage{hyperref}
\usepackage{systeme}
\usepackage{amssymb}
\usepackage{fancyhdr}
\usepackage{pgfplots}
\usepackage{booktabs}
\usepackage{pifont}
\usepackage{amsthm,latexsym,amsfonts,graphicx,epsfig,comment}
\pgfplotsset{compat=1.16}
\usepackage{xcolor}
\usepackage{tikz}
\usetikzlibrary{shapes.geometric}
\usetikzlibrary{arrows.meta,arrows}
\newcommand{\Z}{\mathbb{Z}}
\newcommand{\N}{\mathbb{N}}
\newcommand{\R}{\mathbb{R}}
\newcommand{\Po}{\mathcal{P}}

\author{Alex Valentino}
\title{291 Initial quiz}
\pagestyle{fancy}
\renewcommand{\headrulewidth}{0pt}
\renewcommand{\footrulewidth}{0pt}
\fancyhf{}
\rhead{
	291H\\	
	Alex Valentino \\
	Initial quiz 
}
\begin{document}
Survey:
\begin{enumerate}
	\item \textit{What's the highest level of math course have you taken? When was the last math course you took?}\\
	I took the Rutgers-Newark equivalent of Math 300. I also went through a semester of calculus taught out of Spivak.  
	\item \textit{Have you taken any multi-variable calculus? Are you proficient at working with equations of a plane and a straight-line? Do you feel
confident working with vector projection?}\\
I haven't taken any multi-variable classes in highschool.
I did do a semester of E\&M in highschool where I learned Maxwell's equations.  
I feel I am proficient at working with plane equations and lines.
I am iffy on vector projection, but I am currently refreshing with a linear algebra textbook.
\item \textit{What's your main motivation for taking this course? In other words, what are you hoping to get out of this course?\\
I love math and physics, and to gain a better understanding of either I see that this course is the best one I can take now.}  
\item \textit{What issues do you anticipate in transitioning into learning math at the college level?}\\
I have already done a year at college, so this question isn't exactly geared toward me.  But, I fully understand the time commitment needed to master the subject.  Math is a skill, and to get good one must practice, a lot.  I, for one, can't wait to embrace the challange that 291 will provide.
\item \textit{Should there be a need to combine the two recitation sessions into one (either occasionally or permanently), using the Th5:40–7:00pm period, would this suitable to your schedule?}\\
I don't think so.  By spacing it out I think that leads to better absorption of the material and less wandering eyes during lecture.  

\end{enumerate}

Quiz:
\begin{enumerate}
	\item \textit{(True or False) A straight-line in the three dimensional space is described by
a linear equation of the form $ax+by+cz = d$ in terms of the three rectangular
coordinates $x, y, z$ of points in the three dimensional space.}
	True.  A common form for a line in $\R^2$ is $ax+by=d$, if one was going to fix $x$ and $y$ and add a factor of $cz$ to the left hand side and let $c$ and $d$ vary, it would clearly just be linear in a new plain.  
	\item \textit{(True or False) Given two linear equations $ax+by+cz = d$ and $a'x+b'y+c'z =
d'$in the three variables x, y, z. Then the set of joint solutions to the system \[ \systeme*{ax+by+cz = d, a'x+b'y+c'z =
	d'} \] is either empty or consists of a unique solution. }\\
	False.  If $a=a', b=b', c=c', d=d'$ then there would be an infinite number of solutions to the system of linear equations above. 
	\item \textit{Which of the following does not parametrize a straight-line or some portion
of a line? Briefly explain your answer.}
	\begin{enumerate}
		\item $r(t) = \langle 2+3t, 9-t, 12+7t \rangle$\\
		Since all of the terms are linear, the function in terms of $x,y,z$ will also be linear.
		\item $r(t) = \langle1 - t^2, 3+3t^2, t^3 \rangle$\\
		The $t^3$ term will be "faster" than the $t^2$, thus it won't exhibit linear behavior.
		\item $r(t) = \langle2cos^2(t), 5+3cos^2(t), sin^2(t)\rangle$\\
		Since $sin^2(t) = 1-cos^2(t),$ the equation above is all in terms of $cos^2(t).$  Therefore since all the derivatives are just going to scalar multiples of each other this function, too will exhibit linear behavior within the domain restriction of.
		\item $r(t) = \langle t^3, 4-8t^3, 8+3t^3\rangle$ \\
		Much as the problem before was all in terms of $cos^2(t)$, here it is all in terms of $t^3.$  Better yet since $t^3$ is bijective over $\R$ this function will be a line in $\R^3.$  
	\end{enumerate}
	\item \textit{Determine the value(s) of the parameter $r$ in the following system of two
linear equations in the two variables $x$ and $y$	
\begin{align*}
	x + 2y &= 3\\
	(r^2 - 3r +2)y &= r-1
\end{align*}  } \textit{such that it has }
	\begin{enumerate}
		\item exactly one solution;\\
		Let $r \in \R, r \neq 1,2$.  This will make the system of linear equations linearly independent, as the two occurrences of the system being linearly dependent are below.  
		\item no solution;\\  
		Let $r=2$.  Then the polynomial in the second row evaluates to $0$, and $r-1=1$.  However, it is impossible for $0x+0y=1.$  Therefore for $r=2$ there are no solutions to this equation.  
		\item infinitely many solutions -- try to provide a formula for the general solution in this case.\\
		Let $r=1$, this will result in the 2nd equation will go to zero on both sides.  Therefore the set of all solutions is $\{(x,\frac{3-x}{2}):x \in \R \}$.
	\end{enumerate}
	\item \textit{Is it possible to construct a function $f(x)$ which is continuous on $[0,1]$, differentiable in $(0,1)$, such that $|f(x)| \leq \frac{1}{2}$ for all $x$ in $(0,1),f(0) = 0, and f(1) = -1$?  If your answer is positive, provide an example of such a function;
if your answer is negative, explain why it can't be done. }\\
It is impossible.  By the intermediate value theorem there must exist a $c \in (0,1)$ where $f'(c) = f(1)-f(0) = -1.$  This is a contradiction as $|f(x)| \leq \frac{1}{2}$ but $|-1| > \frac{1}{2}.$  Therefore no function can exist.
\end{enumerate}


\end{document}