\documentclass[12pt, letterpaper]{article}
\date{\today}
\usepackage[margin=1in]{geometry}
\usepackage{amsmath}
\usepackage{hyperref}
\usepackage{cancel}
\usepackage{amssymb}
\usepackage{fancyhdr}
\usepackage{pgfplots}
\usepackage{booktabs}
\usepackage{pifont}
\usepackage{amsthm,latexsym,amsfonts,graphicx,epsfig,comment}
\pgfplotsset{compat=1.16}
\usepackage{xcolor}
\usepackage{tikz}
\usetikzlibrary{shapes.geometric}
\usetikzlibrary{arrows.meta,arrows}
\newcommand{\Z}{\mathbb{Z}}
\newcommand{\N}{\mathbb{N}}
\newcommand{\R}{\mathbb{R}}
\newcommand{\Po}{\mathcal{P}}

\author{Alex Valentino}
\title{PSS3 - 3/21/2023}
\pagestyle{fancy}
\renewcommand{\headrulewidth}{0pt}
\renewcommand{\footrulewidth}{0pt}
\fancyhf{}
\rhead{
	Homework \\
	350H	
}
\lhead{
	Alex Valentino\\
}
\begin{document}
	Let $A \in M_{n\times n} F$ and let $B$ be the matrix obtained from $A$ by adding a scalar multiple of one row to the other, then $det(A) = det(B)$.
	Proof:\\
	Let $A = \begin{bmatrix}
	a_1 \\
	\vdots\\
	a_i\\
	\vdots\\
	a_j\\
	\vdots\\
	a_n
	\end{bmatrix}$ and $B = \begin{bmatrix}
	a_1 \\
	\vdots\\
	a_i\\
	\vdots\\
	ca_i + a_j\\
	\vdots\\
	a_n
	\end{bmatrix}$.  Therefore 
	$$
	det(B) = det(\begin{bmatrix}
	a_1 \\
	\vdots\\
	a_i\\
	\vdots\\
	ca_i + a_j\\
	\vdots\\
	a_n
	\end{bmatrix})
	 = det(\begin{bmatrix}
	a_1 \\
	\vdots\\
	a_i\\
	\vdots\\
	a_j\\
	\vdots\\
	a_n
	\end{bmatrix}) + c \cdot det(\begin{bmatrix}
	a_1 \\
	\vdots\\
	a_i\\
	\vdots\\
	a_i\\
	\vdots\\
	a_n
	\end{bmatrix})
	$$
	Since a determinant of a matrix with identical rows is 0, then we have that $$det(B) = det(\begin{bmatrix}
	a_1 \\
	\vdots\\
	a_i\\
	\vdots\\
	a_j\\
	\vdots\\
	a_n
	\end{bmatrix}) = det(A)$$.
\end{document}