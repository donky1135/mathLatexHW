\documentclass[12pt, letterpaper]{article}
\date{\today}
\usepackage[margin=1in]{geometry}
\usepackage{amsmath}
\usepackage{hyperref}
\usepackage{cancel}
\usepackage{amssymb}
\usepackage{fancyhdr}
\usepackage{pgfplots}
\usepackage{booktabs}
\usepackage{pifont}
\usepackage{bbm}
\usepackage{amsthm,latexsym,amsfonts,graphicx,epsfig,comment}
\pgfplotsset{compat=1.16}
\usepackage{xcolor}
\usepackage{tikz}
\usetikzlibrary{shapes.geometric}
\usetikzlibrary{arrows.meta,arrows}
\newcommand{\Z}{\mathbb{Z}}
\newcommand{\N}{\mathbb{N}}
\newcommand{\R}{\mathbb{R}}
\newcommand{\Po}{\mathcal{P}}
\usepackage[bb=boondox]{mathalfa}
\author{Alex Valentino}
\title{Homework 1}
\pagestyle{fancy}
\renewcommand{\headrulewidth}{0pt}
\renewcommand{\footrulewidth}{0pt}
\fancyhf{}
\rhead{
	Homework 1\\
	350H	
}
\lhead{
	Alex Valentino\\
}
\begin{document}
	


	\begin{enumerate}
		\item Let $Tri_{m\times n} (F)$ be the set of  all upper triangular $m\times n$ matrices.  Prove that $Tri_{m\times n} (F)$ is a subspace of $M_{m\times n} (F)$.
		\begin{itemize}
			\item Let $\mathbb{0}$ denote the 0 $m\times n$ matrix.  We must show that $\mathbb{0} \in Tri_{m\times n} (F)$.  By definition of $\mathbb{0} \in M_{m\times n} (F)$, $\mathbb{0}_{ij} = 0$, for all $i \in [m], j \in [n].$  By definition of being an upper triangular matrix, for a matrix $A$, $A_{ij} = 0$ whenever $i > j$.  Since all entries in $\mathbb{0}_{ij} = 0,$ then it satisfies the requirements of $i > j.$  Thus $\mathbb{0} \in  Tri_{m\times n} (F)$
			\item Suppose $X,Y \in Tri_{m\times n}(F).$  We must show $X+Y \in Tri_{m\times n}(F)$.  Since $X,Y \in Tri_{m\times n}(F)$, then $X_{ij} = 0, Y_{ij} = 0$ whenever $i > j$.  Therefore $(X+Y)_{ij} = X_{ij} + Y_{ij} = 0 + 0 = 0$ whenever $i > j$.  Thus $X+Y \in Tri_{m\times n}(F)$.
			\item Suppose $X \in Tri_{m\times n}(F), c \in F.$  We must show $cX \in Tri_{m\times n}(F)$.  Therefore by definition of being an upper triangular matrix, $c\cdot A_{ij} = c\cdot 0 = 0$ whenever $i > j$.  Therefore $cA \in Tri_{m\times n}(F)$. 
		\end{itemize}
		Therefore $Tri_{m\times n}(F)$ is a subspace of $M_{m\times n}(F)$.
		\newpage
		\item Let $S$ be a non-empty set, and $F$ a field.  Let $C(S,F)$ denote the set of all functions $f \in \mathcal{F}(S,F)$ such that $f(s) = 0$ for all but a finite number of elements in $S$.  Prove that $C(S,F)$ is a subspace of $\mathcal{F}(S,F)$.  
		\begin{itemize}
			\item Let $\mathbb{0}: S \to S$ denote the function which is 0 for all elements of $S$.  We must show $\mathbb{0} \in C(S,F)$.  Since the set of elements for which $\mathbb{0} \neq 0$ is empty, and $|\emptyset| = 0$, then the non-zero element set is finite.  Therefore $\mathbb{0} \in C(S,F)$.
			\item Let $f,g \in C(S,F),$ and let $A,B \subset S, |A| = n, |B| = m, n,m \in \N$ denote the sets for which $f$ and $g$ respectively are non-zero.  We must show $f+g \in C(S,F)$. Since $f+g \in \mathcal{F}(S,F)$, then there exists a set $D$ such that $S \subseteq D, \forall x \in D (f+g)(x) \neq 0$.  We must show that $D$ is finite.  We claim that $D \subseteq A \cup B$.  Suppose $x \in D$.  We have three cases. 
			\begin{enumerate}
				\item Suppose $f(x) \neq 0, g(x) = 0$.  Therefore $(f+g)(x) = f(x) + g(x) = f(x) + 0 = f(x) \neq 0$.  Since $f(x) \neq 0, x \in A$
				\item Suppose $f(x) = 0, g(x) \neq 0$.
				Therefore $(f+g)(x) = f(x) + g(x) = 0 +g(x) = g(x) \neq 0$.  Since $g(x) \neq 0, x \in B$.
				\item Suppose $f(x) \neq 0, g(x) \neq 0$.  Assume $f(x) \neq -g(x)$.  Since both $f,g$ are non-zero, and not inverses of each other, then $(f+g)(x) \neq 0$.  Since both $f,g$ are non-zero, then $x\in A$ and $x \in B$.
				
				  Assume $f(x) = -g(x)$.  Then $f(x) + g(x) = 0$.  This is a contradiction, as $x \in D$.
			\end{enumerate}
			Since for all possible $x \in D, x \in A \cup B$, then we have shown the claim.  Since $|D| \leq |A\cup B|$, $|A\cup B| \leq n + m$, and $n +m \in \N$, then $|D| \leq n+m$.  Thus $D$ is finite.  Therefore $f+g \in C(S,F)$.
			\item Let $f \in C(S,F), c \in F.$  We must show that $c f \in C(S,F).$  If $c = 0$, then $c f = \mathbb{0},$ which $\mathbb{0} \in C(S,F)$.  Suppose $c \neq 0$.  Since $f \in C(S,F)$, then there exists a set $A \subset S, |A| = n, n \in \N$ such that for all $s\in A, f(s) \neq 0.$  We claim that $A$ is the same set of non-zero points for $c f$.  Suppose $x \in S$, we have two cases:
			\begin{itemize}
				\item Suppose $x \in A.$  Since $f(x) \neq 0, c \neq 0,$ then $c f(x) \neq 0$.
				\item Suppose $x  \not \in A.$  Since $f(x) = 0,$ then $c f(x) = c 0 = 0.$    
			\end{itemize}
			Since $A$ is the set of elements in $S$ for which $cf$ is non-zero, and $A$ is finite, then $cf \in C(S,F).$	       
		\end{itemize}
		Therefore $C(S,F)$ is a subspace of $\mathcal{F}(S,F).$
		\newpage
		
Lemma 1:  Suppose $W$ is a subspace of the vector space $V$, $\Vec{x},\Vec{y} \in V, \Vec{x} \in W, \Vec{y} \not \in W$.  We must show $\Vec{x} + \Vec{y} \not \in W$.  Suppose for contradiction that $\Vec{x} + \Vec{y} \in W$.  Since $-\Vec{x} \in W$, then $\Vec{x} - \Vec{x} + \Vec{y} \in W$.  Therefore $\Vec{y} \in W$.  This is a contradiction.  Therefore $\Vec{x} + \Vec{y} \not \in W$.		
		
		
		\item Let $W_1, W_2$ be subspaces of the vector space $V$.  We must show that $W_1 \cup W_2$ is a subspace $V$ if and only if $W_1 \subset W_2$ or $W_2 \subset W_1$.
		\begin{itemize}
			\item ($\Rightarrow$)  Suppose $W_1 \cup W_2$ is a subspace of $V$.  We must show  $W_1 \subseteq W_2$ or $W_2 \subseteq W_1$.  Suppose for contradiction that $W_1 \nsubseteq W_2$ and $W_2 \nsubseteq W_1$.  By definition of $\nsubseteq$, there exists vectors $\Vec{x}, \Vec{y}$ such that $\Vec{x} \in W_1, \Vec{x} \not \in W_2,\Vec{y} \in W_2, \Vec{y} \not \in W_1$.  Since $\Vec{x},\Vec{y} \in W_1 \cup W_2$, then $\Vec{x} + \Vec{y} \in W_1 \cup W_2$ by the closure property for subspaces. Since $\Vec{x} \in W_1, \Vec{y} \not \in W_1$, then $\Vec{x} + \Vec{y} \not \in W_1$   by lemma 1.  Since $\Vec{y} \in W_2, \Vec{x} \not \in W_2$, then $\Vec{x} + \Vec{y} \not \in W_2$ by lemma 1.  Therefore by definition of union $\Vec{x} + \Vec{y} \not \in W_1 \cup W_2$.  This is a contradiction.  Therefore $W_1 \subseteq W_2$ or $W_1 \subseteq W_2$.  
			\item ($\Leftarrow$)  Suppose $W_1 \subseteq W_2$ or $W_2 \subseteq W_1$.  We must show $W_1 \cup W_2$ is a subspace of $V$.  We have two cases.  Suppose  $W_1 \subseteq W_2$.  Then by definition $W_1 \cup W_2 = W_2$.  Since $W_2$ is a vector space of $V$, then the requirements have been satisfied.  The proof is nearly identical for the $W_2 \subseteq W_1$ case.  
		\end{itemize}
		\newpage		  
		\item Show that $P_n(F)$ is generated by $\{1,x,\cdots, x^n\}$.  Suppose $f \in P_n(F)$.  We must show that $f \in Span(\{1,x,\cdots, x^n\})$.  By definition of being a member of $P_n(F)$, $f = a_0 \cdot 1 + a_1 \cdot x + \ldots + a_n \cdot x^n,$ where $a_0,\cdots,a_n \in F$.  Since $a_0, \cdots, a_n \in F,$ and each of those is multiplied by an element in the generating set, then $f \in Span(\{1,x,\cdots, x^n\})$.
		\newpage    




		\item Let $V$ be a vector space, $W \subseteq V$.  We must show that $W \leq V$ if and only if $Span(W) = W$.  
		\begin{itemize}
			\item $(\Rightarrow)$ Suppose $W \leq V$.  We must show $Span(W) = W$.  Therefore we must show $Span(W) \subseteq W, Span(W) \supseteq W$.
			\begin{itemize}
				\item $(\subseteq)$ Suppose $\Vec{x} \in Span(W)$.  We must show $\Vec{x} \in W$.  By definition of being a member of $Span(W),$ there exists vectors $\Vec{w_1},\ldots,\Vec{w_n} \in W, c_1,\ldots c_n \in F$ such that $\Vec{x} = \sum^n_{i=1} c_i \Vec{w_i}$.  By induction, $W$ is closed under successive applications of closure under scalar multiplication and vector addition since $W$ is a subspace.  Therefore $\Vec{x} \in W.$
				\item $(\supseteq)$ Suppose $\Vec{x} \in W$.  We must show $\Vec{x} \in Span(W)$. By the identity property of vector spaces, $\Vec{x} = 1 \cdot \Vec{x}$.  Since $\Vec{x}$ is a scalar multiple of a vector in $W$, then $\Vec{x} \in Span(W)$. 
			\end{itemize}
			\item $(\Leftarrow)$ Suppose $Span(W) = W$.  We must show that $W \leq V$.  
			\begin{itemize}
				\item We must show $\Vec{0} \in W.$  Suppose $\Vec{w} \in W.$  Since $W = Span(W)$, and $0 \cdot \Vec{w} \in Span(W),$ then $0\cdot \Vec{w} = \Vec{0} \in W$.
				\item Suppose $\Vec{x}, \Vec{y} \in W.$  We must show $\Vec{x} + \Vec{y} \in W.$  Since $\Vec{x} + \Vec{y} = 1 \cdot \Vec{x} + 1 \cdot \Vec{y}$, and these are scalar multiples of vectors we know to exists in $W$ which are summed together, then $\Vec{x} + \Vec{y} \in Span(W)$.  Therefore $\Vec{x} + \Vec{y} \in W$.
				\item Suppose $\Vec{x} \in W, c \in F$.  We must show $c \Vec{x} \in W$.  Since $c \Vec{x}$ is a scalar multiple of a vector we know to be within $W$, then $c \Vec{x} \in Span(W)$. Thus $c \Vec{x} \in W$.
			\end{itemize}					   
		\end{itemize}
	\end{enumerate}
\end{document}