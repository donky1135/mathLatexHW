\documentclass[12pt, letterpaper]{article}
\date{\today}
\usepackage[margin=1in]{geometry}
\usepackage{amsmath}
\usepackage{hyperref}
\usepackage{cancel}
\usepackage{amssymb}
\usepackage{fancyhdr}
\usepackage{pgfplots}
\usepackage{booktabs}
\usepackage{pifont}
\usepackage{amsthm,latexsym,amsfonts,graphicx,epsfig,comment}
\pgfplotsset{compat=1.16}
\usepackage{xcolor}
\usepackage{tikz}
\usetikzlibrary{shapes.geometric}
\usetikzlibrary{arrows.meta,arrows}
\newcommand{\Z}{\mathbb{Z}}
\newcommand{\N}{\mathbb{N}}
\newcommand{\R}{\mathbb{R}}
\newcommand{\Po}{\mathcal{P}}

\author{Alex Valentino}
\title{Homework }
\pagestyle{fancy}
\renewcommand{\headrulewidth}{0pt}
\renewcommand{\footrulewidth}{0pt}
\fancyhf{}
\rhead{
	Challange Problem Set 1 \\
	292	
}
\lhead{
	Alex Valentino\\
}
\begin{document}
	\begin{enumerate}
		\item 
		\begin{itemize}
			\item Show that $x(t)$ satisifes $\| A^{-1} x(t)\|^2 = 1$.
			\begin{align*}
				\| A^{-1} x(t)\|^2 &= \| A^{-1} A u(t)\|^2\\
				&= \| u(t)\|^2\\
				&= \| \begin{bmatrix}
				\cos(t)\\
				\sin(t)
				\end{bmatrix}\|^2\\
				&= \cos^2(t) + \sin^2(t)\\
				&= 1.
			\end{align*}
			\item Show that the equation above may be written as $x \cdot M x = 1$.
			\begin{align*}
				x \cdot M x &= x^T M x\\
				&= x^T (A^{-1})^T A^{-1} x\\
				&= (A^{-1}x)^T A^{-1}x\\
				&= A^{-1}x \cdot A^{-1}x\\
				&= \|A^{-1}x\|^2 \\
				&= 1.
			\end{align*}
			\item Show that $M$ is symmetric.
			\begin{align*}
			M^T &= ((A^{-1})^T A^{-1})^T\\
			&= (A^{-1})^T ((A^{-1})^T)^T\\
			&= (A^{-1})^T A^{-1}\\
			&= M
			\end{align*}
			\item Suppose $M = \begin{bmatrix}
			a & c \\ c & b
\end{bmatrix}			 $.  We must show that $x \cdot M x$ can be written as $ax^2 + by^2 + 2cxy$.
			\begin{align*}
				1 &= x \cdot M x\\
				&= \begin{bmatrix}
				x & y
				\end{bmatrix}\begin{bmatrix}
			a & c \\ c & b
\end{bmatrix}\begin{bmatrix}
				x \\ y
				\end{bmatrix}\\
				&= \begin{bmatrix}
				x & y
				\end{bmatrix}\begin{bmatrix}
			ax + cy \\ cx + by
\end{bmatrix}\\
	&= ax^2 + by^2 + 2cxy.
			\end{align*}
		\end{itemize}
		\item 
		\begin{itemize}
			\item Show that both $\lambda_1$ and $\lambda_2$ are positive.  Since $x \cdot M x = \|A^{-1}x\|^2$, then all outputs of $x \cdot M x$ are strictly positive.  Suppose $x = u_1$, then $u_1 \cdot M u_1 = u_1 \cdot \lambda_1 u_1 = \lambda_1 \|u_1\|^2 = \lambda_1 > 0$.  A similar proof exists for $u_2$.  Therefore the eigenvalues are strictly positive.   
			\item Suppose $\lambda_1 = \lambda_2$.  We must show that $\| x(t) \| = \frac{1}{\sqrt{\lambda_1}}$.  Let $V$ be the matrix where $u_1$ and $u_2$ are columns.  Since $u_1, u_2$ are orthonormal, then $V$ is an orthogonal matrix.  Therefore $V^{-1} = V^T$.  Let $D = \begin{bmatrix}
			\lambda_1 & 0\\
			0 & \lambda_2\\
\end{bmatrix}			 $, and let $q = \begin{bmatrix}
q_1(t)\\q_2(t)
\end{bmatrix} = V^T x$.  Therefore we may rewrite $x \cdot M x$ as follows: 
			$$
				1 = x \cdot M x = q \cdot D q = \lambda_1 q_1^2(t) + \lambda_2 q_2^2(t).  			
			$$
			This equation defines an ellipse paramaterized by $q_1(t) = \pm \frac{1}{\lambda_1}\cos(t), q_2(t) = \pm \frac{1}{\lambda_2}\sin(t)$.  Since $q = V^T x$, then we may explicitly solve for $x$ via $x = Vq = \pm (\frac{1}{\lambda_1}\cos(t)u_1 + \frac{1}{\lambda_2}\sin(t)u_2)$.  Therefore if $\lambda_1 = \lambda_2$, then $\|x\| = \sqrt{\frac{\cos^2(t)}{\lambda_1} + \frac{\sin^2(t)}{\lambda_2} } = \sqrt{\frac{\cos^2(t)}{\lambda_1} + \frac{\sin^2(t)}{\lambda_1} } = \frac{1}{\sqrt{\lambda_1}}$.
			\item Note that $\lambda_1 > \lambda_2$
			\begin{itemize}
				\item Suppose $x(t) = \pm \frac{1}{\lambda_2} u_2$.  We must show that $\|x(t)\|$ is maximal.  
				$$
				\|x(t)\| = \sqrt{\frac{\cos^2(t)}{\lambda_1} + \frac{\sin^2(t)}{\lambda_2} } \leq  \sqrt{\frac{\cos^2(t)}{\lambda_2} + \frac{\sin^2(t)}{\lambda_2} } = \frac{1}{\sqrt{\lambda_2}} = \|\pm \frac{1}{\sqrt{\lambda_2}} u_2\|			
				$$
				\item Suppose $\|x(t)\|$ is maximal.  We must show that $x(t) = \pm \frac{1}{\lambda_2} u_2$.  Since $x^2$ is strictly increasing on $\R_+$ then $\|x(t)\|^2$ is maximal.  This is equivalent to $\begin{bmatrix}
			\frac{\lambda_1} & 0\\ 0 & \lambda_2	
\end{bmatrix}				 $
			\end{itemize}
		\end{itemize}
	\end{enumerate}
\end{document}