\documentclass[12pt, letterpaper]{article}
\date{\today}
\usepackage[margin=1in]{geometry}
\usepackage{amsmath}
\usepackage{hyperref}
\usepackage{cancel}
\usepackage{amssymb}
\usepackage{fancyhdr}
\usepackage{pgfplots}
\usepackage{booktabs}
\usepackage{pifont}
\usepackage{amsthm,latexsym,amsfonts,graphicx,epsfig,comment}
\pgfplotsset{compat=1.16}
\usepackage{xcolor}
\usepackage{tikz}
\usetikzlibrary{shapes.geometric}
\usetikzlibrary{arrows.meta,arrows}
\newcommand{\Z}{\mathbb{Z}}
\newcommand{\N}{\mathbb{N}}
\newcommand{\R}{\mathbb{R}}
\newcommand{\Po}{\mathcal{P}}

\author{Alex Valentino}
\title{Homework 7}
\pagestyle{fancy}
\renewcommand{\headrulewidth}{0pt}
\renewcommand{\footrulewidth}{0pt}
\fancyhf{}
\rhead{
	Homework 7\\
	350H	
}
\lhead{
	Alex Valentino\\
}
\begin{document}
\begin{enumerate}
	\item
	\newpage
	 \item Suppose $A \in M_{m \times n} (F), rank(A) = m$.  We must show that there exists $B \in M_{n \times m} (F)$ such that $\mathbb{I}_m = AB$.\\
	 Proof: We know from the first corollary of theorem 3.6 that there exists $L \in GL_m (F), R \in GL_n (F)$ such that $L A R = 
	 \begin{bmatrix} \mathbb{I}_r & O_1 \\ O_2 & O_3\end{bmatrix}$ where $r = rank(A)$ and $O_1,O_2,O_3$ are zero matrices.  Since $rank(A) = m$, and the matrix $LAR$ is $m \times n$ then $LAR = \left[ \begin{array}{c|c} \mathbb{I} & O \end{array} \right]$ where $O$ is a $m \times (n-m)$ 0 matrix. 
Therefore left multplying by $L^{-1}$ yields $AR = \left[ \begin{array}{c|c} L^{-1} & O \end{array} \right]$. 
Let $L' \in M_{n \times m}$ be the matrix given by for all $i \in [n], j \in [m], (L')_{ij} = L_{ij}$ if $j \leq n$ otherwise $(L')_{ij} = 0$.  We claim that $RL' = B$.  Since $L' \in M_{n \times m}$ and $R \in GL_n (F)$ then $RL' \in M_{n \times m}$.  Therefore $AB = ARL' =\left[ \begin{array}{c|c} L^{-1} & O \end{array} \right] L'$.  Note that by the definition of matrix multiplication and the identity matrix $\delta_{ij} = \sum_{k = 1}^m (L^{-1})_{ik} L_{kj}$.  Therefore each entry in the new matrix D is given by $ D_{ij} = \sum_{k = 1}^n \left[ \begin{array}{c|c} L^{-1} & O \end{array} \right]_{ik} L'_{kj}$, since for $\left[ \begin{array}{c|c} L^{-1} & O \end{array} \right]_{ik}$ if $k > m$ then the entry is 0 and similarly $L'_{kj} = 0$ by definition means that the matrix multiply reduces to $ D_{ij} = \sum_{k = 1}^m (L^{-1})_{ik} L_{kj} = \delta_{ij}$.  Therefore $AB = \mathbb{I}_m$.
	\newpage
	\item
	\newpage
	\item
	\newpage
\end{enumerate}
\end{document}
