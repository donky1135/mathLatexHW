\documentclass[12pt, letterpaper]{article}
\date{\today}
\usepackage[margin=1in]{geometry}
\usepackage{amsmath}
\usepackage{hyperref}
\usepackage{cancel}
\usepackage{amssymb}
\usepackage{fancyhdr}
\usepackage{pgfplots}
\usepackage{booktabs}
\usepackage{pifont}
\usepackage{amsthm,latexsym,amsfonts,graphicx,epsfig,comment}
\pgfplotsset{compat=1.16}
\usepackage{xcolor}
\usepackage{tikz}
\usetikzlibrary{shapes.geometric}
\usetikzlibrary{arrows.meta,arrows}
\newcommand{\Z}{\mathbb{Z}}
\newcommand{\N}{\mathbb{N}}
\newcommand{\R}{\mathbb{R}}
\newcommand{\Q}{\mathbb{Q}}
\newcommand{\C}{\mathbb{C}}
\newcommand{\F}{\mathbb{F}}

\newcommand{\Po}{\mathcal{P}}
\newcommand{\Pro}{\mathbb{P}}
\author{Alex Valentino}
\title{503 homework}
\pagestyle{fancy}
\renewcommand{\headrulewidth}{0pt}
\renewcommand{\footrulewidth}{0pt}
\fancyhf{}
\rhead{
	Homework 1\\
	503	
}
\lhead{
	Alex Valentino\\
}
\begin{document}
\begin{enumerate}
	\item[2] Let $z,w \in \C$ with $z = z_1 + i z_2, w = w_1 + i w_2, z_1, z_2, w_1, w_2 \in \R$.  Additionally, let 
	$\langle z,w \rangle = z_1 w_1 + z_2 w_2, (z,w) = z \bar{w}, Re(z,w) = Re(z \bar{w} ) $.  First I must demonstrate that 
	$\frac{1}{2}[(z,w) + (w,z)] =  \langle z,w \rangle$:
	\begin{align*}
		\frac{1}{2}[(z,w) + (w,z)] &= \frac{1}{2} (z \bar{w} + \bar{z} w)\\
		&= \frac{1}{2} [(z_1 + i z_2)(w_1 - i w_2) + (z_1 - i z_2)(w_1 + i w_2)]\\
		&= \frac{1}{2} (z_1 w_1 - i z_1 w_2 + i z_2 w_1 + z_2 w_2 + z_1 w_1 + i z_1 w_2 - i z_2 w_1 + z_2 w_2)\\
		&= \frac{1}{2} (2 z_1 w_1 + 2 z_2 w_2)\\
		&= z_1 w_1 + z_2 w_2\\
		&= \langle z,w \rangle.
	\end{align*}
	Next I must demonstrate that $Re(z,w) =\langle z,w \rangle $:
	\begin{align*}
		Re(z,w) &= Re(z\bar{w})\\
		&= Re((z_1 + iz_2)(w_1 - i w_2))\\
		&= Re(z_1 w_1 - i z_1 w_2 + i z_2 w_1 + z_2 w_2)\\
		&= z_1 w_1 + z_2 w_2\\
		&= \langle z,w \rangle.
	\end{align*}
	Thus $\langle z,w \rangle = \frac{1}{2}[(z,w) + (w,z)] = Re(z,w)$
	\item[3] Let $\omega = s e^{i \phi}, s \geq 0, \phi \in [0,2\pi)$.  To solve $z^n = \omega$ where $z \in \C$ we must note that the polar form of $z = re^{i \theta}$, thus the 
	equation is actually $r^n e^{i n\theta} = s e^{i \phi}$.  Note that by the polar form
	of complex numbers $r\geq 0$.  Therefore $r = \sqrt[n]{|s|}$.  Now to divide out we're 
	left with the equation $e^{i \phi} = e^{i n \theta}$.  Since $e^{i \theta} = 
	e^{i \theta + 2 \pi i}$ then to solve this equation we have to solve that 
	$\phi \mod n \theta \mod{2 \pi}$.  Note that if we add a multiple of $2\pi /n$ to 
	$\theta$ then we have that $n (\theta + \frac{2 \pi }{n}) = n \theta + 2 \pi \equiv n\theta \equiv \phi \mod{2 \pi}$.  Note that $\theta + \frac{2 \pi }{n} \neq \theta \mod 2 \pi$.  Thus $\frac{\phi + 2 \pi a}{n}$ where $a \in \Z$ are all solutions.  But which are unique?  Note that if $a = n + b$ then $\frac{\phi + 2 \pi a}{n} = \frac{\phi + 2 \pi b}{n} + 2 \pi \equiv \frac{\phi + 2 \pi b}{n} \mod{2 \pi}$.  Therefore we have $n$ solutions, where $a \in \{0,\cdots, n-1\}$.
	\item[7]
	\begin{enumerate}
		\item Let $z,w \in \C$ such that $z = re^{i \theta}$, $\bar{z}w \neq 1$, $r = |z| < 1$, $|w| < 1$.  Then we have that 
		\begin{align*}
			1 &\leq \frac{1}{r^2}\\
			r^2 + |w|^2 &\leq \frac{1}{r^2}(r^2 + |w|^2)\\
			r^2 + |w|^2 &\leq  1 + r^2 |w|^2\\
			r^2 - rw - r\bar{w} +  |w|^2 & \leq 1 +- rw - r\bar{w} +  r^2 |w|^2\\
			r^2 - rw - r\bar{w} + (-w) (-\bar{w}) & \leq 1 +- rw - r \bar{w} +  (-r w)(-r \bar{w})\\
			(r - w)(r- \bar{w}) & \leq (1 - rw)(1 - r \bar{w})						
		\end{align*}
		Note the inequality is strict if one assumes that $r^2 < 1$ which implies that $r < 1$.   Thus 
		$$ \frac{r-w}{1-r\bar{w}} \frac{r- \bar{w}}{1 - r w} < 1 $$
		\begin{align*}
		\frac{r-w}{1-r\bar{w}} \frac{r- \bar{w}}{1 - r w} &< 1 \\
		\|\frac{r-w}{1-r\bar{w}} \|^2 &< 1\\
		\|\frac{r-w}{1-r\bar{w}} \| &< 1\\
		\|\frac{r-w}{1-r\bar{w}} \| &< 1\\
		\end{align*}
		Note that our choice of $z$ doesn't matter since if we define $w' = we^{i \theta}$
		$|\frac{w - re^{i\theta}}{1 - \bar{w}re^{i\theta}}| =|e^{i \theta}|  |\frac{w - re^{i\theta}}{1 - \bar{w}re^{i\theta}}| = |\frac{we^{i\theta} - r}{1 - \bar{w'}r} = |\frac{\bar{w} - r}{1 - \bar{w'}r}|$.  Thus we have proven the desired inequality.
		\item For a fixed $w \in \mathbb{D}$ let $F(z) = \frac{w - z}{1 - \bar{w}z}$ 
		\begin{enumerate}
			\item  We know that $F$ maps from $\mathbb{D} \to \mathbb{D}$ by the proof above.  $F$ being holomorphic is 
			equivalent to $\frac{\partial F}{\partial \bar{z}} = 0$ since  $$\frac{\partial F}{\partial \bar{z}} =  \frac{0 \cdot (1 - \bar{w} z) - 0 \cdot (w-z)}{(1 - \bar{w} z)^2} = 0$$ then $F$ is holomorphic.
			\item To show that $F$ swaps 0 and $w$:
			\begin{itemize}
				\item $F(0) = \frac{w- 0}{1 - 0\bar{w} } = \frac{w}{1} = w$.
				\item $F(w) = \frac{w - w}{1 - \bar{w}w} = 0$.  Note that $|w | < 1$ thus $1-\bar{w}w \neq 0$.
			\end{itemize}
			\item Note by the proof in $a$ equality is attained when $r = 1$, which implies if $|z| = 1$ then 
			$|F(z)|= 1$.
			\item Note that 
			\begin{align*}
			F \circ F(z) &= \frac{w - \frac{w - z}{1 - \bar{w} z}}{1 - \bar{w} \frac{w - z}{1 - \bar{w} z}}\\
			&= \frac{w - |w|^2 z - w + z}{1 - |w|^2}\\
			&= \frac{z - |w|^2z}{1 - |w|^2}\\
			&= z.
			\end{align*}
			which holds if $|z| \leq 1$ since $|w| < 1$ and $|\bar{w}^{-1}| > 1$,
			thus ensuring the denominator is never 0.  
			Furthermore this implies that $F$ is bijective since we have found an inverse.
		\end{enumerate}
	\end{enumerate}
	\item[9] Let $f(z) = f(r,\theta) = u(r,\theta) + i v(r,\theta) = u(x(r,\theta),y(r,\theta)) + i v(x(r,\theta),y(r,\theta))$ where $z = re^{i\theta}$.  
	with $- \pi < \theta < \pi$.  Note if we treat 
	$f$ as a function from $\R^2 \to \R^2$ then by the multivariable chain rule we have that
	\begin{align*}
	\begin{bmatrix}
	 \partial_r u & \partial_\theta u\\
	 \partial_r v & \partial_\theta v
	\end{bmatrix} &= \begin{bmatrix}
	\partial_x u & \partial_y u\\
	\partial_x v & \partial_y v
	\end{bmatrix}
	\begin{bmatrix}
	\partial_rx & \partial_\theta x\\
	\partial_ry & \partial_\theta y
	\end{bmatrix}\\
	&= 
	\begin{bmatrix}
	\partial_x u & \partial_y u\\
	\partial_x v & \partial_y v
	\end{bmatrix}	 
	\begin{bmatrix}
	\cos(\theta) & -r \sin(\theta)\\
	\sin(\theta) & r \cos(\theta)
	\end{bmatrix}\\
	&= \begin{bmatrix}
	\partial_x u \cos(\theta) + \partial_y u \sin(\theta) & - \partial_x u r \sin (\theta) + \partial_y u r \cos(\theta)\\
	\partial_x v \cos(\theta) + \partial_y v \sin(\theta) & - \partial_x v r \sin (\theta) + \partial_y v r \cos(\theta)
	\end{bmatrix}
	\end{align*}
	Therefore 
	\begin{align*}
	\partial_r u &= \partial_x u \cos(\theta) + \partial_y u \sin(\theta)\\
	&= \partial_y v \cos(\theta) - \partial_x v \sin(\theta)\\
	&= \frac{1}{r}(\partial_y v r\cos(\theta) - \partial_x v r\sin(\theta)\\
	&= \frac{1}{r}\partial_\theta v
	\end{align*}
	and 
	\begin{align*}
	\frac{1}{r} \partial_\theta u &= \frac{1}{r} (- \partial_x u r \sin (\theta) + \partial_y u r \cos(\theta))\\
	&= - \partial_x u  \sin (\theta) + \partial_y u  \cos(\theta)\\
	&= - \partial_y v \sin(\theta) - \partial_x v \cos(\theta)\\
	&= - \partial_r v
	\end{align*}
	Note that by the requirements above on $z$ we ensure that $z$ is uniquely determined.
	Using this fact we can observe that $\log(z) = \log(r) + i \theta = u + i v$, and 
	we have that $\partial_r \log(r) = \frac{1}{r} = \frac{1}{r} 1 = \frac{1}{r} \partial_\theta \theta$ and $\frac{1}{r} \partial_\theta u = \frac{1}{r} 0 = 0 = -\partial_r \theta$.  Thus $\log(z)$ is holomorphic on the spe
	\item[10] Note that 
	\begin{align*}
		4 \partial_z \partial_{\bar{z}} &= 4 \frac{1}{2} (\partial_x - i \partial_y)\frac{1}{2}(\partial_x + i \partial_y)\\
		&=  \partial_x^2 + \partial_y^2 - i \partial_y \partial_x + i \partial_x \partial_y\\
		&= \partial_x^2 + \partial_y^2 - i \partial_x \partial_y + i \partial_y \partial_x\\
		&= (\partial_x + i \partial_y)(\partial_x - i \partial_y)\\
		&= 4 \frac{1}{2}(\partial_x + i \partial_y)\frac{1}{2}(\partial_x - i \partial_y)\\
		&= 4 \partial_{\bar{z}} \partial_z
	\end{align*}
	Thus $4 \partial_z \partial_{\bar{z}} = 4 \frac{1}{2} (\partial_x - i \partial_y)\frac{1}{2}(\partial_x + i \partial_y) = \partial_x^2 + \partial_y^2 - i \partial_y \partial_x + i \partial_x \partial_y = \partial_x^2 + \partial_y^2 - i \partial_y \partial_x + i \partial_y \partial_x = \partial_x^2 + \partial_y^2$
	\item[13] Assume $f: \Omega \to \C$ is holomorphic and $\Omega$ is an open set.  Let $F:\R^2 \to \C$ where 
	$f(x+iy) = F(x,y) = u(x,y) + i v(x,y)$.
	\begin{enumerate}
		\item If $Re(f)$ is constant then $u$ is constant.  Thus $\partial_x u = \partial_y u = 0$.  This implies by 
		Cauchy-Riemann that $\partial_x v = \partial_y v = 0$.  This implies that $v$ is constant.  Thus $f$ is constant
		\item If $Im(f)$ is constant then $v$ is constant.  Thus $\partial_x v = \partial_y v = 0$.  This implies by 
		Cauchy-Riemann that $\partial_x u = \partial_y u = 0$.  This implies that $u$ is constant.  Thus $f$ is constant
		\item If $|f|$ is constant than $u^2 + v^2$ is constant.  Thus $\partial_x(u^2 + v^2) = \partial_y(u^2 + v^2) = 0$ giving us the equations 
		\begin{align*}
		2 u \partial_x u + 2 v \partial_x v &= 0\\
		2 u \partial_y u + 2 v \partial_y v &= 0
		\end{align*}
		Note that this can be expressed in the form 
$$		
		\begin{bmatrix}
		\partial_x u & \partial_x v\\
		\partial_y u & \partial_y v
		\end{bmatrix}
		\begin{bmatrix}
		u\\ v
		\end{bmatrix}
		= 
		\begin{bmatrix}
		0\\0
		\end{bmatrix}
$$
	Observe that the matrix being used is the transpose of the jacobian, and we have found an element in it's null space.
	Thus $\det J_F = 0$.  Therefore $|f'(z)| = 0$.  Thus $f$ is constant.  
	\end{enumerate}
	\item[14]
	\begin{align*}
		\sum_{n = M}^N a_n b_n &= \sum_{n = M}^N (B_{n} - B_{n-1}) a_n\\
		&= \sum_{n = M}^N B_{n} a_n - \sum_{n = M}^N B_{n-1} a_n\\
		&= \sum_{n = M}^N B_{n} a_n - \sum_{n = M-1}^{N-1} B_{n} a_{n+1}\\
		&= B_N a_N - B_{M-1} a_M + \sum_{n=M}^{N-1} B_n(a_n - a_{n+1}
	\end{align*}
	\item[16]
	\begin{enumerate}
		\item For $a_n = (\log(n))^2$, note that $1 \leq \log(n)^2$ for $n > 2$.  
		Additionally since all polynomials grow faster than log, then $(\log(n))^2 \leq (\sqrt{n})^2 = n$.  Thus $1 \leq (\log(n))^2 \leq n^{\frac{1}{n}} \to 1$ for sufficently large $n$.  Thus the radius of convergence is 1.   
		\item For $a_n = n!$, we know that for sufficently large $n$ that $n! > (n/2)^{n/2}$.  Therefore $\sqrt{n/2} = (n/2)^{\frac{n}{2n}} \leq \sqrt[n]{n!}$, which implies that 
		$\sqrt[n]{n!} = \infty$.  Therefore the radius of convergence is 0.
		\item If $a_n = \frac{n^2}{4^n + 3n}$ then $\frac{n^2}{4^n} \geq a_n$ and
		$\frac{n^2}{2\cdot 4^n} \leq a_n$.  Therefore 
		$\frac{1}{4} \leq \sqrt[n]{\frac{n^2}{2\cdot 4^n}}\frac{n^{2/n}}{\sqrt[n]{2}4}
		\leq a_n \leq \frac{n^{2/n}}{4} \to \frac{1}{4}$.  Thus by squeeze theorem 
		$\lim a_n = \frac{1}{4}$ and the radius of convergence is 4.
	\end{enumerate}
	\item[17] Let $\{a_n\}_{n \in \N} \subset \C$ such that $L = \lim \frac{|a_{n+1}|}{|a_n|}$ and let $\epsilon > 0$.  Note there exists some $N \in \N$ such that for all 
	$n \geq N$ $|\frac{|a_{n+1}|}{|a_n|} - L | < \epsilon$.  Therefore, 
	$ |a_n| (L-\epsilon) < |a_{n+1}| < |a_n| (L+\epsilon)$.  Additionally, 
	we have that $|a_n| = |\frac{a_n}{a_{n-1}}\frac{a_{n-1}}{a_{n-2}}\cdots \frac{a_{N+1}}{a_N}a_N| $ .  Thus, $(L - \epsilon)^{n-N}|a_N| < |a_{n+1}| < (L+\epsilon)^{n - N}|a_N|$.  Taking the $n$-th root we find that 
	$(L - \epsilon)^{1 - \frac{N}{n}}|a_N|^{1/n} < |a_{n+1}|^{1/n} < (L+\epsilon)^{1 - \frac{N}{n}}|a_N|^{1/n}$.  Note that taking the $n$-th root of a constant sends it to 1, thus in the limit $L - \epsilon < |a_{n+1}|^{1/n} < L + \epsilon$.  Note if we reindex $a_n$ we get that $\lim \sqrt[n]{|a_n|} = L$.  
	\item[19]
	\begin{enumerate}
		\item The power series $\sum n z^n$ has the radius of convergence 0 since 
		$\lim n = \infty$.  Thus it doesn't converge for any point on the unit circle
		\item The power series converges for every point on the unit circle since 
		$| \sum_{n=1}^N \frac{z^n}{n^2} | \leq \sum_{n=1}^N \frac{|z^n|}{n^2} = 
		\sum_{n=1}^N \frac{1}{n^2} \to \frac{\pi^2}{6}	$
		\item Note if we fix $z$ such that $|z| = 1$ and $z \neq 1$ that 
		$\frac{1}{|1-z|}$ is bounded, and let it equal $b$.  Additionally, 
		$|1-z^n| \leq |1 - (-1) | = 2$.  Now let $\epsilon > 0$ be fixed.  We know that there exists $N_1 \geq M_1 \in \N$ such that $\frac{8}{M_1 b} < \epsilon$.  Furthermore 
		there exists $N_2 \geq M_2 \in \N$ such that $|\sum_{n=M_2}^{N_2} \frac{1}{n(n+1)}| < \frac{b \epsilon}{4}$.  Finally there exists $N_2 \geq M_3$ such that 
		$|\sum_{n=M_3}^{N_3} \frac{z^2}{n^2}| < \frac{b \epsilon}{4}$.  Now if we take
		$M = \max{M_1, M_2. M_3}$ and $N = \min{N_1, N_2, N_3}$ and that $M < N$, then
		\begin{align*}
		|\sum_{n=M}^N \frac{z^n}{n}| &= |\frac{1}{N}\frac{1-z^N}{1-z} - \frac{1}{M}\frac{1-z^{M-1}}{1-z}+ \sum_{n=M}^{N-1}\left(\frac{1}{n} - \frac{1}{n+1} \right) \frac{1-z^n}{1-z}|\\
		&\leq \frac{1}{N}\frac{2}{b} + \frac{1}{M} \frac{2}{b} + \frac{1}{b} \left( \sum_{n=M}^{N-1} \frac{1}{n(n+1)} +| \sum_{n=M}^{N-1} \frac{z^n}{n(n+1)}| \right)\\
		& \leq \frac{\epsilon}{4} + \frac{\epsilon}{4} + \frac{\epsilon}{4} + 
		\frac{1}{b} | \sum_{n=M}^{N-1} \frac{z^n}{n^2}|\\
		&< \frac{\epsilon}{4} + \frac{\epsilon}{4} + \frac{\epsilon}{4} + \frac{\epsilon}{4} = \epsilon 
		\end{align*}
		Therefore the series converges when $z\neq 1$.  
	\end{enumerate}
	\item[23]
	Let $f(x) = \begin{cases} 0 & x \leq 0\\ e^{\frac{-1}{x^2}} & x > 0\end{cases}$.  
	Note that $\frac{d}{dx} f(x) = \begin{cases} 0 & x \leq 0\\ \frac{2}{x^3}e^{\frac{-1}{x^2}} & x > 0\end{cases}$ and $\frac{d^2}{dx^2} f(x) = \begin{cases} 0 & x \leq 0\\ \frac{4-x^2}{x^6} e^{\frac{-1}{x^2}} & x > 0\end{cases}$.
	I claim that $f^{(n)} (x) = \frac{P_n(x)}{x^{3n}}f(x)$ where $\deg(P_n(x)) = 2(n-1)$.  Note that the base cases have been demonstrated.  Now for my induction step, take
	$f^(n)(x) = \frac{P_n(x)}{x^{3n}}f(x)$.  If $x \leq 0$ then the function should be 0, otherwise if $x>0$ then \begin{align*}
	f^{(n+1)}(x) &= \frac{d}{dx}\frac{P_n(x)}{x^{3n}}e^{\frac{-1}{x^2}}\\
	&= frac{P_n(x)}{x^{3n}} \frac{2}{x^3}e^{\frac{-1}{x^2}} + \frac{P_n' x^{3n} - 3n P_n(x)x^{3n - 1}}{x^{6n}}e^{\frac{-1}{x^2}}\\
	&= \frac{2P_n(x) + x^3 P_n'(x) - 3 n x^2 P_n(x)}{x^{3n+3}}e^{\frac{-1}{x^2}}
	\end{align*}
	which gives us the induction step in the degree of the bottom polynomial, and in 
	the numerator the degree of the polynomial is $2n$ since $\deg(P_n') = 2(n-1) - 1$
	and when multiplied by $x^3$ we get $2(n-1) - 1 + 3 = 2n$.  Since the formula holds,
	then we claim that $\lim_{x \to 0} f^{(n)}(x) = 0$.  Note that if we approach from 
	the left then we trivially get 0.  Therefore we must approch from the right.  
	To do so I will make a change of variables with $x = \frac{1}{y}$.  Now our limit 
	becomes $\lim_{y \to \infty} f^{(n)} (\frac{1}{y}) = \frac{y^{3n} P_n(\frac{1}{y})}{e^{y^2}}$.  Note that since $P_n$ has degree $2(n-1)$ then $y^{3n} P_n(\frac{1}{y})$
	is a polynomial in $y$ with degree $n + 2$.  Thus we're taking the limit to infinity 
	of a polynomial over an exponential.  Thus the limit goes to 0.  Since all derivatives of $f$ vanish at the origin then the power series is just 0.  However that 
	implies that $f = 0$, however since $f \neq 0$ then $f$ isn't represented by it's power series, thus implying that $f$ is not analytic.  
\end{enumerate}
\end{document}
