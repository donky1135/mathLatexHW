\documentclass[12pt, letterpaper]{article}
\date{\today}
\usepackage[margin=1in]{geometry}
\usepackage{amsmath}
\usepackage{hyperref}
\usepackage{cancel}
\usepackage{amssymb}
\usepackage{fancyhdr}
\usepackage{pgfplots}
\usepackage{booktabs}
\usepackage{pifont}
\usepackage{amsthm,latexsym,amsfonts,graphicx,epsfig,comment}
\pgfplotsset{compat=1.16}
\usepackage{xcolor}
\usepackage{tikz}
\usetikzlibrary{shapes.geometric}
\usetikzlibrary{arrows.meta,arrows}
\newcommand{\Z}{\mathbb{Z}}
\newcommand{\N}{\mathbb{N}}
\newcommand{\R}{\mathbb{R}}
\newcommand{\Q}{\mathbb{Q}}
\newcommand{\C}{\mathbb{C}}
\newcommand{\F}{\mathbb{F}}

\newcommand{\Po}{\mathcal{P}}
\newcommand{\Pro}{\mathbb{P}}
\author{Alex Valentino}
\title{503 homework}
\pagestyle{fancy}
\renewcommand{\headrulewidth}{0pt}
\renewcommand{\footrulewidth}{0pt}
\fancyhf{}
\rhead{
	Homework 1\\
	503	
}
\lhead{
	Alex Valentino\\
}
\begin{document}
\begin{enumerate}
	\item[2] Let $z,w \in \C$ with $z = z_1 + i z_2, w = w_1 + i w_2, z_1, z_2, w_1, w_2 \in \R$.  Additionally, let 
	$\langle z,w \rangle = z_1 w_1 + z_2 w_2, (z,w) = z \bar{w}, Re(z,w) = Re(z \bar{w} ) $.  First I must demonstrate that 
	$\frac{1}{2}[(z,w) + (w,z)] =  \langle z,w \rangle$:
	\begin{align*}
		\frac{1}{2}[(z,w) + (w,z)] &= \frac{1}{2} (z \bar{w} + \bar{z} w)\\
		&= \frac{1}{2} [(z_1 + i z_2)(w_1 - i w_2) + (z_1 - i z_2)(w_1 + i w_2)]\\
		&= \frac{1}{2} (z_1 w_1 - i z_1 w_2 + i z_2 w_1 + z_2 w_2 + z_1 w_1 + i z_1 w_2 - i z_2 w_1 + z_2 w_2)\\
		&= \frac{1}{2} (2 z_1 w_1 + 2 z_2 w_2)\\
		&= z_1 w_1 + z_2 w_2\\
		&= \langle z,w \rangle.
	\end{align*}
	Next I must demonstrate that $Re(z,w) =\langle z,w \rangle $:
	\begin{align*}
		Re(z,w) &= Re(z\bar{w})\\
		&= Re((z_1 + iz_2)(w_1 - i w_2))\\
		&= Re(z_1 w_1 - i z_1 w_2 + i z_2 w_1 + z_2 w_2)\\
		&= z_1 w_1 + z_2 w_2\\
		&= \langle z,w \rangle.
	\end{align*}
	Thus $\langle z,w \rangle = \frac{1}{2}[(z,w) + (w,z)] = Re(z,w)$
	\item[3]
	\item[7]
	\begin{enumerate}
		\item Let $z,w \in \C$ such that $z = re^{i \theta}$, $\bar{z}w \neq 1$, $r = |z| < 1$, $|w| < 1$.  Then we have that 
		\begin{align*}
			1 &\leq \frac{1}{r^2}\\
			r^2 + |w|^2 &\leq \frac{1}{r^2}(r^2 + |w|^2)\\
			r^2 + |w|^2 &\leq  1 + r^2 |w|^2\\
			r^2 - rw - r\bar{w} +  |w|^2 & \leq 1 +- rw - r\bar{w} +  r^2 |w|^2\\
			r^2 - rw - r\bar{w} + (-w) (-\bar{w}) & \leq 1 +- rw - r \bar{w} +  (-r w)(-r \bar{w})\\
			(r - w)(r- \bar{w}) & \leq (1 - rw)(1 - r \bar{w})						
		\end{align*}
		Note the inequality is strict if one assumes that $r^2 < 1$ which implies that $r < 1$.   Thus 
		$$ \frac{r-w}{1-r\bar{w}} \frac{r- \bar{w}}{1 - r w} < 1 $$
		\begin{align*}
		\frac{r-w}{1-r\bar{w}} \frac{r- \bar{w}}{1 - r w} &< 1 \\
		\|\frac{r-w}{1-r\bar{w}} \|^2 &< 1\\
		\|\frac{r-w}{1-r\bar{w}} \| &< 1\\
		\|\frac{r-w}{1-r\bar{w}} \| &< 1\\
		\end{align*}
		\item For a fixed $w \in \mathbb{D}$ let $F(z) = \frac{w - z}{1 - \bar{w}z}$ 
		\begin{enumerate}
			\item  We know that $F$ maps from $\mathbb{D} \to \mathbb{D}$ by the proof above.  $F$ being holomorphic is 
			equivalent to $\frac{\partial F}{\partial \bar{z}} = 0$ since  $$\frac{\partial F}{\partial \bar{z}} =  \frac{0 \cdot (1 - \bar{w} z) - 0 \cdot (w-z)}{(1 - \bar{w} z)^2} = 0$$ then $F$ is holomorphic.
			\item To show that $F$ swaps 0 and $w$:
			\begin{itemize}
				\item $F(0) = \frac{w- 0}{1 - 0\bar{w} } = \frac{w}{1} = w$.
				\item $F(w) = \frac{w - w}{1 - \bar{w}w} = 0$.  Note that $|w | < 1$ thus $1-\bar{w}w \neq 0$.
			\end{itemize}
			\item Note by the proof in $a$ equality is attained when $r = 1$, which implies if $|z| = 1$ then 
			$|F(z)|= 1$.
			\item Note that 
			\begin{align*}
			F \circ F(z) &= \frac{w - \frac{w - z}{1 - \bar{w} z}}{1 - \bar{w} \frac{w - z}{1 - \bar{w} z}}\\
			&= \frac{w - |w|^2 z - w + z}{1 - |w|^2}\\
			&= \frac{z - |w|^2z}{1 - |w|^2}\\
			&= z.
			\end{align*}
			which holds if $|z| \leq 1$ since $|w| < 1$ and $|\bar{w}^{-1}| > 1$,
			thus ensuring the denominator is never 0.  
			Furthermore this implies that $F$ is bijective since we have found an inverse.
		\end{enumerate}
	\end{enumerate}
	\item[9]
	\item[10] need to clarify if we can swap the order of the x and y partial derivatives
	\item[13] Assume $f: \Omega \to \C$ is holomorphic and $\Omega$ is an open set.  Let $F:\R^2 \to \C$ where 
	$f(x+iy) = F(x,y) = u(x,y) + i v(x,y)$.
	\begin{enumerate}
		\item If $Re(f)$ is constant then $u$ is constant.  Thus $\partial_x u = \partial_y u = 0$.  This implies by 
		Cauchy-Riemann that $\partial_x v = \partial_y v = 0$.  This implies that $v$ is constant.  Thus $f$ is constant
		\item If $Im(f)$ is constant then $v$ is constant.  Thus $\partial_x v = \partial_y v = 0$.  This implies by 
		Cauchy-Riemann that $\partial_x u = \partial_y u = 0$.  This implies that $u$ is constant.  Thus $f$ is constant
		\item If $|f|$ is constant than $u^2 + v^2$ is constant.  Thus $\partial_x(u^2 + v^2) = \partial_y(u^2 + v^2) = 0$ giving us the equations 
		\begin{align*}
		2 u \partial_x u + 2 v \partial_x v &= 0\\
		2 u \partial_y u + 2 v \partial_y v &= 0
		\end{align*}
		Note that this can be expressed in the form 
$$		
		\begin{bmatrix}
		\partial_x u & \partial_x v\\
		\partial_y u & \partial_y v
		\end{bmatrix}
		\begin{bmatrix}
		u\\ v
		\end{bmatrix}
		= 
		\begin{bmatrix}
		0\\0
		\end{bmatrix}
$$
	Observe that the matrix being used is the transpose of the jacobian, and we have found an element in it's null space.
	Thus $\det J_F = 0$.  Therefore $|f'(z)| = 0$.  Thus $f$ is constant.  
	\end{enumerate}
	\item[14]
	\item[16]
	\begin{enumerate}
		\item
		\item
		\item
	\end{enumerate}
	\item[17]
	\item[19]
	\item[23]
\end{enumerate}
\end{document}
