\documentclass[12pt, letterpaper]{article}
\date{\today}
\usepackage[margin=1in]{geometry}
\usepackage{amsmath}
\usepackage{hyperref}
\usepackage{cancel}
\usepackage{amssymb}
\usepackage{fancyhdr}
\usepackage{pgfplots}
\usepackage{booktabs}
\usepackage{pifont}
\usepackage{amsthm,latexsym,amsfonts,graphicx,epsfig,comment}
\pgfplotsset{compat=1.16}
\usepackage{xcolor}
\usepackage{tikz}
\usetikzlibrary{shapes.geometric}
\usetikzlibrary{arrows.meta,arrows}
\newcommand{\Z}{\mathbb{Z}}
\newcommand{\N}{\mathbb{N}}
\newcommand{\R}{\mathbb{R}}
\newcommand{\Q}{\mathbb{Q}}
\newcommand{\C}{\mathbb{C}}

\newcommand{\Po}{\mathcal{P}}
\newcommand{\Pro}{\mathbb{P}}
\author{Alex Valentino}
\title{452 homework}
\pagestyle{fancy}
\renewcommand{\headrulewidth}{0pt}
\renewcommand{\footrulewidth}{0pt}
\fancyhf{}
\rhead{
	Homework 10\\
	452	
}
\lhead{
	Alex Valentino\\
}
\begin{document}
\begin{enumerate}
	\item[4.3]
	\begin{enumerate}
		\item[a] Let the Klein 4 group be defined as $V_4 = <a,b | a^2 = b^2 = (ab)^2 = e>$, then we have the character 
		table:\\
		\begin{tabular}{c|cccc}
		& $\{e\}$& $\{a,a^2\}$ & $\{b,b^2\}$ & $\{ab,(ab)^2\}$\\
		$\chi_1$ & 1 & 1 & 1 & 1\\
		$\chi_2$ & 1 & 1 & -1 & -1\\
		$\chi_3$ & 1 & -1 & 1 & -1\\
		$\chi_4$ & 1 & -1 & -1 & 1\\		
		\end{tabular}
		\item[c] Let $D_4 = <r,s | r^4 = s^2 = e, srs = r^{-1}>$ and it's character table as follows:\\
		\begin{tabular}{c|ccccc}
		& $\{e\}$ & $\{r^2\}$ & $\{r,r^3\}$ & $\{s, r^2 s\}$ & $\{rs, r^3s\}$\\
		$\chi_1$ & 1 & 1 & 1 & 1 & 1\\
		$\chi_2$ & 1 &  1 & -1 & 1 & -1\\
		$\chi_3$ & 1 &  1 & 1 & -1 & -1\\
		$\chi_4$ & 1 &  1 & -1 & -1 & 1\\
		$\chi_5$ & 2 &  -2 & 0 & 0 & 0\\		
		\end{tabular}
		
	\end{enumerate}
	\item[4.7] Let $G$ be a finite group with $|G| = nm, N \leq G, G' = G / N = \{r_1,\cdots,r_m\}$, with the canonical projection map $\pi : G \to G'$, 
	and let $\rho': G' \to GL(V)$ be an irreducible representation of $G'$, and $\rho = \rho' \circ \pi$.  To directly 
	show that $\rho$ is an irreducible representation of $G$, assume for contradiction that $W$ is a $G$-invariant
	subspace.  Then for all $g \in G, v \in W, \rho_g(v) = v$.  However, consider for each $g$, there exists $r_i$ 
	such that $\pi(g) = r_i$.  This now implies that $\rho_g(v) = \rho_{r_i}'(v) = v$.  This implies that 
	$W$ is a $G'-invariant$ subspace, contradicting that $\rho'$ is irreducible.  Now by using theorem 10.4.6, 
	if we consider $\chi'$ to be the character of $\rho'$ and $\chi$ to be the character of $\rho$, then 
	\begin{align*}
	<\chi,\chi> &= \frac{1}{nm} \sum_{g \in G} \overline{\chi(g)}\chi(g)\\
	&= \frac{1}{nm} \sum_{g \in G} \overline{\chi'(\pi(g))}\chi'(\pi(g))\\
	&= \frac{1}{nm} \sum_{i=1}^m n \overline{\chi'(r_i)}\chi'(r_i)\\
	& \text{since } |gN| = n\\
	&= \frac{1}{m}\sum_{i=1}^m \overline{\chi'(r_i)}\chi'(r_i)\\ 
	&= <\chi',\chi'> \\
	&= 1
	\end{align*}
	Since the inner product of $\chi$ with itself is 1, then it is necessarily irreducible.  
	\item[5.5] Let $G$ be a group.  Let $\chi_1, \chi_2$ by characters for 
	$G$.  If we let $\chi = \chi_1 \chi_2$, then for any $g,h \in G$ then $\chi(gh) = \chi_1(gh)\chi_2(gh) = 
	\chi_1(g)\chi_2(h)\chi_1(g)\chi_2(h) = \chi_1(g)\chi_2(g)\chi_1(h)\chi_2(h) = \chi(g)\chi(h)$.  Since the 
	set of characters is just the set of homomorphisms from $G \to C^{\times}$ then $\chi$ is a homomorphism as well. 
	Additionally, for every $\chi \in \hat{G}, g \in G$, $\chi(g) \neq 0$, therefore 
	$\frac{1}{\chi(g)}$ exists, giving us that, $\frac{1}{\chi(g)}\frac{1}{\chi(h)} = \frac{1}{\chi(g)\chi(h)} = \frac{1}{\chi(gh)}$, therefore
	the inverse of a character is a character as well.  Thus the set of characters and their 
	inverses is closed under 
	multiplication.  Finally, since the trivial representation exists and the characters are functions over complex numbers the $\hat{G}$ has an identity and associativity.  Thus the 
	dual group is a group.\\
	  If $G$ is abelian and finite then by the structure theorem we have that 
	$G \cong \Z_{d_1} \oplus \cdots \oplus \Z_{d_k}$ where $d_1,\cdots,d_k$ are prime powers.  
	Note for cyclic groups, if $g$ generates $\Z_n$, then $1 = \chi(1) = \chi(g^n) = \chi(g)^n$, which implies 
	that $\chi(g) = e^{\frac{2 \pi i k g }{n}}$ where $k$ is some number between $0$ and $n-1$.  Furthermore 
	since the roots of unity for $n$ are cyclic implies that $\hat{Z_n}$ is isomorphic to $Z_n$.  Therefore 
	$\Z_{d_1} \oplus \cdots \oplus \Z_{d_k} \cong \{e^{\frac{2 \pi i j}{d_1}}\}_{j=1}^{d_1} 
	\oplus \cdots \oplus \{e^{\frac{2 \pi i j}{d_k}}\}_{j=1}^{d_k}$.  Furthermore since $d_1,\cdots,d_k$ are 
	prime powers there's an isomorphism from $\{e^{\frac{2 \pi i j}{d_1}}\}_{j=1}^{d_1} 
	\oplus \cdots \oplus \{e^{\frac{2 \pi i j}{d_k}}\}_{j=1}^{d_k}$ to 
	$\{e^{\frac{2 \pi i j_1}{d_1} + \cdots + \frac{2 \pi i j_k}{d_k}}\}_{j_1,\cdots,j_k = 1}^{d_1,\cdots,d_k}$, as 
	the sum of fractions with prime powers can uniquely be written that way, thus each element is uniquely mapped.  
	Thus we have shown that $G$ is isomorphic to it's dual group.  
	\iffalse Note that we can represent elements $(a_1,\cdots,a_k) \in \Z_{d_1} \oplus \cdots \oplus \Z_{d_k}$ as the matrix with the 
	diagonal $e^{\frac{2\pi i a_1}{d_1}}, \cdots, e^{\frac{2\pi i a_k}{d_k}}$, as the roots of unity in each of the 
	diagonals is isomorphic to each individual elements in $(a_1,\cdots,a_k)$.  Therefore 
	\fi
	\iffalse We know that the injective homomorphism from 
	$\chi: \Z_{d_i} \to \C^{\times}$ is $\chi(g) = e^{\frac{2 \pi i g }{d_i}}$, and $\chi:\Z \to \C^{\times}$ is 
	$\chi(n) = e^n$.  Since these also are trivially irreducible implies that by the structure theorem 
	we have an injective map from $G$ to $\Z_{d_1} \oplus \cdots \oplus \Z_{d_k} \oplus \Z^n$ and for each 
	respective component we have an irreducible representation implies that we have an injection from $G$ to a product 
	over the characters specified above.  If $G$ is finite then since we have an injective map then the sets 
	are isomorphic.  Additionally since these are all homomorphisms then $G$ is isomorphic to it's dual.
	\fi
	\item[5.7]
	\begin{enumerate}
		\item Note since $G, G'$ are finite abelian groups then their homomorphism is isomorphic to a linear transformation from $\Z_{d_1} \oplus \cdots \oplus \Z_{d_k}$ to 
		$\Z_{d_1'} \oplus \cdots \oplus \Z_{d_l'}$.  This can be given as a $l \times k$ matrix.  Therefore the 
		induced homomorphism is simply the $d_i'$ component in  $\Z_{d_1'} \oplus \cdots \oplus \Z_{d_l'}$
		after applying $\phi$ to $(a_1,\cdots, a_k) \in \Z_{d_1} \oplus \cdots \oplus \Z_{d_k}$.  
		This is equivalent to $e^{\frac{2 \pi i \phi(a_1,\cdots, a_k)_1}{d_1'} + \cdots + \frac{2 \pi i \phi(a_1,\cdots, a_k)_l}{d_l'}}$.  
		\item We know from linear algebra that the dual $\hat{\phi}$ corresponds to the transpose of the transpose
		of the linear operation $\phi$.  Therefore if $\phi$ is injective implies that $\phi^T$ is surjective, 
		and $\phi$ is surjective implies that $\phi^T$ is injective.  Since one can create an isomorphism 
		from $\phi^T$ and $\hat{\phi}$ implies the desired result.  
		
	\end{enumerate}
	
\end{enumerate}
\end{document}
