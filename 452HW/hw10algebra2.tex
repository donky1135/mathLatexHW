\documentclass[12pt, letterpaper]{article}
\date{\today}
\usepackage[margin=1in]{geometry}
\usepackage{amsmath}
\usepackage{hyperref}
\usepackage{cancel}
\usepackage{amssymb}
\usepackage{fancyhdr}
\usepackage{pgfplots}
\usepackage{booktabs}
\usepackage{pifont}
\usepackage{amsthm,latexsym,amsfonts,graphicx,epsfig,comment}
\pgfplotsset{compat=1.16}
\usepackage{xcolor}
\usepackage{tikz}
\usetikzlibrary{shapes.geometric}
\usetikzlibrary{arrows.meta,arrows}
\newcommand{\Z}{\mathbb{Z}}
\newcommand{\N}{\mathbb{N}}
\newcommand{\R}{\mathbb{R}}
\newcommand{\Q}{\mathbb{Q}}
\newcommand{\C}{\mathbb{C}}

\newcommand{\Po}{\mathcal{P}}
\newcommand{\Pro}{\mathbb{P}}
\author{Alex Valentino}
\title{452 homework}
\pagestyle{fancy}
\renewcommand{\headrulewidth}{0pt}
\renewcommand{\footrulewidth}{0pt}
\fancyhf{}
\rhead{
	Homework 10\\
	452	
}
\lhead{
	Alex Valentino\\
}
\begin{document}
\begin{enumerate}
	\item[4.3]
	\begin{enumerate}
		\item[a] Let the Klein 4 group be defined as $V_4 = <a,b | a^2 = b^2 = (ab)^2 = e>$, then we have the character 
		table:\\
		\begin{tabular}{c|cccc}
		& $\{e\}$& $\{a,a^2\}$ & $\{b,b^2\}$ & $\{ab,(ab)^2\}$\\
		$\chi_1$ & 1 & 1 & 1 & 1\\
		$\chi_2$ & 1 & 1 & -1 & -1\\
		$\chi_3$ & 1 & -1 & 1 & -1\\
		$\chi_4$ & 1 & -1 & -1 & 1\\		
		\end{tabular}
		\item[c] Let $D_4 = <r,s | r^4 = s^2 = e, srs = r^{-1}>$ and it's character table as follows:\\
		\begin{tabular}{c|ccccc}
		& $\{e\}$ & $\{r^2\}$ & $\{r,r^3\}$ & $\{s, r^2 s\}$ & $\{rs, r^3s\}$\\
		$\chi_1$ & 1 & 1 & 1 & 1 & 1\\
		$\chi_2$ & 1 &  1 & -1 & 1 & -1\\
		$\chi_3$ & 1 &  1 & 1 & -1 & -1\\
		$\chi_4$ & 1 &  1 & -1 & -1 & 1\\
		$\chi_5$ & 2 &  -2 & 0 & 0 & 0\\		
		\end{tabular}
		
	\end{enumerate}
	\item[4.7] Let $G$ be a finite group with $|G| = nm, N \leq G, G' = G / N = \{r_1,\cdots,r_m\}$, with the canonical projection map $\pi : G \to G'$, 
	and let $\rho': G' \to GL(V)$ be an irreducible representation of $G'$, and $\rho = \rho' \circ \pi$.  To directly 
	show that $\rho$ is an irreducible representation of $G$, assume for contradiction that $W$ is a $G$-invariant
	subspace.  Then for all $g \in G, v \in W, \rho_g(v) = v$.  However, consider for each $g$, there exists $r_i$ 
	such that $\pi(g) = r_i$.  This now implies that $\rho_g(v) = \rho_{r_i}'(v) = v$.  This implies that 
	$W$ is a $G'-invariant$ subspace, contradicting that $\rho'$ is irreducible.  Now by using theorem 10.4.6, 
	if we consider $\chi'$ to be the character of $\rho'$ and $\chi$ to be the character of $\rho$, then 
	\begin{align*}
	<\chi,\chi> &= \frac{1}{nm} \sum_{g \in G} \overline{\chi(g)}\chi(g)\\
	&= \frac{1}{nm} \sum_{g \in G} \overline{\chi'(\pi(g))}\chi'(\pi(g))\\
	&= \frac{1}{nm} \sum_{i=1}^m n \overline{\chi'(r_i)}\chi'(r_i)\\
	& \text{since } |gN| = n\\
	&= \frac{1}{m}\sum_{i=1}^m \overline{\chi'(r_i)}\chi'(r_i)\\ 
	&= <\chi',\chi'> \\
	&= 1
	\end{align*}
	Since the inner product of $\chi$ with itself is 1, then it is necessarily irreducible.  
	\item[5.5] Let $G$ be a finite group with $|G| = n$ and $k$ conjugacy classes.  Let $\chi_i, \chi_j$ by characters for irreducible representations of $\rho_i, \rho_j$ for 
	$G$.  If we let $\chi = \chi_i \chi_j$, then for any $g,h \in G$ then $\chi(gh) = \chi_i(gh)\chi_j(gh) = 
	\chi_i(g)\chi_i(h)\chi_j(g)\chi_j(h) = \chi_i(g)\chi_j(g)\chi_i(h)\chi_j(h) = \chi(g)\chi(h)$.  Since the 
	set of characters is just the set of homomorphisms from $G \to C^{\times}$ then $\chi$ is a homomorphism as well. 
	Additionally the above proof can be redone with $\frac{1}{\chi}$.  Thus the set of characters and their 
	inverses is closed under 
	multiplication, and since these are functions over complex numbers all other group axioms are satisfied.  Thus the 
	dual group is a group.  If $G$ is abelian then by the structure theorem we have that 
	$G \cong \Z_{d_1} \oplus \cdots \oplus \Z_{d_k} \oplus \Z^n$.  We know that the injective homomorphism from 
	$\chi: \Z_{d_i} \to \C^{\times}$ is $\chi(g) = e^{\frac{2 \pi i g }{d_i}}$, and $\chi:\Z \to \C^{\times}$ is 
	$\chi(n) = e^n$.  Since these also are trivially irreducible implies that by the structure theorem 
	we have an injective map from $G$ to $\Z_{d_1} \oplus \cdots \oplus \Z_{d_k} \oplus \Z^n$ and for each 
	respective component we have an irreducible representation implies that we have an injection from $G$ to a product 
	over the characters specified above.  If $G$ is finite then since we have an injective map then the sets 
	are isomorphic.  Additionally since these are all homomorphisms then $G$ is isomorphic to it's dual.
	\item[5.7]
	\begin{enumerate}
		\item Since a given element $g \in G$ with the isomorphism from $G$ to $\Z_{d_1} \oplus \cdots \oplus \Z_{d_k}$
		implies that $g$ can map into $e^{\frac{2 \pi i v_1}{d_1}} + \cdots + e^{\frac{2 \pi i v_k}{d_k}}$.
		Thus the homomorphism is a homomorphism between direct sums of cyclic groups.  Thus one can take 
		$g$ to $\phi(g)$ with $G'$ to $\Z_{d_1'} \oplus \cdots \oplus \Z_{d_k'}$ and get 
		$e^{\frac{2 \pi i v_1'}{d_1'}} + \cdots + e^{\frac{2 \pi i v_k'}{d_k'}}$
		\item If dual $\phi$ is surjective then that implies that every possible combination over the module 
		 $\Z_{d_1'} \oplus \cdots \oplus \Z_{d_k'}$ is reached.  Therefore every $g' \in G'$ has a unique 
		 representation in $G$ under $\phi^{-1}$
	\end{enumerate}
	
\end{enumerate}
\end{document}
