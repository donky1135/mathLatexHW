\documentclass[12pt, letterpaper]{article}
\date{\today}
\usepackage[margin=1in]{geometry}
\usepackage{amsmath}
\usepackage{hyperref}
\usepackage{cancel}
\usepackage{amssymb}
\usepackage{fancyhdr}
\usepackage{pgfplots}
\usepackage{booktabs}
\usepackage{pifont}
\usepackage{amsthm,latexsym,amsfonts,graphicx,epsfig,comment}
\pgfplotsset{compat=1.16}
\usepackage{xcolor}
\usepackage{tikz}
\usetikzlibrary{shapes.geometric}
\usetikzlibrary{arrows.meta,arrows}
\newcommand{\Z}{\mathbb{Z}}
\newcommand{\N}{\mathbb{N}}
\newcommand{\R}{\mathbb{R}}
\newcommand{\Q}{\mathbb{Q}}
\newcommand{\C}{\mathbb{C}}

\newcommand{\Po}{\mathcal{P}}
\newcommand{\Pro}{\mathbb{P}}
\author{Alex Valentino}
\title{452 homework}
\pagestyle{fancy}
\renewcommand{\headrulewidth}{0pt}
\renewcommand{\footrulewidth}{0pt}
\fancyhf{}
\rhead{
	Homework 4\\
	452	
}
\lhead{
	Alex Valentino\\
}
\begin{document}
\begin{enumerate}
	\item[15.8.1] Since we're doing this proof by induction, we only need to find that for a 
	given finite field $F$, that for $K = F(\alpha,\beta)$, there exists $\gamma \in K$ such 
	that $K = F(\gamma)$.  Note that since $[K : F]$ is finite implies that $K$ is a finite 
	field as well.  We know that $K^{\times}$ is a cyclic group from chapter 15.7.  Furthermore
	we know that cyclic groups have a single generator.  Let $\gamma$ be the generator of 
	$K^{\times}$.  Since $K$ will contain all of the powers of $\gamma$, and there exists 
	$m,n \in \N$ such that $\gamma^m = \alpha, \gamma^n = \beta$, therefore $F(\gamma) = K$.
	\item[16.3.1] Suppose $f$ is degree $n$ with coefficents in the field $F$, and let $L$ be
	the spliting field.  We want to show that $[K:F] \mid n!$.  Note that since $K$ is a
	splitting field then $[K:F]$ is the cardinality of the galois group $G(K/F)$.  Since 
	$G(K/F)$ is a group operating on $n$ elements, then we know by Cayley's theorem that 
	$G(K/F) < S_n$.  Therefore by Lagranges theorem $|G(K/F)| \mid n!  = [K:F] \mid n!$.   
	\iffalse Note that every element in 
	$K$ can be generated by taking a polynomial in the ring $p \in F[u_1,\cdots,u_n]$ and 
	evaluating $p$ on the roots of $f$, $\alpha_1,\cdots, \alpha_n$ (counting multiplicity), 
	$p(\alpha_1,\cdots,\alpha_n)$.  Note that the basis 
	elements of $F[u_1,\cdots,u_n]$ are the monomials $\prod_{i=1}^n u_i^{c_i}, c_i \in
	 \N \cup \{0\}$.  If one evaluates $u_i \mapsto \alpha_i$ we find that the basis now 
	 goes down to at most $n!$ elements, as any power of a root maps to a linear 
	 combination of other roots, giving us a total of $n!$ potential items. 
	\fi 
	\item[16.3.2]
	\begin{enumerate}
		\item[b] Note that  $x^4 -1 =(x^2 + 1)(x-1)(x+1)$ over $\Q$.  Since we only need
		the splitting field of $x^2 + 1$ then we only need to adjoin $i$, making the extension
		degree 2.
		\item[c] For $x^4 +1$, we can see that $\sqrt{i}$ satisfies this equation.  Since 
		$\sqrt{i}$ is not in the extension $\Q(i)$ then we have the further degree two extension
		of $\Q(\sqrt{i})$.  Since we have two degree 2 extensions, then the total extension 
		is of degree 4.   
	\end{enumerate}
	\item[16.3.3] Let $F = \mathbb{F}_2 (u)$, where $F$ is the field of rational functions.  
	We want to show that $x^2 - u$ is irreducible.  Note that if $x^2 - u$ is reducible then 
	there exists $f,g \in F$ such that $(x-f)(x-g) = x^2 - u$, with $fg = -u, f + g = 0$.  
	Therefore $g = \frac{-u}{f}, f^2 = u$.  This implies that $f = \sqrt{u}$, however this is 
	a contradiction as $u$ is a transcendental element, and $\sqrt{u}$ is not defined in 
	$F$.  If we however consider the field extension $F(i\sqrt{u})$ then 
	$(x + i\sqrt{u})^2 = x^2 + 2i\sqrt{u}x - u = x^2 - u$, demonstrating a double root.  
	\item[16.4.1 (a)] For $\Q(\sqrt[3]{2})$, there is either 1 or 3 automorphisms, 
	and since $x^3 - 2$ is the irreducible polynomial of $\sqrt[3]{2}$ then an automorphism 
	must map to another root.  Since the other roots of $x^3 - 2$ are complex, and 
	$\Q(\sqrt[3]{2}) \subset \R$ then the only automorphism is the identity, and there is 
	exactly 1.  For $\Q(\sqrt[3]{2},\omega)$, since this is the splitting field for 
	$x^3 -2$ implies that there must be 6 automorphisms.  Futhermore $\Q(\sqrt[3]{2},\omega)$
	must contain all of the automorphisms of $\Q(\omega)$, in which there are 2, 
	$\{id, \omega \mapsto \omega^2\}$.  Thus to exhaustively find
	the galois group $G(\Q(\sqrt[3]{2},\omega),\Q)$, we need to find the 3 automorphisms 
	within $G(\Q(\sqrt[3]{2},\omega),\Q(\omega)$.  Note that the only automorphism which leaves
	$\{1,\omega,\omega^2\}$ untouched is $\sqrt[3]{2} \mapsto \sqrt[3]{2} \omega$ and composing 		it once more, $\sqrt[3]{2} \mapsto \sqrt[3]{2} \omega^2$.  Note that if we take first 
	$\omega \mapsto \omega^2 $ and then $\sqrt[3]{2} \mapsto \sqrt[3]{2} \omega$ we get 
	that $\sqrt[3]{2} \omega^2 \mapsto \sqrt[3]{2} \omega^2$, however if done in the opposite 
	order we get that $\sqrt[3]{2} \mapsto \sqrt[3]{2} \omega^2$.  Thus our group of 
	automorphisms is isomorphic to the non-abelian group of order 6, which would be $S_3$.  
	This means that all of the automorphisms simply permute the three roots.    
	\iffalse Note that for $\Q(\sqrt[3]{2})$ there are 3 automorphisms, \\
	$G(\Q(\sqrt[3]{2})/\Q) = \{id, \sqrt[3]{2} \mapsto -\sqrt[3]{2}, 
	\sqrt[3]{2^2} \mapsto -\sqrt[3]{2^2} \}$.  For $\Q(\sqrt[3]{2},\omega)$ there are 6 
	automorphisms, as this is the splitting field for $x^3 - 2$ and 
	$[\Q(\sqrt[3]{2},\omega):\Q] = 6$ as seen previously in the textbook.  Note that 
	$\omega \mapsto \omega^2$ fixes $\Q$ as $(\omega^2)^2 = \omega$, thus only flipping the 
	non-rational complex terms.  Therefore 
	$G(\Q(\sqrt[3]{2},\omega)/\Q) = \{id, \sqrt[3]{2} \mapsto -\sqrt[3]{2}, 
	\sqrt[3]{2^2} \mapsto -\sqrt[3]{2^2}, \omega \mapsto \omega^2,  
	\sqrt[3]{2} \mapsto \sqrt[3]{2} \omega, \sqrt[3]{2} \mapsto \sqrt[3]{2^2} \omega\}$
	\fi
\end{enumerate}
\end{document}
