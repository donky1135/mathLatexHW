\documentclass[12pt, letterpaper]{article}
\date{\today}
\usepackage[margin=1in]{geometry}
\usepackage{amsmath}
\usepackage{hyperref}
\usepackage{cancel}
\usepackage{amssymb}
\usepackage{fancyhdr}
\usepackage{pgfplots}
\usepackage{booktabs}
\usepackage{pifont}
\usepackage{amsthm,latexsym,amsfonts,graphicx,epsfig,comment}
\pgfplotsset{compat=1.16}
\usepackage{xcolor}
\usepackage{tikz}
\usetikzlibrary{shapes.geometric}
\usetikzlibrary{arrows.meta,arrows}
\newcommand{\Z}{\mathbb{Z}}
\newcommand{\N}{\mathbb{N}}
\newcommand{\R}{\mathbb{R}}
\newcommand{\Q}{\mathbb{Q}}
\newcommand{\C}{\mathbb{C}}

\newcommand{\Po}{\mathcal{P}}
\newcommand{\Pro}{\mathbb{P}}
\author{Alex Valentino}
\title{452 homework}
\pagestyle{fancy}
\renewcommand{\headrulewidth}{0pt}
\renewcommand{\footrulewidth}{0pt}
\fancyhf{}
\rhead{
	Homework 11\\
	452	
}
\lhead{
	Alex Valentino\\
}
\begin{document}
\begin{enumerate}
	\item[10.4.3b]
	\item[10.7.4] Let $G$ be a group, $\rho$ a representation, $C$ a conjugacy class, and 
	$T = \sum_{g \in C}\rho_g$.  We want to show 
	that $T$ is $G$-invariant.  Therefore 
	for a group element $h\in G$, we have that 
	$\rho_h(T) = \rho_h(\sum_{g \in C} \rho_{g}) = \sum_{g \in C} \rho_{hg}$.  Note for each $hg$, there exists 
	a unique $g' \in C$ such that $hg = g'h$, since $C$ is a conjugacy class.  Therefore 
	the sum is equivalent to 
	$\sum_{g' \in C} \rho_{g'h} = \sum_{g' \in C} \rho_{g'} \rho_h = T(\rho_h)$.  Therefore $T$ is $G$-invariant 
	\item[11.1.8b] What are the units in $\Z/8\Z$?  
	If $n = 2k$, then $4\cdot 2k \equiv 0 \mod{8}$, thus 
	all even elements are zero divisors.  If $\gcd(n,8) = 
	1$, then by Bezout's lemma there exists $x,y \in \Z$
	such that $nx + 8y = 1$, therefore $nx\equiv 1 \mod{8}$.
	Since $8=2^3$, then $n$ must be odd.  Thus the 
	units of $\Z/8\Z$ is $\{1,3,5,7\}$.  
	\item[11.3.2] Let $a \subset \Z[i]$ be a non-zero ideal.
	Then there exists $x,y\in \Z$ with both not 
	equal to 0 such that 
	$x+ iy \in a$.  Therefore 
	$(x-iy)\cdot (x+iy) = x^2 + y^2 \in a$.  Since at least 
	(WLOG) $x$ is non-zero, $x^2$ is a non-zero integer.
	Thus $a$ has a non-zero integer.    	
	\item[11.3.9]
	\begin{enumerate}
		\item Let $x$ be nilpotent, therefore there 
		exists $n \in \N$ such that $x^n = 0$.  We 
		want to find $a \in R$ such that 
		$a(1+x) = 1$.  I claim that 
		$a = 1 - x + x^2 - x^3 + x^4 + \cdots + 
		(-1)^{n-1} x^{n-1} = \sum_{i=0}^{n-1} (-1)^i x^i$.
		Observe that 
		\begin{align*}
		a(1+x) &= (1+x)\sum_{i=0}^{n-1} (-1)^i x^i\\
		&= 
		\sum_{i=0}^{n-1} (-1)^i x^i + \sum_{i=0}^{n-1} (-1)^i x^{i+1}\\
		 &= 1 +  \sum_{i=1}^{n-1} (-1)^i x^i + 
		\sum_{i=1}^{n} (-1)^{i+1} x^i \\
		&= 1 +  \sum_{i=1}^{n-1} (-1)^i x^i + 
		\sum_{i=1}^{n-1} (-1)^{i+1} x^i \\
		&= 1
		\end{align*}
		The final line works since $x^n = 0$.  Thus 
		$1+x$ is a unit.  
		\item Let $R$ be a ring with prime 
		characteristic $p$, and let $a \in R$ be a 
		nilpotent element with $n\in \N$ such that 
		$a^n = 0$.  We want to show there exists 
		$k \in \N$ such that $(1+a)^k = 1$.  
		We claim that $k=n \cdot p$.  Observe that 
		if $0 < l < pp$ then $\binom{p}{l} \mid p$ since 
		both $l, p-l < p$, therefore $l!, (p-l)!$ do not 
		contain the prime factor $p$.  Thus 
		$\frac{p!}{l!(p-l)!} \mid p$.  Therefore 
		$(1+a)^p = \sum_{l=0}^p \binom{p}{l} a^l = 1 + a^p
		$
		
	\end{enumerate}
\end{enumerate}
\end{document}
