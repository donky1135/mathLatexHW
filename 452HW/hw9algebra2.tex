<<<<<<< Updated upstream
\documentclass[12pt, letterpaper]{article}
\date{\today}
\usepackage[margin=1in]{geometry}
\usepackage{amsmath}
\usepackage{hyperref}
\usepackage{cancel}
\usepackage{amssymb}
\usepackage{fancyhdr}
\usepackage{pgfplots}
\usepackage{booktabs}
\usepackage{pifont}
\usepackage{amsthm,latexsym,amsfonts,graphicx,epsfig,comment}
\pgfplotsset{compat=1.16}
\usepackage{xcolor}
\usepackage{tikz}
\usetikzlibrary{shapes.geometric}
\usetikzlibrary{arrows.meta,arrows}
\newcommand{\Z}{\mathbb{Z}}
\newcommand{\N}{\mathbb{N}}
\newcommand{\R}{\mathbb{R}}
\newcommand{\Q}{\mathbb{Q}}
\newcommand{\C}{\mathbb{C}}

\newcommand{\Po}{\mathcal{P}}
\newcommand{\Pro}{\mathbb{P}}
\author{Alex Valentino}
\title{452 homework}
\pagestyle{fancy}
\renewcommand{\headrulewidth}{0pt}
\renewcommand{\footrulewidth}{0pt}
\fancyhf{}
\rhead{
	Homework 9\\
	452	
}
\lhead{
	Alex Valentino\\
}
\begin{document}
\begin{enumerate}
	\item[1.2]
	\begin{enumerate}
		\item We know that $O$ represents all rotational symmetry 
		of the octoheadron.  Additionally $O \cong S_4$, and 
		$S_4$ is generated via $(1234)$ and $(12)$.  Additionally 
		the isomorphism between $O$ and $S_4$ is via the 
		group action of $O$ on the diagonal pairs of faces.  
		Therefore we need two matrices which rotate cycle all 
		four diagonals and one which cycles just two.  
		The 4 cycle is $R_{(1234)} = \begin{bmatrix}
		0 & -1 & 0\\ 1 & 0 & 0\\ 0 & 0 & 1
\end{bmatrix}		 $, which is equvalent to a 90 degree rotation around 
$e_3$ in $\R^3$, and we assume that the octoheadron is the 
$L^1$ sphere in $\R^3$ with the vertices aligned with the standard basis vectors.  For the 2-cycle we need to flip 2 faces, and this 
is achieved via rotating the octoheadron 45 degrees in the 
xy plane, flipping around $e_2$, then rotating 45 degrees back. 
This is represented via
$R_{12} = \begin{bmatrix}
\frac{1}{\sqrt{2}} & -\frac{1}{\sqrt{2}} & 0 \\
\frac{1}{\sqrt{2}} & \frac{1}{\sqrt{2}} & 0\\
0 & 0 & 1
\end{bmatrix} \begin{bmatrix}
-1 & 0 & 0\\ 
0 & 1 & 0\\
0 & 0 & -1
\end{bmatrix}\begin{bmatrix}

\frac{1}{\sqrt{2}} & \frac{1}{\sqrt{2}} & 0 \\
-\frac{1}{\sqrt{2}} & \frac{1}{\sqrt{2}} & 0\\
0 & 0 & 1
\end{bmatrix} = \begin{bmatrix}
0 & -1 & 0\\ -1 & 0 & 0\\ 0 & 0 & -1
\end{bmatrix} $.  Therefore, since we have the generators of $O$, 
then the group generated by $<R_{(1234)}, R_{(12)}>$ is the 
standard representation.  
		\item For $D_n$, we know that we're operating on points in 
		$\R^2$, therefore for generators $x,y$ where 
		$x^n = 1$ and $y^2 =1$ we have that 
		$R_x = \begin{bmatrix}
		\cos(\frac{2\pi}{n}) & -\sin(\frac{2\pi}{n})\\
		\sin(\frac{2\pi}{n}) & \cos(\frac{2\pi}{n})
\end{bmatrix}, R_y = \begin{bmatrix}
1 & 0\\ 0 & -1
\end{bmatrix} $
	\end{enumerate}
	\item[2.1] 
\end{enumerate}
\end{document}
=======
\documentclass[12pt, letterpaper]{article}
\date{\today}
\usepackage[margin=1in]{geometry}
\usepackage{amsmath}
\usepackage{hyperref}
\usepackage{cancel}
\usepackage{amssymb}
\usepackage{fancyhdr}
\usepackage{pgfplots}
\usepackage{booktabs}
\usepackage{pifont}
\usepackage{amsthm,latexsym,amsfonts,graphicx,epsfig,comment}
\pgfplotsset{compat=1.16}
\usepackage{xcolor}
\usepackage{tikz}
\usetikzlibrary{shapes.geometric}
\usetikzlibrary{arrows.meta,arrows}
\newcommand{\Z}{\mathbb{Z}}
\newcommand{\N}{\mathbb{N}}
\newcommand{\R}{\mathbb{R}}
\newcommand{\Q}{\mathbb{Q}}
\newcommand{\C}{\mathbb{C}}

\newcommand{\Po}{\mathcal{P}}
\newcommand{\Pro}{\mathbb{P}}
\author{Alex Valentino}
\title{452 homework}
\pagestyle{fancy}
\renewcommand{\headrulewidth}{0pt}
\renewcommand{\footrulewidth}{0pt}
\fancyhf{}
\rhead{
	Homework 9\\
	452	
}
\lhead{
	Alex Valentino\\
}
\begin{document}
\begin{enumerate}
	\item[1.2]
	\begin{enumerate}
		\item We know that $O$ represents all rotational symmetry 
		of the octoheadron.  Additionally $O \cong S_4$, and 
		$S_4$ is generated via $(1234)$ and $(12)$.  Additionally 
		the isomorphism between $O$ and $S_4$ is via the 
		group action of $O$ on the diagonal pairs of faces.  
		Therefore we need two matrices which rotate cycle all 
		four diagonals and one which cycles just two.  
		The 4 cycle is $R_{(1234)} = \begin{bmatrix}
		0 & -1 & 0\\ 1 & 0 & 0\\ 0 & 0 & 1
\end{bmatrix}		 $, which is equvalent to a 90 degree rotation around 
$e_3$ in $\R^3$, and we assume that the octoheadron is the 
$L^1$ sphere in $\R^3$ with the vertices aligned with the standard basis vectors.  For the 2-cycle we need to flip 2 faces, and this 
is achieved via rotating the octoheadron 45 degrees in the 
xy plane, flipping around $e_2$, then rotating 45 degrees back. 
This is represented via
$R_{12} = \begin{bmatrix}
\frac{1}{\sqrt{2}} & -\frac{1}{\sqrt{2}} & 0 \\
\frac{1}{\sqrt{2}} & \frac{1}{\sqrt{2}} & 0\\
0 & 0 & 1
\end{bmatrix} \begin{bmatrix}
-1 & 0 & 0\\ 
0 & 1 & 0\\
0 & 0 & -1
\end{bmatrix}\begin{bmatrix}

\frac{1}{\sqrt{2}} & \frac{1}{\sqrt{2}} & 0 \\
-\frac{1}{\sqrt{2}} & \frac{1}{\sqrt{2}} & 0\\
0 & 0 & 1
\end{bmatrix} = \begin{bmatrix}
0 & -1 & 0\\ -1 & 0 & 0\\ 0 & 0 & -1
\end{bmatrix} $.  Therefore, since we have the generators of $O$, 
then the group generated by $<R_{(1234)}, R_{(12)}>$ is the 
standard representation.  
		\item For $D_n$, we know that we're operating on points in 
		$\R^2$, therefore for generators $x,y$ where 
		$x^n = 1$ and $y^2 =1$ we have that 
		$R_x = \begin{bmatrix}
		\cos(\frac{2\pi}{n}) & -\sin(\frac{2\pi}{n})\\
		\sin(\frac{2\pi}{n}) & \cos(\frac{2\pi}{n})
\end{bmatrix}, R_y = \begin{bmatrix}
1 & 0\\ 0 & -1
\end{bmatrix} $
	\end{enumerate}
	\item[2.1] 
\end{enumerate}
\end{document}
>>>>>>> Stashed changes
