\documentclass[12pt, letterpaper]{article}
\date{\today}
\usepackage[margin=1in]{geometry}
\usepackage{amsmath}
\usepackage{hyperref}
\usepackage{cancel}
\usepackage{amssymb}
\usepackage{fancyhdr}
\usepackage{pgfplots}
\usepackage{booktabs}
\usepackage{pifont}
\usepackage{amsthm,latexsym,amsfonts,graphicx,epsfig,comment}
\pgfplotsset{compat=1.16}
\usepackage{xcolor}
\usepackage{tikz}
\usetikzlibrary{shapes.geometric}
\usetikzlibrary{arrows.meta,arrows}
\newcommand{\Z}{\mathbb{Z}}
\newcommand{\N}{\mathbb{N}}
\newcommand{\R}{\mathbb{R}}
\newcommand{\Q}{\mathbb{Q}}
\newcommand{\C}{\mathbb{C}}

\newcommand{\Po}{\mathcal{P}}
\newcommand{\Pro}{\mathbb{P}}
\author{Alex Valentino}
\title{452 homework}
\pagestyle{fancy}
\renewcommand{\headrulewidth}{0pt}
\renewcommand{\footrulewidth}{0pt}
\fancyhf{}
\rhead{
	Homework 3\\
	452	
}
\lhead{
	Alex Valentino\\
}
\begin{document}
\begin{enumerate}
	\item[7.5]
	$$
		x^9 - x = x(x+1)(x+2)(x^2+1)(x^2+x-1)(x^2-x-1)
	$$
	\item[7.7] Suppose $K$ is a finite field, and that there exists a prime $p$ and $r \in \N$
	such that $|K| = p^r$.  We must consider $\pi = \prod_{k \in K^*} k$.  Note that for each 
	element $k \in K^*$ which doesn't satisfy $x^2 - 1 = 0$ has a unique inverse other than
	itself.  Therfore $\prod_{k \in K^*, k^2 \neq 1} k = 1$.  Note that if we multiply the
	previous product by $(-1)(1)$ we get $\pi$.  Thus $\pi = -1$. 
	\item[7.8] Let $f(x) = x^3 + x + 1, g(x) = x^3 + x^2 +1, f(\alpha) = 0, g(\beta) = 0, 
	\mathbb{F}_2(\alpha) = K, \mathbb{F}_2(\beta) = L$.  We want to construct 
	$\sigma : K \to L$ such that $\sigma$ is an isomorphism. Note that 
	$$
	(\alpha + 1)^3 + (\alpha + 1)^2 + 1 = 
	\alpha^3 + \alpha^2 + \alpha + 1 + \alpha^2 + 1 + 1 = \alpha^3 + \alpha + 1 = 0.
	$$
	Therefore $g(\alpha+1) = 0$, thus we know by the textbook that $\sigma(\alpha + 1) = \beta$
	is an isomorphism since $\alpha + 1, \beta$ are both roots of the irreducible polynomial of
	$g$.  Furthermore since both $K,L$ are isomorphic to $\mathbb{F}_8$ then we only need to ask
	the number of automorphisms of $\mathbb{F}_8$.  If we consider the basis of $\mathbb{F}_8$
	to be $(1,\alpha,\alpha^2)$, then we just have the isomorphisms of $(1)$ doing nothing, 
	$(2)$, multiplying by $\alpha$ yielding $(\alpha,\alpha^2,\alpha + 1)$, and $(3)$ 
	multiplying by $\alpha^2$ yielding $(\alpha^2, \alpha + 1, \alpha^2 + \alpha)$.  Note 
	that we must maintain at least one root, and since there are $3$ roots, and that all cases
	are covered by swapping the roots which in turn swaps the other implies that there are 3 
	automorphisms.
	\item[Bonus] Let $P(x) = \prod_{i=1}^n (x - \alpha_i)$.  By the defintion of the formal 
	derivative we have by the product rule that 
	$P'(x) = \sum_{i=1}^n \prod_{j=1, i \neq j}^n (x-\alpha_j)$.  Note that 
	$P'(\alpha_i) =  \prod_{j=1, i \neq j}^n (\alpha_i-\alpha_j)$, since all of the other 
	$n-1$ terms in the series contain $(x-\alpha_i)$, thus going to 0.  Therefore 
	$\prod_{i=1}^n P'(\alpha_i) = \prod_{i=1}^n \prod_{j=1, i \neq j}^n (\alpha_i-\alpha_j)$.
	Note that the normal discriminant is strictly positive, therefore for each 
	$(\alpha_i-\alpha_j)$ there exists $(\alpha_j-\alpha_i)$ with a negative sign.  Since there 
	are $nC2 = \frac{n(n+1)}{2}$ pairings one must multiply by $(-1)^{\frac{n(n+1)}{2}}$ to 
	counteract the signs of the negative pairings.  Note that additionally once the sign 
	issue is taken care of that we now have paired every $(\alpha_i-\alpha_j)$ with 
	$(\alpha_j-\alpha_i)$, thus we have eliminated nearly half of the terms in the product 
	therefore the formulation above is equivalent to $\prod_{i < j}^n (\alpha_i - \alpha_j)^2$.
\end{enumerate}
\end{document}
