\documentclass[12pt, letterpaper]{article}
\date{\today}
\usepackage[margin=1in]{geometry}
\usepackage{amsmath}
\usepackage{hyperref}
\usepackage{cancel}
\usepackage{amssymb}
\usepackage{fancyhdr}
\usepackage{pgfplots}
\usepackage{booktabs}
\usepackage{pifont}
\usepackage{amsthm,latexsym,amsfonts,graphicx,epsfig,comment}
\pgfplotsset{compat=1.16}
\usepackage{xcolor}
\usepackage{tikz}
\usetikzlibrary{shapes.geometric}
\usetikzlibrary{arrows.meta,arrows}
\newcommand{\Z}{\mathbb{Z}}
\newcommand{\N}{\mathbb{N}}
\newcommand{\R}{\mathbb{R}}
\newcommand{\Q}{\mathbb{Q}}
\newcommand{\C}{\mathbb{C}}

\newcommand{\Po}{\mathcal{P}}
\newcommand{\Pro}{\mathbb{P}}
\author{Alex Valentino}
\title{452 homework}
\pagestyle{fancy}
\renewcommand{\headrulewidth}{0pt}
\renewcommand{\footrulewidth}{0pt}
\fancyhf{}
\rhead{
	Homework 2\\
	452	
}
\lhead{
	Alex Valentino\\
}
\begin{document}
\begin{enumerate}
	\item[4.1] Note that the smallest number of powers of $(\alpha^2+1)^i$ to be linearly 
	dependent is $4$, with $(\alpha^2+1)^2 = 3\alpha^2 + \alpha + 1, 
	(\alpha^2+1)^3 = 7 \alpha^2 + 5\alpha + 2, (\alpha^2+1)^4 = 16\alpha^2 + 17\alpha +1$.
	In order to figure out the irreducible polynomial of $\alpha^2 + 1$ we need to figure out
	what is the smallest number of powers of our root which make  a non-trivial linear
	combination.  Since the first 3 powers are clearly linearly independent, then the fourth 
	power ensures that they are linearly dependent.  Thus to compute the coefficents of our 
	polynomial we simply row reduce the following matrix to obtain the coefficents:
	$\begin{bmatrix}
		1 & 1 & 2 & 7 \\
 0 & 1 & 5 & 17 \\
 1 & 3 & 7 & 16 \\
	\end{bmatrix}.$
	After reduction we obtain a polynomial with $\alpha^2 + 1$ as  a root, 
	$x^4 -5x^3+8x^2 -5x$.  We know that $deg_\Q(\alpha^2+1)=3$, since $a^2 + 1 \in \Q(\alpha)$,
	and $deg_\Q(\alpha) = 3$.  Therefore if we factor out an $x$ we get a degree three 
	polynomial $x^3-5x^2+8x-5$.  
	\item[4.2a] Note that $
	(\sqrt{3} + \sqrt{5})^2 = 8 + 2\sqrt{15}$, $(\sqrt{3} + \sqrt{5})^3= 
	(8 + 2\sqrt{15})(\sqrt{3} + \sqrt{5}) = 8\sqrt{3} + 8\sqrt{5} + 6\sqrt{5} + 10 \sqrt{3} = 
	18\sqrt{3} + 14\sqrt{5}$, $(\sqrt{3} + \sqrt{5})^4 = (18\sqrt{3} + 14\sqrt{5})(\sqrt{3} + \sqrt{5}) = 54 + 70 + 32\sqrt{15} = 124 + 32\sqrt{15}$.  
	
	Since $(\sqrt{3} + \sqrt{5})^4 - 16(\sqrt{3} + \sqrt{5})^2 = -4$, then $x^4 - 16x^2 + 4$ 
	is a minimal polynomial for $\sqrt{3} + \sqrt{5}$.  Additionally  
	$[\Q(\sqrt{3},\sqrt{5}) : \Q] = 4$ since by the previous homework trivially $x^2 - 5$ is 
	irreducible over $\Q(\sqrt{3})$ then that implies that 
	$[\Q(\sqrt{3},\sqrt{5}) : \Q(\sqrt{3})]=2$.  Therefore $[\Q(\sqrt{3},\sqrt{5}) : \Q] = 4$
	which means that we have found the lowest degree polynomial.  
	\item[5.2a] To construct the regular pentagon is equivalent to constructing the 5th roots 
	of unity, or to find the solutions to the equation $x^5 = 1$.  Note that trivially 1 is 
	a root of unity so we must consider showing that the roots of $x^4 + x^3 + x^2 + x + 1$
	are constructable. We claim that for $\zeta = e^{2\pi i/5}$ that $\Q(\zeta + \bar{\zeta})$ 
	is an intermediate step in between $\Q$ and $\Q(\zeta)$.  Note that since 
	$\bar{\zeta} = \zeta^4$, then $(\zeta + \bar{\zeta})\zeta = \zeta^2 + 1$.  Since we 
	have found a degree two linear combination of $\zeta$ in $\Q(\zeta+\bar{\zeta})$ then 
	we have found a tower whose indices from one field extension to another is 2.  Therefore
	$\zeta$ is constructible. 
	\item[5.4] Given a triangle $ABC$ in the plane, we want to show that it is possible to 
	construct a square with the same area. This is equivalent to taking the square root of 
	the area of the triangle.  For a given triangle, it is always possible to drop an altitude,
	say from point $A$ to the intersection point $P$, as it is always possible to construct a 
	line perpendicular to a given line through a point not on the line.  At this point we have
	lines $BC$ and $AP$, in which $\frac{1}{2}AB*CB$ is the area of the triangle.  Since 
	$\frac{1}{2}$ is a constructible number, and so are the two line segments mentioned before,
	we can multiply the two lengths and then halve them.  Finally we are allowed to take a 
	square root by the method outlined in the artin chapter.  Therefore we can construct a 
	square with the same area as the triangle.  
	\item[6.1] Let $F$ be a field with $char(F) = 0, f \in F[x]$, let $f'$ be the 
	formal derivative of $f$, and let $g \mid f, g \mid f'$.  We want to show that $g^2 \mid f$.
	Assuming that $f$ is non-constant, let $f(x) = a(x)g(x), f'(x) = b(x)g(x)$.  
	Then $b(x)g(x) = f'(x) = (a(x)g(x))' = a'(x)g(x) + a(x)g'(x)$.  Since $g | f'$ then 
	$g|a g'$.  Since $deg(g') < deg(g)$ then $g \nmid g'$.  Additionally since $g$ is 
	irreducible then it is prime since $F[x]$ is a UFD.  Thus $g \mid a$.  Additionally 
	we can have $f' \neq 0$ since $char(F) = 0$ ensuring that $(x^p)' = px^{p-1} \neq 0$.
	Therefore $g^2 \mid f$.  
\end{enumerate}
\end{document}
