\documentclass[12pt, letterpaper]{article}
\date{\today}
\usepackage[margin=1in]{geometry}
\usepackage{amsmath}
\usepackage{hyperref}
\usepackage{cancel}
\usepackage{amssymb}
\usepackage{fancyhdr}
\usepackage{pgfplots}
\usepackage{booktabs}
\usepackage{pifont}
\usepackage{amsthm,latexsym,amsfonts,graphicx,epsfig,comment}
\pgfplotsset{compat=1.16}
\usepackage{xcolor}
\usepackage{tikz}
\usetikzlibrary{shapes.geometric}
\usetikzlibrary{arrows.meta,arrows}
\newcommand{\Z}{\mathbb{Z}}
\newcommand{\N}{\mathbb{N}}
\newcommand{\R}{\mathbb{R}}
\newcommand{\Q}{\mathbb{Q}}
\newcommand{\C}{\mathbb{C}}

\newcommand{\Po}{\mathcal{P}}
\newcommand{\Pro}{\mathbb{P}}
\author{Alex Valentino}
\title{452 homework}
\pagestyle{fancy}
\renewcommand{\headrulewidth}{0pt}
\renewcommand{\footrulewidth}{0pt}
\fancyhf{}
\rhead{
	Homework 5\\
	452	
}
\lhead{
	Alex Valentino\\
}
\begin{document}
\begin{enumerate}
	\item[16.4.1(b)] Note that 
	$$x^2 - 2x - 1 = (x-(1+ \sqrt{2}))(x-(1- \sqrt{2})), 
	x^2 - 2x - 1 = (x-(1+ 2\sqrt{2}))(x-(1- 2\sqrt{2})),$$
	thus the roots are contained within $\Q(\sqrt{2})$, and since 
	$\Q(\sqrt{2})$ is a galois extension for $f$ implies that 
	$[\Q(\sqrt{2}): \Q] = |Gal(\Q(\sqrt{2})/ \Q)|$.   	
	\item[16.6.1] For the equation $x^3 + x + 1$, $\Delta_f = -31$.  
	Since $deg(\alpha) = 3$, this implies that $[\Q(\alpha) : \Q] = 3$,
	which means that the order of the galois group $G(\Q(\alpha)/\Q)$ is either 1 
	or 3.  Since $\sqrt{-31}$ needs an extension of degree 2, then 
	it is not contained in  $\Q(\alpha)$.  However, for the splitting 
	field $K$, the square root of the discriminate is guaranteed to 
	be within as $\sqrt{\Delta_f} = (\alpha_1 - \alpha_2)(\alpha_2 - \alpha_3)(\alpha_1 - \alpha_3)$, which is just the product and 
	difference of the roots.  Therefore  
	$\sqrt{-31} = \sqrt{\Delta_f} \in K$.  
	\item[16.6.2] Note that we have the inherited automorphisms 
	from $\Q(\sqrt{2}),\Q(\sqrt{3}),\Q(\sqrt{5})$ of 
	$\sqrt{p} \mapsto -\sqrt{p}$ where $p = 2,3,5$.  Furthermore, 
	we have $8 = [\Q(\sqrt{2},\sqrt{3},\sqrt{5}):\Q]$ as 
	we have $3$ items being adjoined to $\Q$, and there is $2^3 = 8$
	distinct elements which can be created by multiplying them together,
	represented by $\sqrt{2}^{b_0} \sqrt{3}^{b_1} \sqrt{5}^{b_2}$, 
	where $b_i = 0,1, i=0,1,2$.  Furthermore, we can generate other 
	automorphisms by chaining the swapping of signs of different roots.
	Furthermore, the swapping of signs is commutative.  Finally 
	our 3 inherited automorphisms each generate a subgroup of order 2.
	Therefore the only possible galois group is $(\Z/2\Z)^3$, or
	the field on 8 elements.  	 
	\item[16.7.2]
	\begin{enumerate}
		\item[b] $[F:L] = 9$ cannot occur since 
		$[K : L][L : F] = [K:F] = |G(K/f)| = 24$, and $9 \nmid 24$.
		\item[c] Note that $C_2 \times C_{12} \cong C_2 \times C_3 
		\times C_4$ by the chinese remainder theorem.  
		Since the following is a direct product, 
		this implies that $C_4$ is normal.  Therefore there is exactly
		one copy of $C_4$, implying there is one intermediate field 
		which has galois group $C_4$.  
	\end{enumerate}
	\item[16.7.4] Subfields of  $\Q(\sqrt{2},\sqrt{3},\sqrt{5})$:
	\begin{enumerate}
		\item $\Q	$
		\item $\Q(\sqrt{2})$
		\item $\Q(\sqrt{3})$
		\item $\Q(\sqrt{5})$
		\item $\Q(\sqrt{6})$
		\item $\Q(\sqrt{10})$
		\item $\Q(\sqrt{15})$
		\item $\Q(\sqrt{30})$
		\item $\Q(\sqrt{2},\sqrt{3})$
		\item $\Q(\sqrt{2},\sqrt{5})$
		\item $\Q(\sqrt{3},\sqrt{5})$
		\item $\Q(\sqrt{5},\sqrt{6})$
		\item $\Q(\sqrt{10},\sqrt{3})$
		\item $\Q(\sqrt{15},\sqrt{2})$
		\item $\Q(\sqrt{2},\sqrt{3},\sqrt{5})$
	\end{enumerate}
\end{enumerate}
\end{document}
