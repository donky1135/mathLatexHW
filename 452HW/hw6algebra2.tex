\documentclass[12pt, letterpaper]{article}
\date{\today}
\usepackage[margin=1in]{geometry}
\usepackage{amsmath}
\usepackage{hyperref}
\usepackage{cancel}
\usepackage{amssymb}
\usepackage{fancyhdr}
\usepackage{pgfplots}
\usepackage{booktabs}
\usepackage{pifont}
\usepackage{amsthm,latexsym,amsfonts,graphicx,epsfig,comment}
\pgfplotsset{compat=1.16}
\usepackage{xcolor}
\usepackage{tikz}
\usetikzlibrary{shapes.geometric}
\usetikzlibrary{arrows.meta,arrows}
\newcommand{\Z}{\mathbb{Z}}
\newcommand{\N}{\mathbb{N}}
\newcommand{\R}{\mathbb{R}}
\newcommand{\Q}{\mathbb{Q}}
\newcommand{\C}{\mathbb{C}}

\newcommand{\Po}{\mathcal{P}}
\newcommand{\Pro}{\mathbb{P}}
\author{Alex Valentino}
\title{452 homework}
\pagestyle{fancy}
\renewcommand{\headrulewidth}{0pt}
\renewcommand{\footrulewidth}{0pt}
\fancyhf{}
\rhead{
	Homework 6\\
	452	
}
\lhead{
	Alex Valentino\\
}
\begin{document}
\begin{enumerate}
	\item[7.10] Let $G$ be the galois group of the field extension $K/F$, and let $H$ be a subgroup of $G$.  We know 
	by the main theorem of galois theory that there exists $L$ such that $F \subset L \subset K$ and $H = Gal(K/L)$.  
	 
	\item[7.11(b)]
	\item[8.2(d)]
	\item[9.12]
	\begin{enumerate}
		\item[a]
		\item[e]
	\end{enumerate}
\end{enumerate}
\end{document}
