\documentclass[12pt, letterpaper]{article}
\date{\today}
\usepackage[margin=1in]{geometry}
\usepackage{amsmath}
\usepackage{hyperref}
\usepackage{cancel}
\usepackage{amssymb}
\usepackage{fancyhdr}
\usepackage{pgfplots}
\usepackage{booktabs}
\usepackage{pifont}
\usepackage{amsthm,latexsym,amsfonts,graphicx,epsfig,comment}
\pgfplotsset{compat=1.16}
\usepackage{xcolor}
\usepackage{tikz}
\usetikzlibrary{shapes.geometric}
\usetikzlibrary{arrows.meta,arrows}
\newcommand{\Z}{\mathbb{Z}}
\newcommand{\N}{\mathbb{N}}
\newcommand{\R}{\mathbb{R}}
\newcommand{\Q}{\mathbb{Q}}
\newcommand{\C}{\mathbb{C}}

\newcommand{\Po}{\mathcal{P}}
\newcommand{\Pro}{\mathbb{P}}
\author{Alex Valentino}
\title{452 homework}
\pagestyle{fancy}
\renewcommand{\headrulewidth}{0pt}
\renewcommand{\footrulewidth}{0pt}
\fancyhf{}
\rhead{
	Homework 1\\
	452	
}
\lhead{
	Alex Valentino\\
}
\begin{document}
\begin{enumerate}
	\item[1.1] Consider $y \in R \backslash F$, and the function 
	$f:R \to R$ given by $x \mapsto yx$.  Note that since $deg_F(R) = n$ then 
	this implies that $f$ is a linear operator.  Therefore by rank nullity 
	$rank(f) + 0 = rank(f) + nullity(f) = n$.  Thus $f$ is surjective.  Therefore 
	there exists $x' \in R$ such that $f(x') = 1$.  Thus $y$ has an inverse, making 
	every element in $R$ a unit, thus $R$ is a field.  
	\item[2.1] Let $f(x) = x^3 -3x+4$ and let $\alpha \in \C$ satisfy 
	$f(\alpha) = 0$. Note that $a^3 = 3\alpha - 4$, 
	$\alpha^4 = 3\alpha^2 - 4\alpha$, 	
	therefore to find an inverse  $ a \alpha^2 + b \alpha + c \in \Q(\alpha)$,
	we simply have to define a matrix which encodes 
	$(a\alpha^2 + b\alpha + c)(\alpha^2 + \alpha + 1) = 1$.  Note that since
	$\alpha(\alpha^2 + \alpha + 1) = \alpha^2 + 4 \alpha - 4$ and 
	$\alpha^2 (\alpha^2 + \alpha + 1) = 3 \alpha^2 - 4 \alpha + 3 \alpha - 4 + \alpha^2 = 4 \alpha^2 - \alpha - 4$, then by the linearity of multiplying by 
	$(\alpha^2 + \alpha + 1)$ we have the matrix equation to solve
	$$
	\begin{bmatrix}0 \\ 0 \\ 1\end{bmatrix} = \begin{bmatrix}
	4 & 1 & 1\\ -1 & 4 & 1\\ -4 & -4 & 1
	\end{bmatrix}\begin{bmatrix}a \\ b \\ c\end{bmatrix}
	$$
	With the resulting solution being $a = \frac{-3}{49}, b = \frac{-5}{49},c=\frac{17}{49}$
	\item[2.3] Note that the minimal polynomial for $\beta = \sqrt[3]{2}e^{\frac{2\pi i}{3}}$ is $x^3 - 2$.  Note by theorem 15.2.8 there is an isomorphism
	between $\Q$ adjoined roots of an irreducible polynomial which fix $\Q$.
	Therefore since $\sqrt[3]{2}$ is another root of $x^3 -2 $ then there exists
	an isomorphism $\phi \Q(\beta) \to \Q(\sqrt[3]{2}$ in which $\phi(\Q) = \Q$.
	Therefore for the equation $x_1^2 + \cdots x_k^2 = -1$, by applying $\phi$ 
	to it we get $\phi(x_1)^2 + \cdots + \phi(x_k)^2 = -1$.  Note that 
	$\phi(x_i) \in \Q(\sqrt[3]{2} \subset \R$, meaning that 
	$\phi(x_1)^2 + \cdots + \phi(x_k)^2 > 0 > -1$.  Therefore in $\Q(\beta)$ 
	this equation is impossible.  
	\item[3.1] Let $F$ be a field and let $\alpha$ be an algebraic element over
	$F$ such that $[F(\alpha):F] = 5$.  Since the degree of $\alpha$ over $F$ is 
	prime, and since $\alpha^2 \not \in F$ then by corollary 15.3.7 
	$F(\alpha^2) = F(\alpha)$.  
	\item[3.9] Let $\alpha$ be a complex root of $f(x)$, $\beta$ be is a complex
	root of $g(x)$, $f(x), g(x)$ are both irreducible over $\Q$, and let 
	$K = \Q(\alpha), L = \Q(\beta)$.  $(\Rightarrow)$ Suppose $f(x)$ is 
	irreducible over $L$, then $deg(f) = [K:Q]$ must be equivalent to 
	$[\Q(\alpha,\beta):L]$ since $f$ is irreducible over $L$ by theorem 15.2.7.
	Therefore since $[\Q(\alpha,\beta):\Q] = [\Q(\alpha,\beta):K][K:\Q] = 
	[\Q(\alpha,\beta):L][L:\Q]$ then you can divide out by $[K:\Q]$ and get
	$[\Q(\alpha,\beta):K] = [L:\Q]$.  Thus $g$ is irreducible over $K$.
	The converse is trivial.  
\end{enumerate}
\end{document}
