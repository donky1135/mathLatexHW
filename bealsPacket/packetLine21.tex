\documentclass[12pt, letterpaper]{article}
\date{\today}
\usepackage[margin=1in]{geometry}
\usepackage{amsmath}
\usepackage{hyperref}
\usepackage{cancel}
\usepackage{amssymb}
\usepackage{fancyhdr}
\usepackage{pgfplots}
\usepackage{booktabs}
\usepackage{pifont}
\usepackage{amsthm,latexsym,amsfonts,graphicx,epsfig,comment}
\pgfplotsset{compat=1.16}
\usepackage{xcolor}
\usepackage{tikz}
\usetikzlibrary{shapes.geometric}
\usetikzlibrary{arrows.meta,arrows}
\newcommand{\Z}{\mathbb{Z}}
\newcommand{\N}{\mathbb{N}}
\newcommand{\R}{\mathbb{R}}
\newcommand{\Q}{\mathbb{Q}}
\newcommand{\Po}{\mathcal{P}}

\author{Alex Valentino}
\title{Beals Packet}
\pagestyle{fancy}
\renewcommand{\headrulewidth}{0pt}
\renewcommand{\footrulewidth}{0pt}
\fancyhf{}
\rhead{
	Line \\
	Beals Summer Packet	
}
\lhead{
	Alex Valentino\\
}
\begin{document}
\begin{enumerate}
	\item[4.4.11] Suppose $g$ is defined on all of $\R$. Show that $g$ is continuous if and only if $g^{-1}(O)$ is open whenever $O \subseteq \R$ is an
open set.
	\begin{itemize}
		\item[$\Rightarrow$] Suppose $g$ is continuous, $O$ is an open set.  If $g^{-1}(O)$ is empty then it vacuously open, so assume $g^{-1}(O)$ is non-empty.  Let $x \in g^{-1}(O)$.  Therefore $f(x) \in O$.  Thus by the definition of an open set
		there exists $\epsilon > 0$ such that $V_\epsilon(f(x)) \subseteq O$.  Since $g$ is continuous then there exists 
		$\delta > 0$ such that for all $y \in V_\delta (x)$ implies $f(y) \in V_\epsilon(f(x))$.  Since every $f(y)$ is contained within $O$, then $ V_\delta (x) \subset g^{-1}(O)$.  Therefore $g^{-1}(O)$ is open.  
		\item[$\Leftarrow$] Assume if $O \subseteq \R$ and $O$ is open implies $g^{-1}(O)$ is open.  We must show that
		$g$ is continuous.  Let $c \in g(O), \epsilon > 0$.  Since $V_\epsilon(c)$ is open then by our initial assumption
		$g^{-1}(V_\epsilon(c))$ is open as well.  Since $c$ is defined to be in the image $g(O)$ then there exists 
		$x \in \R$ such that $c = f(x)$.  Since $g^{-1}(V_\epsilon(c))$ is open then there exists $\delta > 0$ such 
		that $V_\delta(x) \subseteq g^{-1}(V_\epsilon(c))$.  Since $V_\delta(x)$ is a subset of $g^{-1}(V_\epsilon(c))$
		then by the definition of $g^{-1}(V_\epsilon(c))$, $y \in V_\delta(x)$ implies $f(y) \in V_\epsilon(c)$.
		Therefore $g$ is continuous.  
	\end{itemize}

\end{enumerate}
\end{document}
