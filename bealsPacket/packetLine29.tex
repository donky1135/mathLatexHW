\documentclass[12pt, letterpaper]{article}
\date{\today}
\usepackage[margin=1in]{geometry}
\usepackage{amsmath}
\usepackage{hyperref}
\usepackage{cancel}
\usepackage{amssymb}
\usepackage{fancyhdr}
\usepackage{pgfplots}
\usepackage{booktabs}
\usepackage{pifont}
\usepackage{amsthm,latexsym,amsfonts,graphicx,epsfig,comment}
\pgfplotsset{compat=1.16}
\usepackage{xcolor}
\usepackage{tikz}
\usetikzlibrary{shapes.geometric}
\usetikzlibrary{arrows.meta,arrows}
\newcommand{\Z}{\mathbb{Z}}
\newcommand{\N}{\mathbb{N}}
\newcommand{\R}{\mathbb{R}}
\newcommand{\Q}{\mathbb{Q}}
\newcommand{\Po}{\mathcal{P}}

\author{Alex Valentino}
\title{Beals Packet}
\pagestyle{fancy}
\renewcommand{\headrulewidth}{0pt}
\renewcommand{\footrulewidth}{0pt}
\fancyhf{}
\rhead{
	Line 29\\
	Beals Summer Packet	
}
\lhead{
	Alex Valentino\\
}
\begin{document}
\begin{enumerate}
    \item[6.5.9] If $\sum_{n=0}^\infty a_nx^n = \sum_{n=0}^\infty b_nx^n $
    prove that $a_n = b_n$ for all $n \in \N$.\\
    Proof.  Consider $0 = h(x) = \sum_{n=0}^\infty (a_n-b_n) x^n$.  Since both power series are continuous on the interval $(-R,R)$ then $h(x)$ is defined for 0.  Therefore $0 = h(0) = (a_0-b_0) + (a_1-b_1)0 + \cdots = a_0-b_0$.
    Therefore $a_0 = b_0$.  Since each power series is differentiable, and the sum of differentiable functions is differentiable, then we have that
    $0 = h'(0) = (a_1 - b_1) + 2(a_2-b_2)0 + \cdots$.  
    Therefore by the principle of mathematical induction, for all $k \in \N$, 
    if $k < n$, then $a_k = b_k$.  Consider the $n$th derivative of $h(x)$,
    therefore $\frac{d^n}{dx^n}h(x) = \sum_{l=n}^\infty \frac{(l)!}{(l-n)!}(a_l-b_l)x^{l-n}$.  We know by theorem 6.5.7 that convergent power series are 
    infinitely differentiable, since $h(x)$ is defined on $(-R,R)$ then 
    $\frac{d^n}{dx^n}h(0)$ is defined.  Therefore 
    $0 =\frac{d^n}{dx^n}h(0) = n!(a_n-b_n) + (n+1)!(a_n - b_n)0 + \cdots$.
    Thus $0= a_n - b_n$, $a_n = b_n$.  Therefore the power series are equivalent.
\end{enumerate}
\end{document}
