\documentclass[12pt, letterpaper]{article}
\date{\today}
\usepackage[margin=1in]{geometry}
\usepackage{amsmath}
\usepackage{hyperref}
\usepackage{cancel}
\usepackage{amssymb}
\usepackage{fancyhdr}
\usepackage{pgfplots}
\usepackage{booktabs}
\usepackage{pifont}
\usepackage{amsthm,latexsym,amsfonts,graphicx,epsfig,comment}
\pgfplotsset{compat=1.16}
\usepackage{xcolor}
\usepackage{tikz}
\usetikzlibrary{shapes.geometric}
\usetikzlibrary{arrows.meta,arrows}
\newcommand{\Z}{\mathbb{Z}}
\newcommand{\N}{\mathbb{N}}
\newcommand{\R}{\mathbb{R}}
\newcommand{\Po}{\mathcal{P}}

\author{Alex Valentino}
\title{Beals Packet}
\pagestyle{fancy}
\renewcommand{\headrulewidth}{0pt}
\renewcommand{\footrulewidth}{0pt}
\fancyhf{}
\rhead{
	Line 6\\
	Beals Summer Packet	
}
\lhead{
	Alex Valentino\\
}
\begin{document}
	\begin{enumerate}
		\item[2.2.1]
		\begin{enumerate}
			\item Let $\epsilon > 0$ be arbitrary.  Choose $N \in \N$ with $N > \sqrt{\frac{\frac{1}{\epsilon} -1 }{6}}$.  To verify that our choice of $N$ is appropriate, let $n \in \N$ satisfy $n > N$.  Therefore 
			\begin{align*}
				n &> \sqrt{\frac{\frac{1}{\epsilon} -1 }{6}}\\
				n^2 &>  \frac{\frac{1}{\epsilon} -1 }{6}\\
				6n^2 + 1 &>  \frac{1}{\epsilon}\\
				\frac{1}{6n^2 + 1} &< \epsilon.
			\end{align*}
			Since $n^2 > 0$ for all $n \in \N$ then we have just proved $|\frac{1}{6n^2 + 1} | < \epsilon$.
		\end{enumerate}
		\item[2.2.7]
		\begin{enumerate}
		\item 		 The convergence to infinity definition is thus: For all $\epsilon > 0$ there exists $B ]in \N$ such that $n > N$ implies $|x_n| > \epsilon$.\\	
		Let $\epsilon > 0$.  Choose $N \in \N$ such that $N > \epsilon^2$.  To verify that our choice of $N$ is correct, let $n \in \N$ satisfy $n > N$.  Therefore $n > \epsilon^2 \Rightarrow \sqrt{n} > \epsilon$.  Since $\sqrt{n} > 0$ for all $n \in \N$ then we have shown that $\lim_{n \to \infty} |\sqrt{n}| = \infty$
		\item The sequence $(1,0,2,0,3,0,4,\ldots)$ does not converge to infinity as choosing $\epsilon = 0.5$ does not satisfy the condition.  As for all $n \in \N, x_{2n} = 0$ thus no matter how large $N$ is choosen to be $x_{2n} < 0.5$.  
\end{enumerate}
		\item[2.2.8]
		\begin{enumerate}
			\item The sequence $(-1)^n$ is frequently in the set $\{1\}$ as given an arbitrary $N \in \N$ if $N$ is even then the choice $n = N$ yields $(-1)^{2N} = 1$ and if $N$ is odd where $N = 2l - 1$ then $n = N + 1$ yields 
			$(-1)^n = (-1)^{2l + 2} = ((-1)^2)^{l+1} = 1^{l+1} = 1$.  \\
			We claim that the sequence is not eventually in the set $\{1\}$ Supposes $N \in \N$.  If $N$ is odd then there exists $l\in \Z_{\leq 0}$ such that $N = 2l+1$.  Since $N \leq N$ then $(-1)^N = (-1)^{2l + 1} = -1 \not \in \{1\}$. If $N$ is even then we take $N + 1 > N$, therefore $(-1)^{N + 1} = (-1)^{2l + 1} = -1 \not \in \{1\}$.  Since every choice of $N$ results in the sequence not being $1$ after a certain point then it is not eventually in $\{1\}$.
			\item Eventually is a stronger definition then frequently.  As shown above we can have sequences frequently dance in and out of sets, but they are not guaranteed to stay inside after a certain point.  We claim that eventually implies frequently and that the converse is not true:
			\begin{itemize}
				\item Suppose $(a_n)$ is eventually in the set $A$.  Therefore
				 there exists $N_e \in \N$ such that for all $n \geq N_e, a_n \in 
				 A$.  We must show
				 that $(a_n)$ is frequently in $A$.  Suppose $N \in \N$.  We must
				 show there exists $n \geq N$ such that $a_n \in A$.  We claim
				 that all $n \geq N_e$ satisfy the definition.    If $N \leq N_e$
				 then the definition is satisfied by $n = N_e$.  If $N > N_e$ then any natural number $k$ such that $k > N$ would suffice as $k > N > N_e$, therefore 
				 $a_k \in A$.  Therefore $(a_n)$ is frequently in $A$.  
				 \item Frequently does not imply eventually as part (a) of this problem serves as a counterexample.  
			\end{itemize}
			\item The sequence $(a_n)$ converges to the real number $r$ if for all 
			$\epsilon > 0$ $(a_n)$ is eventually in $V_{\epsilon}(r)$.  
			We do not use the frequently definition here as we need the sequence 
			to stay inside of $V_{\epsilon}(r)$ for all members of $a_n$ for 
			$n > N$ has been shown to not be satisfied by the frequently defintion.
			\item $(x_n)$ is not necessarily eventually in the set $(1.9,2.1)$.  If $(x_n)$ has an infinite number of terms that are not in the set $(1.9,2.1)$, even if they are exceedingly rare, then for every $N \in \N$ there will always exists $n \geq N$ for which $x_n \not \in (1.9,2.1)$.  \\
			We claim that $(x_n)$ is frequently in the set $(1.9,2.1)$.  Let $A \subseteq \N$ be the set containing all the indicies where $x_n = 2$.  Note that 
			since there is an infinite number of terms where $x_n = 2$ then $A$ is
			a countable subset of $\N$.  Suppose $N \in \N$.  If $N \in A$ then 
			$N \leq N$, $x_N = 2 \in (1.9,2.1)$.  If $N \not \in A$ then since $A$ is infinite there exists $a \in A$ such that $N < a$. By definition of being a 
			member of $A$ then $x_a = 2\in (1.9,2.1)$.  Therefore $(x_n)$ is frequently in $(1.9,2.1)$.     
		\end{enumerate}
	\end{enumerate}
\end{document}
