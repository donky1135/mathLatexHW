\documentclass[12pt, letterpaper]{article}
\date{\today}
\usepackage[margin=1in]{geometry}
\usepackage{amsmath}
\usepackage{hyperref}
\usepackage{cancel}
\usepackage{amssymb}
\usepackage{fancyhdr}
\usepackage{pgfplots}
\usepackage{booktabs}
\usepackage{pifont}
\usepackage{amsthm,latexsym,amsfonts,graphicx,epsfig,comment}
\pgfplotsset{compat=1.16}
\usepackage{xcolor}
\usepackage{tikz}
\usetikzlibrary{shapes.geometric}
\usetikzlibrary{arrows.meta,arrows}
\newcommand{\Z}{\mathbb{Z}}
\newcommand{\N}{\mathbb{N}}
\newcommand{\R}{\mathbb{R}}
\newcommand{\Q}{\mathbb{Q}}
\newcommand{\Po}{\mathcal{P}}

\author{Alex Valentino}
\title{Beals Packet}
\pagestyle{fancy}
\renewcommand{\headrulewidth}{0pt}
\renewcommand{\footrulewidth}{0pt}
\fancyhf{}
\rhead{
	Line 20\\
	Beals Summer Packet	
}
\lhead{
	Alex Valentino\\
}
\begin{document}
\begin{enumerate}
	\item[4.4.6]
	\begin{enumerate}
		\item A continuous function $f:(0,1) \to \R$ and a cauchy sequence $(x_n)$ such that $f(x_n)$ is not a cauchy sequence.\\
		Consider $f(x) = \sin(\frac{1}{x})$ and the sequence \[x_n = 
		\begin{cases} 
		\frac{1}{\frac{\pi}{2}+2 \pi n} &\text{if $n$ odd}\\
		\frac{1}{\frac{3\pi}{2}+2 \pi n} &\text{if $n$ even}\\  
		\end{cases}\] 
		Clearly $(x_n) \to 0$, however $f(x_n) = (-1)^{n+1}$ alternating between $1$ and $-1$ forever, thus never converging, and therefore not cauchy.
		\item A continuous function $f:[0,1] \to \R$ and a cauchy sequence $(x_n)$ such that $f(x_n)$ is not a cauchy sequence.\\
		This is impossible since $f$ is continuous then we know the sequence in the image converges, and as continuous functions map compact sets to compact sets (theorem 4.4.2), therefore guaranteed convergence within $f([0,1])$, thus making it cauchy\\
		\item A continuous function $f:[0,\infty) \to \R$ and a cauchy sequence $(x_n)$ such that $f(x_n)$ is not a cauchy sequence.\\
		This is impossible since a cauchy sequence is bounded by say a constant $M$, and since the cauchy sequence would be
		strictly non-negative then the closed interval $[0,M]$ would entirely contain the sequence.  Therefore using the reasoning above on a larger compact set we arrive at the same conclusion.
		\item A continuous bounded function $f$ on $(0, 1)$ that attains a maximum value
on this open interval but not a minimum value.\\
		Consider $f(x) = \frac{1}{2}-|x-\frac{1}{2}|$.  By repeated applications of the algebraic continuity theorem, $f$ is continuous for all of $\R$.  $f$ attains its maximum at $x=\frac{1}{2}$.  However it's minimum 
		including limit points occur at $x=0,1$ with a value of 0.  However $f$ with the restriction to $(0,1)$ cannot 
		attain the minimum.     
	\end{enumerate}
\end{enumerate}
\end{document}
