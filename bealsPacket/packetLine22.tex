\documentclass[12pt, letterpaper]{article}
\date{\today}
\usepackage[margin=1in]{geometry}
\usepackage{amsmath}
\usepackage{hyperref}
\usepackage{cancel}
\usepackage{amssymb}
\usepackage{fancyhdr}
\usepackage{pgfplots}
\usepackage{booktabs}
\usepackage{pifont}
\usepackage{amsthm,latexsym,amsfonts,graphicx,epsfig,comment}
\pgfplotsset{compat=1.16}
\usepackage{xcolor}
\usepackage{tikz}
\usetikzlibrary{shapes.geometric}
\usetikzlibrary{arrows.meta,arrows}
\newcommand{\Z}{\mathbb{Z}}
\newcommand{\N}{\mathbb{N}}
\newcommand{\R}{\mathbb{R}}
\newcommand{\Q}{\mathbb{Q}}
\newcommand{\Po}{\mathcal{P}}

\author{Alex Valentino}
\title{Beals Packet}
\pagestyle{fancy}
\renewcommand{\headrulewidth}{0pt}
\renewcommand{\footrulewidth}{0pt}
\fancyhf{}
\rhead{
	Line 22\\
	Beals Summer Packet	
}
\lhead{
	Alex Valentino\\
}
\begin{document}
\begin{enumerate}
	\item[4.5.6] 
	Let $f: [a,b] \to \R$ be continuous, $f(a) < 0 < f(b)$.  We must show there
	exists $c \in [a,b]$ such that $f(c) = 0$.  Let $I_0 = [a,b]$, consider the
	midpoint $z_1 = \frac{a+b}{2}$.  If $f(z_1) > 0$, then set 
	$b_1 = z_1, a_1 = a, I_1 = [a_1, b_1]$, and if $f(z_1) \leq 0$ then set 
	$a_1 = z_1, b_1 = b$.  This lends itself to a general formula for $I_n$, 
	where for $I_{n-1}$ with $z_{n-1} = \frac{a_{n-1} + b_{n-1}}{2}$, then
	if $f(z_{n-1}) > 0$, $I_n = [a_{n-1}, z_{n-1}]$, otherwise 
	$I_n = [z_{n-1}, b_{n-1}]$.  Since we have $I_0 \supseteq I_1 \supseteq \cdots$, then there exists $x \in \bigcap_{n=0}^\infty I_n$. We claim that $(a_n)$
	converges to $x$.  Since each interval is half the length of the previous, then for an arbitrary $\epsilon > 0$, there exists $n \in \N$ such that for all 
	$n \geq N$, $|a_n - x| \leq 2^{-n}(b-a) < \epsilon$.  
	A similar argument exists for $b_n \to x$.  Since $f$ is continuous then 
	$\lim f(a_n) = \lim f(b_n) = f(x)$.  We know by the algebraic limit theorem
	that since $f(a_n) \leq 0$ that $f(x) \leq 0$, and similarly $f(b_n) > 0$ 
	implies that $f(x) \geq 0$.  Therefore $f(x) = 0$.  
\end{enumerate}
\end{document}
