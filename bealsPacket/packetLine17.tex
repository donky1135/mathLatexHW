\documentclass[12pt, letterpaper]{article}
\date{\today}
\usepackage[margin=1in]{geometry}
\usepackage{amsmath}
\usepackage{hyperref}
\usepackage{cancel}
\usepackage{amssymb}
\usepackage{fancyhdr}
\usepackage{pgfplots}
\usepackage{booktabs}
\usepackage{pifont}
\usepackage{amsthm,latexsym,amsfonts,graphicx,epsfig,comment}
\pgfplotsset{compat=1.16}
\usepackage{xcolor}
\usepackage{tikz}
\usetikzlibrary{shapes.geometric}
\usetikzlibrary{arrows.meta,arrows}
\newcommand{\Z}{\mathbb{Z}}
\newcommand{\N}{\mathbb{N}}
\newcommand{\R}{\mathbb{R}}
\newcommand{\Q}{\mathbb{Q}}
\newcommand{\Po}{\mathcal{P}}

\author{Alex Valentino}
\title{Beals Packet}
\pagestyle{fancy}
\renewcommand{\headrulewidth}{0pt}
\renewcommand{\footrulewidth}{0pt}
\fancyhf{}
\rhead{
	Line 17\\
	Beals Summer Packet	
}
\lhead{
	Alex Valentino\\
}
\begin{document}
\begin{enumerate}
	\item[4.2.8] Assume $f(x) \geq g(x)$ for all $x \in A \subseteq \R$ on which $f,g$ are defined.  Show that for any limit point $c \in A$ we must have
	$$
		\lim_{x\to c} f(x) \geq \lim_{x\to c} g(x).
	$$\\
	Let $\lim_{x \to c} f(x) = L, \lim_{x \to c} g(x) = M$.  Suppose $(x_n) \subseteq A, (x_n) \to c$.  Therefore by theorem 4.2.3, $\lim f(x_n) \to L, \lim g(x) \to M$.  Since $(x_n) \subseteq A$ then $f(x_n) \geq g(x_n)$ for all $n \in \N$.  Therefore by the order limit theorem, since $f(x_n) \geq g(x_n)$ then $L \geq M$.  Therefore $\lim_{x\to c} f(x) \geq \lim_{x\to c} g(x)$
\end{enumerate}
\end{document}
