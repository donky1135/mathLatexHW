\documentclass[12pt, letterpaper]{article}
\date{\today}
\usepackage[margin=1in]{geometry}
\usepackage{amsmath}
\usepackage{hyperref}
\usepackage{cancel}
\usepackage{amssymb}
\usepackage{fancyhdr}
\usepackage{pgfplots}
\usepackage{booktabs}
\usepackage{pifont}
\usepackage{amsthm,latexsym,amsfonts,graphicx,epsfig,comment}
\pgfplotsset{compat=1.16}
\usepackage{xcolor}
\usepackage{tikz}
\usetikzlibrary{shapes.geometric}
\usetikzlibrary{arrows.meta,arrows}
\newcommand{\Z}{\mathbb{Z}}
\newcommand{\N}{\mathbb{N}}
\newcommand{\R}{\mathbb{R}}
\newcommand{\Po}{\mathcal{P}}

\author{Alex Valentino}
\title{Beals Packet}
\pagestyle{fancy}
\renewcommand{\headrulewidth}{0pt}
\renewcommand{\footrulewidth}{0pt}
\fancyhf{}
\rhead{
	Line 11\\
	Beals Summer Packet	
}
\lhead{
	Alex Valentino\\
}
\begin{document}
\begin{enumerate}
	\item[2.7.13]
	\begin{enumerate}
		\item 
		\begin{align*}
		| \sum_{j = m+1}^n x_n y_n | &= |s_{n}y_{n+1} - s_{m}y_{m+1} + \sum_{j=m+1}^n s_j(y_j - y_{j+1})| &\text{Exercise 2.7.12}\\
		&\leq M|y_{n+1} - y_{m+1} + \sum_{j=m+1}^n (y_j - y_{j+1})| &\text{Upper bound on $(s_n)$}\\
		&\leq M|y_{n+1} + y_{m+1} + \sum_{j=m+1}^n (y_j - y_{j+1})| \\
		&= M|y_{n+1} + y_{m+1} + y_{m+1} - y_{n+1}| &\text{Expanding the telescoping series}\\
		&= 2 M |y_{m+1}|
		\end{align*}
		\item Dirichlet's Test proof:\\
		We will show that the series $t_m = \sum_{j = 1}^m x_j y_j$ converges by 
		the Cauchy Criterion for Series.  Let $\epsilon > 0$. Since $(y_n)$
		converges to $0$ then there exists $N \in \N$ such that for all $n \geq N$
		$|y_n| < \frac{\epsilon}{2M}$.  Therefore for all $n > m \geq N$\\
		$$
			|t_n - t_m| = |\sum_{j = m+1}^n x_j y_j| \leq 2M|y_{m+1}| \leq 2M|y_N|
			< 2M\frac{\epsilon}{2M} = \epsilon.
		$$  
		Therefore by the Cauchy Criterion for Series, $(t_m)$ converges.  
		\item The Alternating Series Test is simply the cases where $x_n = (-1)^{n+1}$, as it is a sequence bounded above and below by 1.  The requirement on
		$y_n$ is the exact same as in the Alternating Series Test
	\end{enumerate}
\end{enumerate}
\end{document}
