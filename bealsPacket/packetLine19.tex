\documentclass[12pt, letterpaper]{article}
\date{\today}
\usepackage[margin=1in]{geometry}
\usepackage{amsmath}
\usepackage{hyperref}
\usepackage{cancel}
\usepackage{amssymb}
\usepackage{fancyhdr}
\usepackage{pgfplots}
\usepackage{booktabs}
\usepackage{pifont}
\usepackage{amsthm,latexsym,amsfonts,graphicx,epsfig,comment}
\pgfplotsset{compat=1.16}
\usepackage{xcolor}
\usepackage{tikz}
\usetikzlibrary{shapes.geometric}
\usetikzlibrary{arrows.meta,arrows}
\newcommand{\Z}{\mathbb{Z}}
\newcommand{\N}{\mathbb{N}}
\newcommand{\R}{\mathbb{R}}
\newcommand{\Po}{\mathcal{P}}

\author{Alex Valentino}
\title{Homework }
\pagestyle{fancy}
\renewcommand{\headrulewidth}{0pt}
\renewcommand{\footrulewidth}{0pt}
\fancyhf{}
\rhead{
	Line 18\\
	Beals Summer Packet	
}
\lhead{
	Alex Valentino\\
}
\begin{document}
\begin{enumerate}
	\item[4.3.11]
	\begin{enumerate}
		\item $A = \Z$\\
		As shown in the previous chapter, the step function $s(x) = [[x]]$, or the floor function has discontinuities at all points in $\Z$, and is otherwise continuous.  
		\item $A = (0,1)$\\
		Consider the function \[ f(x)= \begin{cases}  \infty  & \text{ if } x \in (0,1) \\ 0 & \text{ if } x \not \in (0,1)
		\end{cases}\]
		Clearly at any point in $(0,1)$ the function is not continuous, and any point outside it is simply the constant 0 function, which as shown previously is continuous.
		\item $A = [0,1]$\\
		Nearly identical to the function above, simply replace the open unit interval with it's closure.  
		\item Consider the function \[ f(x)= \begin{cases}  \frac{1}{\lfloor x \rfloor}  & \text{ if } x\geq 1 \\ 0 & \text{ if } x < 1
		\end{cases}\]
	\end{enumerate}
\end{enumerate}
\end{document}
