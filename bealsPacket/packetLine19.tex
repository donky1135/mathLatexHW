\documentclass[12pt, letterpaper]{article}
\date{\today}
\usepackage[margin=1in]{geometry}
\usepackage{amsmath}
\usepackage{hyperref}
\usepackage{cancel}
\usepackage{amssymb}
\usepackage{fancyhdr}
\usepackage{pgfplots}
\usepackage{booktabs}
\usepackage{pifont}
\usepackage{amsthm,latexsym,amsfonts,graphicx,epsfig,comment}
\pgfplotsset{compat=1.16}
\usepackage{xcolor}
\usepackage{tikz}
\usetikzlibrary{shapes.geometric}
\usetikzlibrary{arrows.meta,arrows}
\newcommand{\Z}{\mathbb{Z}}
\newcommand{\N}{\mathbb{N}}
\newcommand{\R}{\mathbb{R}}
\newcommand{\Q}{\mathbb{Q}}
\newcommand{\Po}{\mathcal{P}}

\author{Alex Valentino}
\title{Beals Packet}
\pagestyle{fancy}
\renewcommand{\headrulewidth}{0pt}
\renewcommand{\footrulewidth}{0pt}
\fancyhf{}
\rhead{
	Line 19\\
	Beals Summer Packet	
}
\lhead{
	Alex Valentino\\
}
\begin{document}
\begin{enumerate}
	\item[4.3.11]
	\begin{enumerate}
		\item For $A=\Z$, the floor function, $[x]$
		\item For $A=(0,1)$, the function $f(x) = \begin{cases} \infty & \text{if } x \in (0,1) \\ 0 & \text{ otherwise } \end{cases}$
		\item For $A=[0,1]$, the function $f(x) = \begin{cases} \infty & \text{if } x \in [0,1] \\ 0 & \text{ otherwise } \end{cases}$
		\item For $A = \{\frac{1}{n} : n \in \N\}$, the function $f(x) = \begin{cases} [\frac{1}{x}] & \text{if } x \in (0,1)\\ 0 & \text{if } x \not \in (0,1)\end{cases}$.  
	\end{enumerate}
\end{enumerate}
\end{document}