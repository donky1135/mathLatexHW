\documentclass[12pt, letterpaper]{article}
\date{\today}
\usepackage[margin=1in]{geometry}
\usepackage{amsmath}
\usepackage{hyperref}
\usepackage{cancel}
\usepackage{amssymb}
\usepackage{fancyhdr}
\usepackage{pgfplots}
\usepackage{booktabs}
\usepackage{pifont}
\usepackage{amsthm,latexsym,amsfonts,graphicx,epsfig,comment}
\pgfplotsset{compat=1.16}
\usepackage{xcolor}
\usepackage{tikz}
\usetikzlibrary{shapes.geometric}
\usetikzlibrary{arrows.meta,arrows}
\newcommand{\Z}{\mathbb{Z}}
\newcommand{\N}{\mathbb{N}}
\newcommand{\R}{\mathbb{R}}
\newcommand{\Po}{\mathcal{P}}

\author{Alex Valentino}
\title{Beals Packet}
\pagestyle{fancy}
\renewcommand{\headrulewidth}{0pt}
\renewcommand{\footrulewidth}{0pt}
\fancyhf{}
\rhead{
	Line 2\\
	Beals Summer Packet	
}
\lhead{
	Alex Valentino\\
}
\begin{document}
\begin{enumerate}
	\item[1.3.2]
	\begin{enumerate}
		\item A real number $s$ is a greatest lower bound for a set $A \subseteq \R$ if
		\begin{enumerate}
			\item $s$ is a lower bound for $A$
			\item if for every lower bound $b$, $b \leq s$
		\end{enumerate}
		\item Lemma 1.3.7 for infimums:\\
		Suppose $s \in \R, A \subseteq \R$ where $s$ is a lower bound for $A$, then $inf(A) = s$ if and only if for all $\epsilon > 0$ there exists $a \in A$ such that $a < s + \epsilon$. 
		\begin{enumerate}
			\item ($\Rightarrow$) Suppose $s = inf(A), \epsilon > 0$.  Then the number $s + \epsilon$ is not a lower bound as that would contradict being less than or equal to $s$.  Since it is not a lower bound on $A$, then there is a smaller element $a$ of $A$.  Therefore $a < s + \epsilon$
			\item $(\Leftarrow)$  Suppose for all $\epsilon > 0$ there exists $a \in A$ such that $a < \epsilon + s$.  We must show that $s = inf(A)$.  Since $s$ is already a lower bound, we simply need to show that any lower bound is less than or equal to $s$.  Suppose $b$ is a lower bound on $A$.  Since we have already shown that for any number greater than $s$, it can't be a lower bound, then conversely for $b$, since $b$ is a lower bound then $b \leq s$.     
		\end{enumerate}
	\end{enumerate}
	\item[1.3.3] Suppose $A,B \subseteq \R$, $sup (A) < sup(B)$.  We must show that there exists $b \in B$ such that $b$ is an upper bound of $A$. By theorem 1.3.7 for all $\epsilon > 0$ there exists $b \in B$ such that $sup (b) - \epsilon < b$.  Since $sup (A) < sup(B)$ then $0 <  sup(B)-sup (A)$.  Therefore let $\epsilon = sup(B)-sup (A)$.  Therefore there exists $b \in B$ such that $ sup(A) < b$.  By definition of supremum $b$ is an upper bound on $A$.  
	\item[1.3.9]
	\begin{enumerate}
		\item A finite, non-empty set always contains its supremum \\
		True
		\item If $a < L$ for every element $a$ in $A$ then $sup(A) < L$.\\
		False, if $L = sup(A)$ and $A$ does not contain it's supremum then we satisfy the preposition, however $sup(A) < sup(A)$ is clearly false.  
		\item If $A$ and $B$ are sets with the property that $a < b$ for every $a \in A$ and $b \in B$, then it follows that $sup(A) < inf(B)$.  \\
		False, if $A = [0,1), B = (1,2]$ then it is true for all $a \in A, b \in B$ that $a < b$.  However $sup(A) = 1, inf (B) = 1$ as by theorem 1.3.7 subtracting any $\epsilon > 0,$ there exists $a \in A$ such that  $1 - \epsilon < a$ and similarly for the result proved in exercise 1.3.2.  Therefore $sup(A) = inf(A)$.  
		\item True
		\item True, proved in the previous exercise above.  
	\end{enumerate}
\end{enumerate}
\end{document}
