\documentclass[12pt, letterpaper]{article}
\date{\today}
\usepackage[margin=1in]{geometry}
\usepackage{amsmath}
\usepackage{hyperref}
\usepackage{cancel}
\usepackage{amssymb}
\usepackage{fancyhdr}
\usepackage{pgfplots}
\usepackage{booktabs}
\usepackage{pifont}
\usepackage{amsthm,latexsym,amsfonts,graphicx,epsfig,comment}
\pgfplotsset{compat=1.16}
\usepackage{xcolor}
\usepackage{tikz}
\usetikzlibrary{shapes.geometric}
\usetikzlibrary{arrows.meta,arrows}
\newcommand{\Z}{\mathbb{Z}}
\newcommand{\N}{\mathbb{N}}
\newcommand{\R}{\mathbb{R}}
\newcommand{\Po}{\mathcal{P}}

\author{Alex Valentino}
\title{Beals Packet}
\pagestyle{fancy}
\renewcommand{\headrulewidth}{0pt}
\renewcommand{\footrulewidth}{0pt}
\fancyhf{}
\rhead{
	Line 9\\
	Beals Summer Packet	
}
\lhead{
	Alex Valentino\\
}
\begin{document}
\begin{enumerate}
	\item[2.5.6]
		Note that $\sup S$ exists as $(a_n)$ is bounded above by $M$. 
		Therefore let $\sup S = s$ We claim for arbitrary $\epsilon > 0$ there are
		an infinite number of elements from the 
		sequence $(a_n)$ contained within $V_\epsilon (s)$.  
		By definition of suprememum, for arbitrary $\epsilon > 0, s-\epsilon \in S$.  Therefore by definition there are an infinite number of elements of $(a_n)$
		above $s-\epsilon$.  Similiarly, by definition of supremum $s + \epsilon \not \in S$.  Therefore there are only finitely many elements of $(a_n)$ above $s+ \epsilon$.  Since $(a_n)$ is infinite then there must be infinitely many terms less than $s+ \epsilon$.  Since there are infinitely many terms greater than $s-\epsilon$ and above $s+ \epsilon$ there are only finitely many terms then there
		must be an infinite number of terms between $s-\epsilon$ and $s+\epsilon$.  
		Therefore $V_\epsilon (s)$ has an infinite number of terms from $(a_n)$.  
		We will now construct by induction a subsequence of $(a_n)$ which converges to $s$, where each $a_{n_k}$ sits within $V_{2^{-k}}(s)$.  For $k=1$, since $0 < 1$, we have an infinite 
		number of terms in $V_1 (s)$.  Choose $n_1 \in \N$ satisfying $a_{n_1} \in V_1 (s)$.  By the principle of mathematical induction for all $j \in \N$ if $j < k$ then there exists $n_j \in \N$ satisfying $n_j > n_{j-1} > \cdots > n_1$ and 
		$a_{n_j} \in V_{2^{-j}}(s)$.  By the induction hypothesis $a_{n_{k-1}} \in V_{2^{1-k}}(s)$.  As proved above there is an infinite number of terms in $V_{2^{-k}}(s)$, therefore we can go out far enough and select a $n_k \in \N$ such that 
		$n_k > n_{k-1}, a_{n_k} \in V_{2^{-k}}(s)$.  Now, to prove convergence, 
		suppose $\epsilon > 0$.  Since $(2^{-n})$ goes to 0 then we can find sufficient $k \in \N$ such that $2^{-k} < \epsilon$.  Since $(2^{-n})$ is a decreasing sequence then for any $m > n, V_{2^{-n}}(s) \supseteq V_{2^{-m}}(s)$.  Therefore for all $l \geq k, a_{n_l} \in V_{2^{-k}}(s)$.  Therefore $(a_{n_k})$ converges to $s$.  
		
\end{enumerate}
\end{document}
