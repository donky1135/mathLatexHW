\documentclass[12pt, letterpaper]{article}
\date{\today}
\usepackage[margin=1in]{geometry}
\usepackage{amsmath}
\usepackage{hyperref}
\usepackage{cancel}
\usepackage{amssymb}
\usepackage{fancyhdr}
\usepackage{pgfplots}
\usepackage{booktabs}
\usepackage{pifont}
\usepackage{amsthm,latexsym,amsfonts,graphicx,epsfig,comment}
\pgfplotsset{compat=1.16}
\usepackage{xcolor}
\usepackage{tikz}
\usetikzlibrary{shapes.geometric}
\usetikzlibrary{arrows.meta,arrows}
\newcommand{\Z}{\mathbb{Z}}
\newcommand{\N}{\mathbb{N}}
\newcommand{\R}{\mathbb{R}}
\newcommand{\Q}{\mathbb{Q}}
\newcommand{\Po}{\mathcal{P}}

\author{Alex Valentino}
\title{Beals Packet}
\pagestyle{fancy}
\renewcommand{\headrulewidth}{0pt}
\renewcommand{\footrulewidth}{0pt}
\fancyhf{}
\rhead{
	Line 28\\
	Beals Summer Packet	
}
\lhead{
	Alex Valentino\\
}
\begin{document}
\begin{enumerate}
	\item[6.4.7] Let $h(x) = \sum_{n=1}^\infty \frac{1}{n^2 + x^2}$
	\begin{enumerate}
		\item Show that h is a continuous function defined on all of $\R$\\
		Note that $\frac{1}{n^2 + x^2}$ is a continuous function with the denominator never reaching 0.  Additionally, 
		observing that $\frac{1}{n^2 + x^2} \leq \frac{1}{n^2}$, and as shown previously that $\sum_{n=1}^\infty \frac{1}{n^2}$ converges, then by the Weierstrauss M-test, then $h(x)$ is converges uniformly, and additionally since each $f_n(x)$
		is continuous then $h(x)$ is continuous (theorem 6.4.2).
		\item Is $h$ differentiable?  If so, is the derivative function $h'$ continuous?\\
		Note that $f_n'(x) = \frac{-2x}{(x^2+n^2)^2}$, and since $f_n'(x)$ has a maximum value of $\frac{3\sqrt{3}}{8n^3}$,
		and that $\frac{3\sqrt{3}}{8n^3} \leq \frac{3\sqrt{3}}{8n^2}$, then by the very same reasoning as above, since 
		$\frac{3\sqrt{3}}{8} \sum_{n=1}\frac{1}{n^2}$ converges by the algebraic limit theorem for series, then by the 
		Weierstrauss M-test the sum $\sum_{n=1}^\infty f_n'(x)$ is uniformly convergent.  Thus $h'(x)$ exists.  Since 
		both $-2x$ and $(x^2+n^2)^2$ are continuous, and $(x^2+n^2)^2 \neq 0$, then their quotient is continuous, therefore $h'(x)$ is continuous (theorem 6.4.2).
	\end{enumerate}
\end{enumerate}
\end{document}
