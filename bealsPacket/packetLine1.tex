\documentclass[12pt, letterpaper]{article}
\date{\today}
\usepackage[margin=1in]{geometry}
\usepackage{amsmath}
\usepackage{hyperref}
\usepackage{cancel}
\usepackage{amssymb}
\usepackage{fancyhdr}
\usepackage{pgfplots}
\usepackage{booktabs}
\usepackage{pifont}
\usepackage{amsthm,latexsym,amsfonts,graphicx,epsfig,comment}
\pgfplotsset{compat=1.16}
\usepackage{xcolor}
\usepackage{tikz}
\usetikzlibrary{shapes.geometric}
\usetikzlibrary{arrows.meta,arrows}
\newcommand{\Z}{\mathbb{Z}}
\newcommand{\N}{\mathbb{N}}
\newcommand{\R}{\mathbb{R}}
\newcommand{\Po}{\mathcal{P}}

\author{Alex Valentino}
\title{Beals Packet}
\pagestyle{fancy}
\renewcommand{\headrulewidth}{0pt}
\renewcommand{\footrulewidth}{0pt}
\fancyhf{}
\rhead{
	Line 1\\
	Beals Summer Packet	
}
\lhead{
	Alex Valentino\\
}
\begin{document}
\begin{enumerate}
	\item[1.2.1]
		\begin{enumerate}
			\item We must show that the square root of $3$ is irrational.  Suppose for contradiction that there exists $a,b \in \Z$ such that $\frac{a}{b} = \sqrt{3}$ and $a,b$ are coprime.  Therefore $a^2 = 3 b^2$, 3 is a divisor of $a^2$. By Euclid's lemma 3 is a divisor of $a$.  Therefore there exists $c \in \Z$ such that $a = 3c$.  Thus $3c^2 = b^2$.  By similar reasoning above, $3$ is a divisor of $b$.  This is a contradiction, as $a,b$ are coprime.  Therefore $\sqrt{3}$ is irrational.    A similar argument works for $\sqrt{6}$, one must simply choose a single prime from the prime factorization of 6 and then get the contradiction from $a,b$ not being coprime.  
			\item The argument fails due to 4 having no prime factor of an odd power, no way to apply Euclid's lemma.  
		\end{enumerate}
	\item[1.2.2]
		\begin{enumerate}
			\item If $A_1 \supseteq A_2 \supseteq \cdots$ are all sets containing an infinite number of elements, then the intersection of all of the sets must be infinite as well.  This is false, if we have the sets $A_n = {n, n+1, n+1, \cdots}$ then the intersection $\bigcap_{n = 1}^\infty A_n = \emptyset$
			\item If $A_1 \supseteq A_2 \supseteq \cdots$ are all sets containing a number of reals, then the intersection of all of the sets must be finite and non-empty.  This is true
			\item $A \cap (B \cup C) = (A \cap B) \cup C$.  This is false.  If $A = \{1\}, B = \{2\}, C = \{2,3\}$.  Then $ \emptyset = \{1\} \cap (\{2\} \cup \{2,3\}	) = A \cap (B \cup C) \neq (A \cap B) \cup C = \emptyset \cup \{2,3\} = \{2,3\}$
			\item true
			\item true
		\end{enumerate}
	\item[1.2.10]
	Let $y_1 = 1$ and for each $n \in \N$ define $y_{n+1} = \frac{3y_n + 4}{4}$.
	\begin{enumerate}
		\item Use induction to prove that the sequence satisfies $y_n < 4$ for all $n \in \N$.\\
		Proof:  For the base case, $y_1 = 1 < 4$.  By the principle of mathematical induction for all $k \in \N$ if $k < n$ then $y_k < 4$.  We must show that $y_n < 4$.  Since $n-1 < n$, then by the induction hypothesis $y_{n-1} < 4$.  Therefore, 
		\begin{align*}
			y_{n-1} &< 4\\
			3 y_{n-1} &< 12\\ 
			3 y_{n-1} + 4 &< 16\\ 
			(3 y_{n-1} + 4)/4 &< 4\\ 
			y_n &< 4.
		\end{align*}
		\item We must show that $(y_1,y_2,\cdots)$ is increasing.  For the base case, $y_1 = 1, y_2 = \frac{3 + 4}{4} = \frac{7}{4}, 1 < \frac{7}{4}$.  By PMI for all $k \in \N $ if $k < n$ then $y_k < y_{k+1}$.  Since $n-1 < n$, by the induction hypothesis $y_{n-1} < y_{n}$.  Therefore, 
		\begin{align*}
		y_{n-1} &< y_{n}\\	
		3 y_{n-1} + 4 &< 3 y_{n} + 4\\
		\frac{3 y_{n-1} + 4}{4} &< \frac{3 y_{n} + 4}{4}\\
		y_n &< y_{n+1}	
\end{align*}			
	\end{enumerate}		
\end{enumerate}
\end{document}
