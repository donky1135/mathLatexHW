\documentclass[12pt, letterpaper]{article}
\date{\today}
\usepackage[margin=1in]{geometry}
\usepackage{amsmath}
\usepackage{hyperref}
\usepackage{cancel}
\usepackage{amssymb}
\usepackage{fancyhdr}
\usepackage{pgfplots}
\usepackage{booktabs}
\usepackage{pifont}
\usepackage{amsthm,latexsym,amsfonts,graphicx,epsfig,comment}
\pgfplotsset{compat=1.16}
\usepackage{xcolor}
\usepackage{tikz}
\usetikzlibrary{shapes.geometric}
\usetikzlibrary{arrows.meta,arrows}
\newcommand{\Z}{\mathbb{Z}}
\newcommand{\N}{\mathbb{N}}
\newcommand{\R}{\mathbb{R}}
\newcommand{\Q}{\mathbb{Q}}
\newcommand{\Po}{\mathcal{P}}

\author{Alex Valentino}
\title{Beals Packet}
\pagestyle{fancy}
\renewcommand{\headrulewidth}{0pt}
\renewcommand{\footrulewidth}{0pt}
\fancyhf{}
\rhead{
	Line 23\\
	Beals Summer Packet	
}
\lhead{
	Alex Valentino\\
}
\begin{document}
\begin{enumerate}
	\item[5.2.8]
	\begin{enumerate}
		\item If a derivative function is not constant, then the derivative must take on
some irrational values.\\
	True.  Suppose $f: A \to \R$, $f'(x)$ exists, $f'(x) \neq c$.  Since $A$ by 	 	definition is an interval for $f'$ to be well defined then if $A$ is a closed
	interval $[a,b]$ then by the darboux theorem $f'$ attains all values between
	$f'(a)$ and $f'(b)$.  Since $f'(a)$ and $f'(b)$ are real numbers then by the 
	density of the irrationals in $\R$ there exists $i$ such that (WLOG) 
	$f'(a) < i < f'(b)$.  Therefore $f'$ always attains an irrational.  
	If the interval is open, then we can find the midpoint, $m = \frac{a+b}{2}$,
	take the open ball around $m$, $V_\epsilon(m)$ guaranteed by it's entry in an open set, then take take the set $[m -\epsilon/2,m +\epsilon/2]$, which we know
	contain the end points as $m -\epsilon < m -\epsilon/2$ and $m + \epsilon/2 > m + \epsilon/2$, thus constructing a closed interval on which $f'$ is defined.
	\item If $f'$ exists on an open interval, and there is some point $c$ where $f'(c) > 0$, then there exists a $\delta$-neighborhood $V_\delta (c)$ around $c$ in which $f'(x) > 0$ for all
$x \in V_\delta (c)$).\\
False, consider the function \[f(x) = \begin{cases}
	x+x^2 \sin(e^{1/|x|}) & \text{ if $ x\neq 0$ }\\
	0 & \text{ if $ x = 0$ }
	\end{cases}	
	\] at 0.  If we evaluate the derivative manually we find 
	$\lim_{x \to 0} 1 + x \sin(e^{1/|x|})$, which similar to evaluating $1 + x\sin(1/x)$
	converges to 1 by the squeeze theorem.  However with the actual evaluation of 
	$f'(x) = 1 + 2x\sin(e^{1/|x|}) - e^{1/|x|}\cos(e^{1/|x|})$ clearly the  $e^{1/|x|}\cos(e^{1/|x|})$ 	term will fluctuate wildly when approaching 0, contradicting that an open neighborhood around $0$ will be purely positive.  
	\iffalse 
	True.  Suppose for contradiction that for all $\delta > 0$ there exists
	$x \in V_\delta(c)$ such that $f'(x) \leq 0$.  Therefore we may construct
	a sequence $(x_n) \to c$ where $f'(x_n) \leq 0$.  Thus by the algebraic
	limit theorem for functional limits since $f(x_n) \leq 0$ then $f'(c) \leq 0$.
	This is a contradiction as $f'(c) > 0$.  
	\fi
	\item If $f$ is differentiable on an interval containing zero and if $\lim_{x \to 0}f'(x) =
L$, then it must be that $L = f'(0)$\\
	 True.  Suppose for contradiction that $f'(0) \neq L$.  Then there exists $\epsilon_0 > 0$ such that 
	$|f'(0) - L| > \epsilon_0$.   Since $f'$ converges to $L$, then there exists $\delta_0 > 0$ such that $x \in V_{\delta_0}(0)$ implies $|f(x) - L| < \epsilon_0/2$.  Since $f$ is differentiable on $[-\delta_0/2,0]$, then by darboux's theorem $f'$ attains all values between $f'(-\delta_0/2)$ and $f'(0)$.  This is a contradiction as for every x value in 
	$(-\delta_0/2,0)$, $|f'(x) - L| < |f'(0) - L|/2$, however all values between $f'(-\delta_0/2)$ and $f'(0)$ must be attained.  
	
	
	\item The question above without the requirement of the limit existing\\
	False, take $f(x) = \frac{x^2 - x}{x}$.  Clearly $\lim_{x\to 0} \frac{x^2 - x}{x} = \lim_{x\to 0} \frac{x(x-1)}{x} = -1$, however directly evaluating $f(0)$ is undefined.  
	  
	\end{enumerate}
\end{enumerate}
\end{document}
