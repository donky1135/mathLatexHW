\documentclass[12pt, letterpaper]{article}
\date{\today}
\usepackage[margin=1in]{geometry}
\usepackage{amsmath}
\usepackage{hyperref}
\usepackage{cancel}
\usepackage{amssymb}
\usepackage{fancyhdr}
\usepackage{pgfplots}
\usepackage{booktabs}
\usepackage{pifont}
\usepackage{amsthm,latexsym,amsfonts,graphicx,epsfig,comment}
\pgfplotsset{compat=1.16}
\usepackage{xcolor}
\usepackage{tikz}
\usetikzlibrary{shapes.geometric}
\usetikzlibrary{arrows.meta,arrows}
\newcommand{\Z}{\mathbb{Z}}
\newcommand{\N}{\mathbb{N}}
\newcommand{\R}{\mathbb{R}}
\newcommand{\Q}{\mathbb{Q}}
\newcommand{\Po}{\mathcal{P}}

\author{Alex Valentino}
\title{Beals Packet}
\pagestyle{fancy}
\renewcommand{\headrulewidth}{0pt}
\renewcommand{\footrulewidth}{0pt}
\fancyhf{}
\rhead{
	Line 3\\
	Beals Summer Packet	
}
\lhead{
	Alex Valentino\\
}
\begin{document}
\begin{enumerate}
	\item[1.4.2] Recall that $\mathbb{I}$ stands for the set of irrational numbers.
	\begin{enumerate}
		\item Show that if $a,b \in \Q$, then $a+b, ab \in \Q$.  \\
		Suppose $a,b \in \Q$.  By definition of being members of $\Q$, there exists $c,d,e,f \in \Z$ such that $a = \frac{c}{d}, b = \frac{e}{f}$.  
		\begin{itemize}
			\item We will show that $a + b \in \Q$\\
			\begin{align*}
				a + b &= \frac{c}{d} + \frac{e}{f}\\
				&= \frac{cf + ed}{df}
			\end{align*}
			Since $\Z$ is closed under addition and multiplication, $cf + ed \in \Z, df \in \Z$.  Therefore $a+b$ satisfies the definition of a rational number
			\item We will show that $ab \in \Q$.
		Since $ab = \frac{ce}{df}$ by definition, and $\Z$ is closed under multiplication then $ce \in \Z, df \in \Z$.  Therefore $ab \in \Q$.  
\end{itemize}
	\item Suppose $a \in \Q, t \in \mathbb{I}$.
	\begin{itemize}
		\item We must show that $a + t \in \mathbb{I}$.  Suppose for contradiction that $a + t \in \Q$.  Then there exists $r \in \Q$ such that $a + t = r$.  Since $\Q$ is closed under addition then $t \in \Q$.  This is a contradiction as $t \not \in \Q$.  
		\item We must show that if $a \neq 0$ then $at \in \mathbb{I}$.  Suppose for contradiction that $at \in \Q$.  Then there exists $r \in \Q$ such that $at = r$.  Since $\Q$ is closed under non-zero division then $ t \in \Q$.  This is a contradiction as $t \not \in \Q$.  
	\end{itemize}
	\item Given two irrational numbers $s,t \in \mathbb{I}$ we can say nothing about whether $st \in \mathbb{I}$ or $s + t \in \mathbb{I}$.  As $\sqrt{3} * \sqrt{2} = \sqrt{6} \in \mathbb{I}$, however $\sqrt{2} * \sqrt{2} =  4 \in \Q$.  Similarly $\frac{\sqrt{2}}{2} \in \mathbb{I}$, $\frac{\sqrt{2}}{2} + \frac{\sqrt{2}}{2} = \sqrt{2} \in \mathbb{I}$, however $\sqrt{2}, -\sqrt{2} + 2 \in \mathbb{I}$, $\sqrt{2} -\sqrt{2} + 2 =  2 \in \Q$.  Therefore you can't say anything conclusive about the product and sum of general irrational numbers.   			 
	\end{enumerate}
	\item[1.4.6]	
	\begin{enumerate}
		\item Let $T = \{x \in \R: x^2 < 2\}, \alpha = \sup T$.  Suppose for contradiction that $\alpha^2 > 2$.  Suppose $n \in \N$. Then we have that
		\begin{align*}
		\left(\alpha - \frac{1}{n}\right)^2 &= \alpha^2 -\frac{2\alpha}{n} + \frac{1}{n^2}\\
			&> \alpha^2 -\frac{2\alpha}{n}
		\end{align*}
		By the archimedean principle, we may choose $n_0 \in \N$ such that $\frac{1}{n_0} < \frac{\alpha^2 - 2}{2 \alpha}$.  Therefore if we set $n_0 = n$ we have that $\left(\alpha - \frac{1}{n}\right)^2 > \alpha^2 - \alpha^2 + 2 = 2$.  This contradicts the fact that all upper bounds must be greater than or equal to $\alpha$.
		\item Suppose $b \geq 0$.  Let $T = \{x \in \R: x^2 < b\}, \alpha = \sup T$.  We will show that $\alpha^2 = b$ by cases.  Two notes before proceeding with the proof.  First note that we already know that $0^2 = 0 \in \R$, therefore we will operate on the assumption that $b > 0$.  Second, we claim that $\alpha > 0$.  Since $b > 0$ we can apply the archamedian principle to get $m \in \N$ such that $b > \frac{1}{m}$.  Since $\frac{1}{m^2} < \frac{1}{m} < b$ then $\frac{1}{m} \in T$.  By definition of $\alpha = \sup T$ then $\frac{1}{m} < \alpha$.  Therefore $\alpha > 0$.  We now begin evaluating cases. 
		\begin{itemize}
			\item Suppose $\alpha^2 < b$.  Let $n \in \N$.  Therefore 
			\begin{align*}
				\left( \alpha + \frac{1}{n}\right)^2 &= \alpha^2 + \frac{2\alpha}{n} + \frac{1}{n^2}\\
				&< \alpha^2 + \frac{2\alpha}{n} + \frac{1}{n}\\
				&= \alpha^2 + \frac{2\alpha + 1}{n}
			\end{align*}
			Since $\alpha > 0$ then $\frac{b - \alpha^2}{2\alpha + 1} > 0$.  Therefore by the archamedian principle there exists $n_0 \in \N$ such
			that $\frac{1}{n_0} < \frac{b - \alpha^2}{2\alpha + 1}$.  If we set $n = n_0$ then we have that $(\alpha + \frac{1}{n_0})^2 <\alpha^2 + b - \alpha^2 = b$.  This contradicts the fact that $\alpha = \sup T$ as all elements of $T$ must be less than $\alpha$.
			\item Suppose $\alpha^2 > b$.  Let $n \in \N$.  Therefore 
			\begin{align*}
				\left( \alpha - \frac{1}{n}\right)^2
				&= \alpha^2 - \frac{2\alpha}{n} + \frac{1}{n^2}\\ 
				&>\alpha^2 - \frac{2\alpha}{n}
			\end{align*}  
			By the archimedean principle, we may choose $n_0 \in \N$ such that $\frac{1}{n_0} < \frac{\alpha^2 - b}{2 \alpha}$.  Therefore if we set $n = n_0$ we have that $\left(\alpha - \frac{1}{n}\right)^2 > \alpha^2 - \alpha^2 + b = b$.  This contradicts the fact that all upper bounds must be greater than or equal to $\alpha$.
		\end{itemize}
	\end{enumerate}
	\item[1.4.8]
		\begin{enumerate}
			\item We must show that for two countable sets $A_1, A_2$ that $A_1 \cup A_2$ is countable.  For the proof we will be dealing with the set $B_2 = A_2\backslash A_1$.  We will assume that $B_2$ is countable.  Therefore there exists $f_1 : \N \to A_1, f_2 : \N \to B_2$ such that both are bijections.  We claim that $F: \N \to A_1 \cup A_2$ given by 
			$F(x) = \begin{cases}
				f_1(\frac{x-1}{2}) & x \text{ even}\\
				f_2(x/2) & x \text{ odd}
			\end{cases}$ is a bijection.  
			\begin{itemize}
			\item	Suppose $x_1,x_2 \in \N, F(x_1), F(x_2) \in A_1 \cup A_2, F(x_1) = F(x_2)$.  We must show that $x_1 = x_2$.  Since $A_1, B_2$ are disjoint then either $F(x_1) \in A_1$ or $F(x_1) \in B_2$.  If $F(x_1) \in A_1$ then $f_1(\frac{x_1 - 1}{2}) = f_1(\frac{x_2 - 1}{2})$ where $x_1, x_2$ must be odd, if not $F$ would output in $B_2$ by definition, contradicting the initial assumption.  Since $f_1$ is a bijection then $x_1 = x_2$.  Suppose $F(x_1) \in B_2$.  Then our equation becomes $f_2(x_1/2) = f_2(x_2/2)$ where $x_1, x_2$ must be even by similar reasoning above.  Since $f_2$ is a bijection then $x_1 = x_2$.  Therefore $F$ is an injective function.
			\item Suppose $y \in A_1 \cup A_2$.  We must show there exists $x \in \N$ such that $y = f(x)$.  Since $y \in A_1 \cup A_2$ then either $y \in A_1$ or $y \in B_2$.  Suppose $y \in A_1$.  Then by the definition of countability there exists $n \in \N$ such that $f_1 (n) = y$.  We claim that $x = 2n + 1$.  Observe that $F(2n + 1) = f_1 (\frac{2n + 1 - 1}{2}) = f_1 (2n/2) = f_1 (n) = y$.  Suppose $y \in B_2$.  Then by the definition of countability there exists $m \in \N$ such that $f_2(m) = y$.  We claim that $x = 2m$.  Observe that $F(2m) = f_2 (\frac{2m}{2}) = f_2(m) = y$. Therefore $F$ is surjective.  
		\end{itemize}
		Since $F$ has been shown to be a bijection between $\N$ and $A_1 \cup A_2$ then the union of any two countable sets is countable.\\  If $B_2$ was finite then we could have given an arbitrary indexing to $B_2$ by the bijection $\sigma : \{1,2,\ldots,n\} \to B_2$ then given $F$ as $F(x) = \begin{cases} \sigma(x) & x \leq n\\ f_1(x-n) & x > n \end{cases}$\\
		The greater proof of having $A_1,\ldots,A_m$ countable sets having a countable union is by induction.  Since any two countable sets can be unioned together to be a larger countable set, then we can apply that operation an arbitrary amount of times until we have $A_1 \cup \cdots \cup A_{m-1}$ as a countable set and $A_m$, then union then together and apply what has been proved above. 
		\item Induction fails to prove part (ii) as $\infty$ is not a natural number.  Part (i) is $m$ sets, which is a finite number, and only requires a finite process to achieve. 
		\item The arrangement as shown in the problem lends itself to a bijective function $f: \N \to \N \times \N$.  If one is to take the sets $B_n = A_n \backslash (A_1 \cup A_2 \cup \cdots \cup A_{n-1} \cup A_{n+1} \cup \cdots)$ and assume that they remain countable after performing this process then we can take the respective bijection $f_n : \N \to B_n$ and arrange each function and it's output as so: 
		
		\begin{tabular}{c c c}
		$f_1 (1)$ & $f_1(2)$ & $\cdots$\\
		$f_2 (1)$ & $f_2(1)$ & $\cdots$\\
		$\vdots$ & $\vdots$  & $\ddots$
		\end{tabular}
		Since for each $(m,n) \in \N^2$ we have a unique $f_n (m)$ then we have another bijection $g: \N^2 \to \cup_{n=1}^\infty A_n$.  Since the composition of bijections is a bijection, $g \circ f : \N \to  \cup_{n=1}^\infty A_n$ is a bijection.  Therefore $ \cup_{n=1}^\infty A_n$ is countable.   
		\end{enumerate}
\end{enumerate}
\end{document}
