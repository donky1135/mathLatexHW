\documentclass[12pt, letterpaper]{article}
\date{\today}
\usepackage[margin=1in]{geometry}
\usepackage{amsmath}
\usepackage{hyperref}
\usepackage{cancel}
\usepackage{amssymb}
\usepackage{fancyhdr}
\usepackage{pgfplots}
\usepackage{booktabs}
\usepackage{pifont}
\usepackage{amsthm,latexsym,amsfonts,graphicx,epsfig,comment}
\pgfplotsset{compat=1.16}
\usepackage{xcolor}
\usepackage{tikz}
\usetikzlibrary{shapes.geometric}
\usetikzlibrary{arrows.meta,arrows}
\newcommand{\Z}{\mathbb{Z}}
\newcommand{\N}{\mathbb{N}}
\newcommand{\R}{\mathbb{R}}
\newcommand{\Q}{\mathbb{Q}}

\newcommand{\Po}{\mathcal{P}}

\author{Alex Valentino}
\title{Beals Packet}
\pagestyle{fancy}
\renewcommand{\headrulewidth}{0pt}
\renewcommand{\footrulewidth}{0pt}
\fancyhf{}
\rhead{
	Line 12\\
	Beals Summer Packet	
}
\lhead{
	Alex Valentino\\
}
\begin{document}
	\begin{enumerate}
		\item[3.2.10]
			\begin{itemize}
				\item[i] A countable set contained in $[0,1]$ with no limit points\\
				This can't occur, as if we enumarate a countable set, then we have
				a sequence, and since it's contained within $[0,1]$ then we have
				a bounded sequence.  Therefore by the Bolzano-Weierstrauss theorem
				we have a convergent subsequence, i.e. a limit point of the set.  
				Thus this description can't occur
				\item[ii]  A countable set contained in $[0,1]$ with no isolated points\\
				This can occur!  Take $A = [0,1] \cap \Q$.  We know by the density of $\Q$ in $\R$ that we have a sequence of rational numbers to every
				real number.  Since a rational number is a real number, then for every element in the set $A$ we can generate a sequence with a sufficent $\epsilon$ contained within $[0,1]$ for which an arbitrary rational number satisfies the definition of a limit point.  And the countablility of the rationals
				is a given.  
				\item[iii]
				% Let $A$ be this uncountable set of isolated points.  
				%Then there exists a bijection between $\R$ and $A$.  
				
				The issue arises with the fact that for each $x \in A$, there is an associated $\epsilon_x > 0$ in which $V_{\epsilon_x} (x) \cap A = \{x\}$. If we have some minimum $\epsilon_x > 0$ then we have a most a collection of neighborhoods which cover the entire real line.  Arguing from geometry, it's clear that only a countable number of these uncountable "tiles" need to overlay the 1d bathroom floor that is the real line.  Therefore, alike in the example showing all of the points of $\{\frac{1}{n}: n \in \N\}$ are isolated, we must find a formula for $\epsilon_x$ based on $x$.  The issue arises with the fact that $x$ is an uncountable variable, where we have an uncountable set of elements from $x$ with progressively smaller $\epsilon_x$'s. Once again, arguing from geometry, one can see how one can set up our $\epsilon_x$'s in order to turn one of our allegedly isolated points into a limit point.  
				
				%If we let $f$ be the bijection from $\R$ to $A$ then we have the issue of 
				
				%$f^{-1}(V_{\epsilon_x}(x)) = f^{-1}(x)$.  Note the uncountable neighborhood just mapped there to a single real number.  This violates the 1-1 nature of the bijection.  
				
				%  Therefore we have a minimum distance required for each point in $A$ to maintain the fact that they're isolated.  If we 
				
				
				%For previous examples we've seen how to define $\epsilon$ to show that
				%each point in the sequence is isolated, however those have been countable.  The issue arises with choosing some set of points in $\R$ which converges to a number.  
				
				
				\iffalse A set with an uncountable number of isolated points.\\
				 This can't occur, as for every isolated point $x$ in our supposed set $A$, we have an associated $\epsilon_x$ for which $V_{\epsilon_x}(x)$ contains only $x$ and no other points in $A$.  Therefore thinking geometrically, there is a 
				 "minimum spacing" between the points.  This has only been seen so 
				 far with countable sets.  
				 \fi 
			\end{itemize}
	\end{enumerate}
\end{document}
