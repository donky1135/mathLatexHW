\documentclass[12pt, letterpaper]{article}
\date{\today}
\usepackage[margin=1in]{geometry}
\usepackage{amsmath}
\usepackage{hyperref}
\usepackage{cancel}
\usepackage{amssymb}
\usepackage{fancyhdr}
\usepackage{pgfplots}
\usepackage{booktabs}
\usepackage{pifont}
\usepackage{amsthm,latexsym,amsfonts,graphicx,epsfig,comment}
\pgfplotsset{compat=1.16}
\usepackage{xcolor}
\usepackage{tikz}
\usetikzlibrary{shapes.geometric}
\usetikzlibrary{arrows.meta,arrows}
\newcommand{\Z}{\mathbb{Z}}
\newcommand{\N}{\mathbb{N}}
\newcommand{\R}{\mathbb{R}}
\newcommand{\Q}{\mathbb{Q}}
\newcommand{\Po}{\mathcal{P}}

\author{Alex Valentino}
\title{Beals Packet}
\pagestyle{fancy}
\renewcommand{\headrulewidth}{0pt}
\renewcommand{\footrulewidth}{0pt}
\fancyhf{}
\rhead{
	Line 15\\
	Beals Summer Packet	
}
\lhead{
	Alex Valentino\\
}
\begin{document}
\begin{enumerate}
	\item[3.5.3]
	\begin{enumerate}
		\item Show that a closed interval $[a,b]$ is a $G_{\delta}$ set.\\
		Suppose $r \in (0,1)$.  We claim that $[a,b] = \bigcap_{n=1}^\infty (a-r^n, b+r^n)$.  
		\begin{itemize}
			\item We want to show that $[a,b] \subseteq \bigcap_{n=1}^\infty (a-r^n, b+r^n)$.  \\
		Since for all $n \in \N, [a,b] \subseteq (a-r^n, b+r^n)$, then by 
			the definition of set intersection, $[a,b] \subseteq \bigcap_{n=1}^\infty (a-r^n, b+r^n)$.  
			\item We want to show that $ \bigcap_{n=1}^\infty (a-r^n, b+r^n) \subseteq  [a,b]$.  Suppose $x \in \bigcap_{n=1}^\infty (a-r^n, b+r^n)$.  We
			must show that $a \leq x \leq b$.  First to show that $a \leq x$.
			We know that for all $n \in \N, a-r^n < x$, therefore $x$ is an upper bound on $\{a-r^n: n \in \N\}$.  Therefore all we need to prove is that $a \in \bigcap_{n=1}^\infty (a-r^n, b+r^n)$.  Since $a$ is the limit point of the sequence $(a - r^n)_{n=1}^\infty$, then there is no set $(a-r^n, b + r^n)$ which 
			excludes it.  Therefore $a \in \bigcap_{n=1}^\infty (a-r^n, b+r^n)$.
			Thus, since $x$ is an upper bound on $(a - r^n)_{n=1}^\infty$, and our
			set contains $a$, then $a \leq x$.\\
			Next we must show that $x \leq b$.  We know that for all $n \in \N, 
			x < b + r^n$.  Thus $x$ is a lower bound on $(b + r^n)$.  Therefore,
			similar to above, $x \leq b$.
			Therefore $a \leq x \leq b$.  Thus $x \in [a,b]$
		\end{itemize}
		\item Show that the half open interval $(a,b]$ is both $G_\delta$ and $F_\sigma$ set
		\begin{itemize}
			\item Show that $(a,b]$ is a $G_\delta$ set\\
			We claim that $\cap_{n=1}^\infty (a,b+r^n) = (a,b]$.  As shown in the proof above, for the closed end point, the set above converges, and for the open right endpoint, it is trivial.    
			\item Show that $(a,b]$ is a $F_\sigma$ set\\\
			We claim that $\cup_{n=1}^\infty [a + \frac{1}{n}, b] = (a,b]$.  \\
			\begin{itemize}
				\item We must show that $\cup_{n=1}^\infty [a-\frac{1}{n}, b] \subseteq (a,b]$. Suppose $x \in \cup_{n=1}^\infty [a + \frac{1}{n}, b]$   We must show that $a < x \leq b$.  Since $x \leq b$ is trivial, we must show that $a < x$.
				Since $x$ is in the union, there must be a smallest $n_0 \in \N$ in which $x \in [a + \frac{1}{n_0}, b]$.  Therefore $a < a +  \frac{1}{n_0} \leq x$.  
				\item We must show that $(a,b] \subseteq  \cup_{n=1}^\infty [a-\frac{1}{n}, b]$.  Since $\{b\}$ is trivially in both, we must show that $(a,b) \subseteq  \cup_{n=1}^\infty [a-\frac{1}{n}, b]$.  Suppose $x \in (a,b)$.  Therefore there exists $\epsilon > 0$ such that $V_\epsilon(x) \subset (a,b)$.
				Since the neighborhood is contained within $(a,b)$ then 
				$a < x - \epsilon$.  Therefore $0 < x - \epsilon - a$.  Thus by the archimedean principle there exists $n' \in \N$ such that $\frac{1}{n'} < x - \epsilon - a$.  Therefore $a + \frac{1}{n'} < x - \epsilon$, and therefore $V_\epsilon(x) \subset [a+ \frac{1}{n'}, b] \subseteq \cup_{n=1}^\infty [a+ \frac{1}{n}, b]$. \\
				Thus $(a,b] \subseteq \cup_{n=1}^\infty [a+ \frac{1}{n}, b]$
			\end{itemize}
			Making $(a,b]$ a $F_\sigma$ set.     
		\end{itemize}
		\item 
		\begin{itemize}
			\item Show that $\Q$ is a $F_\sigma$ set.\\
			Since $\Q$ is countable, then there exists a bijection between $\N$ and $\Q$.  Therefore let $f: \N \to \Q$ be a bijection.  Then we claim that $\bigcup_{n=1}^\infty [f(n),f(n)]$ is $\Q$.  Since we're guaranteed to uniquely attain every rational number, then we have exactly $\Q$.  Since we have a countable union of closed intervals, then we have satisifed the definition of $F_\sigma$
			\item Show that the set of irrationals forms a $G_\delta$ set.\\
			Let $L_n = \displaystyle \bigcup_{x \in \Z} \left( \frac{x}{n},\frac{x+1}{n}\right)$.  We claim that $\R \backslash \Q = \displaystyle \bigcap_{n=1}^\infty L_n$.  
			\begin{itemize}
				\item Suppose $x \in \R \backslash \Q$.  We must show that $x \in \bigcap_{n=1}^\infty L_n$.  Suppose for contradiction that $x \not \in \bigcap_{n=1}^\infty L_n$.  Therefore by the definition of set compliment and  DeMorgan's law, $x \in \cup_{n=1}^\infty L_n^c$.  Since $x$ is in the union, there
				must exists $n_0 \in \N$ where $x \in L_{n_0}^c$.  Therefore $x \in \{\frac{y}{n_0}: y \in \Z\}$. Thus there exists a $y_0 \in \Z$ such that $x = \frac{y_0}{n_0}$.  This contradicts $x$ being irrational.  Therefore $x$ is in the intersection.
				\item Suppose $x \in \bigcap_{n=1}^\infty L_n$.  Then we must show that $x \in \R \backslash \Q$.  Suppose for contradiction that $x \in \Q$.  Then there exists $a,b \in \Z$ such that $x = \frac{a}{b}$.  However, since $x$ is in the intersection of $L_n$'s, then $x \in L_b$.  This is a contradiction as $L_b$ excludes all rational numbers with a denominator of $b$.  Thus $x$ is irrational.
			\end{itemize}
			Therefore $\R \backslash \Q = \displaystyle \bigcap_{n=1}^\infty L_n$,
			thus making the irrationals $G_\delta$.			 
		\end{itemize}
	\end{enumerate}
\end{enumerate}
\end{document}
