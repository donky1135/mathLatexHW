\documentclass[12pt, letterpaper]{article}
\date{\today}
\usepackage[margin=1in]{geometry}
\usepackage{amsmath}
\usepackage{hyperref}
\usepackage{cancel}
\usepackage{amssymb}
\usepackage{fancyhdr}
\usepackage{pgfplots}
\usepackage{booktabs}
\usepackage{pifont}
\usepackage{amsthm,latexsym,amsfonts,graphicx,epsfig,comment}
\pgfplotsset{compat=1.16}
\usepackage{xcolor}
\usepackage{tikz}
\usetikzlibrary{shapes.geometric}
\usetikzlibrary{arrows.meta,arrows}
\newcommand{\Z}{\mathbb{Z}}
\newcommand{\N}{\mathbb{N}}
\newcommand{\R}{\mathbb{R}}
\newcommand{\Q}{\mathbb{Q}}
\newcommand{\Po}{\mathcal{P}}

\author{Alex Valentino}
\title{Beals Packet}
\pagestyle{fancy}
\renewcommand{\headrulewidth}{0pt}
\renewcommand{\footrulewidth}{0pt}
\fancyhf{}
\rhead{
	Line 13\\
	Beals Summer Packet	
}
\lhead{
	Alex Valentino\\
}
\begin{document}
	\begin{enumerate}
		\item[3.3.10] The only sets which are "clompact" are finite collections of
		real number singletons and the empty set.  Otherwise if a set has a non-zero length 
		then one can scale a copy of $\cup_{n = 0}^\infty [2^{-(n+1)}, 2^{-n}]$ to be shorter than whatever length given and translated to be inside the set.  Once 
		inside the set, there is no finite sub cover of the sets given which can 
		be chosen to still cover the set, as they fit together exactly to form the 
		interval $[0,1/2]$.    
	\end{enumerate}
\end{document}
