\documentclass[12pt, letterpaper]{article}
\date{\today}
\usepackage[margin=1in]{geometry}
\usepackage{amsmath}
\usepackage{hyperref}
\usepackage{cancel}
\usepackage{amssymb}
\usepackage{fancyhdr}
\usepackage{pgfplots}
\usepackage{booktabs}
\usepackage{pifont}
\usepackage{amsthm,latexsym,amsfonts,graphicx,epsfig,comment}
\pgfplotsset{compat=1.16}
\usepackage{xcolor}
\usepackage{tikz}
\usetikzlibrary{shapes.geometric}
\usetikzlibrary{arrows.meta,arrows}
\newcommand{\Z}{\mathbb{Z}}
\newcommand{\N}{\mathbb{N}}
\newcommand{\R}{\mathbb{R}}
\newcommand{\Q}{\mathbb{Q}}
\newcommand{\Po}{\mathcal{P}}

\author{Alex Valentino}
\title{Beals Packet}
\pagestyle{fancy}
\renewcommand{\headrulewidth}{0pt}
\renewcommand{\footrulewidth}{0pt}
\fancyhf{}
\rhead{
	Line 18\\
	Beals Summer Packet	
}
\lhead{
	Alex Valentino\\
}
\begin{document}
\begin{enumerate}
	\item[4.3.6] 
	\begin{enumerate}
		\item Suppose $c \in \Q$.  Thus $h(c) = 1$.  Consider $y_n = \frac{1}{\pi^n} + c$.  By the fact that $\pi$ is transcendental, each $y_n$ is irrational, and the sequence converges to $c$.  
		Therefore, since each $y_n$ is irrational we have that $h(y_n) \to 0$.  \\
		Suppose $i \not \in \Q$.  Therefore $h(i) = 0$.  If we consider the 
		sequence $(a_n)$ given by the truncated decimal expansion of $i$,
		clearly each $a_n$ is rational.  Therefore $h(a_n) \to 1$.\\
		Therefore $h(x)$ is a nowhere continuous function.
		\item Suppose $c \in \Q$.  Consider the sequence once more of 
		$y_n = \frac{1}{\pi^n} + c$.  Since each $y_n$ is irrational, then $t(y_n) \to 0$. This goes against $h(c) = \frac{1}{n}$. Therefore $t(x)$ is not continuous at every rational number.
		\item Consider $i \in \R \backslash \Q$, and let $\epsilon > 0$.  If we consider the set 
		$T = \{x \in \R: t(x) \geq \epsilon\}$, we note that since each $t(x)$ is positive, then $T$ is a set of rational numbers.  If we apply the archamedian principle to $\epsilon$, we find that $m \in \N, \epsilon > \frac{1}{m}$.  Therefore, for all $x \in T$, $V_{\frac{1}{2m}}(x) \cap T = \{x\}$, otherwise if two numbers from $T$ were in the neighborhood then one would be guarenteed to have a larger denominator than $m$, which would contradict being a member of $T$.  Therefore if we choose $\delta < \frac{1}{2m}$ then $x \in T$ implies that $x \not \in V_{\delta}(i)$.  Therefore if $x \in V_{\delta}(i)$, then $t(x) \in V_\epsilon (t(i))$.  Therefore $t(x)$ converges for every irrational number. 
	\end{enumerate}
\end{enumerate}
\end{document}
