\documentclass[12pt, letterpaper]{article}
\date{\today}
\usepackage[margin=1in]{geometry}
\usepackage{amsmath}
\usepackage{hyperref}
\usepackage{cancel}
\usepackage{amssymb}
\usepackage{fancyhdr}
\usepackage{pgfplots}
\usepackage{booktabs}
\usepackage{pifont}
\usepackage{amsthm,latexsym,amsfonts,graphicx,epsfig,comment}
\pgfplotsset{compat=1.16}
\usepackage{xcolor}
\usepackage{tikz}
\usetikzlibrary{shapes.geometric}
\usetikzlibrary{arrows.meta,arrows}
\newcommand{\Z}{\mathbb{Z}}
\newcommand{\N}{\mathbb{N}}
\newcommand{\R}{\mathbb{R}}
\newcommand{\Q}{\mathbb{Q}}
\newcommand{\Po}{\mathcal{P}}

\author{Alex Valentino}
\title{Beals Packet}
\pagestyle{fancy}
\renewcommand{\headrulewidth}{0pt}
\renewcommand{\footrulewidth}{0pt}
\fancyhf{}
\rhead{
	Line 18\\
	Beals Summer Packet	
}
\lhead{
	Alex Valentino\\
}
\begin{document}
\begin{enumerate}
	\item[4.3.6]
	\begin{enumerate}
		\item Suppose $c \in \Q$.  Thus $h(c) = 1$.  Consider $\pi^{-n} + c  = y_n$.  Since $\pi$ is transcendental then 
		every $y_n$ is irrational.  Therefore $h(y_n) \to 0$.\\
		
		Suppose $c \not \in \Q$.  Therefore $h(c) = 0$.  Consider $x_n = \frac{p_n}{q_n}$, which is the $nth$ decimal
		expansion of $c$.  Since each $x_n \in \Q$, then $h(x_n) \to 1$. 
		
		Since every possible $\R$ value is disconntinous, then $h$ is a nowhere continuous function.  
		\item Suppose $c \in \Q$.  Consider $x_n = c - \pi^{-n}$.  As shown above, each $x_n$ is irrational, thus $t(x_n) \to 0$, not to the non-zero value of $t(c)$.
		\item Let $\epsilon > 0, i \in \R \backslash \Q$, and consider the set $T = \{x \in \R: t(x) \geq \epsilon\}$.  Since all elements in $T$ are rational, then we know something about their denominators, in particular that they are bounded below by $\epsilon$.  Looking at $[i - \frac{1}{2}, i + \frac{1}{2}]\cap T$, we have a finite number of elements, therefore looking to $\min t([i - \frac{1}{2}, i + \frac{1}{2}]\cup T) = \frac{1}{m}$, then we have for each $x \in [i - \frac{1}{2}, i + \frac{1}{2}]\cap T$, $V_{\frac{1}{2m}}(x) \cap T = \{x\}$, otherwise multiple points would imply denominators smaller than $\frac{1}{m}$.  Therefore if we choose $\delta < \frac{1}{2m}$, then for all $x \in [i - \frac{1}{2}, i + \frac{1}{2}]\cap T$, $x \not \in V_\delta(c)$.  Therefore if $y \in V_\delta(c), y \not \in T$, thus 
		$t(y) < \epsilon, y \in V_\epsilon(0)$.  Thus we have convergence for $t(i)$. 
	\end{enumerate}
\end{enumerate}
\end{document}
