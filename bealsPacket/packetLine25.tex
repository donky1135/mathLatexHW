\documentclass[12pt, letterpaper]{article}
\date{\today}
\usepackage[margin=1in]{geometry}
\usepackage{amsmath}
\usepackage{hyperref}
\usepackage{cancel}
\usepackage{amssymb}
\usepackage{fancyhdr}
\usepackage{pgfplots}
\usepackage{booktabs}
\usepackage{pifont}
\usepackage{amsthm,latexsym,amsfonts,graphicx,epsfig,comment}
\pgfplotsset{compat=1.16}
\usepackage{xcolor}
\usepackage{tikz}
\usetikzlibrary{shapes.geometric}
\usetikzlibrary{arrows.meta,arrows}
\newcommand{\Z}{\mathbb{Z}}
\newcommand{\N}{\mathbb{N}}
\newcommand{\R}{\mathbb{R}}
\newcommand{\Q}{\mathbb{Q}}
\newcommand{\Po}{\mathcal{P}}

\author{Alex Valentino}
\title{Beals Packet}
\pagestyle{fancy}
\renewcommand{\headrulewidth}{0pt}
\renewcommand{\footrulewidth}{0pt}
\fancyhf{}
\rhead{
	Line 25\\
	Beals Summer Packet	
}
\lhead{
	Alex Valentino\\
}
\begin{document}
\begin{enumerate}
	\item[5.4.5]
	\begin{enumerate}
		\item Show that $g'(1)$ does not exists:\\
		Consider the sequence $x_m = 1 + 2^{-m}$ where $m \in \N$.  We will show that 
		$$\frac{g(x_m)-g(1)}{x_m - 1} = m + 1 - 2^{m+1}.$$ \\
		Note that 
		\begin{align*}
		g(x_m) &= \sum_{n=0}^\infty \frac{h(2^n(1 + 2^{-m}))}{2^n}\\
		&= \sum_{n=0}^\infty \frac{h(2^{n} + 2^{n-m})}{2^n}\\
		&= h(1+2^{-m}) + \sum_{n=1}^\infty \frac{h(2^{n} + 2^{n-m})}{2^n}\\
		&= 1-2^{m} + \sum_{n=1}^\infty \frac{h(2^{n-m})}{2^n} & \text{ periodicity of $h(x)$}\\ 
		&= 1-2^{m} + \sum_{n=1}^m \frac{h(2^{n-m})}{2^n} & \text{ if $n > m$ then $2^{n-m}$ is a whole multiple of 2}\\
		&= 1-2^{m} + \sum_{n=1}^m \frac{2^{n-m}}{2^n}\\		
		&= 1-2^{m} + m2^{-m}\\
		&= 1 + (m-1)2^{-m}. 
		\end{align*}
		Therefore, $$\frac{g(x_m)-g(1)}{x_m - 1} = \frac{1 + (m-1)2^{-m} - 1}{1 + 2^{-m} - 1}
		= \frac{(m-1)2^{-m}}{2^{-m}} = m-1.$$
		Since the limit diverges to infinity, then $g'(1)$ does not exists.\\
		Show that $g'(\frac{1}{2})$ does not exists.\\
		Consider the sequence $x_m = 1 + 2^{-m}$ where $m \in \N$.  We will show that 
		$$ \frac{g(x_m)-g(1)}{x_m - 1} = m-3$$
		for a sufficently large $m$.  		
		\\
		Note that 
		\begin{align*}
			g(x_m) &= \sum_{n=0}^\infty \frac{h(2^n(\frac{1}{2} + 2^{-m}))}{2^n}\\
			&= \sum_{n=0}^\infty \frac{h(2^{n-1}+ 2^{n-m})}{2^n}\\
			&= \sum_{n=0}^m \frac{h(2^{n-1}+ 2^{n-m})}{2^n}\\
			&= h(2^{-1}+ 2^{-m}) + h(1+ 2^{1-m}) + \sum_{n=2}^m \frac{h(2^{n-m})}{2^n} & \text{ $n$ is large enough that 
			$2 \mid 2^{n-1}$}\\	
			&= h(2^{-1}+ 2^{-m}) + h(1+ 2^{1-m}) + (m-2)2^{-m} &\\
			&= 2^{-1}+ 2^{-m} + 1 - 2^{1-m} + (m-2)2^{-m} & \text{ take $m>2$}\\
			&= 2^{-1} + 1 + (m-3)2^{-m}
		\end{align*}
		Noting that $h(\frac{1}{2}) = 2^{-1} + 1$ we can compute:
		$$
		\frac{g(x_m)-g(\frac{1}{2})}{x_m - \frac{1}{2}} = \frac{2^{-1} + 1 + (m-3)2^{-m} - 2^{-1} - 1}{\frac{1}{2} + 2^{-m} - \frac{1}{2}} = m - 3.
		$$
		Therefore $g'(\frac{1}{2})$ diverges and is not well defined.  
		\item For $p \in \Z, k \in \N_0, x = \frac{p}{2^k}$ show that $g'(x)$ does not exists. \\
		Consider the sequence $x_m = \frac{p}{2^k} + \frac{1}{2^m}$ where $m \in \N$.  We will show that
		$$
		\frac{g(x_m)-g(\frac{p}{2^k})}{x_m - \frac{p}{2^k}} = 
		$$ 		
		Note that 
		\begin{align*}
		g(\frac{p}{2^k}) &= \sum_{n=0}^\infty \frac{h(2^{n-k}p)}{2^n}\\
		&= \sum_{n=0}^m \frac{h(2^{n-k}p)}{2^n} & \text{ if $n > k$ then $h(2^{n-k}p) = 0$}\\
		\end{align*}
		Furthermore we can compute 
		\begin{align*}
		g(x_m) &=  \sum_{n=0}^\infty \frac{h(2^{n-k}p + 2^{n-m})}{2^n}\\
		&=  \sum_{n=0}^m \frac{h(2^{n-k}p + 2^{n-m})}{2^n} & \text{ Assume $m > k$}\\
		&= \sum_{n=0}^k \frac{h(2^{n-k}p + 2^{n-m})}{2^n} + \sum_{n=k+1}^m \frac{h(2^{n-k}p + 2^{n-m})}{2^n}\\
		&= \sum_{n=0}^k \frac{h(2^{n-k}p + 2^{n-m})}{2^n} + \sum_{n=k+1}^m \frac{h(2^{n-m})}{2^n} & 
		\text{ if $n > k$ then $2\mid 2^{n-k}p$ }\\
		&= \sum_{n=0}^k \frac{h(2^{n-k}p + 2^{n-m})}{2^n} + (m-k-1)2^{-m}.
\end{align*}				 
		Now we must estimate $\sum_{n=0}^k \frac{h(2^{n-k}p + 2^{n-m})}{2^n}$.  Since we have finite $n$ then we can 
		simply make $m$ big enough such that for all $0\leq n \leq k,h(2^{n-k}p) \not \in \Z , |h(2^{n-k}p + 2^{n-m}) - h(2^{n-k}p)| < \min\{1-h(2^{n-k}p),h(2^{n-k}p)\}$ (note that if every $h(2^{n-k}p) \in \Z $ then $g(x)$ is either $g(1)$ or $g(0)$, both of which have been shown to be nondifferentiable ).  Therefore $h(2^{n-k}p + 2^{n-m}) = h(2^{n-k}p) \pm 2^{n-m}$, as 
		our specification on $m$ put $h(2^{n-k}p + 2^{n-m})$ on the same line segment as $h(2^{n-k}p)$.
		Therefore, 
		\begin{align*}
		\sum_{n=0}^k \frac{h(2^{n-k}p + 2^{n-m})}{2^n} + (m-k-1)2^{-m} &= 
		\sum_{n=0}^k \frac{h(2^{n-k}p)\pm 2^{n-m}}{2^n} + (m-k-1)2^{-m}\\
		&= g(\frac{p}{2^k}) +(m-k-1)2^{-m} + \sum_{n=0}^k \pm 2^{m-n}. 
		\end{align*}
		Therefore we may compute 
		\begin{align*}
			\frac{g(x_m)-g(\frac{p}{2^k})}{x_m - \frac{p}{2^k}} &= \frac{g(\frac{p}{2^k}) +(m-k-1)2^{-m} + \sum_{n=0}^k \pm 2^{m-n}-g(\frac{p}{2^k})}{2^{-m}}\\
			&= \frac{ +(m-k-1)2^{-m} + \sum_{n=0}^k \pm 2^{m-n}}{2^{-m}}\\
			&= (m-k-1) + \sum_{n=0}^k \pm 1\\
			&\geq  m-k-1 - k - 1 = m -2k -2.
		\end{align*}
		Therefore $g'(p/2^k)$ diverges, and does not exists.  
	\end{enumerate}
\end{enumerate}
\end{document}
