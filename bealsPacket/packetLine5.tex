\documentclass[12pt, letterpaper]{article}
\date{\today}
\usepackage[margin=1in]{geometry}
\usepackage{amsmath}
\usepackage{hyperref}
\usepackage{cancel}
\usepackage{amssymb}
\usepackage{fancyhdr}
\usepackage{pgfplots}
\usepackage{booktabs}
\usepackage{pifont}
\usepackage{amsthm,latexsym,amsfonts,graphicx,epsfig,comment}
\pgfplotsset{compat=1.16}
\usepackage{xcolor}
\usepackage{tikz}
\usetikzlibrary{shapes.geometric}
\usetikzlibrary{arrows.meta,arrows}
\newcommand{\Z}{\mathbb{Z}}
\newcommand{\N}{\mathbb{N}}
\newcommand{\R}{\mathbb{R}}
\newcommand{\Po}{\mathcal{P}}

\author{Alex Valentino}
\title{Beals Packet}
\pagestyle{fancy}
\renewcommand{\headrulewidth}{0pt}
\renewcommand{\footrulewidth}{0pt}
\fancyhf{}
\rhead{
	Line 5\\
	Beals Summer Packet	
}
\lhead{
	Alex Valentino\\
}
\begin{document}
	\begin{enumerate}
		\item[1.5.7]
			\begin{itemize}
				\item $\{a,b,c\}$
				\item $\{a,b,c\}$
				\item $\{1,2,3,4\}$
			\end{itemize}
		\item[1.5.8]
			\begin{enumerate}
				\item Suppose $a' \in B$.  Then by definition of $B$ $a' \not \in f(a') = B$.  This is a contradiction
				\item Suppose $a' \not \in B$.  This means by the construction of $B$, $a' \in f(a')$.  However $f(a') = B$.  This is a contradiction
			\end{enumerate}
		\item[1.5.9]
			\begin{enumerate}
				\item We claim that the set of all functions from $\{0,1\}$ to $\N$ is countable.  Each function can be represented as $\{(0,a),(1,b)\}$ where $a,b \in \N$.  Note that there is an obvious bijection between the set $\{(0,a),(1,b)\}$ and $(a,b)$.  Therefore the set of all functions from $\{0,1\}$ to $\N$ has the same cardinality as $\N^2$.  Since $\N^2$ has the same cardinality as $\N$, then we have shown that the set of all function from $\{0,1\}$ to $\N$ is countable.  
				\item We claim that the set of all functions from $\N$ to $\{0,1\}$ is uncountable.  We claim that there is a bijection between the set of all functions from $\N$ to $\{0,1\}$ and the set $S = \{(a_1,a_2,\ldots):a_n = 0 \text{ or } a_n = 1\}$ as defined in exercise 1.5.4 which was proven to be uncountable.  Let $f: \{0,1\}^\N \to S$ be given by $g(f) = (f(1),f(2),f(3),\ldots)$.  Suppose $f_1,f_2 \in \{0,1\}^\N, g(f_1) = g(f_2)$.  We must show that $f_1 = f_2$.  Since $f_1, f_2$ are over the natural numbers, it suffices to show that for all $n \in \N, f_1(n) = f_2(n)$. Since $g(f_1) = g(f_2)$, then $f_1(n) = f_2(n)$ for all $n \in \N$.  Therefore $f_1 = f_2$.  Since $\{0,1\}^\N$ injects into $S$, and $S$ is uncountable
				\item Let the set $S$ be as described in problem 1.5.4, the set of all binary sequences.  Let the function $f: \N \times S \to \N$ be given by 
				$$
				f(n,s) := \begin{cases} 2n & s_n = 0\\ 2n -1 & s_n = 1. \end{cases}
				$$
				For a sequence $s \in S$ let $A_s = \{f(n,s) : n \in \N\}$.  We claim that the set $K = \{A_s : s \in S\}$ is an uncountable antichain.  
				\begin{itemize}
					\item From 1.5.4 we know that $S$ is uncountable.  We claim that there is a 1-1 correspondence with $S$ and $K$.  Let the function $g: S \to K$ be given by $g(s) = A_s$. We claim that $g$ is 1-1.  Suppose $r,s \in S, g(r) = g(s)$.  We must show that $r = s$.  To show that $r=s$, it suffices to show for all $n \in \N$, $r_n = s_n$.  Suppose $n \in \N, s_n = 1$.  Then $2n-1 \in g(s), g(r)$.  Since $2n-1 \in g(r)$, then $r_n = 1$ as this is the only condition under which $2n-1 \in g(r)$.  A similar proof exists for $s_n = 0$.  Since $s_n = r_n$ for arbitrary $n$, then $s = r$.
					\item We claim that for arbitrary distinct $r,s \in S$ that $A_s \not \subset A_r, A_r \not \subset A_s$.  Suppose $r,s \in S, r \neq s$.  Then by definition there must exists $n \in \N$ such that $r_n \neq s_n$.  Suppose WLOG $r_n = 1$.  Therefore $s_n = 0, 2n - 1\in A_r, 2n \in A_s$.  Since $r_n = 1$ then $2n \not \in A_r$ as if it was then that would contradict $2n -1 \in A_r$.  Similarly, there would be a contradiction if $2n - 1 \in A_s$.  Therefore $A_s \not \subset A_r, A_r \not \subset A_s$.
				\end{itemize}
				Since there is a 1-1 correspondence with an uncountable set, and $K$ is an antichain then we have satisfied finding an uncountable subset of $\Po(\N)$.   
			\end{enumerate}
	\end{enumerate}
\end{document}
