\documentclass[12pt, letterpaper]{article}
\date{\today}
\usepackage[margin=1in]{geometry}
\usepackage{amsmath}
\usepackage{hyperref}
\usepackage{cancel}
\usepackage{amssymb}
\usepackage{fancyhdr}
\usepackage{pgfplots}
\usepackage{booktabs}
\usepackage{pifont}
\usepackage{amsthm,latexsym,amsfonts,graphicx,epsfig,comment}
\pgfplotsset{compat=1.16}
\usepackage{xcolor}
\usepackage{tikz}
\usetikzlibrary{shapes.geometric}
\usetikzlibrary{arrows.meta,arrows}
\newcommand{\Z}{\mathbb{Z}}
\newcommand{\N}{\mathbb{N}}
\newcommand{\R}{\mathbb{R}}
\newcommand{\Po}{\mathcal{P}}

\author{Alex Valentino}
\title{Beals Packet}
\pagestyle{fancy}
\renewcommand{\headrulewidth}{0pt}
\renewcommand{\footrulewidth}{0pt}
\fancyhf{}
\rhead{
	Line 8\\
	Beals Summer Packet	
}
\lhead{
	Alex Valentino\\
}
\begin{document}
	\begin{enumerate}
		\item[2.4.1]
		\item[2.4.6]
		\begin{enumerate}
			\item Suppose $(a_n)$ is a bounded sequence.  Prove that the sequence $y_n = \sup\{a_k: k \geq n\}$ converges.\\
			Proof:  We claim that $(y_n)$ is a decreasing and bounded sequence.  
			\begin{itemize}
				\item We claim that $(y_n)$ is decreasing.  Suppose $n \in \N$.  We must show that $y_n \geq y_{n+1}$.  We have two cases, $y_n = a_n, y_n \neq a_n$.
				\begin{itemize}
					\item Suppose $a_n = y_n$.  Therefore for all other elements in the sequence after $a_n$, $a_n \geq a_k$ where $k > n$.  Therefore $a_n$ is an upper bound on $\{a_k : k \geq n + 1\}$.  Since $y_{n+1}$ is the supremum of the set mentioned before, and we have established that $a_n$ is an upper bound, then by definition $y_n = a_n \geq y_{n+1}$.  
					\item Suppose $a_n \neq y_n$.  Since $a_n$ is already not a suprememum of the set $\{a_k : k \geq n\}$ then computing the supremum of the set excluding $a_n, \{a_k : k \geq n + 1\}$ should not change the supremum value.  Therefore $y_n = y_{n+1}, y_n \geq y_{n+1}$.
				\end{itemize}
				Therefore $(y_n)$ is a decreasing sequence. 
				\item We claim that $(y_n)$ is bounded.  Since $(y_n)$ is decreasing we simply need to show that there is a quantity larger than $y_1$.  Since $y_1 = \sup \{a_k : k \geq 1\}$ then $y_1$ is the supremum for $(a_n)$.  
				Since $(a_n)$ is bounded, suppose for some quantity $M \in \R$,	then by definition of supremum $y_1 \leq M$.  Therefore $(y_n)$ is bounded				  
			\end{itemize}
			Therefore by the monotone convergences theorem $(y_n)$ converges.
			\item Let $z_n = \inf \{ a_k : k \geq n\}$.  Then $\lim z_n = \liminf a_n$.  This should converge since $(z_n)$ can easily be proved 
			to be increasing and bounded.
			\item 
			\begin{itemize}
				\item Prove that $\liminf a_n \leq \limsup a_n$\\
				Proof:  Suppose $n \in \N$. For an arbitrary element $e \in \{a_k : k \geq n\}$, $e \leq y_n, e \geq z_n$ by the respective definitions of supremum and infimum.  Therefore for all $n \in \N, z_n \leq y_n$.  Since we know that $\liminf a_n, \limsup a_n$ exists, then by the algebraic order theorem $\liminf a_n \leq \limsup a_n$.
				\item An example of a strict inequality between $\liminf a_n$ and $\limsup a_n$ is the sequence $a_n = \frac{1}{n} + (-1)^{n+1}$.  
				This is because $\liminf a_n = \lim \frac{1}{n} - 1 = -1 < 1 = \lim \frac{1}{n} + 1 = \limsup a_n$
			\end{itemize}
			\item We must prove that $\liminf a_n = \limsup a_n$ if and only if $\lim a_n$ exists
			\begin{itemize}
				\item[$\Rightarrow$] Suppose $\liminf a_n = \limsup a_n$.  We must show that $\lim a_n$ exists.  Note that by definition of infimum and supremum for all $k \in \N, z_k \leq a_k, a_k \leq y_k$.  Therefore for all $k \in \N, z_k \leq a_k \leq y_k$.  Since $\lim z_n = \lim y_n$, then by the squeeze theorem $\lim a_n = \liminf a_n = \limsup a_n$.
				\item[$\Leftarrow$] Suppose $\lim a_n$ exists and suppose for contradiction that $\liminf a_n \neq \limsup a_n$.  
				Since $\liminf a_n \leq \limsup a_n$ then $\liminf a_n < \limsup a_n$.  Therefore $0 < \liminf a_n - \limsup a_n$.  Let $l$ be the limit point of $(a_n)$.  Note that since the inequality between $\liminf a_n$ and $\limsup a_n$ is
				strict then $l$ can at most converge to either $\liminf a_n$ or 
				$\limsup a_n$, therefore we assume WLOG $l < \limsup a_n$.
				Since $(a_n)$ converges to $l$ and $0 < \limsup a_n - l$ then 
				by the definition of convergence there exists $N\in \N$ such that 
				for all $n \geq N, |a_n - l| < \limsup a_n - l$.  Furthermore by 
				the definition of convergence there exists $M \in \N$ such that 
				for all $m \geq N + M$, $|a_m - l| < (\limsup a_n - l)/2$.  Since
				there are only finitely many points in the sequence which are
				closer to $\limsup a_n$ than $l$ for the sequence starting at $N$ 
				then we can upper bound this sequence by taking 
				$M = \max(\{a_N, a_{N+1}, \cdots, a_{N+M-1}, \frac{\limsup a_n + l}{2}\})$.  
				Note that 
				\begin{align*}
					l &< \limsup a_n\\
					l + \limsup a_n &< 2 * \limsup a_n\\
					\frac{l + \limsup a_n}{2} &< \limsup a_n
				\end{align*}
				therefore $M < \limsup a_n$.
				Since $|a_n - l| < \limsup a_n -l$ for all $n \geq N$, then $a_n < \limsup a_n$.  By the definition of supremum, $y_N \leq M$.  Therefore $y_N < \limsup a_n$.  This is a contradiction as $(y_n)$ is a decreasing sequence, and it would be expected that $y_N \geq \limsup a_n$.  Therefore $\liminf a_n = \limsup a_n$.  
			\end{itemize}
		\end{enumerate}
	\end{enumerate}
\end{document}
