\documentclass[12pt, letterpaper]{article}
\date{\today}
\usepackage[margin=1in]{geometry}
\usepackage{amsmath}
\usepackage{hyperref}
\usepackage{cancel}
\usepackage{amssymb}
\usepackage{fancyhdr}
\usepackage{pgfplots}
\usepackage{booktabs}
\usepackage{pifont}
\usepackage{amsthm,latexsym,amsfonts,graphicx,epsfig,comment}
\pgfplotsset{compat=1.16}
\usepackage{xcolor}
\usepackage{tikz}
\usetikzlibrary{shapes.geometric}
\usetikzlibrary{arrows.meta,arrows}
\newcommand{\E}{\mathbb{E}}
\newcommand{\Z}{\mathbb{Z}}
\newcommand{\N}{\mathbb{N}}
\newcommand{\R}{\mathbb{R}}
\newcommand{\Q}{\mathbb{Q}}
\newcommand{\Po}{\mathcal{P}}
\newcommand{\Pro}{\mathbb{P}}
\author{Alex Valentino}
\title{477 homework}
\pagestyle{fancy}
\renewcommand{\headrulewidth}{0pt}
\renewcommand{\footrulewidth}{0pt}
\fancyhf{}
\rhead{
	Homework 9\\
	477	
}
\lhead{
	Alex Valentino\\
}
\begin{document}
\begin{enumerate}
	\item[8.2] Let $X$ be the value of the 4 sided die roll, let $Y$ be the value of the 6 sided die roll, let $Z$ be the value of the 12 sided die roll, and let 
	$W=X+Y+Z$ be the collective value from the three die rolls.
	Thus $\E W = \E[X+Y+Z] = \E X + \E Y + \E Z = \frac{5}{2} + \frac{7}{2} + \frac{13}{2} = 12.5$
	\item[8.4] Let $X$ be the value of the 4 sided die roll, let $Y$ be the value of the 6 sided die roll, let $Z$ be the value of the 12 sided die roll, and let $V$ represent the number of fours.  Since $X,Y,Z$ are independent die rolls, then 
	we can write $V$ as the sum of indicators: $V = \mathbb{I}_{X = 4} + \mathbb{I}_{Y = 4} +\mathbb{I}_{Z = 4}$.  Thus $\E V = \E \mathbb{I}_{X = 4} + \E \mathbb{I}_{Y = 4} + \mathbb{I}_{Z = 4} = \frac{1}{4} + \frac{1}{6} + \frac{1}{12} = \frac{1}{2}$.  
	\item[8.8] Let $X \sim Unif[1,7], Y \sim Exp[2], Z = X + Y$.  We want to find $\E Z$ 
	and $Var(Z)$.  By the linearity of expectation, $\E Z = \E X + \E Y = 4 + 0.5 = 6, Var(X+Y) = Var(X) + Var(Y) + 2Cov(X,Y) = 3 + 0.25 + 0 = 3.25$
	\item[8.24] Note that $N \in \{10,\cdots,40\}$ since there are a minimum of 10 matching pair and a max of 40 pairs where Jane follows Sam with the correct color.
	Note that $N$ can be written as the sum of indicator variables 
	$N = \sum_{n=1}^{40} \mathbb{I}_n$, where $\mathbb{I}_k$ represents the $k$th pairing.  Note that by exchangability that all $\mathbb{I}_k$ are the same.  
	There probabilities evaluate to $\Pro(\mathbb{I}_k) = \frac{1}{79 * 80} \left(
	50*49 + 30 * 29 \right) = \frac{83}{158}$.  Thus $\E N = \sum_{n=1}^{40} \E \mathbb{I}_n = 40 * \frac{83}{158}\approx 21.01 $
	\item[8.42]  
	\begin{align*}
		\E \bar{X}_n^4 &= \E \left( \frac{X_1 + \cdots + X_n}{n} \right)^4\\
		&= \frac{1}{n^4} \E [(X_1 + \cdots + X_n)^4]\\
		&= \frac{1}{n^4} \E [ \frac{4!}{1!1!1!1!}\sum_{i < j < k < l}X_i X_j X_k X_l + \frac{4!}{2!1!1!}\sum_{i<j, k \neq i,j} X_k^2 X_i X_j\\
		&+ \frac{4!}{2!2!} \sum_{i \neq j}X_i^2 X_j^2 + \frac{4!}{3!}\sum_{i\neq j}X_i^3 X_j + \frac{4!}{4!}\sum_{i} X_i^4]\\
		&= \frac{1}{n^4} \left( 6\sum_{i \neq j} \E[X_i^2] \E[X_j^2] + \sum_i \E[X_i^4] \right)\\
		&\text{ by linearity of expectation, independence of the variables, and that } \E X_i = 0 \text{ for all } i\\
		&= \frac{1}{n^4} (6 \binom{n}{2}a^2 + nc)\\
		&= \frac{1}{n^3} (3(n-1)a^2 + c)
	\end{align*}
	\item[8.48]  Note that the joint pmf of $X$ and $Y$ is given by 
	$$
		\begin{matrix}
			\frac{X}{Y} & 1 & 2 & 3\\
			0 & \frac{9}{100} & \frac{81}{100} & 0\\
			1 & 0 & \frac{9}{100} & 0\\
			2 & 0 & 0 & \frac{1}{100}
		\end{matrix}			
	$$
	Therefore computing the expectations required to solve covariance are easy:\\
	$$\E X = \frac{9}{100} + 2 \cdot \frac{90}{100} + \frac{3}{100} = \frac{192}{100} = 1.92,$$ $$\E Y = \frac{9}{100} + \frac{2}{100} = \frac{11}{100} = 0.11,$$ $$\E XY = 2 \cdot 1 \cdot \frac{9}{100} + 3 \cdot 2 \cdot \frac{1}{100} = \frac{24}{100} = 0.24.$$  Therefore $Cov(X,Y) = \E[XY] - \E[X]\E[Y] = 0.24 - 1.92\cdot 0.11 = 0.0288$. 
\end{enumerate}
\end{document}
