\documentclass[12pt, letterpaper]{article}
\date{\today}
\usepackage[margin=1in]{geometry}
\usepackage{amsmath}
\usepackage{hyperref}
\usepackage{cancel}
\usepackage{amssymb}
\usepackage{fancyhdr}
\usepackage{pgfplots}
\usepackage{booktabs}
\usepackage{pifont}
\usepackage{amsthm,latexsym,amsfonts,graphicx,epsfig,comment}
\pgfplotsset{compat=1.16}
\usepackage{xcolor}
\usepackage{tikz}
\usetikzlibrary{shapes.geometric}
\usetikzlibrary{arrows.meta,arrows}
\newcommand{\E}{\mathbb{E}}
\newcommand{\Z}{\mathbb{Z}}
\newcommand{\N}{\mathbb{N}}
\newcommand{\R}{\mathbb{R}}
\newcommand{\Q}{\mathbb{Q}}
\newcommand{\Po}{\mathcal{P}}
\newcommand{\Pro}{\mathbb{P}}
\author{Alex Valentino}
\title{477 homework}
\pagestyle{fancy}
\renewcommand{\headrulewidth}{0pt}
\renewcommand{\footrulewidth}{0pt}
\fancyhf{}
\rhead{
	Homework \\
	477	
}
\lhead{
	Alex Valentino\\
}
\begin{document}
\begin{enumerate}
	\item[4.10] Since scoring goals is a "somewhat" rare event, we can 
	model this process with a Poisson distribution.  Let $g$ denote 
	the number of goals scored in a game.  Since $\Pro(g\geq 1) = 0.5$, then 
	$0.5=1-\Pro(g\geq 1) = \Pro(g = 0)$.  Therefore $0.5 = e^{-\lambda},\lambda = \log(2).$\\  Thus $$\Pro(g \geq 3) = 1 - \Pro(g < 3) = 1 - (\Pro(g=0)  + \Pro(g=1)+\Pro(g=2))$$ $$=  1 - (\frac{1}{2} + \frac{1}{2}\log(2) + \frac{1}{2}\frac{\log(2)^2}{2}) \approx 0.033$$.  
	\item[4.14] Note that since the expectation is $1000$, and the formula
	for the expectation of an exponential variable is $\frac{1}{\lambda}$ then $\lambda = \frac{1}{1000}$
	\begin{enumerate}
		\item $\Pro(t > 2000) = e^{-\frac{2000}{1000}} = e^{-2} \approx 0.1353 $
		\item $\Pro(t > 2000 | t > 500) = \Pro(t >1500) = e^{-\frac{3}{2}} \approx 0.2231$
	\end{enumerate}
	\item[4.34] Since the average can be interpreted as the mean, and 
	one can view 3 times a week as being somewhat rare, makes the Poisson distribution an ideal model.  Since the mean of the poisson distribution is the rate, then assuming $\lambda = 3$ gives the probability of at most 2 
	accidents happening next week is $e^{-3} (1 + 3 + \frac{9}{2} ) \approx 0.42319$
	\item[5.2]
	\begin{enumerate}
		\item 
		$$
			\E[X] = M'(0) = \frac{5}{6} - \frac{4}{3} = -1/2
		$$
		$$
			\E[X^2] = M''(0) = \frac{25}{6} + \frac{16}{3} = \frac{57}{6}
		$$
		$$
			Var(X) = \E[X^2] - (\E[X])^2  = \frac{57}{6} - \frac{1}{4} = \frac{38-1}{4} = \frac{37}{4}
		$$
	\end{enumerate}
	\item[5.6]
	$$
	\Pro(X=4) = \frac{1}{7},\Pro(X=1) = \frac{2}{7},\Pro(X=9) = \frac{4}{7}
	$$
	\item[5.12] We are assuming $t < 1$.  Therefore
	\begin{align*}
		\E[e^{tX}] &= \int_\infty^\infty e^{tx} f(x)dx\\
		&= \int_0^\infty \frac{1}{2} x^2 e^{(t-1)x}dx\\
		&= \frac{1}{2}\frac{1}{t-1}x^2 e^{(t-1)x}\bigg\rvert_{0}^\infty + \frac{1}{1-t}\int_0^\infty xe^{(t-1)x}dx\\
		&= \frac{-1}{(1-t)^2} xe^{(1-t)x} \bigg\rvert_{0}^\infty + \frac{1}{(1-t)^2}\int_0^\infty e^{(t-1)x}dx\\
		&= \frac{-1}{(1-t)^3} e^{(t-1)x} \bigg\rvert_0^\infty\\
		&= \frac{1}{(1-t)^3}
	\end{align*}
	\item[5.26] Let $Y=X(X-3)$.  We want to find $Y$'s pdf.  Thus we must compute $\Pro(Y \leq a)$.  Note that the inequality $X(X-3) \leq a$ 
	is equivalent to $X \in [\frac{3-\sqrt{9+4a}}{2},\frac{3+\sqrt{9+4a}}{2}]$.  Note that since $X$ is non-zero in $[0,3]$ $a$ must have the 
	restriction $[\frac{-9}{4},0]$
	Thus we can compute the cdf of $Y$ via $\int_{\frac{3-\sqrt{9+4a}}{2}}^{\frac{3+\sqrt{9+4a}}{2}} \frac{2}{9}xdx = \frac{\sqrt{9+4a}}{3}$.
	Thus the pdf of $Y$ is given by $f_y(a) = \begin{cases} \frac{2}{3\sqrt{9+4a}} & a \in [\frac{-9}{4},0]\\ 0 & \text{otherwise} \end{cases}$
\end{enumerate}
\end{document}
