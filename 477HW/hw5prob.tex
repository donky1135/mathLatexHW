\documentclass[12pt, letterpaper]{article}
\date{\today}
\usepackage[margin=1in]{geometry}
\usepackage{amsmath}
\usepackage{hyperref}
\usepackage{cancel}
\usepackage{amssymb}
\usepackage{fancyhdr}
\usepackage{pgfplots}
\usepackage{booktabs}
\usepackage{pifont}
\usepackage{amsthm,latexsym,amsfonts,graphicx,epsfig,comment}
\pgfplotsset{compat=1.16}
\usepackage{xcolor}
\usepackage{tikz}
\usetikzlibrary{shapes.geometric}
\usetikzlibrary{arrows.meta,arrows}
\newcommand{\Z}{\mathbb{Z}}
\newcommand{\N}{\mathbb{N}}
\newcommand{\R}{\mathbb{R}}
\newcommand{\Q}{\mathbb{Q}}
\newcommand{\Po}{\mathcal{P}}
\newcommand{\Pro}{\mathbb{P}}
\author{Alex Valentino}
\title{477 homework}
\pagestyle{fancy}
\renewcommand{\headrulewidth}{0pt}
\renewcommand{\footrulewidth}{0pt}
\fancyhf{}
\rhead{
	Homework 5\\
	477	
}
\lhead{
	Alex Valentino\\
}
\begin{document}
\begin{enumerate}
	\item[3.34]\textit{Let $X$ be a random variable with probability mass function$$
\Pro (X = 1) = \frac{1}{
2} , \Pro (X = 2) = \frac{1
}{3} \text{ and } \Pro (X = 5) = \frac{1}
{6}$$}
\begin{enumerate}
	\item \textit{Find a function $g$ such that $\mathbb{E}[g(X )] = 1
3 \ln 2 + 1
6 \ln 5$. Your answer should
give at least the values $g(k)$ for all possible values $k$ of $X$ , but you can also
specify $g$ on a larger set if possible.}\\
	$g(x) = \begin{cases} 26 ln(2) & x=1\\ 0 & x=2\\ 72ln(5) & x=5  \end{cases}$
	\item \textit{Suppose $t \in \R$, then find a function which satisfies
	$\mathbb{E}[g(X)] = \frac{1}{2}e^t + \frac{2}{3}e^{2t} + \frac{5}{6}e^{5t}$}\\
	Let $g(x) = e^{xt}$
	\item \textit{Find a function $g$ such that $\mathbb{E}[g(X )] = 2$.}\\
	Let $g(x) = 2$
\end{enumerate} 
	\item[3.36] if $p_X(y) = \begin{cases} \frac{2}{y^2} & 1 \leq y \leq 2\\
	0 &  \text{otherwise}\end{cases}$ then 
	$\mathbb{E}[X^4] = \int_{-\infty}^\infty y^4 p_X(y) dy = 
	\int_1^2 2y^2dy = \frac{2}{3}(8 - 1) = \frac{14}{3}.$
	\item[3.44] 
	\begin{enumerate}
		\item The cdf is given by $F_S(a) = \int_0^1 \int_{-\frac{\pi}{2}}^{\arctan(a)} d\theta dr = \frac{1}{\pi}\arctan(a) + \frac{1}{2}$, which one finds by taking the area of the sector from $-\frac{\pi}{2}$ to the angle of the slope of the line $ax$.  
		\item The pdf is given by $\frac{d}{da} (\frac{1}{\pi}\arctan(a) + \frac{1}{2}) = \frac{1}{\pi} \frac{1}{1+a^2}$
	\end{enumerate}
	\item[3.52]  
	\begin{align*}
		\sum_{k=1}^\infty P(X \geq k) &= \sum_{k=1}^\infty \sum_{i=k}^\infty \Pro(X=i) \\
		&= \sum_{i=1}^\infty \sum_{k=1}^i \Pro(X=i)\\
		&= \sum_{i=1}^\infty i\Pro(X=i)\\
		&= \mathbb{E}X
	\end{align*}
	\item[3.54]
	\begin{enumerate}
		\item  $\Pro(X \geq k) = 1 - \Pro(X < k) = 1 - \sum_{i=1}^{k-1} (1-p)^{i-1}p = 1 - (1-(1- p)^{k-1}) = (1-p)^{k-1} $
		\item $\mathbb{E}X = \sum_{k=1}^\infty \Pro(X \geq k) = \sum_{k=1}^\infty (1-p)^{k-1} = \frac{1}{1-(1-p)} = p^{-1}$
\end{enumerate}
	\item[3.62] 
	\begin{enumerate}
		\item $F_X(a) = \begin{cases} 0 & a < 0 \\a & 0 \leq  a < 0.75\\ 1 & a \geq 0.75\end{cases}$
		\item $\mathbb{E}(X) = \int_0^{0.75} xdx + \int_{0.75}^1 0.75 dx = \frac{9}{32} + \frac{1}{4} = \frac{15}{32}$.
		\item $Var(X) = \mathbb{E}(X^2) - (\mathbb{E}X)^2 = \int_0^{0.75} x^2dx + \int_{0.75}^1 \frac{9}{32} dx - \frac{225}{1024} = 0.0615234$
	\end{enumerate}		 
\end{enumerate}
\end{document}
