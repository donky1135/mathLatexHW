\documentclass{exam}
\usepackage[utf8]{inputenc}
\usepackage{amsmath}
\usepackage{amssymb}

\title{Math 292 Homework 1}
\author{Ish Shah}
\date{January 31, 2023}

\begin{document}

\maketitle

\section{Exercises from section 1.3}
\begin{enumerate}
    \item[1.2] Divide both sides of the differential equation by $1+t^2$ to yield
    \begin{equation*}
        x' + \frac{2t}{1+t^2}x = \frac{\cot t}{1+t^2}
    \end{equation*}
    An integrating factor is $\displaystyle e^{\int \frac{2t}{1+t^2}\,dt} = 1+t^2$. Multiplying by the integrating factor results in
    \begin{equation*}
        \frac{d}{dt}\left(\left(1+t^2\right)x\right) = \cot t
    \end{equation*}
    and integrating both sides in the $t$ variable can be used to find the general solution
    \begin{equation*}
        x(t) = \frac{\ln(\sin t) + C}{1+t^2}
    \end{equation*}
    To find the corresponding flow transformation, let $x(t_0) = x_0$ for some $t_0$ and solve for $C$ to get $C = \left(1+t_0^2\right)x_0-\ln(\sin t_0)$. By substitution, the flow transformation is
    \begin{equation*}
        \Phi_{t_1,t_0}(x) = \frac{\ln(\sin t_1) + \left(1+t_0^2\right)x_0-\ln(\sin t_0)}{1+t_1^2}
    \end{equation*}
    In the case $\displaystyle x\left(\frac{\pi}{2}\right) = 2$, $C = \frac{4+\pi^2}{2}$, so the solution is
    \begin{equation*}
        x(t) = \frac{2\ln(\sin t)+4+\pi^2}{2\left(1+t^2\right)}
    \end{equation*}
    \item[1.3] Replace $x'$ with $\frac{1}{t'}$ and multiply both sides by $t'$ to yield $e^x-2tx=x^2t'$. Adding $2tx$ to both sides gives $e^x = x^2t' + 2tx$, and the right hand side is clearly the derivative of $x^2t$ with respect to $x$. Integrating both sides in $x$ gives $e^x + C  =x^2t$. Solving for $t$ gives
    \begin{equation*}
        t(x) = \frac{e^x + C}{x^2}
    \end{equation*}
    \item[1.6] $x' = \frac{1}{3}x + e^{-2t}x^{-2}$ is a Bernoulli equation, so make the substitution $V = x^{1-(-2)} = x^3$ to get a new differential equation
    \begin{equation*}
        V' = V + 3e^{-2t} 
    \end{equation*}
    which can be multiplied by $e^{-t}$ on both sides to give
    \begin{equation*}
        \left(e^{-t}V\right)' = 3e^{-3t}
    \end{equation*}
    and integrated to yield
    \begin{equation*}
        e^{-t}V = C-e^{-3t}
    \end{equation*}
    so $x(t) = \sqrt[3]{Ce^{t}-e^{-2t}}$. To find the corresponding flow transformation, first solve for $C$ using $x(t_0) = x_0$ to yield
    \begin{equation*}
        C = x_0^3e^{-t_0} + e^{-3t_0}
    \end{equation*}
    and replace $t$ with $t_1$ to get
    \begin{equation*}
        \Phi_{t_1, t_0}(x) = \sqrt[3]{\left(x_0^3e^{-t_0}+e^{-3t_0}\right)e^{t_1}-e^{-2t_1}}
    \end{equation*}
    If $x(0) = 2$, then $C = 9$, so
    \begin{equation*}
        x(t) = \sqrt[3]{9e^t-e^{-2t}}
    \end{equation*}
    \item[1.8] The differential equation $x' = x(1-x) - c$ with initial condition $x(0) = x_0$ is separable:
    \begin{equation*}
        \int_{x_0}^{x(t)}\frac{dx}{x(1-x)-c} = \int_{0}^{t}dt'
    \end{equation*}
    Performing partial fraction decomposition:
    \begin{equation*}
        \frac{1}{x(1-x)-c} = \frac{1}{\sqrt{1-4c}}\left(\frac{1}{\sqrt{\frac{1}{4}-c}+\frac{1}{2}-x}+\frac{1}{\sqrt{\frac{1}{4}-c}-\frac{1}{2}+x}\right)
    \end{equation*}
    which when integrated yields
    \begin{equation*}
        \left.\frac{1}{\sqrt{1-4c}}\ln\left(\frac{\sqrt{\frac{1}{4}-c}-\frac{1}{2}+x}{\sqrt{\frac{1}{4}-c}+\frac{1}{2}-x}\right)\right\vert_{x_0}^{x(t)} = t
    \end{equation*}
    so
    \begin{equation*}
        e^{t\sqrt{1-4c}}=\frac{\left(\sqrt{\frac{1}{4}-c}-\frac{1}{2}+x(t)\right)\left(\sqrt{\frac{1}{4}-c}+\frac{1}{2}-x_{0}\right)}{\left(\sqrt{\frac{1}{4}-c}+\frac{1}{2}-x(t)\right)\left(\sqrt{\frac{1}{4}-c}-\frac{1}{2}+x_{0}\right)}
    \end{equation*}
    and solving for $x(t)$ yields
    \begin{equation*}
        x(t) = \frac{\left(\frac{1}{2}x_{0}-c+x_{0}\sqrt{\frac{1}{4}-c}\right)e^{t\sqrt{1-4c}}-\left(\frac{1}{2}x_{0}-c-x_{0}\sqrt{\frac{1}{4}-c}\right)}{\sqrt{\frac{1}{4}-c}+\frac{1}{2}-x_{0}-e^{t\sqrt{1-4c}}\left(\frac{1}{2}-x_{0}-\sqrt{\frac{1}{4}-c}\right)}
    \end{equation*}
    Properties of the solution can be determined using Barrow's theorem. The solutions in $x$ to the equation $x(1-x) - c = 0$ are $\frac{1}{2} \pm \sqrt{\frac{1}{4} - c}$, so the maximal intervals of $x$ are $\left(-\infty, \frac{1}{2} - \sqrt{\frac{1}{4} - c}\right)$, $\left(\frac{1}{2} - \sqrt{\frac{1}{4} - c}, \frac{1}{2} + \sqrt{\frac{1}{4} - c}\right)$, and $\left(\frac{1}{2} + \sqrt{\frac{1}{4} - c}, \infty\right)$. So, for $\frac{1}{2} - \sqrt{\frac{1}{4} - c} < x_0 < \frac{1}{2} + \sqrt{\frac{1}{4} - c}$, $x(t)$ is defined over $t \in \mathbb{R}$, and $\lim_{t\to\infty}x(t) = \frac{1}{2} + \sqrt{\frac{1}{4} - c}$. When $x_0 > \frac{1}{2} + \sqrt{\frac{1}{4} - c}$, $x(t)$ is not defined everywhere, but it is defined for $t \geq 0$ with $\lim_{t\to\infty}x(t) = \frac{1}{2} + \sqrt{\frac{1}{4} - c}$. Meanwhile, for small $x_0$ in the range $0 < x_0 < \frac{1}{2} - \sqrt{\frac{1}{4} - c}$, $x(t)$ is defined only up to $t = \frac{1}{\sqrt{1-4c}}\ln\left(\frac{\frac{1}{2}-x_0+\sqrt{\frac{1}{4}-c}}{\frac{1}{2}-x_0-\sqrt{\frac{1}{4}-c}}\right)$. The limit as $t$ approaches this time (from the left) is $-\infty$.
\end{enumerate}

\section{Exercises from section 2.5}
\begin{enumerate}
    \item[2.1] The differential equation $x'(t) = \sin(x(t))\text{, }x(0) = x_0$ is separable, with
    \begin{equation*}
        \int_{0}^{t}dt' = \int_{x_0}^{x(t)}\frac{1}{\sin{x}}\,dx = \int_{u_0}^{u(t)}\frac{1+u^2}{2u}\frac{2}{1+u^2}du
    \end{equation*}
    where $u = \tan\left(\frac{x}{2}\right)$, so $u(t) = \tan\left(\frac{x(t)}{2}\right)$ and $u_0 = \tan\left(\frac{x_0}{2}\right)$. Then it follows that
    \begin{equation*}
        t = \ln{\left(\frac{u(t)}{u_0}\right)}
    \end{equation*}
    so $u(t) = u_0e^t$, and hence
    \begin{equation*}
        x(t) = 2\arctan\left(e^t\tan\left(\frac{x_0}{2}\right)\right) + 2\pi k
    \end{equation*}
    where $k = \lfloor\frac{x_0}{2\pi}+\frac{1}{2}\rfloor$. (This adjustment is necessary due to the range of the arctangent function. Since $k$ is a constant that takes strictly integer values, this adjustment disappears when differentiating $x(t)$, and also disappears in $\sin(x(t))$ because the $2\pi k$ term simply adds $k$ periods. This justifies its validity.) From the the formula for $x(t)$, it can be seen that $x(t)$ is defined for all $t \in \mathbb{R}$.
    \item[2.4]\begin{enumerate}
        \item Consider the second equation $\left(x'\right)^2-x^2=1$. Solving for $x'$ yields $x' = \pm\sqrt{1+x^2}$. It can be shown that the expression on the right-hand side is Lipschitz continuous. Differentiating in terms of $x$ on the right hand side gives $\pm\frac{x}{\sqrt{1+x^2}}$. Taking the absolute value gives $\left|\frac{x}{\sqrt{1+x^2}}\right| = \sqrt{\frac{x^2}{x^2+1}} < \sqrt{\frac{x^2+1}{x^2+1}} = 1$. So, absolute value of the derivative of this right-hand expression is bounded by 1. Letting $F(x) = \pm\sqrt{1+x^2}$, it follows that
        \begin{equation*}
            \left|F(x_2) - F(x_1)\right| = \left|F'(c)\left(x_2-x_1\right)\right| \leq \left|x_2-x_1\right|
        \end{equation*}
        where $x_1 < c < x_2$, showing that $F$ is Lipschitz with contant 1. Since it has been shown that equation (2) is Lipschitz continuous at all points, (2) is the equation that has a unique solution given the initial condition $x(0) = x_0$.
        \item Differentiate both sides of equation (1) to get
        \begin{equation*}
            2{x'}{x''}+2{x'}{x} = 0
        \end{equation*}
        so either $x'(t) = 0$ or $x''(t) + x(t) = 0$. The first possibility leaves $x(t) = c$ for some constant $c$, and plugging it into the original equation gives $c^2 = 1$, so $x(t) = \pm 1$. However, since $-1 < x_0 < 1$, discard this solution. Instead, consider $x''(t) + x(t) = 0$. The function $x(t) = \sin\left(t+c\right)$ satisfies this relation, and plugging it into $\left(x'\right)^2 + x^2$ gives $\cos\left(t+c\right) + \sin\left(t+c\right) = 1$, satisfying the original differential equation. Because of the initial condition $x(0) = x_0$, it is necessary to find some $c$ such that $x_0 = \sin c$. There are infinitely many such $c$ of the form $c = \arcsin{x_0} + 2\pi k$ and $c = \pi-\arcsin{x_0} + 2\pi k$, where $k$ is an integer; $\arcsin{x_0}$ must be defined because $-1 < x_0 < 1$. For example, if $x_0 = \frac{1}{\sqrt{2}}$, then both $x(t) = \sin\left(t+\frac{\pi}{4}\right)$ and $x(t) = \sin\left(t + \frac{3\pi}{4}\right)$ are distinct solutions of $\left(x'\right)^2 + x^2 = 1$ and $x(0) = \frac{1}{\sqrt{2}}$. These lead to two different solutions of the differential equation with the same $x_0$. However, it is possible to generate infinitely many solutions using piecewise functions. The sinusoidal solution around $t = 0$ can be used to satisfy the initial condition with $-1 < x(0) = x_0 < 1$, and outside of some region including $t = 0$, $x(t)$ can be constant (either $1$ or $-1$), as this is also a solution to the differential equation. For example, the piecewise function
        $$x(t) = 
        \begin{cases}
            \sin\left(-\pi k_1 -\frac{\pi}{2}\right) & -\pi k_1 > t + \arcsin{x_0} + \frac{\pi}{2} \\
            \sin\left(t+\arcsin\left(x_{0}\right)\right) & -\pi k_{1}<t+\arcsin\left(x_{0}\right)+\frac{\pi}{2}<\pi k_{2} \\
            \sin\left(\pi k_{2}-\frac{\pi}{2}\right) & \pi k_{2}<t+\arcsin\left(x_{0}\right)+\frac{\pi}{2}
        \end{cases}$$
        where $k_1\text{ and } k_2$ are two positive integers is constructed this way, and it describes infinitely many solutions to $\left(x'\right)^2 + x^2 = 1$ and $x(0) = x_0$ because there are infinitely many pairs of positive integers (note that this piecewise does not describe all of the infinite solutions for $x(t)$).
    \end{enumerate}
\end{enumerate}

\end{document}
