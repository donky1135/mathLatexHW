\documentclass[12pt, letterpaper]{article}
\date{\today}
\usepackage[margin=1in]{geometry}
\usepackage{amsmath}
\usepackage{hyperref}
\usepackage{cancel}
\usepackage{amssymb}
\usepackage{fancyhdr}
\usepackage{pgfplots}
\usepackage{booktabs}
\usepackage{pifont}
\usepackage{amsthm,latexsym,amsfonts,graphicx,epsfig,comment}
\pgfplotsset{compat=1.16}
\usepackage{xcolor}
\usepackage{tikz}
\usetikzlibrary{shapes.geometric}
\usetikzlibrary{arrows.meta,arrows}
\newcommand{\Z}{\mathbb{Z}}
\newcommand{\N}{\mathbb{N}}
\newcommand{\R}{\mathbb{R}}
\newcommand{\Po}{\mathcal{P}}

\author{Alex Valentino}
\title{Homework 7}
\pagestyle{fancy}
\renewcommand{\headrulewidth}{0pt}
\renewcommand{\footrulewidth}{0pt}
\fancyhf{}
\rhead{
	Homework 7\\
	292	
}
\lhead{
	Alex Valentino\\
}

% $\begin{bmatrix} \end{bmatrix}$

\begin{document}
	\begin{enumerate}
		\item[4.8]
		\begin{enumerate}
			\item Solving for $v(x,y) = (0,0)$ yields two solutions of the form $(\frac{-1}{2},\frac{1}{2})$ and $(\frac{3}{2},\frac{1}{2})$.  Putting these values into the jacobian of $v$ corresponding to $\begin{bmatrix} 2 x-1 & -2 y-1 \\ 2 x-1 & 3-2 y \\\end{bmatrix}$ yields the matrices $\begin{bmatrix} -2 & -2 \\ -2 & 2 \\ \end{bmatrix}$ and $\begin{bmatrix} 2 & -2 \\ 2 & 2 \\ \end{bmatrix}$ respectively.  These have eigenvalues of $\pm \sqrt{2}$ and $2 \pm 2 i$ respectively, therefore since both have eigenvalues with a real part greater than 0 then they are both unstable.  
			\item Solving for $v(x,y) = (0,0)$ yields two solutions of the form  $(\frac{-1}{2},\frac{1}{2})$ and $(\frac{3}{2},\frac{1}{2})$. Putting these values into the jacobian of $v$ corresponding to $\begin{bmatrix} 2 x-1 & 3-2 y \\ 2 x-1 &  -2 y-1 \\\end{bmatrix}$ yields the matrices $\begin{bmatrix} -2 & 2 \\ -2 & -2 \\ \end{bmatrix}$ and $\begin{bmatrix} 2 & 2 \\ 2 & -2 \\ \end{bmatrix}$.  These have eigenvalues of $\pm 2 \sqrt{2}$ and $-2\pm 2 i$.  Therefore $(-\frac{1}{2},\frac{1}{2})$ is stable and $(\frac{3}{2},\frac{1}{2})$ is unstable.  
		\end{enumerate}
		\item[4.10]
		\begin{enumerate}
			\item 		 Solving for $v(x,y) = (0,0)$ yields three solutions of the form $(0,0), (1,-1), (1,-2)$.  Putting these values into the jacobian of $v$ corresponding to $\begin{bmatrix} -2 (y+2) & -2 (y+2)-2 (x+y) \\ y & x-1 \\ \end{bmatrix}$ yields $\begin{bmatrix}  -4 & -4 \\ 0 & -1 \\\end{bmatrix}$,
	$\begin{bmatrix} -2 & -2 \\ -1 & 0 \\ \end{bmatrix}$, $\begin{bmatrix} 0 & 2 \\ -2 & 0 \\ \end{bmatrix}$ respectively.  These correspond with the eigenvalues $\{-4,-1\}$, $-1\pm \sqrt{3}$, and $\pm 2i$.  Therefore $(0,0)$ is stable, $(1,-1)$ is unstable.  However $(1,-2)$ has unknown behavior.  
			\item  Solving for $v(x,y) = (0,0)$ yields three solutions of the form $(0,0), (1,-1), (1,-2)$.  Putting these values into the jacobian of $v$ corresponding to $\begin{bmatrix} 2 (y+2) & 2 (y+2) + 2 (x+y) \\ y & x-1 \\ \end{bmatrix}$ yields  
			$\begin{bmatrix}  4 & 4 \\ 0 & -1 \\\end{bmatrix}$,	$\begin{bmatrix} 2 & 2 \\ -1 & 0 \\ \end{bmatrix}$, 
			$\begin{bmatrix} 0 & -2 \\ -2 & 0 \\ \end{bmatrix}$ respectively.   These correspond with the eigenvalues $\{4,-1\}$, $1\pm i$, and $\pm 2$.  Therefore all of the points are unstable.  
		\end{enumerate}				
		\item[5.1] 
		\begin{enumerate}
			\item[1] $\textbf{X}_1 = \Psi(\textbf{X}_0) = x_0 + \int_0^t v(x_0,s)ds = \int_0^t 2s ds = t^2$
			\item[2] $\textbf{X}_2 = \Psi(\textbf{X}_1) = x_0 + \int_0^t v(x_1,s)ds = \int_0^t 2s(1+s^2) ds = t^2 + \frac{1}{2}t^4$
			\item[3] $\textbf{X}_3 = \Psi(\textbf{X}_2) = x_0 + \int_0^t v(x_2,s)ds = \int_0^t 2s(1+s^2 + \frac{1}{2}s^4) ds = t^2 + \frac{1}{2}t^4 + \frac{1}{6} t^6$
			\item[4] $\textbf{X}_4 = \Psi(\textbf{X}_3) = x_0 + \int_0^t v(x_3,s)ds = \int_0^t 2s(1+s^2 + \frac{1}{2}s^4 + \frac{1}{6} s^6) ds = t^2 + \frac{1}{2}t^4 + \frac{1}{6} t^6 + \frac{1}{24} t^8$
		\end{enumerate}
		These terms correspond exactly with the solution of $e^{2t} -1$ as 
		$$ -1 + e^{2t} = -1 + \sum_{k=0}^\infty \frac{t^{2k}}{k!} = \sum_{k=1}^\infty \frac{t^{2k}}{k!} = t^2 + \frac{1}{2}t^4 + \frac{1}{6} t^6 + \frac{1}{24} t^8 + \cdots $$\\
		which shows the first four terms of the series we've found manually via Picard iteration.
		\item[5.2]
	\end{enumerate}
\end{document}