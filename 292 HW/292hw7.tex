\documentclass[12pt, letterpaper]{article}
\date{\today}
\usepackage[margin=1in]{geometry}
\usepackage{amsmath}
\usepackage{hyperref}
\usepackage{cancel}
\usepackage{amssymb}
\usepackage{fancyhdr}
\usepackage{pgfplots}
\usepackage{booktabs}
\usepackage{pifont}
\usepackage{amsthm,latexsym,amsfonts,graphicx,epsfig,comment}
\pgfplotsset{compat=1.16}
\usepackage{xcolor}
\usepackage{tikz}
\usetikzlibrary{shapes.geometric}
\usetikzlibrary{arrows.meta,arrows}
\newcommand{\Z}{\mathbb{Z}}
\newcommand{\N}{\mathbb{N}}
\newcommand{\R}{\mathbb{R}}
\newcommand{\Po}{\mathcal{P}}

\author{Alex Valentino}
\title{Homework 7}
\pagestyle{fancy}
\renewcommand{\headrulewidth}{0pt}
\renewcommand{\footrulewidth}{0pt}
\fancyhf{}
\rhead{
	Homework 7\\
	292	
}
\lhead{
	Alex Valentino\\
}

% $\begin{bmatrix} \end{bmatrix}$

\begin{document}
	\begin{enumerate}
		\item[4.8]
		\begin{enumerate}
			\item Solving for $v(x,y) = (0,0)$ yields two solutions of the form $(\frac{-1}{2},\frac{1}{2})$ and $(\frac{3}{2},\frac{1}{2})$.  Putting these values into the jacobian of $v$ corresponding to $\begin{bmatrix} 2 x-1 & -2 y-1 \\ 2 x-1 & 3-2 y \\\end{bmatrix}$ yields the matrices $\begin{bmatrix} -2 & -2 \\ -2 & 2 \\ \end{bmatrix}$ and $\begin{bmatrix} 2 & -2 \\ 2 & 2 \\ \end{bmatrix}$ respectively.  These have eigenvalues of $\pm \sqrt{2}$ and $2 \pm 2 i$ respectively, therefore since both have eigenvalues with a real part greater than 0 then they are both unstable.  
			\item Solving for $v(x,y) = (0,0)$ yields two solutions of the form  $(\frac{-1}{2},\frac{1}{2})$ and $(\frac{3}{2},\frac{1}{2})$. Putting these values into the jacobian of $v$ corresponding to $\begin{bmatrix} 2 x-1 & 3-2 y \\ 2 x-1 &  -2 y-1 \\\end{bmatrix}$ yields the matrices $\begin{bmatrix} -2 & 2 \\ -2 & -2 \\ \end{bmatrix}$ and $\begin{bmatrix} 2 & 2 \\ 2 & -2 \\ \end{bmatrix}$.  These have eigenvalues of $\pm 2 \sqrt{2}$ and $-2\pm 2 i$.  Therefore $(-\frac{1}{2},\frac{1}{2})$ is stable and $(\frac{3}{2},\frac{1}{2})$ is unstable.  
		\end{enumerate}
		\item[4.10]
		\begin{enumerate}
			\item 		 Solving for $v(x,y) = (0,0)$ yields three solutions of the form $(0,0), (1,-1), (1,-2)$.  Putting these values into the jacobian of $v$ corresponding to $\begin{bmatrix} -2 (y+2) & -2 (y+2)-2 (x+y) \\ y & x-1 \\ \end{bmatrix}$ yields $\begin{bmatrix}  -4 & -4 \\ 0 & -1 \\\end{bmatrix}$,
	$\begin{bmatrix} -2 & -2 \\ -1 & 0 \\ \end{bmatrix}$, $\begin{bmatrix} 0 & 2 \\ -2 & 0 \\ \end{bmatrix}$ respectively.  These correspond with the eigenvalues $\{-4,-1\}$, $-1\pm \sqrt{3}$, and $\pm 2i$.  Therefore $(0,0)$ is stable, $(1,-1)$ is unstable.  However $(1,-2)$ has unknown behavior.  
			\item  Solving for $v(x,y) = (0,0)$ yields three solutions of the form $(0,0), (1,-1), (1,-2)$.  Putting these values into the jacobian of $v$ corresponding to $\begin{bmatrix} 2 (y+2) & 2 (y+2) + 2 (x+y) \\ y & x-1 \\ \end{bmatrix}$ yields  
			$\begin{bmatrix}  4 & 4 \\ 0 & -1 \\\end{bmatrix}$,	$\begin{bmatrix} 2 & 2 \\ -1 & 0 \\ \end{bmatrix}$, 
			$\begin{bmatrix} 0 & -2 \\ -2 & 0 \\ \end{bmatrix}$ respectively.   These correspond with the eigenvalues $\{4,-1\}$, $1\pm i$, and $\pm 2$.  Therefore all of the points are unstable.  
		\end{enumerate}				
		\item[5.1] 
		\begin{enumerate}
			\item[1] $\textbf{X}_1 = \Psi(\textbf{X}_0) = x_0 + \int_0^t v(x_0,s)ds = \int_0^t 2s ds = t^2$
			\item[2] $\textbf{X}_2 = \Psi(\textbf{X}_1) = x_0 + \int_0^t v(x_1,s)ds = \int_0^t 2s(1+s^2) ds = t^2 + \frac{1}{2}t^4$
			\item[3] $\textbf{X}_3 = \Psi(\textbf{X}_2) = x_0 + \int_0^t v(x_2,s)ds = \int_0^t 2s(1+s^2 + \frac{1}{2}s^4) ds = t^2 + \frac{1}{2}t^4 + \frac{1}{6} t^6$
			\item[4] $\textbf{X}_4 = \Psi(\textbf{X}_3) = x_0 + \int_0^t v(x_3,s)ds = \int_0^t 2s(1+s^2 + \frac{1}{2}s^4 + \frac{1}{6} s^6) ds = t^2 + \frac{1}{2}t^4 + \frac{1}{6} t^6 + \frac{1}{24} t^8$
		\end{enumerate}
		These terms correspond exactly with the solution of $e^{2t} -1$ as 
		$$ -1 + e^{2t} = -1 + \sum_{k=0}^\infty \frac{t^{2k}}{k!} = \sum_{k=1}^\infty \frac{t^{2k}}{k!} = t^2 + \frac{1}{2}t^4 + \frac{1}{6} t^6 + \frac{1}{24} t^8 + \cdots $$\\
		which shows the first four terms of the series we've found manually via Picard iteration.
		\item[5.2] Credit to CT Lim for telling me about the series expansion trick.\\
		Note that after performing the substitution $s=t e^q$, the first three terms of our flow transformation's power series in $q$ are: $$\begin{bmatrix}
		1 & -q t \\
 -\frac{q}{t} & q+1 \\
\end{bmatrix},
\begin{bmatrix}
		 \frac{q^2}{2}+1 & -q^2 t-q t \\
 -\frac{q}{t} & q^2+q+1 \\
\end{bmatrix},
\begin{bmatrix}
 \frac{q^2}{2}+1 & -\frac{2 q^3 t}{3}-q^2 t-q t \\
 -\frac{q^3}{6 t}-\frac{q}{t} & \frac{2 q^3}{3}+q^2+q+1 \\
\end{bmatrix}.
 $$
 Also note that the first 3 terms in the power series expansions of $e^q$ and $e^{-q}$ around 0 are:
 $
 \frac{q^3}{6}+\frac{q^2}{2}+q+1
 $
		and
 $
 -\frac{q^3}{6}+\frac{q^2}{2}-q+1
 $ respectively.  
		\begin{enumerate}
			\item[1] $\textbf{X}_1 = \Psi(\textbf{X}_0) = \begin{bmatrix}  1 & t-e^q t \\
 \frac{e^{-q}}{t}-\frac{1}{t} & q+1 \\\end{bmatrix} \textbf{x}_0$.  Note that taking the first term of the power series of $\textbf{X}_1$ we get
 $\begin{bmatrix}
  1 & t-(1+q) t \\
 \frac{1-q}{t}-\frac{1}{t} & q+1 \\
 \end{bmatrix}\textbf{x}_0 = \begin{bmatrix}
		1 & -q t \\
 -\frac{q}{t} & q+1 \\
\end{bmatrix}\textbf{x}_0$, exactly the first term in the power series of $[\Phi_{t,s}]$.
		\item[2] $\textbf{X}_2 = \Psi(\textbf{X}_1) = \begin{bmatrix}
		 q+e^{-q} & -2 e^q t+q t+2 t \\
 \frac{e^{-q} q}{t}+\frac{2
   e^{-q}}{t}-\frac{2}{t} &
   \frac{q^2}{2}+e^q \\
\end{bmatrix}\textbf{x}_0$.   Note that taking up to the second term of the power series of $\textbf{X}_2$ we get
$$
	\begin{bmatrix}
	 q+\frac{q^2}{2}-q+1 & -2 ( \frac{q^2}{2}+q+1) t+q t+2 t \\
 \frac{q-q^2}{t}+\frac{2
   (\frac{q^2}{2}-q+1)}{t}-\frac{2}{t} &
   \frac{q^2}{2}+\frac{q^2}{2}+ q+1 \\
	\end{bmatrix} \textbf{x}_0
	=	\begin{bmatrix}
		 \frac{q^2}{2}+1 & -q^2 t-q t \\
 -\frac{q}{t} & q^2+q+1 \\
\end{bmatrix}\textbf{x}_0,
$$ exactly the second term in the power series of $[\Phi_{t,s}]$.
		\item[3] $\textbf{X}_3 = \Psi(\textbf{X}_2) = \begin{bmatrix}
		 e^{-q} q+2 q+3 e^{-q}-2 & \frac{t q^2}{2}-e^q t q+t
   q-e^q t+t \\
 \frac{e^{-q} q^2}{2 t}+\frac{e^{-q}
   q}{t}-\frac{q}{t}+\frac{e^{-q}}{t}-\frac{1}{t} &
   \frac{q^3}{6}-\frac{q^2}{2}-2 q+3 e^q-2 \\
\end{bmatrix}\textbf{x}_0 $. Note that taking up to the third term of the power series of $\textbf{X}_3$ we get
$$
	\begin{bmatrix}
		 q-q^2+\frac{q^3}{2} +2 q+3(-\frac{q^3}{6}+\frac{q^2}{2}-q+1)-2 & \frac{t q^2}{2}-(q+q^2+\frac{q^3}{2})t+t
   q-( \frac{q^3}{6}+\frac{q^2}{2}+q+1
) t+t \\
 \frac{q^2 - q^3}{2 t}+\frac{q-q^2+\frac{q^3}{2}}{t}-\frac{q}{t}+\frac{ -\frac{q^3}{6}+\frac{q^2}{2}-q+1
}{t}-\frac{1}{t} &
   \frac{q^3}{6}-\frac{q^2}{2}-2 q+3( \frac{q^3}{6}+\frac{q^2}{2}+q+1
)-2 \\
\end{bmatrix}\textbf{x}_0
$$
$$= \begin{bmatrix}
 \frac{q^2}{2}+1 & -\frac{2 q^3 t}{3}-q^2 t-q t \\
 -\frac{q^3}{6 t}-\frac{q}{t} & \frac{2 q^3}{3}+q^2+q+1 \\
\end{bmatrix}\textbf{x}_0,
$$
exactly the second term in the power series of $[\Phi_{t,s}]$.
		\end{enumerate}
	\end{enumerate}
\end{document}