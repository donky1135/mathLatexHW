\documentclass[12pt, letterpaper]{article}
\date{\today}
\usepackage[margin=1in]{geometry}
\usepackage{amsmath}
\usepackage{hyperref}
\usepackage{cancel}
\usepackage{amssymb}
\usepackage{fancyhdr}
\usepackage{pgfplots}
\usepackage{booktabs}
\usepackage{pifont}
\usepackage{amsthm,latexsym,amsfonts,graphicx,epsfig,comment}
\pgfplotsset{compat=1.16}
\usepackage{xcolor}
\usepackage{tikz}
\usetikzlibrary{shapes.geometric}
\usetikzlibrary{arrows.meta,arrows}
\newcommand{\Z}{\mathbb{Z}}
\newcommand{\N}{\mathbb{N}}
\newcommand{\R}{\mathbb{R}}
\newcommand{\Po}{\mathcal{P}}

\author{Alex Valentino}
\title{Homework }
\pagestyle{fancy}
\renewcommand{\headrulewidth}{0pt}
\renewcommand{\footrulewidth}{0pt}
\fancyhf{}
\rhead{
	Homework 8\\
	292	
}
\lhead{
	Alex Valentino\\
}
\begin{document}
\begin{enumerate}
	\item[3] In order to find the solution of the non-homogeneous equation, we must first find a solution to the homogeneous one.  Guessing the form of the solution to be $t^\alpha$ we get that
	\begin{align*}
		\alpha(\alpha - 1) t^\alpha - 2 t^\alpha &= 0\\
		\alpha(\alpha - 1) - 2 &= 0\\
		(\alpha -2)(\alpha +1) &= 0\\
		\alpha &\in \{-1,2\}
	\end{align*}
	Since both $t^2$ and $1/t$ satisfy the differential equation, then $x_1(t) = t^2, x_2 (t) = \frac{1}{t}$.  Thus we have two linearly independent solutions. For the given initial condition we have the system $\begin{bmatrix} 1 & 1 \\ 2 & -1\\ \end{bmatrix} \begin{bmatrix} c_1 \\ c_2 \end{bmatrix} = \begin{bmatrix} 1 \\ 2 \end{bmatrix}$, which when solved gives the solution $c_1 =1, c_2 = 0$.  Computing $\det (M(s)) = -3, \det (N(t,s))) = \frac{s^2}{t} - \frac{t^2}{s}$, we are now ready to apply the variation of constants formula:
	\begin{align*}
		x(t) &= t^2 +  \int_1^t \frac{\det(N(t,s))}{\det(M(s))} r(s) ds\\
		&= t^2 -\int_1^t (\frac{s^2}{t} - \frac{t^2}{s}) \frac{1}{s^3}ds\\
		&= t^2 + \frac{t^3 -3 \log(t) -1}{3t}\\ &\text{ t is strictly positive, no absolute value}\\
		&= \frac{4t^3 - 3\log(t) -1}{3t}.
	\end{align*}
	\item[4] 
	In order to find the solution of the non-homogeneous equation, we must first find a solution to the homogeneous one.  Guessing the form of the solution to be $e^{\alpha t}$ we get that 
	\begin{align*}
		\alpha^2 e^{\alpha t} - \alpha \frac{t+2}{t}e^{\alpha t} +\frac{2}{t} e^{\alpha t} &= 0\\
		\alpha^2 - \alpha \frac{t+2}{t}+\frac{2}{t} &= 0\\
		\alpha^2 -3\alpha +2 &= 0 & \text{evaluating at } t=1\\
		(\alpha - 1)(\alpha - 2) &= 0\\
		\alpha &\in \{ 1,2 \}
	\end{align*}
	After testing the values of $\alpha$ we find that only $\alpha = 1$ is a valid value.  Therefore $x_1 (t) = e^t$.
	Therefore $$x_2(t) = x_1(t) v(t) = x_1(t) \int \frac{1}{x_1^2(t)} e^{P(t)} dt = e^t \int \frac{t^2 e^t}{e^{2t}}dt = - (t^2 + 2t +2).$$
	Thus we have two linearly independent solutions. For the given initial condition we have the system $\begin{bmatrix} e & -5 \\ e & -4\\ \end{bmatrix} \begin{bmatrix} c_1 \\ c_2 \end{bmatrix} = \begin{bmatrix} 1 \\ 3 \end{bmatrix}$, which when solved gives the solution $c_1 = \frac{11}{e}, c_2 = 2$.  
	Computing $\det (M(s)) = s^2 e^s, \det (N(t,s))) = e^t(s^2 + 2s + 2) - e^2(t^2 +2t+2)$, we are now ready to apply the variation of constants formula:
	\begin{align*}
		x(t) &= 11e^{t-1} -2(t^2 + 2t +2) + \int_1^t \frac{\det(N(t,s))}{\det(M(s))} r(s) ds\\
		&= 11e^{t-1} -2(t^2 + 2t +2) + \int_1^t \frac{e^t(s^2 + 2s + 2) - e^2(t^2 +2t+2)}{s^2 e^s}s^2 e^s ds\\
		&= 11e^{t-1} -2(t^2 + 2t +2) + e^t(\frac{1}{3}t^3 - \frac{16}{3}) + e(t^2+2t+2).
	\end{align*}	  
	\item[7]
	 In order to find the solution of the non-homogeneous equation, we must first find a solution to the homogeneous one.  Guessing the form of the solution to be $e^{\alpha t}$ we get that 
	 \begin{align*}
	 \alpha^2 e^{\alpha t} - \alpha e^{\alpha t} -6 e^{\alpha t} &= 0\\
	 \alpha^2 - \alpha - 6 &= 0\\
	 (\alpha + 2)(\alpha - 3) &= 0
	 \end{align*}
	 Since both $e^{-2t}$ and $e^{3t}$ satisfy the differential equation, then $x_1(t) = e^{-2t}, x_2 = e^{3t}$.   
	 Computing $\det (M(s)) = 5 e^s, \det (N(t,s))) = e^{3t -2s} -e^{3s-2t}$, we are now ready to apply the variation of constants formula:
	 \begin{align*}
	 	x(t) &= c_1 e^{-2t} + c_2 e^{3t} +  \int_{t_0}^t \frac{\det(N(t,s))}{\det(M(s))} r(s) ds\\
	 	&= c_1 e^{-2t} + c_2 e^{3t} + \frac{1}{5} \int_{t_0}^te^{3t -2s} -e^{3s-2t} ds\\
	 	&=  c_1 e^{-2t} + c_2 e^{3t} + \frac{1}{5}  (\frac{1}{2} e^{3t - 2 t_0} + \frac{1}{3} e^{3 t_0 - 2 t} - \frac{5}{6} e^t)\\
	 	&\text{since the two terms with } t_0 \text{ are multiples of } x_1, x_2 \text{ our solution becomes}\\
	 	&= c_1 e^{-2t} + c_2 e^{3t} -\frac{1}{6} e^t.
	 \end{align*}
\end{enumerate}
\end{document}