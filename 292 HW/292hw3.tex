\documentclass[12pt, letterpaper]{article}
\date{\today}
\usepackage[margin=1in]{geometry}
\usepackage{amsmath}
\usepackage{hyperref}
\usepackage{cancel}
\usepackage{amssymb}
\usepackage{fancyhdr}
\usepackage{pgfplots}
\usepackage{booktabs}
\usepackage{pifont}
\usepackage{amsthm,latexsym,amsfonts,graphicx,epsfig,comment}
\pgfplotsset{compat=1.16}
\usepackage{xcolor}
\usepackage{tikz}
\usetikzlibrary{shapes.geometric}
\usetikzlibrary{arrows.meta,arrows}
\newcommand{\Z}{\mathbb{Z}}
\newcommand{\N}{\mathbb{N}}
\newcommand{\R}{\mathbb{R}}
\newcommand{\Po}{\mathcal{P}}

\author{Alex Valentino}
\title{Homework 3}
\pagestyle{fancy}
\renewcommand{\headrulewidth}{0pt}
\renewcommand{\footrulewidth}{0pt}
\fancyhf{}
\rhead{
	Homework 3\\
	292	
}
\lhead{
	Alex Valentino\\
}
\begin{document}
\begin{enumerate}
	\item[2.5]
	\item[2.6]
	\item[2.10]
	\begin{enumerate}
		\item The inverse transform is given by:
		$$
		(x,y) = (e^{-u}, \frac{v}{e^{2u}}).
		$$
		\item Time derivatives of $u,v$:
			$$
				\frac{d}{dt}u = \frac{-x'}{x} = \frac{-x}{x} = 1                                                                              
			$$
			\begin{align*}
			\frac{d}{dt}v &= 2xy x' + x^2 y'\\
			&= 2x^2 y + x^2((xy-\frac{1}{x})^2 -\frac{2}{x^2})\\	
			&= 2x^2 y + (x^2y -1)^2 - 1\\
			&= 2x^2 y + x^4y^2 -2x^2y + 1 - 2\\
			&= x^4y^2 -1\\
			&= v^2 - 1
			\end{align*}
			Thus the vector field $\vec{w}$ for the system $\Vec{u}' = \Vec{w}(\Vec{u})$ is given by:
			$\Vec{w} = (-1,v^2 -1)$.  This is clearly decoupled as specified.
		\item Solving the decoupled system for $\Vec{u}(0) = (u_0,v_0)$.  
			Since $u' = -1$, then $u = u_0 - t.$  For $v' = v^2 - 1$, by barrow's formula we get the equation
			$$
				t = \int_{v_0}^v \frac{dz}{z^2 - 1}.		
			$$
			Splitting $\frac{1}{z^2 - 1}$ apart by partial fraction decomposition yields
			$$
				\frac{1}{z^2 - 1} = -1 (\frac{1}{2(1-v)} + \frac{1}{2(v+1)}).
			$$
			This results in the integral being evaluated as 
			$$
				t = -ln(\sqrt{\frac{v+1}{1-v}}) + ln(\sqrt{\frac{v_0+1}{1-v_0}}).			
			$$
			Note that $ln\left(\sqrt{\frac{v+1}{1-v}}\right) = artanh(v)$, therefore we can directly express $v$ now:
			$$
				v = tanh(artanh(v_0) - t).			
			$$
			We must show that this solution for $\Vec{u}$ with $\Vec{u}(0) = (u_0,v_0)$ exists uniquely for all $t$ if and only if $|v_0| \leq 1$.
			\begin{itemize}
				\item $(\Rightarrow)$  Suppose the solution given above at $\Vec{u} = (u_0, v_0)$ exists for all $t$ and is unique.  Then we must show $|v_0| \leq 1$.  Suppose for contradiction that $|v_0| > 1$.  Then we must evaluate $artanh(v_0)$, however for $v_0>1$, artanh is not defined.  This contradicts $u$ existing for all $t$.  Therefore $|v_0| \leq 1$.
				\item $(\Leftarrow)$  Suppose $|v_0| \leq 1$.  We must show there exists a unique solution for all $t$ at $\Vec{u} = (u_0, v_0)$. Note that by definition we're operating inside of the maximal interval $(-1,1)$ and the endpoints $\{-1,1\}$.  First for the cases where $v_0 \in (-1,1)$.  Since we need to show the existence and uniqueness of a solution, we simply need to show that $\Vec{w}$ is Lipschitz on $(-1,1)$.  Note that since $w_1 = -1$, that for any value of $v_0$, $w_1$ is always bounded.  For $v' = w_2 = v^2 -1$, since $v\in (-1,1)$, then $max(|w_2(v)|) = 1$.  Then we have the inequality 
				$$				
				|x^2 - 1 -y^2 + 1| \leq |x^2 - y^2| \leq |x+y| |x-y| \leq 2|x-y|
				$$
				Therefore on $(-1,1)$ we have each component of $\Vec{w}$ lipschitz continuous, thus $\|\Vec{w}\|$ is lipschitz.  For the case of $v_0 = 1$, we must show that the constant solution is the only one for $v' = v^2 - 1$.  Since $\lim_{\delta \to 0} \int_{1-\delta}^1 \frac{dz}{|z^2-1|} \geq \lim_{\delta \to 0} |artanh(1-\delta) - artanh(1)| = \lim_{\delta \to 0} \infty = |artanh(1) - artanh(1+\delta)| \leq \lim_{\delta \to 0} \int_{1}^{1+\delta} \frac{dz}{|z^2-1|} $, then the times for which v leaves $1$ is infinite, therefore the constant solution is the unique solution when $v_0 = 1$.  Also note that $|artanh(x)| = |artanh(-x)|$, therefore these inequalities can be converted to also show the uniqueness of the steady state solution for $v=-1$. 
			\end{itemize}			 
			\item 
	\end{enumerate}
\end{enumerate}
\end{document}