\documentclass[12pt, letterpaper]{article}
\date{\today}
\usepackage[margin=1in]{geometry}
\usepackage{amsmath}
\usepackage{hyperref}
\usepackage{cancel}
\usepackage{amssymb}
\usepackage{fancyhdr}
\usepackage{pgfplots}
\usepackage{booktabs}
\usepackage{pifont}
\usepackage{amsthm,latexsym,amsfonts,graphicx,epsfig,comment}
\pgfplotsset{compat=1.16}
\usepackage{xcolor}
\usepackage{tikz}
\usetikzlibrary{shapes.geometric}
\usetikzlibrary{arrows.meta,arrows}
\newcommand{\Z}{\mathbb{Z}}
\newcommand{\N}{\mathbb{N}}
\newcommand{\R}{\mathbb{R}}
\newcommand{\Po}{\mathcal{P}}

\author{Alex Valentino}
\title{Homework 3}
\pagestyle{fancy}
\renewcommand{\headrulewidth}{0pt}
\renewcommand{\footrulewidth}{0pt}
\fancyhf{}
\rhead{
	Homework 3\\
	292	
}
\lhead{
	Alex Valentino\\
}
\begin{document}
\begin{enumerate}
	\item[2.5] 
		Note that the $x_1(t), x_2(t)$ are flow transformations, and can be given as $x_1 = \Psi_t (x_1), x_2 = \Psi_t (x_2)$.  Thus the "catching up" condition can be interpreted as $\Psi_0(x_2) = \Psi_T (x_1)$.  Therefore we have:
		$$
			\Psi_{(k+1)T} (x_1) = \Psi_{kT} \circ \Psi_T (x_1) = \Psi_{kT} \circ \Psi_0(x_2) = \Psi_{kT}(x_2).		
		$$  
	\item[2.6]
		\begin{enumerate}
			\item showing that $v(x) = tanh(x)$ is Lipschitz:
			$$
			tanh(x) = \frac{e^x - e^{-x}}{e^x+e^{-x}} < \frac{e^x + e^{-x}}{e^x+e^{-x}}	= 1		
			$$
			Therefore $|tanh(x) - tanh(y)| < 1 + 1 = 2$.  \\
			Finding the flow transformation:
			\begin{align*}
				t - t_0 &= \int_{x_0}^x \frac{dx}{tanh(x)}\\
				&= ln(sinh(x)) - ln(sinh(x_0))\\
				t- t_0 + ln(sinh(x_0)) &= ln(sinh(x))\\
				sinh(x) &= e^{t- t_0 + ln(sinh(x_0))}\\
				x &= arsinh(e^{t- t_0 + ln(sinh(x_0))}).
			\end{align*}
			Verifying that $\frac{d}{dx_0}\Psi_t (x_0) = \frac{v(\Psi_t (x_0))}{v(x_0)}$:
			\begin{align*}
			\frac{d}{dx}\Psi_t &= \frac{d}{dx}arsinh(e^{t- t_0 + ln(sinh(x_0))})\\
			&=  \frac{e^{t- t_0 + ln(sinh(x_0))}}{\sqrt{1 + e^{2(t- t_0 + ln(sinh(x_0)))}}}cosh(x_0)\\
			&= \frac{e^{t- t_0 + ln(sinh(x_0))}}{\sqrt{1 + e^{2(t- t_0 + ln(sinh(x_0)))}}}\frac{1}{tanh(x_0)}\\
			&= \frac{tanh(arsinh(e^{t- t_0 + ln(sinh(x_0))}))}{tanh(x_0)}\\
			&= \frac{v(\Psi_t(x_0))}{v(x_0)}.
			\end{align*}
			Verifying that $\frac{d}{dt} \Psi_t (x_0) = v(\Psi_t (x_0))$:
			\begin{align*}
				\frac{d}{dt} \Psi_t &= \frac{d}{dt}arsinh(e^{t- t_0 + ln(sinh(x_0))})\\
				&= \frac{e^{t- t_0 + ln(sinh(x_0))}}{\sqrt{1 + e^{2(t- t_0 + ln(sinh(x_0)))}}} \\
				&= tanh(arsinh(e^{t-t_0 + ln(sinh(x_0))})\\
				&= v(\Psi_t(x_0))
			\end{align*}
			\item 
			Verifying that $\lim_{t \to \infty} x_2(t) - x_1(t) = \int_{x_1}^{x_2} \frac{1}{v(x)}dx$:
			\begin{align*}
				\lim_{t \to \infty} x_2(t) - x_1(t) &= \lim_{t \to \infty} \int_{x_1}^{x_2} \frac{d}{dx} \Psi_t(x)dx\\
				&= \int_{x_1}^{x_2}\frac{\lim_{t \to \infty} v(\Psi_t (x))}{v(x)}dx\\
				&= \int_{x_1}^{x_2}\frac{\lim_{t \to \infty} \frac{e^{t- t_0 + ln(sinh(x_0))}}{\sqrt{1 + e^{2(t- t_0 + ln(sinh(x_0)))}}}}{v(x)}dx\\
				&= \int_{x_1}^{x_2}\frac{1}{v(x)}dx.\\
			\end{align*}
		\end{enumerate}
	\item[2.10]
	\begin{enumerate}
		\item The inverse transform is given by:
		$$
		(x,y) = (e^{-u}, \frac{v}{e^{2u}}).
		$$
		\item Time derivatives of $u,v$:
			$$
				\frac{d}{dt}u = \frac{-x'}{x} = \frac{-x}{x} = 1                                                                              
			$$
			\begin{align*}
			\frac{d}{dt}v &= 2xy x' + x^2 y'\\
			&= 2x^2 y + x^2((xy-\frac{1}{x})^2 -\frac{2}{x^2})\\	
			&= 2x^2 y + (x^2y -1)^2 - 1\\
			&= 2x^2 y + x^4y^2 -2x^2y + 1 - 2\\
			&= x^4y^2 -1\\
			&= v^2 - 1
			\end{align*}
			Thus the vector field $\vec{w}$ for the system $\Vec{u}' = \Vec{w}(\Vec{u})$ is given by:
			$\Vec{w} = (-1,v^2 -1)$.  This is clearly decoupled as specified.
		\item Solving the decoupled system for $\Vec{u}(0) = (u_0,v_0)$.  
			Since $u' = -1$, then $u = u_0 - t.$  For $v' = v^2 - 1$, by barrow's formula we get the equation
			$$
				t = \int_{v_0}^v \frac{dz}{z^2 - 1}.		
			$$
			Splitting $\frac{1}{z^2 - 1}$ apart by partial fraction decomposition yields
			$$
				\frac{1}{z^2 - 1} =  \frac{1}{2(v-1)} + \frac{-1}{2(v+1)}.
			$$
			This results in the integral being evaluated as 
			$$
				t - t_0 = ln\left(\sqrt{\frac{v-1}{v+1}}\right) - ln \left( \sqrt{\frac{v_0-1}{v_0 + 1}} \right).			
			$$
			\iffalse
			Note that $ln\left(\sqrt{\frac{v+1}{1-v}}\right) = artanh(v)$, therefore we can directly express $v$ now:
			$$
				v = tanh(artanh(v_0) - t).			
			$$
			\fi
			Inverting to get $v$ yields:
		$$
			v(t) = \frac{v_0+1+(v_0-1)e^{2t}}{v_0+1-(v_0-1)e^{2t}}		
		$$
			We must show that this solution for $\Vec{u}$ with $\Vec{u}(0) = (u_0,v_0)$ exists uniquely for all $t$ if and only if $|v_0| \leq 1$.
			\begin{itemize}
				\item $(\Rightarrow)$  Suppose the solution given above at $\Vec{u} = (u_0, v_0)$ exists for all $t$ and is unique.  Then we must show $|v_0| \leq 1$.  Suppose for contradiction that $|v_0| > 1$. Then let's see if we can get the denominator of $v$ to be $0$.  
				\begin{align*}
					0 &= v_0 + 1 - (v_0-1)e^{2t}\\
					(v_0-1)e^{2t} &= v_0 + 1\\
					2t &= ln \left( \frac{v_0 + 1}{v_0-1} \right)\\
					t &=\frac{1}{2} ln \left( \frac{v_0 + 1}{v_0-1} \right)
				\end{align*}										Since $v_0 > 1$, then $v_0 - 1 > 0$, therefore the natural log is defined, and we can get a value for t, which means that $v$ has a singularity, which contradicts the solution existing for all t.
				  Therefore $|v_0| \leq 1$.
				\item $(\Leftarrow)$  Suppose $|v_0| \leq 1$.  We must show there exists a unique solution for all $t$ at $\Vec{u} = (u_0, v_0)$. Note that by definition we're operating inside of the maximal interval $(-1,1)$ and the endpoints $\{-1,1\}$.  First for the cases where $v_0 \in (-1,1)$.  Since we need to show the existence and uniqueness of a solution, we simply need to show that $\Vec{w}$ is Lipschitz on $(-1,1)$.  Note that since $w_1 = -1$, that for any value of $v_0$, $w_1$ is always bounded.  For $v' = w_2 = v^2 -1$, since $v\in (-1,1)$, then $max(|w_2(v)|) = 1$.  Then we have the inequality 
				$$				
				|x^2 - 1 -y^2 + 1| \leq |x^2 - y^2| \leq |x+y| |x-y| \leq 2|x-y|
				$$
				Therefore on $(-1,1)$ we have each component of $\Vec{w}$ lipschitz continuous, thus $\|\Vec{w}\|$ is lipschitz.  For the case of $v_0 = 1$, we must show that the constant solution is the only one for $v' = v^2 - 1$.  Note that $\int_{v_0}^v \frac{dv}{v^2 -1} = -artanh(v) + artanh(v_0)$, and that $\int |f(x)| dx \geq | \int f(x) dx |$.
				
				Since: $$\lim_{\delta \to 0} \int_{1-\delta}^1 \frac{dz}{|z^2-1|} \geq \lim_{\delta \to 0} |artanh(1-\delta) - artanh(1)| = \infty 
				$$
				$$ \lim_{\delta \to 0} \int_{1}^{1+\delta} \frac{dz}{|z^2-1|} \geq \lim_{\delta \to 0} |artanh(1) - artanh(1+\delta)| = \infty  $$, then the times for which v leaves $1$ is infinite, therefore the constant solution is the unique solution when $v_0 = 1$.  Similarly for $v_0 = -1$ we have 
				
				$$
				\lim_{\delta \to 0} \int_{-1-\delta}^{-1} \frac{dz}{|z^2-1|} \geq \lim_{\delta \to 0} |artanh(-1-\delta) - artanh(-1)| = \infty 
				$$
				$$
				\lim_{\delta \to 0} \int_{-1}^{-1+\delta} \frac{dz}{|z^2-1|} \geq \lim_{\delta \to 0} |-artanh(-1+\delta) + artanh(-1)| = \infty. 
				$$
				Since the time it takes for the end points to be reached from both sides is bounded below by infinity, thus it takes infinite time to leave, therefore the only solution for the end points is constant.   
			\end{itemize}			 
			\item 
			$$			
				\Vec{u}(t) = \left( -1,	\frac{x_0^2 y_0+1+(x_0^2 y_0-1)e^{2t}}{x_0^2 y_0+1-(x_0^2 y_0-1)e^{2t}}\right)		
			$$
			$$\Vec{x}(t) = \left( x_0 e^t, \displaystyle \frac{1}{x_0^2 e^{2t}}\frac{x_0^2 y_0+1+(x_0^2 y_0-1)e^{2t}}{x_0^2 y_0+1-(x_0^2 y_0-1)e^{2t}}\right)$$
			
				\begin{align*}
				\Vec{x}' &= (x_0 e^t, \frac{-2}{x_0^2 e^{2t}}\frac{x_0^2 y_0+1+(x_0^2 y_0-1)e^{2t}}{x_0^2 y_0+1-(x_0^2 y_0-1)e^{2t}}\\
				&+ \frac{2}{x_0^2}\frac{x_0^2 y_0 - 1}{-e^{2t}(x_0^2 y_0-1)+x_0^2 y_0+1}+\frac{2(x_0^2 y_0-1)(e^{2t}(x_0^2 y_0-1)+x_0^2 y_0+1)}{x_0^2(-e^{2t}(x_0^2 y_0-1)+x_0^2 y_0+1)^2})\\
				&= (x,-2y - \frac{1}{x^2} + x^2 y^2) 		
				\end{align*}
			\item Skipped
			\item To solve the Ricatti equation, we assume  there exists a solution for $y$, denoted $y_1$ given by $y_1 =c e^{\alpha t}$.  Thus our equation becomes:
			$$
			\alpha c e^{\alpha t} = c^2 x_0^2 e^{2(\alpha + 1)t} - 2 ce^{\alpha t} - x_0^2 e^{2t}.			
			$$
			To remove the $e$ terms, we must solve $\alpha = 2\alpha + 2, \alpha = -2$.  Since $-2 = -4 + 2 = 2(-2) + 2$ then $\alpha = -2$.  Thus our equation becomes:
			$$
			-2c = c^2 x_0^2 -2c -x_0^2.			
			$$
			Thus $c = x_0^{-2}$.  Let $g := y - y_1$.  Solving for $g'$ :
			\begin{align*}
				g' &= g^2 x_0^2 e^{2t}\\
				\int \frac{dg}{g^2} &= x_0^2 \int e^{2t}dt\\
				\frac{-1}{g} &= \frac{x_0^2}{2}(e^{2t} + c_1)\\
				g &= \frac{1}{x_0^2}\frac{-2}{e^{2t}+c_1}\\ 
				y - x_0^{-2}e^{-2t}  &= \\
				y &= \frac{1}{x_0^2} \left( e^{-2t} - \frac{2}{e^{2t}+c_1}\right).
			\end{align*}
			
			We can solve for $c_1$ by evaluating $y$ at $0$:
			\begin{align*}
				y_0 &= \frac{1}{x_0^2} \left( 1 - \frac{2}{1+c_1}\right)\\
				x_0^2 y_0 &= 1 - \frac{2}{1+c_1}\\
				v_0 &= 1 - \frac{2}{1+c_1}\\
				v_0 -1 &= - \frac{2}{1+c_1}\\
				1-v_0 &= \frac{2}{1+c_1}\\
				\frac{2}{1-v_0} &= 1+c_1\\
				\frac{v_0 +1}{1-v_0} &= c_1.
			\end{align*}
			Therefore $y$ is given by:
			\begin{align*}
				y &= \frac{1}{x_0^2}\left( e^{-2t} - \frac{2}{e^{2t} + 	\frac{v_0 +1}{1-v_0}} \right)\\
				  &= \frac{1}{x_0^2}\left( \frac{-2 + 1 + \frac{v_0 +1}{1-v_0}e^{-2t}}{e^{2t} + 	\frac{v_0 +1}{1-v_0}}\right)\\
				  &= \frac{1}{x_0^2} \frac{\frac{v_0-1}{v_0 +1} + e^{-2t}}{\frac{1-v_0}{v_0 +1}e^{2t} + 1}\\
				  &= \frac{1}{x_0^2e^{2t}}\frac{\frac{v_0-1}{v_0 +1} + e^{-2t}}{\frac{1-v_0}{v_0 +1} + e^{-2t}}\\
				  &= \frac{1}{x_0^2e^{2t}}\frac{v_0 -1 + (v_0 + 1)e^{-2t}}{v_0 -1 + (v_0 + 1)e^{-2t}}\\
				  &= \frac{1}{x_0^2e^{2t}}\frac{v_0+1+(v_0-1)e^{2t}}{v_0+1-(v_0-1)e^{2t}}			
			\end{align*}
			Since this is the same as the formula we found via the change of variables, this formula is correct.  

			
	\end{enumerate}
\end{enumerate}
\end{document}