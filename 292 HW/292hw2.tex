\documentclass[12pt, letterpaper]{article}
\date{\today}
\usepackage[margin=1in]{geometry}
\usepackage{amsmath}
\usepackage{hyperref}
\usepackage{cancel}
\usepackage{amssymb}
\usepackage{fancyhdr}
\usepackage{pgfplots}
\usepackage{booktabs}
\usepackage{pifont}
\usepackage{amsthm,latexsym,amsfonts,graphicx,epsfig,comment}
\pgfplotsset{compat=1.16}
\usepackage{xcolor}
\usepackage{tikz}
\usetikzlibrary{shapes.geometric}
\usetikzlibrary{arrows.meta,arrows}
\newcommand{\Z}{\mathbb{Z}}
\newcommand{\N}{\mathbb{N}}
\newcommand{\R}{\mathbb{R}}
\newcommand{\Po}{\mathcal{P}}

\author{Alex Valentino}
\title{Homework 2}
\pagestyle{fancy}
\renewcommand{\headrulewidth}{0pt}
\renewcommand{\footrulewidth}{0pt}
\fancyhf{}
\rhead{
	Homework 2\\
	292	
}
\lhead{
	Alex Valentino\\
}
\begin{document}
\begin{enumerate}
	\item[2.4] 
	\begin{enumerate}
		\item For equation (2), we can define $x'$ explicitly by the following:
		\begin{equation*}
		x' = \pm \sqrt{1+x^2}.
		\end{equation*}		 
		Therefore taking the absolute value of $x'$ yields:
		\begin{equation*} |x'| = \sqrt{1+x^2} \leq \sqrt{x^2 + x^2} = \sqrt{2x^2} = \sqrt{2}|x|.
		\end{equation*}		 
		Since (2) has a lipschitz constant of $\sqrt{2}$, then by theorem 5 (2) has unique solutions for all $x_0 \in (-1,1)$.\\
		Turning our attention to (1), then we have the equation 
		\begin{equation*}
			x' = \pm \sqrt{1-x^2}
		\end{equation*}
		Taking the absolute value of the derivative of $x'$ yields 
		\begin{equation*}
		x'' = \frac{|x|}{\sqrt{1-x^2}}.
		\end{equation*}
		Taking the limit as $x$ approaches $-1$ yields:
		\begin{equation*}
		\lim_{x \to -1} |x''| = \frac{1}{\sqrt{1 - 1}} = \frac{1}{0} = \infty.
		\end{equation*}
		Since $|x'|$ is continuous on $(-1,1)$ and is not lipschitz then $(1)$ has infinite solutions.  
		\item Since (1) does not have a unique solutions we must show an infinite number of solutions to $(x')^2 + x^2 = 1, x(0) = x_0$.  Since $x(t) = 1$ solves the equation as 
		\begin{equation*}
		(x_0')^2 + x_0 = 0^2 + 1^2 = 1
		\end{equation*}
		but not the initial value of $x_0 \in (-1,1)$, we must solve the differential equations by other means to give another solution to interpolate with.\\
		Solving for non-steady state:
		\begin{align*}
			x' &= \pm \sqrt{1-x^2}\\
			1 &= \frac{\pm x'}{\sqrt{1-x^2}}\\
			\int_{t_0}^t dt &= \pm \int_{x_0}^x \frac{dz}{\sqrt{1-z^2}}\\
			t - t_0 &= \pm(\arcsin(x) - \arcsin(x_0))\\
			\pm (t-t_0 + \arcsin(x_0)) &= \arcsin(x)\\
			x(t) &= \pm \sin(t - t_0 + arcsin(x_0))
		\end{align*}
		Since $x(t) = 1, x(t) = \sin(t - t_0 + arcsin(x_0))$ both solve the differential equation, then we may create a new solution 
		\[
			x(t) = \begin{cases}
				1 & t > t_0 -arcsin(x_0) + 2 \pi a + \frac{\pi}{2}\\
				\sin(t - t_0 + arcsin(x_0)) & t \leq 	t_0 -arcsin(x_0) + 2 \pi a + \frac{\pi}{2}
			\end{cases}
		\]
		where $a \in \N\cup \{0\}$.
		Note that at $t=t_0 -arcsin(x_0+2\pi a + \frac{\pi}{2}$, that for $\sin$ we have:
		\begin{equation*}
			\sin(t - t_0 + arcsin(x_0)) = \sin(2\pi a + \frac{\pi}{2}) = \sin(\frac{\pi}{2}) = 1
		\end{equation*}
		and for the derivative
		\begin{equation*}
		cos(\frac{\pi}{2}) = 0
		\end{equation*}
		Which exactly aligns with the value and derivative of the constant function $x(t) = 1$.  Therefore since our solutions are continuous, and there exist one for each natural number, then we have found an infinite number of solutions.  
	\end{enumerate}
	\item[2.9]
		\begin{enumerate}
			\item Note that different classes of solutions are had for $\alpha = 1$ and $\alpha \neq 1$.  Proceeding with the $\alpha \neq 1$ case:
			\begin{equation}
			
			\end{equation}
		\end{enumerate}
\end{enumerate}
\end{document}