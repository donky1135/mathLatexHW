\documentclass[12pt, letterpaper]{article}
\date{\today}
\usepackage[margin=1in]{geometry}
\usepackage{amsmath}
\usepackage{hyperref}
\usepackage{cancel}
\usepackage{amssymb}
\usepackage{fancyhdr}
\usepackage{pgfplots}
\usepackage{booktabs}
\usepackage{pifont}
\usepackage{amsthm,latexsym,amsfonts,graphicx,epsfig,comment}
\pgfplotsset{compat=1.16}
\usepackage{xcolor}
\usepackage{tikz}
\usetikzlibrary{shapes.geometric}
\usetikzlibrary{arrows.meta,arrows}
\newcommand{\Z}{\mathbb{Z}}
\newcommand{\N}{\mathbb{N}}
\newcommand{\R}{\mathbb{R}}
\newcommand{\Po}{\mathcal{P}}

\author{Alex Valentino}
\title{Homework 2}
\pagestyle{fancy}
\renewcommand{\headrulewidth}{0pt}
\renewcommand{\footrulewidth}{0pt}
\fancyhf{}
\rhead{
	Homework 2\\
	292	
}
\lhead{
	Alex Valentino\\
}
\begin{document}
\begin{enumerate}
	\item[2.4] 
	\begin{enumerate}
		\item For equation (2), we can define $x'$ explicitly by the following:
		\begin{equation*}
		x' = \pm \sqrt{1+x^2}.
		\end{equation*}		 
		Therefore taking the absolute value of $x'$ yields:
		\begin{equation*} |x'| = \sqrt{1+x^2} \leq \sqrt{x^2 + x^2} = \sqrt{2x^2} = \sqrt{2}|x|.
		\end{equation*}		 
		Since (2) has a lipschitz constant of $\sqrt{2}$, then by theorem 5 (2) has unique solutions for all $x_0 \in (-1,1)$.\\
		Turning our attention to (1), then we have the equation 
		\begin{equation*}
			x' = \pm \sqrt{1-x^2}
		\end{equation*}
		Taking the absolute value of the derivative of $x'$ yields 
		\begin{equation*}
		x'' = \frac{|x|}{\sqrt{1-x^2}}.
		\end{equation*}
		Taking the limit as $x$ approaches $-1$ yields:
		\begin{equation*}
		\lim_{x \to -1} |x''| = \frac{1}{\sqrt{1 - 1}} = \frac{1}{0} = \infty.
		\end{equation*}
		Since $|x'|$ is continuous on $(-1,1)$ and is not lipschitz then $(1)$ has infinite solutions.  
		\item Since (1) does not have a unique solutions we must show an infinite number of solutions to $(x')^2 + x^2 = 1, x(0) = x_0$.  Since $x(t) = 1$ solves the equation as 
		\begin{equation*}
		(x_0')^2 + x_0 = 0^2 + 1^2 = 1
		\end{equation*}
		but not the initial value of $x_0 \in (-1,1)$, we must solve the differential equations by other means to give another solution to interpolate with.\\
		Solving for non-steady state:
		\begin{align*}
			x' &= \pm \sqrt{1-x^2}\\
			1 &= \frac{\pm x'}{\sqrt{1-x^2}}\\
			\int_{t_0}^t dt &= \pm \int_{x_0}^x \frac{dz}{\sqrt{1-z^2}}\\
			t - t_0 &= \pm(\arcsin(x) - \arcsin(x_0))\\
			\pm (t-t_0 + \arcsin(x_0)) &= \arcsin(x)\\
			x(t) &= \pm \sin(t - t_0 + arcsin(x_0))
		\end{align*}
		Since $x(t) = 1, x(t) = \sin(t - t_0 + arcsin(x_0))$ both solve the differential equation, then we may create a new solution 
		\[
			x(t) = \begin{cases}
				1 & t > t_0 -arcsin(x_0) + 2 \pi a + \frac{\pi}{2}\\
				\sin(t - t_0 + arcsin(x_0)) & t \leq 	t_0 -arcsin(x_0) + 2 \pi a + \frac{\pi}{2}
			\end{cases}
		\]
		where $a \in \N\cup \{0\}$.
		Note that at $t=t_0 -arcsin(x_0+2\pi a + \frac{\pi}{2})$, that for $\sin$ we have:
		\begin{equation*}
			\sin(t - t_0 + arcsin(x_0)) = \sin(2\pi a + \frac{\pi}{2}) = \sin(\frac{\pi}{2}) = 1
		\end{equation*}
		and for the derivative
		\begin{equation*}
		cos(\frac{\pi}{2}) = 0
		\end{equation*}
		Which exactly aligns with the value and derivative of the constant function $x(t) = 1$.  Therefore since our solutions are continuous, and there exist one for each natural number, then we have found an infinite number of solutions.  
	\end{enumerate}
	\item[2.9]
		\begin{enumerate}
			\item Note that different classes of solutions are had for $\alpha = 1$ and $\alpha \neq 1$.  Proceeding with the $\alpha \neq 1$ case:
			\begin{equation*}
			x' = x|ln|x||^\alpha
			\end{equation*}
			Applying barrow's formula yields:
			\begin{equation*}
			\int_{x_0}^x \frac{dx}{x|ln|x||^\alpha} = t-t_0
			\end{equation*}
			Note that $x$ is within the maximal interval $(0,1)$, therefore $|x| = x, |ln(x)| = -ln(x)$.  Thus the substitutions of $u = |ln|x||$ and $du = \frac{-1}{x}dx$ may be made:
			\begin{equation*}
				- \int_{x_0}^x u^{-\alpha}du = t - t_0. 
			\end{equation*}			   
			Therefore after evaluation we have:
			\begin{equation*}
				\frac{-1}{1-\alpha}(u(x)^{1-\alpha}-u(x_0)^{1-\alpha}) = t - t_0
			\end{equation*}
			Let $k = 1 - \alpha$, therefore by algebraic manipulation we have
			\begin{align*}
				\frac{-1}{k}(u(x)^{k}-u(x_0)^{k}) &= t - t_0\\
				u(x)^{k}-u(x_0)^{k} &= k(t_0-t)\\
				u(x)^k &= k(t_0-t) + u(x_0)^{k}\\
				u(x) &= (k(t_0-t) + u(x_0)^{k})^\frac{1}{k}\\
				-ln(x) &= (k(t_0-t) + u(x_0)^{k})^\frac{1}{k}\\
				ln(x) &= -(k(t_0-t) + u(x_0)^{k})^\frac{1}{k}\\
				x &= e^{-(k(t_0-t) + u(x_0)^{k})^\frac{1}{k}}.
			\end{align*}
			Some observations:  if $\alpha > 1$ then $1-\alpha = k < 1$.  Since $k$ is negative, then $u(0)^k = (-ln(0))^k = \infty^k = 0, u(1)^k = (-ln(1))^k = 0^k = \infty$.  On the other hand if $\alpha < 1$ then $1-\alpha = k > 1$.  Since $k$ is positive, then $u(1)^k = (-ln(1))^k = 0^k = 0, u(0)^k = (-ln(0))^k = \infty^k = \infty$.  Therefore evaluating $T_0, T_1$ for the cases of $\alpha >1, \alpha < 1$ yields:
			\begin{enumerate}
				\item $T_0, \alpha > 1$: $$t_0 + -\int_{x_0}^0 u^{-\alpha} = t_0 + \frac{-1}{k}(u(0)^{k}-u(x_0)^{k}) = t_0 + \frac{u(x_0)^{k}}{k} < \infty$$
				\item $T_1, \alpha > 1$: $$t_0 + -\int_{x_0}^1 u^{-\alpha} = t_0 + \frac{-1}{k}(u(1)^{k}-u(x_0)^{k}) =  -\infty$$
				\item $T_0, \alpha < 1$: $$t_0 + -\int_{x_0}^0 u^{-\alpha} = t_0 + \frac{-1}{k}(u(0)^{k}-u(x_0)^{k}) =  -\infty$$
				\item $T_1, \alpha < 1$: $$t_0 + -\int_{x_0}^1 u^{-\alpha} = t_0 + \frac{-1}{k}(u(1)^{k}-u(x_0)^{k}) = t_0 + \frac{u(x_0)^{k}}{k} < \infty$$
\end{enumerate}			 
			Since $T_0$ is finite for $\alpha > 1$, then the solution does not hold for all $t$, however since $T_1$ is infinite, then we have found the solutions which are valid for $t>t_0$.  Similarly for $\alpha < 1, T_1$ is finite, and $T_0$ is infinite, giving us another partial solution for $t < t_0$. Let us evaluate the $\alpha = 1$ case.  The integral in terms of $u$ from barrow's formula is still valid, but it's evaluation is different:
			\begin{align*}
				-\int_{x_0}^x u^{-1}du &= t-t_0\\
				ln|u(x)| - ln|u(x_0)| &= t_0-t\\
				ln|u(x)| & = t_0 -t + ln|u(x_0)|\\
				u(x) &= e^{t_0 -t + ln|u(x_0)|}\\
				-ln(x) &= e^{t_0 -t + ln|u(x_0)|}\\
				x &= e^{-e^{t_0 -t + ln|u(x_0)|}}				
			\end{align*}
			Evaluating on the endpoints:
			\begin{enumerate}
				\item $$T_0 = t_0 + -\int_{x_0}^0 u^{-1}du = t_0 - ln|u(0)| + ln|u(x_0)| = t_0 - ln(\infty) + ln|u(x_0)| = -\infty$$
				\item $$T_1 = t_0 + -\int_{x_0}^1 u^{-1}du = t_0 - ln|u(1)| + ln|u(x_0)| = t_0 - ln(0) + ln|u(x_0)|= \infty$$
			\end{enumerate}
			Since the endpoints take an infinite amount of time to achieve, we have found the unique solution for all $t$.
			\iffalse
			However, if $\alpha > 1$, by the equation we got via barrows formula, then $1-\alpha = k < 1$.  Since $k$ is negative, then $u(0)^k = (-ln(0))^k = \infty^k = 0$, therefore: 
			\begin{equation*}
				t_0 + -\int_{x_0}^0 u^{-\alpha} = t_0 + \frac{-1}{k}(u(0)^{k}-u(x_0)^{k}) = t_0 + \frac{u(x_0)^{k}}{k} < \infty
			\end{equation*}
			On the other hand										
			\fi
			\item
			\item For which values of $\alpha$ is $v$ Lipschitz? 	Note that since there does not exists solutions for all values of $\alpha \neq 1$, then $x|ln|x||^\alpha$ is not Lipschitz for those values.  We must test for $\alpha = 1$.  Note that since $x \in (0,1)$ then $v(x) = x|ln|x||= -x ln(x)$.  Now testing the boundedness of $|v'|$ at $0$:
			\begin{equation*}
				\lim_{x \to 0} |v'| =\lim_{x \to 0} |-1-ln(x)| = |-1-ln(0)|  = \infty.
			\end{equation*}
			Since $v'$ is unbounded
		\end{enumerate}
		\item[2.10]
		
\end{enumerate}
\end{document}