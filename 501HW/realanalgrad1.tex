\documentclass[12pt, letterpaper]{article}
\date{\today}
\usepackage[margin=1in]{geometry}
\usepackage{amsmath}
\usepackage{hyperref}
\usepackage{cancel}
\usepackage{amssymb}
\usepackage{fancyhdr}
\usepackage{pgfplots}
\usepackage{booktabs}
\usepackage{pifont}
\usepackage{amsthm,latexsym,amsfonts,graphicx,epsfig,comment}
\pgfplotsset{compat=1.16}
\usepackage{xcolor}
\usepackage{tikz}
\usetikzlibrary{shapes.geometric}
\usetikzlibrary{arrows.meta,arrows}
\newcommand{\Z}{\mathbb{Z}}
\newcommand{\N}{\mathbb{N}}
\newcommand{\R}{\mathbb{R}}
\newcommand{\Q}{\mathbb{Q}}
\newcommand{\C}{\mathbb{C}}
\newcommand{\F}{\mathbb{F}}

\newcommand{\Po}{\mathcal{P}}
\newcommand{\Pro}{\mathbb{P}}
\author{Alex Valentino}
\title{503 homework}
\pagestyle{fancy}
\renewcommand{\headrulewidth}{0pt}
\renewcommand{\footrulewidth}{0pt}
\fancyhf{}
\rhead{
	Homework 1\\
	501
}
\lhead{
	Alex Valentino\\
}
\begin{document}
\begin{enumerate}
	\item[1] Let $(x,d)$ be a seperable metric space, and $Y \subseteq X$ endowed with the 
	metric $d_Y$.  Let $S$ be the seperable subset of $X$.  Then we can define the 
	topology of $X$, $\mathcal{O}_X$ as the arbitrary union of sets in $\mathcal{B} = \{B_{r}(x): r \in \Q_{>0}, x \in S\}$.  Note this definition holds because irrational from rational 
	ball radiuses one can put together irrational ball radiuses, and then from there 
	construct arbitrary open sets.  Now to define a countable dense subset of $Y$, one 
	just has to construct the set $\{Y \cap B: B \in \mathcal{B}, Y \cap B \neq \emptyset \}$.  Note that this serves as an effective basis for the subset topology of $Y$, therefore
	for each $B_Y$ in our basis if we pick one element, then we have found a countable dense subset for $Y$, thus making $Y$ seperable.
	
	\iffalse Note that if $Y = \emptyset$ then $Y$ fails to have a dense set by definition, so assume $Y \subseteq X, Y \neq \emptyset$.  Since $(X,d)$ is seperable then there exists $S \subseteq X$ such that $cl(S) \supseteq X$ and $|S| \leq \aleph_0$.   Note that if $Y$ is not the empty set then $S \cap Y$ must be 
	non-empty.  Otherwise if $S \cap Y = \emptyset$, then either $Y = \emptyset$ or 
	$Y =  cl(S) \backslash X$.  Note that the second option is a contradiction since 
	by definition $Y \not \subseteq X$.  The first option is eliminated by the first note of the proof.  Thus let $S_Y = S \cap Y$.  We claim that $S_Y$ is dense in $Y$.  Let $y \in Y$.
	For all $O_Y \subseteq Y$ with $y \in O_Y$ which are open under the subset topology (where $O_Y = O \cap Y, O \subseteq X$, $O$ is open), we just need to show that 
	$S_Y \cap O_Y \neq \emptyset$.  This means we must show that $S \cap Y \cap O \neq \emptyset$.   Since we're assuming that $Y \cap O \neq \emptyset$ then all conditions hold.  
	Therefore $S_Y \cap O_Y \neq \emptyset$.  
	\fi
	\iffalse Note that if $Y = \emptyset$ then $Y$ fails to have a dense set by definition, so assume $Y \subseteq X, Y \neq \emptyset$.  Since $(X,d)$ is seperable then there exists $S \subseteq X$ such that $cl(S) = X$ and $|S| \leq \aleph_0$.  We 
	claim that $S_Y = S \cap Y$ is a coutable dense subset of $Y$. Trivially $S_Y = S \cap Y \subseteq Y$.  Note that $S_Y$ is still countable since $|S_Y| = |S \cap Y|$ which implies that $|S_Y| \leq \aleph_0$ 
	and $|S_Y| \leq |Y|$. 
	 \fi
	\item[3] Let $(X,d)$ be a compact metric space
	\begin{enumerate}
		\item Suppose not, then $f: X \to X$ is continuous and not onto, and has the 
		property that for all possible $x_0 \in X$ and $r > 0$ there exists $x \in X$
		such that $d(f(x),x_0) < r$.  Since $f$ is not onto then there exists a set 
		$E \subset X$ such that for all $x \in X, f(x) \not \in E$.  Choose $e \in E$, 
		then for all $n \in \N$, we can define $x_n$ such that $d(f(x_n),x_0) < \frac{1}{n}$.  Clearly $\lim f(x_n) = x_0$, and since $X$ is a compact metric space then there 
		should exists $x' \in X$ such that $\lim x_n = x'$ and $f(x') = x_0$.  However 
		this contradicts $f$ being not onto, demonstrating the desired result.  
		\item Let $f: X \to X$ be an isometry.  We want to show that it is bijective, and thus need to show that is is injective and surjective.
		\begin{itemize}
			\item To show that $f$ is injective, suppose for $x,y \in X$ that 
			$f(x) = f(y)$.  Then $d(f(x),f(y)) = 0$.  Since $f$ is an isometry then
			$d(x,y) = 0$.  Note this is only true if $x = y$.  Therefore $f$ is 
			injective.
			\item It is important to note that by the definition given above $f$ must be 
			continuous, as for all $\epsilon > 0$, if $x,y \in X, d(x,y) < \epsilon$ 
			then $d(f(x),f(y)) < \epsilon$.  If we then assume for contradiction that 
			$f$ is not onto then the theorem proved above holds and there exists 
			$x_0 \in X, r > 0$ such that for all $x \in X, d(f(x), x_0) \geq r$.  Let the 
			sequence $x_n$ be given by $x_{n+1} = f(x_n)$.  Note that since compactness
			implies sequential compactness in a metric space then there exists $x_{n_k}$ 
			such that $\lim x_{n_k} = l$.  Therefore there exists $p,q\in \N, q < p$ such that 
			$d(x_{n_p}, L), d(x_{n_q}, L) < \frac{r}{2}$.  Thus $d(x_0, x_{p - q}) = d(x_q, x_p) \leq d(x_{n_p}, L) + d(x_{n_q}, L) < r$.  This contradicts the above statement since
			$x_{p-q} \in Im(f)$.  
		\end{itemize}
	\end{enumerate}
	\item[4] Let $X$ be a compact metric space with metric $d$ and $f: X \to \C$ be continuous. Let $\epsilon > 0$ be given. We know by the Heine-Cantor theorem that there exists $1 > \delta > 0$ such that if $x,y \in X, d(x,y) < \delta, |f(x) - f(y)| < \epsilon/2$.
	Therefore for each point $x \in X$ we can cover $X$ with $\bigcup_{x \in X} B_\delta(x)$
	and since $X$ is compact we can choose $x_1,\cdots,x_n$ such that $X \subseteq \bigcup_{i = 1}^n B_\delta(x_i)$.  Furthermore we can arrange our found points $x_1,\cdots, x_n$ by which points are closest, where we can define to be  $z_1 = x_1, z_i = \min_{x \in \{x_1,\cdots,x_n\}\backslash \{z_1,\cdots, z_{i-1}\}} d(x, z_{i-1})$.  Furthermore, $d(z_i,z_{i-1}) < \delta$ since the balls are guarenteed to overlap to enclose $X$.  Thus 
	given an $x,y \in X$, we can choose the $z_i, z_j$ balls to which $x$ and $y$ belong respectively, then define $|f(x) - f(y)| \leq |f(x) - f(z_i)| + \sum_{l = i + 1}^{j} |f(z_l) - f(z_{l-1})| + |f(z_j) - f(y)| < \epsilon |i - j + 1| < \frac{n\epsilon}{\delta}d(x,y) + \epsilon$.  Note that the path taken through the balls might not be optimal, so our bets 
	can be hedged via assuming the worse case scenario of having to pass through all the balls when the ideal path travels through one.  
	\item[5]
	\begin{enumerate}
		\item Let $(X,d)$ be a complete metric space where bounded sets are totally bounded and $A \subset X$ to be bounded and $B \subset X$ to be compact. 		Note that if one takes $A \cap B \neq \emptyset$ then the minimum is given 
		by $x_1 \in A \cap B$ with $d(x_1,x_1) = 0 \leq d(x,y)$ for all $x \in A, y \in B$.
		Therefore one can assume that $A \cap B = \emptyset$.  Additionally since $X$ is 
		a complete metric space then $B$ is bounded and since $B$ is bounded then 
		$B$ is totally bounded.  Therefore $B$ has sequential compactness.  We want to 
		show that there exists $x_1 \in A, x_2 \in B$ such that $d(x_1,x_2) \leq d(x,y)$
		for all $x \in A, y \in B$.  Assume not.  Then there exists $(a_n) \subset A$
		and $(b_n) \subset B$ such that $\lim d(a_n,b_n) = 0$.  However, this implies that 
		$(b_n)$ has a convergent subsequence, $(b_{n_k})$ with $\lim b_{n_k} = b$, and since $n_k$ can be chosen such that $d(a_{n_k}, b_{n_k}) < \frac{\epsilon}{2}$ and $d(b_{n_k}, b) < \frac{\epsilon}{2}$ then $d(a_{n_k}, b) \leq d(a_{n_k}, b_{n_k}) + d(b_{n_k}, b) < \epsilon$.  This implies that $A$ has a limit point contained within $B$, and since $A$ is 
		closed then that limit point is contained within $A$.  Thus $A \cap B \neq \emptyset$ which is a contradiction. 
		\item The sequences $(2^k + \frac{1}{k})_{k \in \N}$ and $(2^k)_{k \in \N}$ are 
		examples, as the terms of the sequence will get arbitrarily close, but both diverge.  
	\end{enumerate}
\end{enumerate}
\end{document}
