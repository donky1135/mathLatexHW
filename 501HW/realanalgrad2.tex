\documentclass[12pt, letterpaper]{article}
\date{\today}
\usepackage[margin=1in]{geometry}
\usepackage{amsmath}
\usepackage{hyperref}
\usepackage{cancel}
\usepackage{amssymb}
\usepackage{fancyhdr}
\usepackage{pgfplots}
\usepackage{booktabs}
\usepackage{pifont}
\usepackage{amsthm,latexsym,amsfonts,graphicx,epsfig,comment}
\pgfplotsset{compat=1.16}
\usepackage{xcolor}
\usepackage{tikz}
\usetikzlibrary{shapes.geometric}
\usetikzlibrary{arrows.meta,arrows}
\newcommand{\Z}{\mathbb{Z}}
\newcommand{\N}{\mathbb{N}}
\newcommand{\R}{\mathbb{R}}
\newcommand{\Q}{\mathbb{Q}}
\newcommand{\C}{\mathbb{C}}
\newcommand{\F}{\mathbb{F}}

\newcommand{\Po}{\mathcal{P}}
\newcommand{\Pro}{\mathbb{P}}
\author{Alex Valentino}
\title{501 homework}
\pagestyle{fancy}
\renewcommand{\headrulewidth}{0pt}
\renewcommand{\footrulewidth}{0pt}
\fancyhf{}
\rhead{
	Homework 2\\
	501
}
\lhead{
	Alex Valentino\\
}
\begin{document}
\begin{enumerate}
	\item[2] Counterexample: Consider $f(x) = x$ from $\R \to \R$, however the 
	domain is imbued with the topology $d(x,y) = \min{|x-y|, 1}$.  (expand and solidfy proof) Note that this metric
	on $\R$ since \begin{enumerate}
		\item $|x-y|,1\geq 0$ and $|x-y| = 0$ implies $x=y$
		\item if $|x-y| > 1$ then $|y-x| > 1$.  
		\item $1 < 1 + 1 = 2$.  
	\end{enumerate}
	Which in conjuction with the standard $l_1$ distance on $\R$ being a metric implies $d$ is a metric.  Additionally, the topology generated by $d$ is equivalent to the 
	standard topology on $\R$ since for any (traditional) open set $U \subseteq \R$
	and $x \in U$ there exists $\epsilon >0 $ such that $x \in B_\epsilon(x) \subseteq U$.  If $\epsilon < 1$ then $B_\epsilon(x)$ is in our constructed topology.  Otherwise
	if $\epsilon \geq 1$ then we can trivially find $\epsilon' < 1$ such that 
	$B_{\epsilon'}(x) \subset B_{\epsilon}(x)$.  Therefore any open ball in the standard 
	topology on $\R$ can be constructed as a union of balls contained within our constructed topology.  Therefore, our constructed topology works.  Note that in our
	new topology for the domain that $\R$ is a bounded set, as for $x,y \in \R$, 
	$d(x,y) \leq 1$.  Therefore $\sup_{x,y \in \R} d(x,y) = 1$, thus making $\R$ have a 
	diameter of 1.  Additionally since $\R$ is complete then it satisfies the conditions.  However $f(\R) = \R$, and in the image topology $\R$ is trivially unbounded.  Therefore we have found a counterexample.  
	\item[7] Let $X$ be a bounded subset of $\ell_2$, with bound $D$. 
	\begin{itemize}
		\item Suppose $X \subset \ell_2$ is totally bounded, that 
		$f \in X$, and $\epsilon > 0$.  Then there exists $B_{\epsilon,d_{l_2}}(f_i)$ for 
		$i \in [n]$ such that $X \subseteq \cup_{i=1}^nB_{\epsilon,d_{l_2}}(f_i) $.
		Therefore, there exists $f_j$ such that $\sum_{i=1}^\infty | f_j(i) - f(i)|^2 < \epsilon^2$.  Therefore 
		$$
		\sqrt{\sum_{n = N_\epsilon}^\infty |f(n)|^2} \leq \sqrt{\sum_{n=N_\epsilon}^\infty |f(n) - f_i(n)|^2} + \sqrt{\sum_{n = N_\epsilon}^\infty |f_i(n)|^2} < \epsilon + \epsilon
		$$
		Therefore $\sum_{n = N_\epsilon}^\infty |f(n)|^2 \leq 4 \epsilon^2$.  
		Therefore we can choose $\epsilon/4 = \epsilon'$ and we're done.
		\item We want to show that $X$ is totally bounded.  Suppose for $\epsilon > 0$.  Then there exists $N_\epsilon \in \N$ 
		such that for all $f \in X$, $\sum_{n = N_\epsilon}^\infty |f(n)|^2 \leq \epsilon^2 $.  Note that $\|f\|_2 \leq \sqrt{\sum_{n = N_\epsilon}^\infty |f(n)|^2 }  + \sqrt{\sum_{n =1}^{N_\epsilon - 1}|f(n)|^2 } \leq D + \epsilon$.  Note that if we 
		take the closure of the ball $B$ with radius $D + \epsilon$ in $\C^{N_\epsilon - 1}$ and 
		cover $B$ with balls $x \in B, B_{\epsilon}(x) \subset \C^{N_\epsilon - 1}$, then we can find an open subcover $B_\epsilon(x_i)$ where $i \in [n]$. Now for 
		an arbitrary $y \in X$, for the first $N_\epsilon - 1$ coordinates, we can find 
		a $j$ such that $\{y_n\}_{n=1}^{N_\epsilon - 1} \in B_{\epsilon}(x_j)$.  
		Therefore $d_2(y,x_j) \leq d(x_j, \{y_n\}_{n=1}^{N_\epsilon - 1}) + \sqrt{\sum_{n=N_\epsilon}^\infty |y_n|^2} < 2 \epsilon$.   Now note that we can find 
		$N_{\epsilon/2}$ for $\epsilon/2$ and this completes the proof.
 	\end{itemize}
	\item[8] 
	\begin{enumerate}
		\item Suppose for contradiction that $\{b_{n_k}\}$ is a sequence which converges 
		pointwise for all $x \in [0,1]$.  Then one can construct the number $a \in [0,1]$
		with the binary representation being $a_n = \begin{cases}0 & n = n_k, k \equiv 0 \mod{2} \\ 1 & n = n_k, k \equiv 1 \mod{2} \\ 0 & \text{otherwise} \end{cases}$.  
		Then for the subsequence of our subsequence $b_{n_{2k}}(a) = 0$ and 
		$b_{n_{2k+1}}(a) = 1$.  Since we have two subsequences which converge to different values, then our original subsequence of functions does not converge pointwise.
		\item Note that convergence in the product topology is convergence for each 
		projection.  However a projection is simply evaluating a function at a 
		specific $x \in [0,1]$.  Since we can always construct a number which every
		subsequence of $(b_n)$ fails on then necessarily no subsequence converges 
		in the product topology.  
	\end{enumerate}
	\item[9] Let $(X,d)$ be a compact metric space, and define
	$d_s:X^{\N} \times X^{\N} \to [0,\infty)$ to be a function on the space of sequences 
	of $X^\N$ by $$
	(x_n), (y_n) \in X^\N, d_s((x_n), (y_n)) = \sum_{i=1}^\infty d(x_n,y_n)2^{-n}
	$$
	\begin{enumerate}
		\item Show that $d_s$ is a metric:
		\begin{enumerate}
			\item Since $d(x_n,y_n) \geq 0$ and $2^{-n}\geq 0$ then $d(x_n,y_n)2^{-n} \geq 0$ and $\sum_{i=1}^\infty d(x_n,y_n)2^{-n} \geq 0$.  If $d_s((x_n),(y_n)) = 0$, 
			then each $d(x_n,y_n) = 0$ since all terms are positive, therefore if all
			$d(x_n, y_n) = 0$ for all $n$ then $x_n = y_n$ for all $n$, thus 
			$(x_n)_{n \in \N} = (y_n)_{n \in \N}$.
			\item Note that since $d(x_n, y_n) = d(y_n,x_n)$ for all $n\in \N$ then 
			$$
			d_s((x_n),(y_n)) = \sum_{n = 1}^\infty d(x_n, y_n)2^{-n} = \sum_{n = 1}^\infty d(y_n, x_n)2^{-n} = d_s((y_n),(x_n))
			$$
			\item Note for $(x_n), (y_n), (z_n) \in X^\N$, $d(x_n,z_n) \leq d(x_n,y_n) + d(y_n,z_n)$ holds for all $n \in \N$.  Additionally since $|d(x_n, y_n)| = d(x_n, y_n)$
			then the series $\sum_{n = 1}^\infty (d(x_n,y_n) + d(y_n,z_n))2^{-n}$
			converges absolutely if it doesn't diverge to $\infty$.  Therefore 
			if it doesn't diverge then via arbitrary rearrangement we have that 
			$\sum_{i=1}^n d(x_n,z_n)2^{-n} \leq \sum_{i=1}^n d(x_n,y_n)2^{-n} + \sum_{i=1}^n d(y_n,z_n)2^{-n}$.  If it does diverge then it goes to positive infinity.  
			If $d_s((x_n),(y_n))$ is finite then the inequality holds.  If $d_s((x_n),(y_n))$ is infinite then the inequality still holds.  Therefore $d_s$ obeys the triangle inequality
		\end{enumerate}
		\item Consider an open set $U \in X^\N$.  Then $U$ is a sequence 
		of open sets in $X$ of which finitely many of them aren't $X$.  Therefore 
		let $U_{k_1},\cdots, U_{n_k}$ be the sets which aren't $X$ in $U$.  Then select 
		$x_{k_i} \in U_{k_i}$, and find $\epsilon_{k_i}$ such that $x_{k_i} \in B_{\epsilon_{k_i}}(x_{k_i}) \subseteq U_{k_i}$.  Then, one can find the smallest $\epsilon_{k_i}, \epsilon'$ from our set, and take the radius of our open ball in $d_s$ to be 
		of radius $\frac{\epsilon'}{2^{k_n + 1}}$.  Note for any $x_i$ which aren't in 
		the finite set of non-$X$ open sets in the sequence $U$, then any radius works,
		as any other point chosen will be contained in $X$.  If $x_i = x_{k_j}$, then 
		the worst case scenario where the sequence $y$ with $d(y_{k_j},x_{k_j}) = \epsilon_{k_j}*2^{k_j}$ and all other $y_i = x_i$ is 
		avoided as: 
		\begin{align*}
		\frac{d(y_{k_j},x_{k_j})}{2^{k_J}} &< \frac{\epsilon'}{2^{k_n + 1}}\\
		d(y_{k_j},x_{k_j}) &< \epsilon' 2^{k_j - k_n - 1}\\
		&< \epsilon_{k_j}
		\end{align*}
		Note that $2^{k_j - k_n - 1}$ is at most $0.5$ by construction.  Therefore every 
		element is guarenteed to be within the required open balls, thus our topology 
		induced by $d_s$ is atleast as strong as the product topology.
		\item Note that since $(X,d)$ is compact then it is sequentially compact.  Also 
		let the sequence of sequences be denoted $(x_{i j})_{i \in \N, j \in \N}$, where
		$x_i$ is a sequence in $X^\N$.  Note that $x_{0j}$ is a sequence in $X^\N$ as 
		well.  By the sequential compactness of $X$ than $x_{0j}$ has a convergent 
		subsequence $x_{0j_{k}}$, which we shall say converges to $l_1$.  Note that 
		$x_{1j_{k}}$ is a sequence in $X$, which has a convergent subsequence, say,
		$x_{1j_{k_l}}$.  Note these subscripts are getting out of hand, therefore 
		we will define $f_{m,n}$ to be $n$-th term in the subsequence starting at 
		$x_{0j_{k}}$ within $x_{mj}$.  We will denote $l_i$ by $\lim_{n \to \infty}f_{i,n}$.  Now for $\epsilon > 0$, there exists some $r \in \N$ where by the diameter of 
		$X$, $D$, is $\frac{D}{2^{2r}} < \frac{\epsilon}{2}$.  Therefore we can choose
		the largest $N$ such that for each $d(l_i, f_{i,N}) < \frac{\epsilon}{1 - 2^{1 - 2r}}$.
		Therefore $$\sum_{i=1}^\infty \frac{d(f_{i,N},l_i)}{2^i} = \sum_{i=1}^{2r-1} \frac{d(f_{i,N},l_i)}{2^i} + \sum_{2r}^\infty\frac{d(f_{i,N},l_i)}{2^i} = \frac{\epsilon}{2} + \frac{\epsilon}{2}  = \epsilon$$
		\item We will show that $X^\N$ is compact under the product topology.  Let $\{U_\alpha\}_{\alpha \in I}$  be a cover where $I$ is an index set, and all $U_\alpha \subset X^\N$ are open in the product topology.  We know for each $x \in U_\alpha$, 
		there exists $\epsilon > 0$ such that $B_{\epsilon,d_s}(x) \subset U$.  
		Therefore we can write $U_\alpha$ as the union of $d_s$ balls.  Thus each
		$U_\alpha$ is open in the topology induced by $d_s$.  Therefore since $(X^\N, d_s)$ is sequentially compact then $X^\N$ is compact with respect to $d_s$, and since
		our open cover is open with respect to $d_s$, there exists a subcover 
		$U_{\alpha_1},\cdots, U_{\alpha_n}$ such that $X \subseteq \cup_{i = 1}^n U_{\alpha_i}$.  Thus $X^\N$ is compact with respect to the product topology.  
	\end{enumerate}
	\item[12] Let $A,B$ be compact, non-empty, and disjoint in the topological space $X$, and for every $b \in B$, there exists $f_b : X \to [0,1]$ continuous such that
	$f_b(b) = 1$, and vanishes on $A$.  We want to show that there exists $U \subset A,
	V \subset B$ open such that $U \cap V = \emptyset$.  Take the sets $V_b = 
	\{f_b(x) > \frac{2}{3}: x \in X\}$ and $U_b = \{f(b) < \frac{1}{3}: x \in X\}$.  
	Note that each $U_b, V_b$ are open since $f_b$ is open.  Additionally by the 
	compactness of $B$ we can choose $V_{b_1},\ldots,V_{b_n}$ such that 
	$B \subseteq \cup_{i=1}^n V_{b_i}$.  Let this subcover be denoted $B$.  Let 
	$U = \cap_{i=1}^n U_{b_i}$.  Note that $U$ is non-empty since each $U_b$ is 
	guarenteed to contain $A$ as $0 < \frac{1}{3} < f_b(x), x \in U_b$ by construction.
	Therefore if $x \in U \cap V$, then there exists $j$ such that 
	$x \in V_{b_j}$.  However, $U_{b_j} \cap V_{b_j}$ have an empty intersection 
	since they're two different level sets of $f_{b_j}$, a continuous function.  
	Thus $U \cap V = \emptyset$.    
\end{enumerate}
\end{document}
