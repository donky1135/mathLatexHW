\documentclass[12pt, letterpaper]{article}
\date{\today}
\usepackage[margin=1in]{geometry}
\usepackage{amsmath}
\usepackage{hyperref}
\usepackage{cancel}
\usepackage{amssymb}
\usepackage{fancyhdr}
\usepackage{pgfplots}
\usepackage{booktabs}
\usepackage{pifont}
\usepackage{amsthm,latexsym,amsfonts,graphicx,epsfig,comment}
\pgfplotsset{compat=1.16}
\usepackage{xcolor}
\usepackage{tikz}
\usetikzlibrary{shapes.geometric}
\usetikzlibrary{arrows.meta,arrows}

\usepackage{algpseudocode}
\usepackage{tabto}
\newcommand{\Z}{\mathbb{Z}}
\newcommand{\N}{\mathbb{N}}
\newcommand{\R}{\mathbb{R}}
\newcommand{\Q}{\mathbb{Q}}
\newcommand{\C}{\mathbb{C}}
\newcommand{\F}{\mathbb{F}}

\newcommand{\Po}{\mathcal{P}}
\newcommand{\Pro}{\mathbb{P}}
\author{Alex Valentino}
\title{algos homework}
\pagestyle{fancy}
\renewcommand{\headrulewidth}{0pt}
\renewcommand{\footrulewidth}{0pt}
\fancyhf{}
\rhead{
	Homework 4\\
	CS 344	
}
\lhead{
	Alex Valentino\\
}
\begin{document}
\begin{enumerate}
	\item[3.1] $A: 1,12, B:2,11, C:3,10, D:13,18,E:5,6, F:4,9 ,G:14,17, H: 15,16, I:7,8$, $A-B$ tree, $B-C$ tree, $B-E$ back, 
	$C-F$ tree, $E-F$ tree, $F-I$ tree, $D-G$ tree, $G-H$ tree
	$D-H$ back.
	\item[3.2]
	\begin{enumerate}
		\item $A: 1,16, B:,2,15,C:3,14,D:4,13,E:8,9,F:7,10, G:6,11,
		H:5,12$, $A->B$ tree, $A->F$ forward, $B-> C$ tree, 
		$B->E$ forward, $C->D$ tree, $D-> H$ tree, $G -> F$ 
		tree, $F -> G$ backward, $F-E$ tree, $E -> G$ backwards.
		\item $A:1,16, B:2,11, C:4,5, D:6,9, E:7,8, F:3,10, 
		G:13,14, H:12,15$, $A->B$ t, $A->H$ t, $B->F$ t, 
		$C->B$ b, $D->C$ c, $F->E$ f, $F->D$ t, $F->C$ t, 
		$G->F$ c, $G-> B$ c, $G-> A$ b, $H-> G$ t. 
	\end{enumerate}
	\item[3.3]
	\begin{enumerate}
		\item $A:1,14,B:15,16,C:2,13,D:3,10,E:11,12, F:4,9, 
		G:5,6,H:7,8$
		\item the sources are $A,B$ and the sinks are 
		$G,H$
		\item B A C E D F H G
		\item Since $\{A,B\},\{D,E\},\{G,H\}$ are all 
		antichains, then their order can be swapped to produce 
		a valid topological ordering.  Since we have 3 binary 
		choices to make, then that implies that we have $2^3$
		possibilities, thus there are 8 topological orderings. 
	\end{enumerate}
	\item[3.5] To reverse a graph, we're going to implement 
	a form of DFS, where in the explore routine one takes 
	in the edge $(u,v)$, then inserts $(v,u)$ into the 
	adjacency list $E^R$.  Note that explore will be ran once 
	per node, thus we will find all edges and flip them.  
	\item[3.9] To compute twodegree, one first iterates through the 
	linked list by counting the degree by just counting the number 
	of elements in the linked list per node.  Let these numbers be put into an 
	array denoted as $degrees[]$.   Then one iterates through 
	the nodes again, where instead one goes through the adjacency 
	list, and for each node $u$ in the adjacency list for $v$, add $degrees[u]$ to the total $twodegree[v]$.  
	
\end{enumerate}
\end{document}
=======
\documentclass[12pt, letterpaper]{article}
\date{\today}
\usepackage[margin=1in]{geometry}
\usepackage{amsmath}
\usepackage{hyperref}
\usepackage{cancel}
\usepackage{amssymb}
\usepackage{fancyhdr}
\usepackage{pgfplots}
\usepackage{booktabs}
\usepackage{pifont}
\usepackage{amsthm,latexsym,amsfonts,graphicx,epsfig,comment}
\pgfplotsset{compat=1.16}
\usepackage{xcolor}
\usepackage{tikz}
\usetikzlibrary{shapes.geometric}
\usetikzlibrary{arrows.meta,arrows}

\usepackage{algpseudocode}
\usepackage{tabto}
\newcommand{\Z}{\mathbb{Z}}
\newcommand{\N}{\mathbb{N}}
\newcommand{\R}{\mathbb{R}}
\newcommand{\Q}{\mathbb{Q}}
\newcommand{\C}{\mathbb{C}}
\newcommand{\F}{\mathbb{F}}

\newcommand{\Po}{\mathcal{P}}
\newcommand{\Pro}{\mathbb{P}}
\author{Alex Valentino}
\title{algos homework}
\pagestyle{fancy}
\renewcommand{\headrulewidth}{0pt}
\renewcommand{\footrulewidth}{0pt}
\fancyhf{}
\rhead{
	Homework 4\\
	CS 344	
}
\lhead{
	Alex Valentino\\
}
\begin{document}
\begin{enumerate}
	\item[3.1] $A: 1,12, B:2,11, C:3,10, D:13,18,E:5,6, F:4,9 ,G:14,17, H: 15,16, I:7,8$, $A-B$ tree, $B-C$ tree, $B-E$ back, 
	$C-F$ tree, $E-F$ tree, $F-I$ tree, $D-G$ tree, $G-H$ tree
	$D-H$ back.
	\item[3.2]
	\begin{enumerate}
		\item $A: 1,16, B:,2,15,C:3,14,D:4,13,E:8,9,F:7,10, G:6,11,
		H:5,12$, $A->B$ tree, $A->F$ forward, $B-> C$ tree, 
		$B->E$ forward, $C->D$ tree, $D-> H$ tree, $G -> F$ 
		tree, $F -> G$ backward, $F-E$ tree, $E -> G$ backwards.
		\item $A:1,16, B:2,11, C:4,5, D:6,9, E:7,8, F:3,10, 
		G:13,14, H:12,15$, $A->B$ t, $A->H$ t, $B->F$ t, 
		$C->B$ b, $D->C$ c, $F->E$ f, $F->D$ t, $F->C$ t, 
		$G->F$ c, $G-> B$ c, $G-> A$ b, $H-> G$ t. 
	\end{enumerate}
	\item[3.3]
	\begin{enumerate}
		\item $A:1,14,B:15,16,C:2,13,D:3,10,E:11,12, F:4,9, 
		G:5,6,H:7,8$
		\item the sources are $A,B$ and the sinks are 
		$G,H$
		\item B A C E D F H G
		\item Since $\{A,B\},\{D,E\},\{G,H\}$ are all 
		antichains, then their order can be swapped to produce 
		a valid topological ordering.  Since we have 3 binary 
		choices to make, then that implies that we have $2^3$
		possibilities, thus there are 8 topological orderings. 
	\end{enumerate}
	\item[3.5] To reverse a graph, we're going to implement 
	a form of DFS, where in the explore routine one takes 
	in the edge $(u,v)$, then inserts $(v,u)$ into the 
	adjacency list $E^R$.  Note that explore will be ran once 
	per node, thus we will find all edges and flip them.  
	\item[3.9] To compute twodegree, one first iterates through the 
	linked list by counting the degree by just counting the number 
	of elements in the linked list per node.  Let these numbers be put into an 
	array denoted as $degrees[]$.   Then one iterates through 
	the nodes again, where instead one goes through the adjacency 
	list, and for each node $u$ in the adjacency list for $v$, add $degrees[u]$ to the total $twodegree[v]$.  
	
\end{enumerate}
\end{document}
>>>>>>> Stashed changes
