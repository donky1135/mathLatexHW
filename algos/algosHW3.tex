\documentclass[12pt, letterpaper]{article}
\date{\today}
\usepackage[margin=1in]{geometry}
\usepackage{amsmath}
\usepackage{hyperref}
\usepackage{cancel}
\usepackage{amssymb}
\usepackage{fancyhdr}
\usepackage{pgfplots}
\usepackage{booktabs}
\usepackage{pifont}
\usepackage{amsthm,latexsym,amsfonts,graphicx,epsfig,comment}
\pgfplotsset{compat=1.16}
\usepackage{xcolor}
\usepackage{tikz}
\usetikzlibrary{shapes.geometric}
\usetikzlibrary{arrows.meta,arrows}
\newcommand{\Z}{\mathbb{Z}}
\newcommand{\N}{\mathbb{N}}
\newcommand{\R}{\mathbb{R}}
\newcommand{\Q}{\mathbb{Q}}
\newcommand{\C}{\mathbb{C}}
\newcommand{\F}{\mathbb{F}}

\newcommand{\Po}{\mathcal{P}}
\newcommand{\Pro}{\mathbb{P}}
\author{Alex Valentino}
\title{algos homework}
\pagestyle{fancy}
\renewcommand{\headrulewidth}{0pt}
\renewcommand{\footrulewidth}{0pt}
\fancyhf{}
\rhead{
	Homework 3\\
	CS 344	
}
\lhead{
	Alex Valentino\\
}
\begin{document}
\begin{enumerate}
	\item[1.27] Note that $(p-1)(q-1) = 352$.  We can find the secret key by applying 
	the extended euclidean algorithm to $352,3$.  This yields $235$ as the multiplicative 
	inverse.  Therefore the message $M = 41$ is encrypted via $41^{253} \mod{351} \equiv 153$.
	Furthermore $153^3 \mod{352} \equiv 41$.  
	\item[1.28] For $p=11, q=7$, $(p-1)(q-1) = 60$.  Thus we just need to pick $e$ such that 
	$\gcd(e,60) = 1$.   Since the factorization of $60 = 2^2 \cdot 3 \cdot 5$ then one of the
 	first prime which is not included is $11$, which I will pick as my $e$.  Now I need to find $d$
	such that $11d \equiv 1 \mod{60}$.  Note that $11^2 = 121 = 120 + 1 = 2*60 + 1$.  Thus 
	$d=11$ also works.  
	\item[1.29]
	\begin{enumerate}
		\item The hash function 
		We claim that $h_{a_1, a_2}(x_1, x_2) = a_1 x_1 + a_2 x_2 \mod{m}$ 
		is a universal hash function.  Suppose $a_1, a_2$ are 
		not known yet, and $(x_1,x_2),(y_1,y_2) \in [m]^2$, and 
		wlog assume $x_2 \neq y_2$.  If the following 2 are equal 
		under $h$ then by definition $a_1(x_1 - y_1) \equiv a_2(y_2 - x_2) \mod{m}$.   Under our assumption $y_2 \neq x_2$, since $\Z_m$ is a field, then $a_2$ is exactly decided to be $a_1 (x_1 - y_1)(y_2 - x_2)^{-1} \equiv a_2 \mod{m}$.  Note that there is exactly a 
		$\frac{1}{m}$ probability of $a_2$ being chosen.  Thus the 
		collision of two arbitrary elements in $[m]$ is $\frac{1}{m}$,
		therefore $h$ is a universal hashing function.
		The number of bits required is 
		$2*(\lfloor \log(m)\rfloor  + 1)$
		\item The same hash function $H$ with $2^k = m$ is not 
		universal since if one has $a_1 = a_2 = 2^{k-1}$ then 
		if $x_1 + x_2 = 2n+1$, then we have $a_1 x_1 + a_2 x_2 = 
		(x_1 + x_2)2^k = (2n+1)2^k \equiv 2^k \mod{m}$, and if 
		$x_1 + x_2 = 2n$ then $a_1 x_1 + a_2 x_2 \equiv 0 \mod{2^k}$.
		Additionally to specify a function one needs $2k$ bits.  
		\item Note that there are $(m-1)^m$ functions 
		$f: [m] \to [m-1]$.  If we consider $x \in [m], y \in [m-1]$,
		and all the functions which fix $f(x) = y$ implies that 
		they all have $(x,y)$ in their definition.  Since $f$ is a 
		set of $m$ ordered pairs, and we take away one gives us 
		$(m-1)^{m-1}$ functions with our desired ordered pair.  Thus 
		the probability that $f(x) = y$ is 
		$\frac{(m-1)^{m-1}}{(m-1)^m} = \frac{1}{m-1}$, which is exactly
		the uniform probability of choosing an item in $[m-1]$.  
		The number of bits required is exactly the number of bits to 
		specify $m$ distinct elements in $[m-1]$.  Since an element in 
		$[m-1]$ has up to $\lfloor \log (m-1) \rfloor + 1$ bits in 
		it's definition.  Therefore we need 
		$m(\lfloor \log (m-1) \rfloor + 1)$ bits.  
	\end{enumerate}
	\item[1.31]   
	\begin{enumerate}
		\item The number of bits in $N!$ is going to be given by $\lfloor \log(N!) \rfloor + 1$.  
		We showed in a previous homework that $\log(n!) = \Theta(n \log (n))$.  In the limit 
		$\lfloor \log(N!) \rfloor + 1 \approx \log(N!)$.  Thus the same $\Theta$ as above works.
		\item I propose the algorithm in which one computes 
		$n!$ as $n*(n-1)!$.  Therefore it takes exactly
		$T(n) = T(n-1) + (\log(n) + 1)(\log ((n-1)!) + 1)$.  
		The multiply is approximate $O(\log(n) * n \log(n))$ as 
		$\log((n-1)!) = O(n \log n)$, shown on a previous homework.
		Therefore $T(n) = O((n\log n)^2)$ as 
		$T(n) = \sum_{i=1}^n O(i (\log i)^2) \leq O((\log n)^2)\sum_{i=1}^n i = O((\log n)^2)\frac{n(n+1)}{2} = O((n \log n)^2)$.
		Thus $T(n) = O((n \log n)^2)$.   
	\end{enumerate}
	\item[1.31e] If we have Karatsuba's algorithm then our $T(n)$ 
	becomes 
	$$T(n) = T(n-1) + (\log(n) + 1)^d (\log((n-1)!) + 1)
	= T(n-1) + O(n (\log n)^{d+1}) \leq O(n^2 (\log n)^{d+1})$$
	Where we apply the same summation trick as before, and 
	$d = 0.585$.  
	\item[2.4]
	\begin{itemize}
		\item Algorithm $A$ has the recurrence relation $T(n) = 5T(n/2) + O(n)$, thus we have that $\log_2(5)$ for the log term, and $d=1$ since $O(n)$
		is a first degree polynomial.  Since $\log_2(5) > \log_2(2) = 2 > 1$ then by the masters
		theorem the programs run time is $O(n^{\log(5)/\log(2)})$.
		\item Algorithm $B$ has the recurrence relation $T(n) = 2T(n-1) + O(1)$.  Since the
		combination time is constant then there exists $c$ which is the constant time.  
		Therefore $T(n) = 2T(n-1) + c = 4T(n-2)+ 2c + c = 8T(n-3) + 4c + 2c + c = \cdots = 
		c\sum_{i=1}^n 2^i  = c(2^{n+1} - 2)$.  Thus the algorithms runtime is $O(2^n)$.  
		\item Algorithm $C$ has the recurrence relation $T(n) = 9T(n/3) + O(n^2)$.  Therefore
		by the Master's theorem we have that $\log_3(9) = 2 = d$.  Thus algorithm
		C has a runtime of $O(n^2 \log(n))$.  
	\end{itemize}
	\item[2.5] Observe that by the proof of the masters theorem, since all of the following 
	recurenses have an exact amount of time being added, then our solutions are exact, which 
	satisfy bounds from both above and below.  
	\begin{enumerate}
		\item $T(n) = 2T(n/3) + 1$.  Since $\log(2)/\log(3) > 0 = d$, then $T(n) = \Theta(n^{\log(2)/\log(3)})$
		\item $T(n) = 5T(n/4) + n$.  Since 
		$\log_4(5) > 1$ then $T(n) = \Theta(n^{\log_4(5)})$
		\item $T(n) = 7T(n/7) + n$.  Since
		$\log_7(7) = 1 = d$ then 
		$T(n) = \Theta(n \log (n))$.
		\item $T(n) = 9T(n/3) + n^2$.  
		Since $\log_3(9) = 2 = d$ then 
		$T(n) = \Theta(n^2 \log(n))$
		\item $T(n) = 8T(n/2) + n^3$.  
		Since $\log_2(8) = 3 = d$ then 
		$T(n) = \Theta(n^3 \log(n))$
		\item $T(n) = 49T(n/25) + n^{3/2}\log n$.  
		Since $\log(7) < \log(8) = 3, 
		\log(5) > \log(4) = 2$, then $\log(7)/\log(5) <
		3/2$.  Therefore $T(n) = \Theta(n^{3/2}\log n)$,
		as the dominating term will be the constant 
		multiple of $n^{3/2} \log n$.  
		\item $T(n) = T(n-1) + 2$.  Since each 
		additional $n$ adds a single 2, then 
		$T(n) = 2n = \Theta(n)$.  
		\item $T(n) = T(n-1) + n^c$.  
		Observe that 
		$T(n) = \sum_{i=1}^n i^c \leq n*n^c = n^{c+1}$. 
		Furthermore, 
		$\frac{n}{2}^{c+1}  =\frac{n}{2}\frac{n}{2}^{c} < \sum_{i=n/2}^n i^c < T(n)$.  Thus 
		$T(n) = \Theta(n^{c+1})$. 
		\item
		$T(n) = T(n-1) + c^n = \sum_{i=1}^n c^i = 
		\frac{c^{n+1} - 1}{c - 1} = \Theta(c^{n})$, as 
		$\frac{c^{n+1} - 1}{c-1} < \frac{c^{n+1}}{c-1} = \frac{c}{c-1}c^n$ and 
		$\frac{c^{n+1} - 1}{c-1} > c^n - \frac{1}{c} > c^n - c > c^n - 
		\frac{c^n}{2} = \frac{1}{2}c^n$ for sufficently large $n$.  
		Thus $T(n) = \Theta(c^n)$.  
		\item $T(n) = 2T(n-1) + 1 = \sum_{i=1}^n 2^{i-1} = 2^{n} - 1$.
		Clearly $T(n) = \Theta(2^n)$.  As seen before in algorithm B, 
		problem 2.4.  
		\item $T(n) = T(\sqrt{n}) + 1$.  We claim that 
		$T(n) = \Theta(\log \log n)$.  Observe for $2^{2^k}$ that 
		$T(2^{2^k}) = k + T(\sqrt{2}) = k + 1 + T(1) = k + 2$.  
		Since for arbitrary $n$ there exists $k$ such that 
		$2^{2^{k-1}} \leq n \leq 2^{2^{k}}$, then 
		$k+1 = \log \log 2^{2^{k-1}} \leq T(n) \leq \log \log 2^{2^{k}} = k+2$.  Thus for well chosen multiples one has $T(n) = \Theta (\log \log n)$.  
	\end{enumerate}
	\item[2.8]
	\begin{enumerate}
		\item The appropriate choice for $\omega$ is $\omega = i$.  
		Since $i^4 = 1$, thus making it the fourth root of unity.  
		Thus the fourier tranform for $[1,0,0,0]$ at the first phase is
		simply mapping $1,0 \mapsto 1,1$ and $0,0 \mapsto 0,0$.  
		Furthermore we have the first coordinate being 
		$1 + 0 = 1$ the second being $1 + i*0 = i$, the third being
		$1 - 1 * 0 = 1$, and the forth being $1 - i * 0 = 1$.  Thus the fourier transform is 
		$[1,1,1,1]$.  Since the inverse FFT is $\frac{1}{n} M_n(\omega^{-1})$ and there was no interaction with the complex parts 
		implies that multiplying by the inverse will yield the same 
		result, times $\frac{1}{4}$, $\frac{1}{4}[1,1,1,1]$.
		\item Note that for $[1,0,1,-1]$ we have
		$s_0 = 2, s_1 = 0, s_0' = -1, s_1' = 1$. Thus we have 
		that $[2 +(-1), 0 - i (-1), 2 - (-1), 0 + i(-1)] = [1,i,3,-i]$.  Going in reverse we get that 
		$\frac{1}{4}[1,-i,3,i]$ is the inverse transform of 
		$[1,0,1,-1]$.  	
	\end{enumerate}
\end{enumerate}
\end{document}
