\documentclass[12pt, letterpaper]{article}
\date{\today}
\usepackage[margin=1in]{geometry}
\usepackage{amsmath}
\usepackage{hyperref}
\usepackage{cancel}
\usepackage{amssymb}
\usepackage{fancyhdr}
\usepackage{pgfplots}
\usepackage{booktabs}
\usepackage{pifont}
\usepackage{amsthm,latexsym,amsfonts,graphicx,epsfig,comment}
\pgfplotsset{compat=1.16}
\usepackage{xcolor}
\usepackage{tikz}
\usepackage{algpseudocode}
\usepackage{tabto}
\usetikzlibrary{shapes.geometric}
\usetikzlibrary{arrows.meta,arrows}
\newcommand{\Z}{\mathbb{Z}}
\newcommand{\N}{\mathbb{N}}
\newcommand{\R}{\mathbb{R}}
\newcommand{\Q}{\mathbb{Q}}
\newcommand{\C}{\mathbb{C}}
\newcommand{\F}{\mathbb{F}}

\newcommand{\Po}{\mathcal{P}}
\newcommand{\Pro}{\mathbb{P}}
\author{Alex Valentino}
\title{algos homework}
\pagestyle{fancy}
\renewcommand{\headrulewidth}{0pt}
\renewcommand{\footrulewidth}{0pt}
\fancyhf{}
\rhead{
	Homework 6\\
	CS 344	
}
\lhead{
	Alex Valentino\\
}
\begin{document}
\begin{enumerate}
	\item[6.1] Given an array $a_1,\cdots,a_n$, to compute the longest contiguous subsequence sum, we first 
	consider the best sum, $s_b$ and the current sum at that variable, $s_c$, and set them respectively to 
	$s_b = -\infty, s_c = 0$, additionally we define the start and end indices of the ideal sum to be and the current 
	sum as  
	$i_s, i_e, c_s, c_e$.  Then we iterate through the array, where for each $i \in [n]$, we first check 
	if $a_i > a_i + s_c$, if this statement is true we set $c_s = i$ and $s_c = a_i$.  If $a_i = a_i + s_c$ we set 
	$c_s = i$.  If $a_i < a_i + s_c$ we set $s_c = s_c + a_i$ and $c_e = i$.  Next, if $s_c \geq s_b$ is true then we set 
	$s_b = s_c$ and $i_e = i, i_s = c_s$.  Note the algorithm mentioned before trivially runs in $O(n)$ time, and 
	it will arrive at the best solution because in every moment it trades up to the maximum current sum/element, which 
	will identify when we need to look at a new contiguous array, then after it will swap up to the best sum.  
	\item[6.2] To figure out the best journey, predicated on for $x$ miles driven in a day 
	$(200-x)^2$ to be the penalty.  For the array of $a_1,\cdots,a_n$ hotel distances, we will define 
	$a_0 = 0$, and the arrays $dp, prev$, where $dp[i]$ is the optimal penalty up to hotel $i$, and 
	$prev[i]$ is the optimal last hotel.  We start by setting $prev[0] = 0$ and $dp[0] = 0$.  Then 
	for each $i \in [n]$, we compute $dp[i] = \min(dp[1] + (200 - (a_i - a_1))^2, dp[2] + (200 - (a_i - a_2))^2, 
	\cdots, dp[i-1] + (200 - (a_i - a_{i-1}))^2)$, then store $prev[i]$ based on which hotel was picked previously.  This will find the least penalty to get to hotel $i$.  By returning the $prev$ array and $dp$ we can do $dp[n]$ to 
	get the best penalty to hotel $n$ and then all the path took via looking up indicies of the last hotel 
	in prev recursively.  
	\item[6.3] Similarly as before, one has the $dp$ array which is maximum profit including the $i$-th location. 
	This is done via first setting $dp[1] = \max_{i \in \N} p_i$, then iterating through for each $i \in [n]\backslash \{1\}$ 
	computing $dp[i] = \max(p_i, \{p_j + dp[j]: m_j - m_i > k, j \in [n] \backslash [i]\})$, where the previous 
	most profitable location is stored in a seperate prev array.  Finally one returns the $prev$ and $dp$ arrays, 
	where one can extract the most profitable wackdonalds locations for $n$ potential locations and their positions via 
	$dp[n]$ and working backwards through the $prev$ array.  
\end{enumerate}
\end{document}
