\documentclass[12pt, letterpaper]{article}
\date{\today}
\usepackage[margin=1in]{geometry}
\usepackage{amsmath}
\usepackage{hyperref}
\usepackage{cancel}
\usepackage{amssymb}
\usepackage{fancyhdr}
\usepackage{pgfplots}
\usepackage{booktabs}
\usepackage{pifont}
\usepackage{amsthm,latexsym,amsfonts,graphicx,epsfig,comment}
\pgfplotsset{compat=1.16}
\usepackage{xcolor}
\usepackage{tikz}
\usetikzlibrary{shapes.geometric}
\usetikzlibrary{arrows.meta,arrows}
\newcommand{\Z}{\mathbb{Z}}
\newcommand{\N}{\mathbb{N}}
\newcommand{\R}{\mathbb{R}}
\newcommand{\Q}{\mathbb{Q}}
\newcommand{\C}{\mathbb{C}}
\newcommand{\F}{\mathbb{F}}

\newcommand{\Po}{\mathcal{P}}
\newcommand{\Pro}{\mathbb{P}}
\author{Alex Valentino}
\title{algos homework}
\pagestyle{fancy}
\renewcommand{\headrulewidth}{0pt}
\renewcommand{\footrulewidth}{0pt}
\fancyhf{}
\rhead{
	Homework 2\\
	CS 344	
}
\lhead{
	Alex Valentino\\
}
\begin{document}
\begin{enumerate}
	\item[1.2] Let $l \in \N$.  Note that the number of bits needed to represent $l$ is
	$[\ln(l)/\ln(2)] + 1$ and the number of decimal places needed to represent
	$l$ is $[\ln(l)/\ln(10)] + 1$ where $[]$ is the floor function. Therefore, 
	\begin{align*}
		\frac{[\ln(l)/\ln(2)] + 1}{[\ln(l)/\ln(10)] + 1} &=
		\frac{[\ln(l)/\ln(2)] + 1}{[\ln(l)/\ln(10)] + 1} \frac{\frac{1}{\ln(l)}}{\frac{1}{\ln(l)}}\\
		&\leq \frac{\frac{1}{\ln(l)}[\ln(l)/\ln(2)] + \frac{1}{\ln(l)}}{\frac{1}{\ln(10)}}\\
		&= \frac{\log(10)}{\ln(l)}[\ln(l)/\ln(2)] + \frac{\log(10)}{\ln(l)}\\
		&\leq \frac{\ln(10) + 1}{\ln(2)}
	\end{align*}
	
	
	the ratio
	is approximately $\ln(10)/\ln(2)$.  Note that since $16 > 10 > 8$, then the ratio 
	is bounded above by $\ln(16)/\ln(2) = 4 (\ln(2/\ln(2)) = 4$.  In the limit, the 
	ratio should approximate the quantity $\ln(10)/\ln(2)$.  
	\item[1.3] The $i$-th layer in the tree can store $d^{i-1}$ by itself.  Therefore 
	if $n$, where $n$ is the number of nodes in this rooted tree, is equal to 
	$1 + d + d^2 + \cdots + d^l$, then each one will be fully contained.  Furthermore,
	each one will be contained in exactly 1 layer.  Since we have $l+1$ layers, then 
	this will exactly be $\log(d^l) + 1$.  Furthermore if a number $n$ is between 
	$d^l < n < d^{l+1}$, then it will half occupy the $l$-th layer.  Therefore the 
	most efficent, and lower bounded, packing is $\log$.  A number must be in the 
	partial sum $\sum_{i=0}^n d^0$.  This is exactly equal to $\frac{d^{n+1}-1}{d-l}$.
	Therefore for a given number $n$, one can find the largest $l$ such that 
	$\frac{d^{l+1}-1}{d-l} < n$.  Then one can find the smallest tree height via 
	$\log_d(\frac{d^{l+1}-1}{d-l})$.    
	
	\item[1.4] 
	\begin{itemize}
		\item To show that $\log(n!) = O(n \log n)$, note that $n!$ = $1\cdot2\cdots n
		\leq n \cdot n \cdots n$ where it's $n$ n's, since $i \leq n$ for all 
		$i \in [n]$.  Since $\log$ is increasing then $\log(n!) \leq \log(n^n) = n \log(n)$.  Therefore $\log(n!) = O(n \log n)$.
		\item To show that $\log(n!) = \Omega(n \log n)$, note that $(n/2)^{n/2}\leq n!$
		since $(n/2)^{n/2} \leq \frac{n}{2}(\frac{n}{2} +1)\cdots n \leq n!$.  Therefore
		$\frac{n}{2}(\log(n)-\log(2)) \leq \log(n!)$.  Since this is a constant factor 
		off of $n \log  n$, and $\frac{\log 2}{n \log n} \to 0$ as $n \to \infty$ 
		implies that $\log(n!) = \Omega(n \log n)$.
	\end{itemize}
	Since $\log n! = O(n \log n) = \Omega(n \log n)$ implies that $\log n! = 
	\Theta(n \log n)$.
	\item[1.10]Suppose $a \equiv b \mod{N}$, then there exists $k \in \Z$ such that 
	$kN = b - a$.  Since $M \mid N$, then there exists $n \in \Z$ such that 
	$nM = N$, therefore $(kn)M = b - a$.  Thus $M \mid b - a$ by definition.
	\item[1.16]	If $b = 2^l$, then $a^b = a^{2^l} = (a^2)^l \mod{c}$.  Therefore 
	the number of multiplications needed is $\log(b) + 1$.  
	\item[1.18]
	\begin{itemize}
		\item Factorization: $210 = 2 \cdot 3 \cdot 5 \cdot 7$, 
		$588 = 2^2 \cdot 3 \cdot 7^2$.  Note that $\gcd(a,b) = p_1^{\min(\alpha_1,\beta_1)}\cdots p_n^{\min(\alpha_n,\beta_n)}$ where $a = p_1^{\alpha_1}\cdots p_n^{\alpha_n}, b = p_1^{\beta_1}\cdots p_n^{\beta_n}$.  Therefore 
		$\gcd(a,b) = 2 \cdot 3 \cdot 7 = 42$
		\item Euclidean algorithm: 
		\begin{align*}
			588 &= 2 \cdot 210 + 168\\
			210 &= 1\cdot 168 + 42\\
			168 &= 4 \cdot 42 + 0
		\end{align*}
		Therefore $42$ is the $\gcd$.
	\end{itemize}
	\item[1.18e] Extended Euclidean algorithm: 
	\begin{align*}
		42 &= (210 - 168)\\
		&= (210 - (588 - 2 * 210))\\
		&= 3*210 + (-1)*588
	\end{align*}
	Note that this isn't exactly the algorithm, but the algebra which the algorithm
	abstracts away and recurses through.  
	\item[1.26] We want to compute $17^{17^{17}} \mod 10$.  Since 
	$\phi(10) = \phi(2)\phi(5) = 4$ then we need to figure out what $17^{17} \mod 4$ is.
	Note that $\phi(4) = 4 - 2$ therefore we have to compute $17\mod 2 \equiv 1$.  
	Thus $17^{17} \mod 4 \equiv 17 \mod 4 \equiv 3 \mod 4$.  Therefore 
	$17^{17^{17}} \equiv 17^3 \mod 10$.   Note that 
	$17^3 \equiv (17 \mod 10)^3 \equiv 7^3 = 343 \equiv 3 \mod{10}$.  
	Therefore $17^{17^{17}} \equiv 3 \mod 10$.  
\end{enumerate}
\end{document}
